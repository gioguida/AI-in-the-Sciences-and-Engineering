% =============================================
% SECTION 5: DATA LOADING & PREPROCESSING
% =============================================

\section{Data Loading \& Preprocessing}

\subsection{Introduction: Why Proper Data Loading Matters}

So far, we've used simple tensors for data. In practice, you'll have:
\begin{itemize}
    \item Large datasets that don't fit in memory
    \item Complex preprocessing pipelines
    \item Data on disk (files, databases)
    \item Need for efficient batching and shuffling
    \item Different data types (images, time series, point clouds, meshes)
\end{itemize}

PyTorch provides \texttt{Dataset} and \texttt{DataLoader} to handle this elegantly:
\begin{itemize}
    \item \textbf{Dataset:} Defines how to access individual samples
    \item \textbf{DataLoader:} Handles batching, shuffling, parallel loading
\end{itemize}

\subsection{Theory: Dataset and DataLoader Architecture}

\begin{theorybox}[The Data Pipeline]
\begin{verbatim}
Raw Data → Dataset → DataLoader → Model
           (access)  (batch, shuffle, parallel)
\end{verbatim}

\textbf{Dataset} answers: "How do I get sample $i$?"

\textbf{DataLoader} answers: "How do I create batches efficiently?"
\end{theorybox}

\textbf{Dataset Requirements:}

A valid dataset must implement:
\begin{enumerate}
    \item \texttt{\_\_len\_\_()}: Return total number of samples
    \item \texttt{\_\_getitem\_\_(idx)}: Return sample at index \texttt{idx}
\end{enumerate}

\textbf{DataLoader Benefits:}
\begin{itemize}
    \item Automatic batching
    \item Shuffling for training
    \item Parallel data loading (multiple workers)
    \item Automatic memory pinning for GPU transfer
    \item Handling variable-length sequences
\end{itemize}

\subsection{Implementation: Basic Usage}

\subsubsection{Using Built-in Datasets}

PyTorch provides common datasets in \texttt{torchvision.datasets}:

\begin{lstlisting}
import torch
from torch.utils.data import DataLoader
import torchvision
import torchvision.transforms as transforms

# Load MNIST dataset
transform = transforms.ToTensor()  # Convert PIL Image to tensor

train_dataset = torchvision.datasets.MNIST(
    root='./data',
    train=True,
    download=True,
    transform=transform
)

test_dataset = torchvision.datasets.MNIST(
    root='./data',
    train=False,
    download=True,
    transform=transform
)

# Create DataLoaders
train_loader = DataLoader(
    train_dataset,
    batch_size=64,
    shuffle=True,      # Shuffle training data
    num_workers=2,     # Parallel loading
    pin_memory=True    # Faster GPU transfer
)

test_loader = DataLoader(
    test_dataset,
    batch_size=64,
    shuffle=False,     # Don't shuffle test data
    num_workers=2
)

# Iterate through batches
for batch_idx, (images, labels) in enumerate(train_loader):
    # images: (64, 1, 28, 28) - batch of 64 grayscale 28x28 images
    # labels: (64,) - batch of 64 labels
    print(f"Batch {batch_idx}: {images.shape}, {labels.shape}")
    break
\end{lstlisting}

\begin{pytorchtip}[DataLoader Parameters]
Key parameters:
\begin{itemize}
    \item \textbf{batch\_size:} Number of samples per batch (32-256 typically)
    \item \textbf{shuffle:} Randomize order (True for training, False for evaluation)
    \item \textbf{num\_workers:} Parallel processes for loading (0=main process, 2-4 for speed)
    \item \textbf{pin\_memory:} Faster CPU→GPU transfer (True if using GPU)
    \item \textbf{drop\_last:} Drop incomplete last batch (useful for batch norm)
\end{itemize}
\end{pytorchtip}

\clearpage
\subsubsection{Creating Custom Datasets}

For your own data, create a custom \texttt{Dataset}:

\begin{lstlisting}
from torch.utils.data import Dataset

class SimpleDataset(Dataset):
    """Example: Dataset from numpy arrays."""
    
    def __init__(self, X, y):
        """
        Args:
            X: numpy array or tensor of features
            y: numpy array or tensor of labels
        """
        self.X = torch.FloatTensor(X) if not isinstance(X, torch.Tensor) else X
        self.y = torch.FloatTensor(y) if not isinstance(y, torch.Tensor) else y
    
    def __len__(self):
        """Return total number of samples."""
        return len(self.X)
    
    def __getitem__(self, idx):
        """Return sample at index idx."""
        return self.X[idx], self.y[idx]

# Usage
import numpy as np

X = np.random.randn(1000, 10)
y = np.random.randn(1000, 1)

dataset = SimpleDataset(X, y)
loader = DataLoader(dataset, batch_size=32, shuffle=True)

for batch_X, batch_y in loader:
    print(batch_X.shape, batch_y.shape)
    break
\end{lstlisting}

\subsubsection{Dataset from Files}

More realistic: loading data from disk:

\begin{lstlisting}
import os
import pandas as pd
from PIL import Image

class ImageDataset(Dataset):
    """Load images from a directory with CSV labels."""
    
    def __init__(self, csv_file, img_dir, transform=None):
        """
        Args:
            csv_file: Path to CSV with columns [filename, label]
            img_dir: Directory with images
            transform: Optional transform to apply
        """
        self.data = pd.read_csv(csv_file)
        self.img_dir = img_dir
        self.transform = transform
    
    def __len__(self):
        return len(self.data)
    
    def __getitem__(self, idx):
        # Get image filename and label
        img_name = self.data.iloc[idx, 0]
        label = self.data.iloc[idx, 1]
        
        # Load image
        img_path = os.path.join(self.img_dir, img_name)
        image = Image.open(img_path).convert('RGB')
        
        # Apply transforms
        if self.transform:
            image = self.transform(image)
        
        return image, label

# Usage with transforms
transform = transforms.Compose([
    transforms.Resize((224, 224)),
    transforms.ToTensor(),
    transforms.Normalize(mean=[0.485, 0.456, 0.406],
                        std=[0.229, 0.224, 0.225])
])

dataset = ImageDataset('labels.csv', 'images/', transform=transform)
loader = DataLoader(dataset, batch_size=32, shuffle=True)
\end{lstlisting}

\subsubsection{Dataset for Time Series}

For scientific computing, time series are common:

\begin{lstlisting}
class TimeSeriesDataset(Dataset):
    """
    Create windows from a long time series.
    Useful for forecasting, where we predict future from past.
    """
    
    def __init__(self, data, window_size, forecast_horizon):
        """
        Args:
            data: 1D array of time series values
            window_size: Length of input sequence
            forecast_horizon: How many steps ahead to predict
        """
        self.data = torch.FloatTensor(data)
        self.window_size = window_size
        self.forecast_horizon = forecast_horizon
    
    def __len__(self):
        # Number of valid windows
        return len(self.data) - self.window_size - self.forecast_horizon + 1
    
    def __getitem__(self, idx):
        # Input: window_size points starting at idx
        x = self.data[idx : idx + self.window_size]
        
        # Target: forecast_horizon points after the input
        y = self.data[idx + self.window_size : 
                     idx + self.window_size + self.forecast_horizon]
        
        return x, y

# Example: predict next 10 steps from past 50
data = np.sin(np.linspace(0, 100, 1000))
dataset = TimeSeriesDataset(data, window_size=50, forecast_horizon=10)

print(f"Dataset size: {len(dataset)}")
x, y = dataset[0]
print(f"Input shape: {x.shape}, Target shape: {y.shape}")
# Input shape: torch.Size([50]), Target shape: torch.Size([10])

loader = DataLoader(dataset, batch_size=32, shuffle=False)
# Note: Don't shuffle time series if order matters!
\end{lstlisting}

\clearpage
\subsubsection{Dataset for Point Clouds}

For 3D data (molecular structures, point clouds):

\begin{lstlisting}
class PointCloudDataset(Dataset):
    """Dataset for 3D point clouds."""
    
    def __init__(self, data_dir, num_points=1024):
        """
        Args:
            data_dir: Directory with .npy files, each containing Nx3 points
            num_points: Sample this many points from each cloud
        """
        self.files = [os.path.join(data_dir, f) 
                     for f in os.listdir(data_dir) if f.endswith('.npy')]
        self.num_points = num_points
    
    def __len__(self):
        return len(self.files)
    
    def __getitem__(self, idx):
        # Load point cloud (N x 3)
        points = np.load(self.files[idx])
        
        # Sample fixed number of points
        if len(points) > self.num_points:
            # Random sampling
            indices = np.random.choice(len(points), self.num_points, 
                                      replace=False)
            points = points[indices]
        elif len(points) < self.num_points:
            # Pad with zeros
            padding = np.zeros((self.num_points - len(points), 3))
            points = np.vstack([points, padding])
        
        return torch.FloatTensor(points)

# Usage
# dataset = PointCloudDataset('point_clouds/')
# loader = DataLoader(dataset, batch_size=8, shuffle=True)
# for batch in loader:
#     print(batch.shape)  # (8, 1024, 3)
\end{lstlisting}

\subsubsection{Data Transforms}

Transforms process data on-the-fly:

\begin{lstlisting}
from torchvision import transforms

# Common image transforms
transform = transforms.Compose([
    # Resize to fixed size
    transforms.Resize((224, 224)),
    
    # Data augmentation (for training)
    transforms.RandomHorizontalFlip(p=0.5),
    transforms.RandomRotation(degrees=15),
    transforms.ColorJitter(brightness=0.2, contrast=0.2),
    
    # Convert to tensor
    transforms.ToTensor(),  # Converts to [0, 1] and (C, H, W)
    
    # Normalize (important!)
    transforms.Normalize(mean=[0.485, 0.456, 0.406],  # ImageNet stats
                        std=[0.229, 0.224, 0.225])
])

# For evaluation/test: no data augmentation
test_transform = transforms.Compose([
    transforms.Resize((224, 224)),
    transforms.ToTensor(),
    transforms.Normalize(mean=[0.485, 0.456, 0.406],
                        std=[0.229, 0.224, 0.225])
])

# Custom transform as a callable
class AddGaussianNoise:
    """Add Gaussian noise to tensor."""
    def __init__(self, mean=0., std=0.1):
        self.mean = mean
        self.std = std
    
    def __call__(self, tensor):
        return tensor + torch.randn(tensor.size()) * self.std + self.mean

# Use in Compose
transform = transforms.Compose([
    transforms.ToTensor(),
    AddGaussianNoise(std=0.05)
])
\end{lstlisting}

\begin{pytorchtip}[When to Apply Transforms]
Transforms can be applied in two places:

\textbf{In Dataset (\_\_getitem\_\_):}
\begin{itemize}
    \item Pro: Each sample transformed on-the-fly
    \item Pro: Can use data augmentation
    \item Con: Adds computation during training
\end{itemize}

\textbf{Pre-computed and saved:}
\begin{itemize}
    \item Pro: Faster training (no transform overhead)
    \item Con: More disk space
    \item Con: Can't use random augmentation
\end{itemize}

For training with augmentation: use Dataset transforms.
For inference: consider pre-computing.
\end{pytorchtip}

\clearpage
\subsubsection{Normalization Strategies}

Normalization is crucial for training stability:

\begin{lstlisting}
# Method 1: Standardization (zero mean, unit variance)
def compute_mean_std(dataset):
    """Compute mean and std for normalization."""
    loader = DataLoader(dataset, batch_size=64, shuffle=False)
    
    mean = 0.0
    std = 0.0
    total_samples = 0
    
    for data, _ in loader:
        batch_samples = data.size(0)
        data = data.view(batch_samples, data.size(1), -1)
        mean += data.mean(2).sum(0)
        std += data.std(2).sum(0)
        total_samples += batch_samples
    
    mean /= total_samples
    std /= total_samples
    
    return mean, std

# Usage
# mean, std = compute_mean_std(train_dataset)
# transform = transforms.Normalize(mean=mean, std=std)

# Method 2: Min-Max scaling to [0, 1] or [-1, 1]
class MinMaxScaler:
    def __init__(self, data):
        self.min = data.min()
        self.max = data.max()
    
    def transform(self, data):
        return (data - self.min) / (self.max - self.min)
    
    def inverse_transform(self, data):
        return data * (self.max - self.min) + self.min

# Method 3: Per-feature normalization (for tabular data)
class StandardScaler:
    def __init__(self, data):
        """
        data: (N, D) tensor
        """
        self.mean = data.mean(dim=0)
        self.std = data.std(dim=0)
    
    def transform(self, data):
        return (data - self.mean) / (self.std + 1e-8)
    
    def inverse_transform(self, data):
        return data * self.std + self.mean

# Usage
X_train = torch.randn(1000, 10)
scaler = StandardScaler(X_train)
X_train_normalized = scaler.transform(X_train)
X_test_normalized = scaler.transform(X_test)  # Use training statistics!
\end{lstlisting}

\begin{warningbox}[Always Use Training Statistics]
When normalizing:
\begin{enumerate}
    \item Compute mean/std from \textbf{training data only}
    \item Apply the same transformation to validation and test data
    \item \textbf{Never} compute statistics from test data!
\end{enumerate}

Why? Using test statistics is a form of data leakage and gives overly optimistic results.
\end{warningbox}

\subsubsection{Handling Variable-Length Sequences}

For sequences of different lengths (text, time series with variable duration):

\begin{lstlisting}
from torch.nn.utils.rnn import pad_sequence, pack_padded_sequence

def collate_fn(batch):
    """
    Custom collate function for variable-length sequences.
    
    batch: List of (sequence, label) tuples
    sequences: List of tensors with different lengths
    """
    sequences, labels = zip(*batch)
    
    # Get lengths
    lengths = torch.LongTensor([len(seq) for seq in sequences])
    
    # Pad sequences to same length
    sequences_padded = pad_sequence(sequences, batch_first=True, 
                                    padding_value=0)
    
    labels = torch.LongTensor(labels)
    
    return sequences_padded, labels, lengths

# Example dataset with variable lengths
class VariableLengthDataset(Dataset):
    def __init__(self, num_samples=100):
        self.data = []
        for _ in range(num_samples):
            # Random length between 10 and 50
            length = torch.randint(10, 50, (1,)).item()
            seq = torch.randn(length, 5)  # 5 features
            label = torch.randint(0, 2, (1,)).item()
            self.data.append((seq, label))
    
    def __len__(self):
        return len(self.data)
    
    def __getitem__(self, idx):
        return self.data[idx]

# Use with custom collate_fn
dataset = VariableLengthDataset()
loader = DataLoader(dataset, batch_size=8, collate_fn=collate_fn)

for sequences, labels, lengths in loader:
    print(f"Padded sequences: {sequences.shape}")  # (8, max_len, 5)
    print(f"Lengths: {lengths}")  # (8,)
    break
\end{lstlisting}

\clearpage
\subsubsection{Train/Validation/Test Split}

Proper data splitting:

\begin{lstlisting}
from torch.utils.data import random_split

# Method 1: random_split
dataset = SimpleDataset(X, y)
train_size = int(0.7 * len(dataset))
val_size = int(0.15 * len(dataset))
test_size = len(dataset) - train_size - val_size

train_dataset, val_dataset, test_dataset = random_split(
    dataset, [train_size, val_size, test_size]
)

# Method 2: Using indices (more control)
from torch.utils.data import Subset
import numpy as np

indices = np.arange(len(dataset))
np.random.shuffle(indices)

train_idx = indices[:train_size]
val_idx = indices[train_size:train_size+val_size]
test_idx = indices[train_size+val_size:]

train_dataset = Subset(dataset, train_idx)
val_dataset = Subset(dataset, val_idx)
test_dataset = Subset(dataset, test_idx)

# Method 3: sklearn for stratified split (classification)
from sklearn.model_selection import train_test_split

X = np.random.randn(1000, 10)
y = np.random.randint(0, 3, 1000)  # 3 classes

X_train, X_temp, y_train, y_temp = train_test_split(
    X, y, test_size=0.3, stratify=y, random_state=42
)
X_val, X_test, y_val, y_test = train_test_split(
    X_temp, y_temp, test_size=0.5, stratify=y_temp, random_state=42
)

train_dataset = SimpleDataset(X_train, y_train)
val_dataset = SimpleDataset(X_val, y_val)
test_dataset = SimpleDataset(X_test, y_test)
\end{lstlisting}

\subsubsection{Efficient Data Loading}

For large datasets that don't fit in RAM:

\begin{lstlisting}
class LazyLoadDataset(Dataset):
    """Load data on-demand, don't store in memory."""
    
    def __init__(self, file_list):
        """
        file_list: List of file paths
        """
        self.file_list = file_list
    
    def __len__(self):
        return len(self.file_list)
    
    def __getitem__(self, idx):
        # Load from disk only when needed
        data = np.load(self.file_list[idx])
        return torch.FloatTensor(data)

# For even larger datasets: memory-mapped arrays
class MemMapDataset(Dataset):
    """Use memory-mapped arrays for huge datasets."""
    
    def __init__(self, memmap_file, shape, dtype):
        """
        memmap_file: Path to .npy file
        shape: Shape of the data array
        dtype: Data type
        """
        self.data = np.memmap(memmap_file, dtype=dtype, 
                             mode='r', shape=shape)
    
    def __len__(self):
        return self.data.shape[0]
    
    def __getitem__(self, idx):
        # Read only the needed slice from disk
        return torch.FloatTensor(self.data[idx])
\end{lstlisting}

\begin{pytorchtip}[Optimizing DataLoader Performance]
If data loading is slow:

\begin{enumerate}
    \item \textbf{Increase num\_workers:} 2-4 usually optimal
    \item \textbf{Use pin\_memory=True:} If using GPU
    \item \textbf{Prefetch:} DataLoader automatically prefetches next batch
    \item \textbf{Reduce transforms:} Pre-compute expensive transforms
    \item \textbf{Use SSD:} Faster disk I/O
    \item \textbf{Check CPU bottleneck:} If GPU is idle, data loading is the issue
\end{enumerate}

Warning: Too many workers can hurt performance (overhead).
\end{pytorchtip}

\clearpage
\subsection{Exercises}

\begin{exercise}[5.1: Simple Dataset - $\bigstar\bigstar$]
\textbf{Goal:} Create your first custom dataset.

\begin{enumerate}
    \item Create a synthetic dataset: $y = 2x_1 + 3x_2 + \epsilon$
    \item Generate 1000 samples with $x_1, x_2 \sim \mathcal{N}(0, 1)$ and noise $\epsilon \sim \mathcal{N}(0, 0.1)$
    \item Implement a custom \texttt{Dataset} class
    \item Create a \texttt{DataLoader} with batch\_size=32
    \item Iterate through one epoch and print batch shapes
\end{enumerate}
\end{exercise}

\begin{exercise}[5.2: Train/Val/Test Split - $\bigstar\bigstar$]
\textbf{Goal:} Practice proper data splitting.

\begin{enumerate}
    \item Generate a classification dataset (1000 samples, 3 classes)
    \item Split into train (70\%), validation (15\%), test (15\%)
    \item Ensure the split is stratified (balanced classes in each split)
    \item Create separate DataLoaders for each split
    \item Verify class distributions are similar
\end{enumerate}

\textbf{Hint:} Use \texttt{sklearn.model\_selection.train\_test\_split} with \texttt{stratify}.
\end{exercise}

\begin{exercise}[5.3: Time Series Dataset - $\bigstar\bigstar\bigstar$]
\textbf{Goal:} Handle sequential data.

\begin{enumerate}
    \item Generate a noisy sine wave with 10,000 points
    \item Create a dataset that returns windows of size 50 to predict the next 10 points
    \item Implement the \texttt{TimeSeriesDataset} class
    \item Train a simple MLP to predict future values
    \item Visualize predictions vs ground truth
\end{enumerate}

\textbf{Challenge:} Try with a real time series (stock prices, weather data).
\end{exercise}

\begin{exercise}[5.4: Data Normalization - $\bigstar\bigstar\bigstar$]
\textbf{Goal:} Understand normalization impact.

\begin{enumerate}
    \item Create a dataset with features on different scales:
    \begin{itemize}
        \item Feature 1: range [0, 1]
        \item Feature 2: range [0, 1000]
        \item Feature 3: range [-100, 100]
    \end{itemize}
    \item Train a model \textbf{without} normalization
    \item Train a model \textbf{with} standardization
    \item Compare:
    \begin{itemize}
        \item Training speed (epochs to converge)
        \item Final loss
        \item Gradient magnitudes
    \end{itemize}
\end{enumerate}
\end{exercise}

\begin{exercise}[5.5: Data Augmentation - $\bigstar\bigstar\bigstar$]
\textbf{Goal:} Implement custom transforms.

\begin{enumerate}
    \item Create a simple image dataset (or use MNIST)
    \item Implement custom transforms:
    \begin{itemize}
        \item Add Gaussian noise
        \item Random scaling
        \item Random shifts
    \end{itemize}
    \item Train models with and without augmentation
    \item Compare validation performance
    \item Visualize augmented samples
\end{enumerate}
\end{exercise}

\begin{exercise}[5.6: Variable-Length Sequences - $\bigstar\bigstar\bigstar\bigstar$]
\textbf{Goal:} Handle sequences of different lengths.

\begin{enumerate}
    \item Create a dataset of sequences with random lengths (10-100)
    \item Implement a custom \texttt{collate\_fn} to pad sequences
    \item Create a DataLoader using this collate function
    \item Train an RNN that handles the padded sequences
    \item Use \texttt{pack\_padded\_sequence} for efficiency (we'll learn this properly in the RNN section)
\end{enumerate}

\textbf{Hint:} The collate function should return sequences, lengths, and labels.
\end{exercise}

\begin{exercise}[5.7: Complete Data Pipeline - $\bigstar\bigstar\bigstar\bigstar$]
\textbf{Goal:} Build a production-ready data pipeline.

Create a complete pipeline with:
\begin{enumerate}
    \item Custom dataset loading from CSV files
    \item Train/val/test split with stratification
    \item Different transforms for train (with augmentation) and test
    \item Normalization using training statistics
    \item Efficient DataLoaders with multiple workers
    \item Save and load normalization parameters
    \item Handle class imbalance with weighted sampling (optional)
\end{enumerate}

Test on a real dataset (e.g., UCI datasets, Kaggle).
\end{exercise}

\clearpage