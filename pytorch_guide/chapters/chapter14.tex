% =============================================
% SECTION 14: TRAINING BEST PRACTICES
% =============================================

\section{Training Best Practices}

\subsection{Introduction: From Working to Working Well}

You've built a model that trains. Now: how do you make it train \textbf{well}?

This section covers practical strategies for:
\begin{itemize}
    \item Choosing hyperparameters
    \item Learning rate strategies
    \item Efficient training
    \item Avoiding common pitfalls
\end{itemize}

\subsection{Hyperparameter Tuning}

\subsubsection{Which Hyperparameters Matter Most?}

\textbf{Priority order:}

\textbf{1. Learning rate (MOST IMPORTANT)}

Can make 10× difference in performance. Get this right first.

\textbf{2. Batch size}

Affects both training speed and generalization.

\textbf{3. Architecture (width, depth)}

Number of layers and neurons per layer.

\textbf{4. Regularization (dropout, weight decay)}

Prevents overfitting.

\textbf{5. Learning rate schedule}

How LR changes over time.

\textbf{6. Everything else}

Optimizer choice, activation functions, initialization—usually matter less.

\subsubsection{Learning Rate Strategies}

\textbf{Strategy 1: Learning Rate Finder}

Find optimal LR before training:

\begin{lstlisting}
from torch_lr_finder import LRFinder

model = YourModel()
optimizer = torch.optim.Adam(model.parameters(), lr=1e-7)
criterion = nn.CrossEntropyLoss()

lr_finder = LRFinder(model, optimizer, criterion, device="cuda")
lr_finder.range_test(train_loader, end_lr=10, num_iter=100)
lr_finder.plot()  # Shows loss vs LR

# Pick LR where loss decreases fastest (steepest slope)
# Usually 10× smaller than where loss explodes
optimal_lr = 1e-3  # Read from plot

lr_finder.reset()  # Reset model and optimizer
\end{lstlisting}

\textbf{Strategy 2: Start Large, Decay}

\begin{lstlisting}
# Common schedules

# 1. Step decay (reduce every N epochs)
scheduler = torch.optim.lr_scheduler.StepLR(
    optimizer, step_size=30, gamma=0.1
)
# LR × 0.1 every 30 epochs

# 2. Exponential decay
scheduler = torch.optim.lr_scheduler.ExponentialLR(
    optimizer, gamma=0.95
)
# LR × 0.95 every epoch

# 3. Cosine annealing (smooth decay)
scheduler = torch.optim.lr_scheduler.CosineAnnealingLR(
    optimizer, T_max=100, eta_min=1e-6
)
# Smooth cosine decay from initial_lr to eta_min over T_max epochs

# 4. Reduce on plateau (adaptive)
scheduler = torch.optim.lr_scheduler.ReduceLROnPlateau(
    optimizer, mode='min', factor=0.5, patience=10
)
# Reduce LR by 0.5 if val loss doesn't improve for 10 epochs

# Usage
for epoch in range(num_epochs):
    train_loss = train_epoch()
    val_loss = validate()
    
    # For ReduceLROnPlateau, pass validation loss
    scheduler.step(val_loss)  # or scheduler.step() for others
\end{lstlisting}

\textbf{Strategy 3: Warmup + Cosine Decay}

State-of-the-art for Transformers:

\begin{lstlisting}
from torch.optim.lr_scheduler import LambdaLR

def get_cosine_schedule_with_warmup(optimizer, num_warmup_steps, 
                                    num_training_steps):
    """Warmup + cosine decay."""
    
    def lr_lambda(current_step):
        # Warmup
        if current_step < num_warmup_steps:
            return float(current_step) / float(max(1, num_warmup_steps))
        
        # Cosine decay
        progress = float(current_step - num_warmup_steps) / \
                  float(max(1, num_training_steps - num_warmup_steps))
        return max(0.0, 0.5 * (1.0 + math.cos(math.pi * progress)))
    
    return LambdaLR(optimizer, lr_lambda)

# Usage
num_epochs = 100
steps_per_epoch = len(train_loader)
num_training_steps = num_epochs * steps_per_epoch
num_warmup_steps = 5 * steps_per_epoch  # 5 epochs warmup

optimizer = torch.optim.AdamW(model.parameters(), lr=1e-3)
scheduler = get_cosine_schedule_with_warmup(
    optimizer, num_warmup_steps, num_training_steps
)

for epoch in range(num_epochs):
    for batch in train_loader:
        # Training step
        ...
        optimizer.step()
        scheduler.step()  # Update LR every step, not epoch!
\end{lstlisting}

\clearpage
\subsubsection{Batch Size Selection}

\textbf{Trade-offs:}

\begin{table}[h]
\centering
\begin{tabular}{lll}
\toprule
\textbf{Batch Size} & \textbf{Pros} & \textbf{Cons} \\
\midrule
Small (8-32) & Better generalization & Slower, noisy gradients \\
             & Less memory & Batch norm unstable \\
Medium (64-256) & Good balance & Standard choice \\
Large (512+) & Faster training & Worse generalization \\
             & More stable & Requires careful tuning \\
\bottomrule
\end{tabular}
\caption{Batch size trade-offs}
\end{table}

\textbf{Rules of thumb:}
\begin{itemize}
    \item \textbf{Start with 32-128:} Usually works well
    \item \textbf{Larger batches need higher LR:} Scale by $\sqrt{\text{batch size}}$
    \item \textbf{Use gradient accumulation} if memory limited
    \item \textbf{Don't go below 8} if using batch norm
\end{itemize}

\textbf{Linear scaling rule:}

If you double batch size, double learning rate (approximately).

\begin{lstlisting}
# Base setup
batch_size = 64
lr = 0.001

# Double batch size
batch_size = 128
lr = 0.002  # Double LR

# But: this is approximate! Always validate on your task
\end{lstlisting}

\subsubsection{Model Architecture Selection}

\textbf{Width vs Depth:}

\begin{lstlisting}
# Wider networks (fewer layers, more neurons)
model_wide = MLP([100, 1024, 1024, 10])  # 2 hidden layers, 1024 neurons

# Deeper networks (more layers, fewer neurons)
model_deep = MLP([100, 256, 256, 256, 256, 10])  # 4 hidden layers, 256 neurons

# Modern preference: deeper is usually better (with skip connections)
\end{lstlisting}

\textbf{Start small, scale up:}

\begin{enumerate}
    \item Start with small model that trains fast
    \item Check if it underfits (high train loss)
    \item If underfitting: increase capacity
    \item If overfitting: add regularization
\end{enumerate}

\subsection{Regularization Strategies}

\subsubsection{Dropout}

\begin{lstlisting}
# Standard dropout
model = nn.Sequential(
    nn.Linear(100, 256),
    nn.ReLU(),
    nn.Dropout(0.5),  # Drop 50% during training
    nn.Linear(256, 10)
)

# Guidelines:
# - Start with 0.5, tune if needed
# - Higher dropout = more regularization
# - Use lower dropout (0.1-0.2) for convolutional layers
# - Higher dropout (0.5-0.7) for large fully connected layers
\end{lstlisting}

\subsubsection{Weight Decay (L2 Regularization)}

\begin{lstlisting}
# Add weight decay to optimizer
optimizer = torch.optim.Adam(
    model.parameters(),
    lr=1e-3,
    weight_decay=1e-4  # L2 penalty
)

# Typical values: 1e-5 to 1e-3
# Larger weight decay = stronger regularization

# Note: Use AdamW for better weight decay
optimizer = torch.optim.AdamW(model.parameters(), lr=1e-3, 
                             weight_decay=1e-4)
\end{lstlisting}

\subsubsection{Data Augmentation}

\begin{lstlisting}
# For images
train_transform = transforms.Compose([
    transforms.RandomHorizontalFlip(),
    transforms.RandomCrop(32, padding=4),
    transforms.ColorJitter(brightness=0.2, contrast=0.2),
    transforms.RandomRotation(15),
    transforms.ToTensor(),
    transforms.Normalize(mean=[0.485, 0.456, 0.406],
                        std=[0.229, 0.224, 0.225])
])

# NO augmentation for validation/test
val_transform = transforms.Compose([
    transforms.ToTensor(),
    transforms.Normalize(mean=[0.485, 0.456, 0.406],
                        std=[0.229, 0.224, 0.225])
])
\end{lstlisting}

\clearpage
\subsection{Training Monitoring}

\subsubsection{What to Track}

\textbf{Essential metrics:}
\begin{itemize}
    \item Training loss (per epoch)
    \item Validation loss (per epoch)
    \item Validation accuracy/metric (per epoch)
    \item Learning rate (per epoch or step)
\end{itemize}

\textbf{Useful but optional:}
\begin{itemize}
    \item Gradient norms
    \item Weight norms
    \item Training time per epoch
    \item GPU memory usage
\end{itemize}

\begin{lstlisting}
# Complete training loop with monitoring
from torch.utils.tensorboard import SummaryWriter

writer = SummaryWriter('runs/experiment')

for epoch in range(num_epochs):
    # Training
    model.train()
    train_loss = 0
    train_correct = 0
    train_total = 0
    
    for batch in train_loader:
        optimizer.zero_grad()
        
        x, y = batch
        x, y = x.to(device), y.to(device)
        
        output = model(x)
        loss = criterion(output, y)
        loss.backward()
        
        # Track gradient norm
        grad_norm = torch.nn.utils.clip_grad_norm_(
            model.parameters(), max_norm=1.0
        )
        
        optimizer.step()
        
        train_loss += loss.item()
        pred = output.argmax(dim=1)
        train_correct += (pred == y).sum().item()
        train_total += y.size(0)
    
    # Validation
    model.eval()
    val_loss = 0
    val_correct = 0
    val_total = 0
    
    with torch.no_grad():
        for batch in val_loader:
            x, y = batch
            x, y = x.to(device), y.to(device)
            
            output = model(x)
            loss = criterion(output, y)
            
            val_loss += loss.item()
            pred = output.argmax(dim=1)
            val_correct += (pred == y).sum().item()
            val_total += y.size(0)
    
    # Compute metrics
    train_loss /= len(train_loader)
    train_acc = train_correct / train_total
    val_loss /= len(val_loader)
    val_acc = val_correct / val_total
    
    # Log to TensorBoard
    writer.add_scalar('Loss/train', train_loss, epoch)
    writer.add_scalar('Loss/val', val_loss, epoch)
    writer.add_scalar('Accuracy/train', train_acc, epoch)
    writer.add_scalar('Accuracy/val', val_acc, epoch)
    writer.add_scalar('LR', optimizer.param_groups[0]['lr'], epoch)
    writer.add_scalar('GradNorm', grad_norm, epoch)
    
    # Print
    print(f"Epoch {epoch}: Train Loss={train_loss:.4f}, "
          f"Val Loss={val_loss:.4f}, Val Acc={val_acc:.4f}")
    
    # Step scheduler
    scheduler.step()

writer.close()
\end{lstlisting}

\subsubsection{Recognizing Training Issues from Curves}

\textbf{Healthy training:}
\begin{verbatim}
Train loss: Decreasing smoothly
Val loss: Decreasing, tracking train loss
Gap: Small (< 10% difference)
\end{verbatim}

\textbf{Overfitting:}
\begin{verbatim}
Train loss: Very low
Val loss: High and increasing
Gap: Large and growing
→ Solution: More regularization, more data
\end{verbatim}

\textbf{Underfitting:}
\begin{verbatim}
Train loss: High
Val loss: High (similar to train)
Gap: Small
→ Solution: Bigger model, train longer
\end{verbatim}

\textbf{Unstable training:}
\begin{verbatim}
Both losses: Oscillating wildly
→ Solution: Lower learning rate, gradient clipping
\end{verbatim}

\subsection{Model Selection and Evaluation}

\subsubsection{Proper Train/Val/Test Split}

\begin{lstlisting}
# WRONG: Tune on test set
# This gives overly optimistic results!

# RIGHT: Three separate sets
# 1. Training set (70%): Train model
# 2. Validation set (15%): Tune hyperparameters
# 3. Test set (15%): Final evaluation (ONCE!)

from sklearn.model_selection import train_test_split

# Split data
X_train, X_temp, y_train, y_temp = train_test_split(
    X, y, test_size=0.3, stratify=y, random_state=42
)
X_val, X_test, y_val, y_test = train_test_split(
    X_temp, y_temp, test_size=0.5, stratify=y_temp, random_state=42
)

# Training process:
# 1. Train many models with different hyperparameters
# 2. Evaluate all on validation set
# 3. Pick best model
# 4. Train best model on train+val (optional)
# 5. Evaluate ONCE on test set
# 6. Report test performance
\end{lstlisting}

\subsubsection{Cross-Validation for Small Datasets}

\begin{lstlisting}
from sklearn.model_selection import KFold

kfold = KFold(n_splits=5, shuffle=True, random_state=42)

scores = []

for fold, (train_idx, val_idx) in enumerate(kfold.split(X)):
    print(f"Fold {fold + 1}")
    
    # Split data
    X_train, X_val = X[train_idx], X[val_idx]
    y_train, y_val = y[train_idx], y[val_idx]
    
    # Create datasets
    train_dataset = YourDataset(X_train, y_train)
    val_dataset = YourDataset(X_val, y_val)
    
    # Train model (reset for each fold!)
    model = YourModel()
    optimizer = torch.optim.Adam(model.parameters())
    
    # Train...
    val_score = train_and_evaluate(model, train_dataset, val_dataset)
    scores.append(val_score)

# Report mean and std
print(f"CV Score: {np.mean(scores):.4f} ± {np.std(scores):.4f}")
\end{lstlisting}

\clearpage
\subsection{Practical Tips}

\subsubsection{Start with Good Defaults}

\begin{lstlisting}
# Proven configuration for most tasks:

optimizer = torch.optim.AdamW(
    model.parameters(),
    lr=1e-3,          # Will likely need tuning
    weight_decay=1e-4,
    betas=(0.9, 0.999)
)

scheduler = torch.optim.lr_scheduler.CosineAnnealingLR(
    optimizer, T_max=num_epochs
)

# Initialization: PyTorch defaults are usually good
# But for RNNs, use orthogonal init for hidden-to-hidden

# Batch size: 64 or 128
# Epochs: Start with 100, use early stopping
# Gradient clipping: max_norm=1.0 (for RNNs)
\end{lstlisting}

\subsubsection{Hyperparameter Search Strategies}

\textbf{Grid search (exhaustive):}
\begin{lstlisting}
lrs = [1e-4, 1e-3, 1e-2]
weight_decays = [0, 1e-5, 1e-4, 1e-3]
dropouts = [0.2, 0.5]

best_val_loss = float('inf')
best_config = None

for lr in lrs:
    for wd in weight_decays:
        for dropout in dropouts:
            val_loss = train_and_evaluate(lr, wd, dropout)
            
            if val_loss < best_val_loss:
                best_val_loss = val_loss
                best_config = (lr, wd, dropout)

print(f"Best: LR={best_config[0]}, WD={best_config[1]}, "
      f"Dropout={best_config[2]}")
\end{lstlisting}

\textbf{Random search (more efficient):}
\begin{lstlisting}
import random

def random_config():
    lr = 10 ** random.uniform(-5, -2)  # 1e-5 to 1e-2
    wd = 10 ** random.uniform(-6, -3)  # 1e-6 to 1e-3
    dropout = random.uniform(0.1, 0.7)
    return lr, wd, dropout

num_trials = 20
results = []

for trial in range(num_trials):
    lr, wd, dropout = random_config()
    val_loss = train_and_evaluate(lr, wd, dropout)
    results.append((val_loss, lr, wd, dropout))

# Sort by validation loss
results.sort()
best = results[0]
print(f"Best: Val Loss={best[0]:.4f}, LR={best[1]:.2e}, "
      f"WD={best[2]:.2e}, Dropout={best[3]:.2f}")
\end{lstlisting}

\textbf{Bayesian optimization (most efficient):}
\begin{lstlisting}
# Use libraries like Optuna or Ray Tune
import optuna

def objective(trial):
    # Suggest hyperparameters
    lr = trial.suggest_loguniform('lr', 1e-5, 1e-2)
    wd = trial.suggest_loguniform('weight_decay', 1e-6, 1e-3)
    dropout = trial.suggest_uniform('dropout', 0.1, 0.7)
    
    # Train and return validation loss
    val_loss = train_and_evaluate(lr, wd, dropout)
    return val_loss

# Optimize
study = optuna.create_study(direction='minimize')
study.optimize(objective, n_trials=50)

print(f"Best hyperparameters: {study.best_params}")
print(f"Best validation loss: {study.best_value:.4f}")
\end{lstlisting}

\subsection{Key Takeaways}

\textbf{Hyperparameter priorities:}
\begin{enumerate}
    \item Learning rate (MOST IMPORTANT)
    \item Batch size
    \item Architecture (depth, width)
    \item Regularization (dropout, weight decay)
    \item Everything else
\end{enumerate}

\textbf{Learning rate strategies:}
\begin{itemize}
    \item Use LR finder to find good starting point
    \item Start high, decay over time
    \item Cosine annealing works well
    \item Warmup helps for Transformers
    \item ReduceLROnPlateau is safe fallback
\end{itemize}

\textbf{Batch size selection:}
\begin{itemize}
    \item Start with 32-128
    \item Larger batch = faster but may hurt generalization
    \item Scale LR with batch size (approximately)
    \item Use gradient accumulation if memory limited
\end{itemize}

\textbf{Regularization:}
\begin{itemize}
    \item Dropout: 0.5 for FC layers, 0.1-0.2 for conv layers
    \item Weight decay: 1e-5 to 1e-3
    \item Data augmentation: Always for images
    \item Early stopping: Always use
\end{itemize}

\textbf{Training monitoring:}
\begin{itemize}
    \item Track train/val loss and metrics
    \item Use TensorBoard for visualization
    \item Watch for overfitting (gap between train/val)
    \item Watch for underfitting (both losses high)
\end{itemize}

\textbf{Model selection:}
\begin{itemize}
    \item Always use separate test set
    \item Tune on validation set only
    \item Test set evaluated ONCE at the end
    \item Use cross-validation for small datasets
\end{itemize}

\textbf{Hyperparameter search:}
\begin{itemize}
    \item Random search better than grid search
    \item Bayesian optimization most efficient
    \item Start with good defaults
    \item Log everything
\end{itemize}

\textbf{Common mistakes:}
\begin{itemize}
    \item Not tuning learning rate
    \item Evaluating on test set during development
    \item Using validation statistics to normalize test data
    \item Not using early stopping
    \item Training for too many epochs
\end{itemize}

\clearpage