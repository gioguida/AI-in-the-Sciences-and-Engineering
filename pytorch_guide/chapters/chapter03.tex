% =============================================
% SECTION 3: BUILDING MODELS WITH NN.MODULE
% =============================================

\section{Building Models with \texttt{nn.Module}}

\subsection{Introduction: Why \texttt{nn.Module}?}

In Section 2, we manually created tensors with \texttt{requires\_grad=True} and computed gradients. While this works for simple cases, real neural networks have hundreds of layers and millions of parameters. We need a way to organize this complexity.

\texttt{nn.Module} is PyTorch's base class for all neural network components. It provides:

\begin{itemize}
    \item \textbf{Automatic parameter management:} Tracks all learnable parameters
    \item \textbf{GPU/CPU transfer:} Move entire models between devices with one command
    \item \textbf{Training/eval modes:} Control behavior of layers like dropout and batch norm
    \item \textbf{State management:} Save and load model weights
    \item \textbf{Composability:} Build complex models from simple modules
\end{itemize}

Every neural network layer (\texttt{nn.Linear}, \texttt{nn.Conv2d}, etc.) and every model you build inherits from \texttt{nn.Module}.

\subsection{Theory: The Module Hierarchy}

A \texttt{nn.Module} is a container that can hold:
\begin{itemize}
    \item \textbf{Parameters:} Learnable tensors (weights, biases)
    \item \textbf{Buffers:} Non-learnable tensors (running statistics, constants)
    \item \textbf{Sub-modules:} Other \texttt{nn.Module} objects
\end{itemize}

\textbf{Example hierarchy:}

\begin{verbatim}
MyModel (nn.Module)
├── layer1 (nn.Linear)
│   ├── weight (Parameter)
│   └── bias (Parameter)
├── layer2 (nn.Linear)
│   ├── weight (Parameter)
│   └── bias (Parameter)
└── activation (nn.ReLU)
\end{verbatim}

When you call \texttt{model.parameters()}, PyTorch automatically finds all parameters in the entire tree.

\begin{theorybox}[Key Principle: The Forward Method]
Every \texttt{nn.Module} must implement a \texttt{forward()} method that defines the computation. You never call \texttt{forward()} directly—instead, you call the module as a function: \texttt{output = model(input)}. This triggers PyTorch's hooks and tracking mechanisms.
\end{theorybox}

\subsection{Implementation: Creating Your First Module}

\subsubsection{A Simple Linear Model}

Let's build a simple linear regression model from scratch:

\begin{lstlisting}
import torch
import torch.nn as nn

class LinearRegression(nn.Module):
    def __init__(self, input_dim, output_dim):
        """
        Initialize the module.
        Always call super().__init__() first!
        """
        super().__init__()  # Initialize parent class
        
        # Define learnable parameters
        self.linear = nn.Linear(input_dim, output_dim)
    
    def forward(self, x):
        """
        Define the forward pass.
        x: input tensor of shape (batch_size, input_dim)
        returns: output tensor of shape (batch_size, output_dim)
        """
        return self.linear(x)

# Create the model
model = LinearRegression(input_dim=10, output_dim=1)

# Use the model
x = torch.randn(32, 10)  # Batch of 32 samples
y = model(x)             # Call the model (invokes forward())
print(y.shape)           # torch.Size([32, 1])
\end{lstlisting}

\begin{pytorchtip}[Always Call super().\_\_init\_\_()]
Forgetting \texttt{super().\_\_init\_\_()} is a common mistake. Without it, PyTorch can't track parameters properly. Always put it as the first line of \texttt{\_\_init\_\_()}.
\end{pytorchtip}

\subsubsection{Accessing Parameters}

\begin{lstlisting}
model = LinearRegression(10, 1)

# Get all parameters
for name, param in model.named_parameters():
    print(f"{name}: {param.shape}")
# Output:
# linear.weight: torch.Size([1, 10])
# linear.bias: torch.Size([1])

# Count parameters
total_params = sum(p.numel() for p in model.parameters())
print(f"Total parameters: {total_params}")  # 11 (10 weights + 1 bias)

# Access specific parameters
print(model.linear.weight)  # The weight matrix
print(model.linear.bias)    # The bias vector
\end{lstlisting}

\subsubsection{Using \texttt{nn.Sequential}}

For simple feed-forward architectures, \texttt{nn.Sequential} is convenient:

\begin{lstlisting}
# Method 1: Pass modules as arguments
model = nn.Sequential(
    nn.Linear(10, 50),
    nn.ReLU(),
    nn.Linear(50, 20),
    nn.ReLU(),
    nn.Linear(20, 1)
)

# Method 2: Use OrderedDict for named layers
from collections import OrderedDict

model = nn.Sequential(OrderedDict([
    ('fc1', nn.Linear(10, 50)),
    ('relu1', nn.ReLU()),
    ('fc2', nn.Linear(50, 20)),
    ('relu2', nn.ReLU()),
    ('output', nn.Linear(20, 1))
]))

# Use it
x = torch.randn(32, 10)
y = model(x)

# Access layers
print(model[0])  # First layer (fc1)
print(model.fc1)  # Same, if using OrderedDict
\end{lstlisting}

\textbf{When to use Sequential:}
\begin{itemize}
    \item Simple feed-forward architecture
    \item No branching or skip connections
    \item No custom logic in forward pass
\end{itemize}

\textbf{When to use custom Module:}
\begin{itemize}
    \item Complex architectures (ResNets, U-Nets)
    \item Multiple inputs/outputs
    \item Custom forward logic (if statements, loops)
    \item Need to access intermediate values
\end{itemize}

\clearpage
\subsubsection{Custom Modules with Multiple Layers}

\begin{lstlisting}
class MLP(nn.Module):
    """Multi-Layer Perceptron with configurable hidden layers."""
    
    def __init__(self, input_dim, hidden_dims, output_dim):
        """
        Args:
            input_dim: Input feature dimension
            hidden_dims: List of hidden layer dimensions
            output_dim: Output dimension
        """
        super().__init__()
        
        # Build layers dynamically
        layers = []
        prev_dim = input_dim
        
        for hidden_dim in hidden_dims:
            layers.append(nn.Linear(prev_dim, hidden_dim))
            layers.append(nn.ReLU())
            prev_dim = hidden_dim
        
        # Output layer (no activation)
        layers.append(nn.Linear(prev_dim, output_dim))
        
        # Store as Sequential
        self.network = nn.Sequential(*layers)
    
    def forward(self, x):
        return self.network(x)

# Create a 3-layer network
model = MLP(input_dim=10, hidden_dims=[50, 30], output_dim=1)

# The architecture is: 10 -> 50 -> 30 -> 1
# With ReLU after first two layers
\end{lstlisting}

\subsubsection{Parameters vs Buffers}

\textbf{Parameters:} Learnable tensors (updated by optimizer)

\textbf{Buffers:} Non-learnable tensors (not updated by optimizer, but saved with model)

\begin{lstlisting}
class MyModule(nn.Module):
    def __init__(self):
        super().__init__()
        
        # Learnable parameter
        self.weight = nn.Parameter(torch.randn(10, 10))
        
        # Non-learnable buffer (e.g., running statistics)
        self.register_buffer('running_mean', torch.zeros(10))
    
    def forward(self, x):
        # Use both in computation
        return x @ self.weight + self.running_mean

model = MyModule()

# Parameters (will be updated by optimizer)
print(len(list(model.parameters())))  # 1 (weight only)

# Buffers (won't be updated by optimizer)
print(len(list(model.buffers())))     # 1 (running_mean)

# Both are saved in state_dict
print(model.state_dict().keys())
# dict_keys(['weight', 'running_mean'])
\end{lstlisting}

\textbf{When to use buffers:}
\begin{itemize}
    \item Running statistics in BatchNorm
    \item Fixed embeddings
    \item Constants that should be saved with the model
    \item Anything that needs to move with the model to GPU but shouldn't be trained
\end{itemize}

\subsubsection{Training vs Evaluation Mode}

Some layers behave differently during training and evaluation:

\begin{lstlisting}
model = nn.Sequential(
    nn.Linear(10, 50),
    nn.ReLU(),
    nn.Dropout(0.5),  # Randomly drops 50% of values during training
    nn.Linear(50, 1)
)

# Training mode (default)
model.train()
x = torch.randn(5, 10)
print(model(x))  # Dropout is active

# Evaluation mode
model.eval()
print(model(x))  # Dropout is disabled

# Check current mode
print(model.training)  # False when in eval mode
\end{lstlisting}

\begin{warningbox}[Always Set Mode Correctly]
Forgetting to call \texttt{model.eval()} during inference can lead to incorrect results (dropout still active, batch norm using batch statistics instead of running statistics). Always:
\begin{lstlisting}
model.train()  # Before training
model.eval()   # Before evaluation/inference
\end{lstlisting}
\end{warningbox}

\clearpage
\subsubsection{Moving Models to GPU}

\begin{lstlisting}
device = torch.device('cuda' if torch.cuda.is_available() else 'cpu')

# Create model
model = MLP(10, [50, 30], 1)

# Move to GPU
model = model.to(device)

# All parameters and buffers are now on GPU
print(next(model.parameters()).device)  # cuda:0

# Input must also be on GPU
x = torch.randn(32, 10, device=device)
y = model(x)  # Works! Both model and input on same device

# Moving back to CPU
model = model.to('cpu')
\end{lstlisting}

\begin{pytorchtip}[Device Management Pattern]
Common pattern for device-agnostic code:
\begin{lstlisting}
device = torch.device('cuda' if torch.cuda.is_available() else 'cpu')
model = model.to(device)

for batch in dataloader:
    x, y = batch
    x, y = x.to(device), y.to(device)
    output = model(x)
    # ... rest of training loop
\end{lstlisting}
\end{pytorchtip}

\subsubsection{Saving and Loading Models}

\begin{lstlisting}
# Save model weights
model = MLP(10, [50, 30], 1)
torch.save(model.state_dict(), 'model_weights.pth')

# Load weights into new model
new_model = MLP(10, [50, 30], 1)  # Must have same architecture!
new_model.load_state_dict(torch.load('model_weights.pth'))

# Save entire model (architecture + weights)
torch.save(model, 'entire_model.pth')
loaded_model = torch.load('entire_model.pth')

# For inference, set to eval mode
loaded_model.eval()
\end{lstlisting}

\textbf{Best practice:} Save \texttt{state\_dict()} rather than entire model. This is more flexible and portable.

\begin{lstlisting}
# Better: Save with additional info
checkpoint = {
    'model_state_dict': model.state_dict(),
    'optimizer_state_dict': optimizer.state_dict(),
    'epoch': epoch,
    'loss': loss,
}
torch.save(checkpoint, 'checkpoint.pth')

# Load
checkpoint = torch.load('checkpoint.pth')
model.load_state_dict(checkpoint['model_state_dict'])
optimizer.load_state_dict(checkpoint['optimizer_state_dict'])
\end{lstlisting}

\subsubsection{Parameter Initialization}

PyTorch layers have default initialization, but you often want custom initialization:

\begin{lstlisting}
import torch.nn.init as init

class InitializedMLP(nn.Module):
    def __init__(self, input_dim, hidden_dim, output_dim):
        super().__init__()
        self.fc1 = nn.Linear(input_dim, hidden_dim)
        self.fc2 = nn.Linear(hidden_dim, output_dim)
        self.relu = nn.ReLU()
        
        # Custom initialization
        self._initialize_weights()
    
    def _initialize_weights(self):
        # Xavier initialization for fc1
        init.xavier_uniform_(self.fc1.weight)
        init.zeros_(self.fc1.bias)
        
        # He initialization for fc2 (better for ReLU)
        init.kaiming_normal_(self.fc2.weight, nonlinearity='relu')
        init.zeros_(self.fc2.bias)
    
    def forward(self, x):
        x = self.relu(self.fc1(x))
        x = self.fc2(x)
        return x
\end{lstlisting}

\textbf{Common initialization strategies:}

\begin{table}[h]
\centering
\begin{tabular}{lll}
\toprule
\textbf{Method} & \textbf{Use Case} & \textbf{Function} \\
\midrule
Xavier/Glorot & Sigmoid/Tanh & \texttt{init.xavier\_uniform\_} \\
He/Kaiming & ReLU & \texttt{init.kaiming\_normal\_} \\
Zeros & Biases & \texttt{init.zeros\_} \\
Constant & Set to specific value & \texttt{init.constant\_} \\
Normal & Custom std dev & \texttt{init.normal\_} \\
Orthogonal & RNNs & \texttt{init.orthogonal\_} \\
\bottomrule
\end{tabular}
\caption{Common weight initialization methods}
\end{table}

\begin{pytorchtip}[Why Initialization Matters]
Poor initialization can cause:
\begin{itemize}
    \item \textbf{Vanishing gradients:} Weights too small, gradients die out
    \item \textbf{Exploding gradients:} Weights too large, gradients explode
    \item \textbf{Slow convergence:} Network takes forever to learn
    \item \textbf{Dead neurons:} ReLUs output zero and never recover
\end{itemize}
Use Xavier for sigmoid/tanh activations, He for ReLU activations.
\end{pytorchtip}

\clearpage
\subsubsection{Module Hooks for Debugging}

Hooks let you inspect intermediate values during forward/backward pass:

\begin{lstlisting}
model = nn.Sequential(
    nn.Linear(10, 50),
    nn.ReLU(),
    nn.Linear(50, 1)
)

# Forward hook: called after forward pass
def forward_hook(module, input, output):
    print(f"Forward: {output.shape}")

# Register hook on second layer
handle = model[2].register_forward_hook(forward_hook)

# Forward pass
x = torch.randn(5, 10)
y = model(x)  # Prints: Forward: torch.Size([5, 1])

# Remove hook when done
handle.remove()

# Backward hook: called during backward pass
def backward_hook(module, grad_input, grad_output):
    print(f"Grad output shape: {grad_output[0].shape}")

handle = model[2].register_backward_hook(backward_hook)
y = model(x)
y.sum().backward()  # Prints grad shape

handle.remove()
\end{lstlisting}

\textbf{Use cases for hooks:}
\begin{itemize}
    \item Debugging shape mismatches
    \item Visualizing activations
    \item Gradient flow analysis
    \item Feature extraction from intermediate layers
\end{itemize}

\subsection{Implementation: Practical Patterns}

\subsubsection{Freezing Layers}

Sometimes you want to freeze part of the model (transfer learning):

\begin{lstlisting}
# Freeze all parameters
for param in model.parameters():
    param.requires_grad = False

# Freeze specific layer
for param in model.layer1.parameters():
    param.requires_grad = False

# Only train the last layer
model = nn.Sequential(
    nn.Linear(10, 50),
    nn.ReLU(),
    nn.Linear(50, 1)
)

# Freeze first two layers
for param in model[:2].parameters():
    param.requires_grad = False

# Now only model[2].parameters() will be updated
optimizer = torch.optim.Adam(
    filter(lambda p: p.requires_grad, model.parameters()),
    lr=0.001
)
\end{lstlisting}

\subsubsection{Multiple Inputs/Outputs}

\begin{lstlisting}
class MultiInputModel(nn.Module):
    def __init__(self):
        super().__init__()
        self.branch1 = nn.Linear(10, 50)
        self.branch2 = nn.Linear(5, 50)
        self.combine = nn.Linear(100, 1)
    
    def forward(self, x1, x2):
        """
        x1: shape (batch, 10)
        x2: shape (batch, 5)
        """
        out1 = self.branch1(x1)
        out2 = self.branch2(x2)
        combined = torch.cat([out1, out2], dim=1)  # (batch, 100)
        return self.combine(combined)

model = MultiInputModel()
x1 = torch.randn(32, 10)
x2 = torch.randn(32, 5)
y = model(x1, x2)
\end{lstlisting}

\subsubsection{Custom Layer with Parameters}

\begin{lstlisting}
class ScaledLinear(nn.Module):
    """Linear layer with learnable scaling factor."""
    
    def __init__(self, in_features, out_features):
        super().__init__()
        self.linear = nn.Linear(in_features, out_features)
        # Custom learnable parameter
        self.scale = nn.Parameter(torch.ones(1))
    
    def forward(self, x):
        return self.scale * self.linear(x)

layer = ScaledLinear(10, 5)

# Parameters include linear weights/bias AND scale
for name, param in layer.named_parameters():
    print(f"{name}: {param.shape}")
# linear.weight: torch.Size([5, 10])
# linear.bias: torch.Size([5])
# scale: torch.Size([1])
\end{lstlisting}

\clearpage
\subsection{Exercises}

\begin{exercise}[3.1: Simple Module - $\bigstar\bigstar$]
\textbf{Goal:} Create your first custom module.

Implement a \texttt{TwoLayerNet} module:
\begin{enumerate}
    \item Two linear layers: input\_dim → hidden\_dim → output\_dim
    \item ReLU activation between them
    \item No activation after the output layer
    \item Test with input\_dim=20, hidden\_dim=50, output\_dim=10
    \item Verify the output shape for a batch of 16 samples
\end{enumerate}

\textbf{Starter code:}
\begin{lstlisting}
import torch
import torch.nn as nn

class TwoLayerNet(nn.Module):
    def __init__(self, input_dim, hidden_dim, output_dim):
        super().__init__()
        # Your code here
    
    def forward(self, x):
        # Your code here
        pass

# Test
model = TwoLayerNet(20, 50, 10)
x = torch.randn(16, 20)
y = model(x)
print(y.shape)  # Should be torch.Size([16, 10])
\end{lstlisting}
\end{exercise}

\begin{exercise}[3.2: Parameter Counting - $\bigstar\bigstar$]
\textbf{Goal:} Understand model size.

\begin{enumerate}
    \item Create a model: 100 → 500 → 200 → 50 → 10 (all linear, ReLU between)
    \item Count total parameters programmatically
    \item Calculate manually and verify:
    \begin{itemize}
        \item Layer 1: $100 \times 500 + 500 = 50{,}500$
        \item Layer 2: $500 \times 200 + 200 = 100{,}200$
        \item ... (continue)
    \end{itemize}
    \item Print the number of parameters in each layer
\end{enumerate}

\textbf{Hint:} Use \texttt{param.numel()} to count elements in each parameter.
\end{exercise}

\begin{exercise}[3.3: Sequential vs Custom - $\bigstar\bigstar\bigstar$]
\textbf{Goal:} Compare two approaches.

Build the same network two ways:
\begin{enumerate}
    \item Using \texttt{nn.Sequential}
    \item Using a custom \texttt{nn.Module} with explicit layers
    \item Both should have: 10 → 50 → 30 → 1 with ReLU activations
    \item Load the same random seed before creating each
    \item Verify they produce identical outputs for the same input
\end{enumerate}

\textbf{Challenge:} Can you implement a method to copy weights from one to the other?
\end{exercise}

\begin{exercise}[3.4: Custom Initialization - $\bigstar\bigstar\bigstar$]
\textbf{Goal:} Implement custom weight initialization.

\begin{enumerate}
    \item Create a 3-layer MLP
    \item Initialize all weights with He initialization
    \item Initialize all biases to zero
    \item Create another identical model with default initialization
    \item Train both on a simple task (e.g., fit $y = x^2$ for x in [-1, 1])
    \item Compare convergence speed (plot losses)
\end{enumerate}

\textbf{Hint:} Use \texttt{nn.init.kaiming\_normal\_()} for He initialization.
\end{exercise}

\begin{exercise}[3.5: Save and Load - $\bigstar\bigstar\bigstar$]
\textbf{Goal:} Practice model persistence.

\begin{enumerate}
    \item Create a model and train it for a few steps (any simple task)
    \item Save the state dict
    \item Create a new model instance
    \item Load the weights
    \item Verify both models produce identical outputs
    \item Try saving/loading with \texttt{torch.save(model, ...)} and compare
\end{enumerate}
\end{exercise}

\begin{exercise}[3.6: Multi-Input Network - $\bigstar\bigstar\bigstar\bigstar$]
\textbf{Goal:} Build a more complex architecture.

Implement a model that takes two inputs and combines them:
\begin{enumerate}
    \item Input 1: shape (batch, 20)
    \item Input 2: shape (batch, 10)
    \item Process each through separate 2-layer networks
    \item Concatenate the results
    \item Pass through a final layer to get output shape (batch, 1)
    \item Test with random inputs
\end{enumerate}

\textbf{Extra challenge:} Add skip connections (concatenate the original inputs with processed features before the final layer).
\end{exercise}

\begin{exercise}[3.7: Frozen Layers - $\bigstar\bigstar\bigstar\bigstar$]
\textbf{Goal:} Practice transfer learning patterns.

\begin{enumerate}
    \item Create a 4-layer network
    \item "Pretrain" it on a simple task
    \item Freeze the first two layers (\texttt{requires\_grad=False})
    \item "Fine-tune" on a different task (only last two layers train)
    \item Verify the first two layers' weights don't change
    \item Compare with training all layers from scratch
\end{enumerate}

\textbf{Task idea:} First task: classify points in quadrants, second task: classify by distance from origin.
\end{exercise}

\clearpage