% =============================================
% SECTION 8: RESIDUAL NETWORKS & SKIP CONNECTIONS
% =============================================

\section{Residual Networks \& Skip Connections}

\subsection{Introduction: The Deep Network Problem}

In the early 2010s, researchers discovered something puzzling: making networks deeper didn't always make them better. Sometimes, a 56-layer network performed \textbf{worse} than a 20-layer network—even on training data!

This wasn't overfitting (test loss was also worse). It was a fundamental optimization problem.

\textbf{The Degradation Problem:}
\begin{itemize}
    \item Very deep networks are harder to optimize
    \item Gradients vanish or explode
    \item Training error increases with depth (counterintuitively)
    \item Not caused by overfitting—happens on training set too
\end{itemize}

\textbf{Residual Networks (ResNets) solved this in 2015,} enabling networks with 100+ layers that actually train better than shallow ones. The key insight: \textbf{skip connections} (also called \textbf{residual connections}).

\textbf{Impact:}
\begin{itemize}
    \item ResNet-152 won ImageNet 2015 (3.57\% error, human-level performance)
    \item Now standard in almost all vision architectures
    \item Fundamental principle used across all deep learning (Transformers, etc.)
\end{itemize}

\subsection{Theory: Why Skip Connections Work}

\subsubsection{The Problem: Vanishing Gradients}

In a deep network without skip connections:

\[
\mathbf{x}_L = f_L(f_{L-1}(...f_2(f_1(\mathbf{x}_0))))
\]

During backpropagation, gradients multiply through the chain rule:

\[
\frac{\partial \mathcal{L}}{\partial \mathbf{x}_0} = \frac{\partial \mathcal{L}}{\partial \mathbf{x}_L} \cdot \frac{\partial \mathbf{x}_L}{\partial \mathbf{x}_{L-1}} \cdot ... \cdot \frac{\partial \mathbf{x}_2}{\partial \mathbf{x}_1} \cdot \frac{\partial \mathbf{x}_1}{\partial \mathbf{x}_0}
\]

If each term is less than 1, the gradient \textbf{vanishes} exponentially as it propagates backward.

\textbf{Example:} If each layer multiplies gradient by 0.9:
\begin{itemize}
    \item After 10 layers: $0.9^{10} \approx 0.35$
    \item After 20 layers: $0.9^{20} \approx 0.12$
    \item After 50 layers: $0.9^{50} \approx 0.005$ (essentially zero!)
\end{itemize}

Early layers barely learn, making deep networks ineffective.

\subsubsection{The Solution: Residual Connections}

Instead of learning a direct mapping $\mathbf{y} = \mathcal{F}(\mathbf{x})$, learn a \textbf{residual}:

\[
\mathbf{y} = \mathcal{F}(\mathbf{x}) + \mathbf{x}
\]

The network learns the \textbf{difference} (residual) from the identity mapping.

\begin{theorybox}[Key Insight: Identity Mapping]
With a skip connection, the network can learn the identity function by simply setting $\mathcal{F}(\mathbf{x}) = 0$ (all weights to zero).

\textbf{Why this matters:}
\begin{itemize}
    \item Identity is easy to learn (do nothing)
    \item Network starts with identity, then learns refinements
    \item Even if $\mathcal{F}$ is poorly initialized, gradients still flow through skip connection
    \item Optimization becomes: "How should I modify the input?" instead of "What should the output be?"
\end{itemize}
\end{theorybox}

\subsubsection{Gradient Flow Through Skip Connections}

During backpropagation with skip connections:

\[
\frac{\partial \mathcal{L}}{\partial \mathbf{x}} = \frac{\partial \mathcal{L}}{\partial \mathbf{y}} \left( \frac{\partial \mathcal{F}}{\partial \mathbf{x}} + 1 \right)
\]

The $+1$ term creates a \textbf{gradient highway}:
\begin{itemize}
    \item Gradients always have a direct path backward (through the $+1$)
    \item Even if $\frac{\partial \mathcal{F}}{\partial \mathbf{x}} \approx 0$, gradients still flow
    \item No vanishing gradient problem!
\end{itemize}

\textbf{Analogy:} Regular network = narrow mountain trail (one path, easy to get stuck)

Skip connections = trail + highway (gradient can always get through)

\clearpage
\subsubsection{Residual Block Variants}

\textbf{Basic Residual Block (Original):}

\begin{verbatim}
Input (x)
   |
   ├─────────────┐  (skip connection)
   |             |
 Conv 3x3        |
   |             |
 BatchNorm       |
   |             |
  ReLU           |
   |             |
 Conv 3x3        |
   |             |
 BatchNorm       |
   |             |
   +─────────────┘  (addition)
   |
  ReLU
   |
Output (y)
\end{verbatim}

\textbf{Pre-Activation Residual Block (Improved):}

\begin{verbatim}
Input (x)
   |
   ├─────────────┐  (skip connection)
   |             |
 BatchNorm       |
   |             |
  ReLU           |
   |             |
 Conv 3x3        |
   |             |
 BatchNorm       |
   |             |
  ReLU           |
   |             |
 Conv 3x3        |
   |             |
   +─────────────┘  (addition)
   |
Output (y)
\end{verbatim}

\textbf{Difference:}
\begin{itemize}
    \item \textbf{Post-activation:} Conv → BN → ReLU (then add)
    \item \textbf{Pre-activation:} BN → ReLU → Conv (then add)
\end{itemize}

\textbf{Pre-activation advantages:}
\begin{itemize}
    \item Better gradient flow (identity mapping is truly unimpeded)
    \item Easier optimization
    \item Better for very deep networks (>100 layers)
\end{itemize}

\subsubsection{When Dimensions Don't Match}

If input and output have different dimensions, use a \textbf{projection shortcut}:

\[
\mathbf{y} = \mathcal{F}(\mathbf{x}) + \mathbf{W}_s\mathbf{x}
\]

where $\mathbf{W}_s$ is a linear projection (1×1 convolution).

\textbf{Two approaches:}

\begin{enumerate}
    \item \textbf{Identity shortcut + zero padding:} Pad channels with zeros (parameter-free)
    \item \textbf{Projection shortcut:} Use 1×1 conv to match dimensions (adds parameters)
\end{enumerate}

Most implementations use projection shortcuts when dimensions change.

\subsection{Implementation: Building Residual Blocks}

\subsubsection{Basic Residual Block}

\begin{lstlisting}
import torch
import torch.nn as nn

class BasicBlock(nn.Module):
    """Basic residual block with two 3x3 convolutions."""
    
    def __init__(self, in_channels, out_channels, stride=1):
        super().__init__()
        
        # Main path
        self.conv1 = nn.Conv2d(in_channels, out_channels, 
                              kernel_size=3, stride=stride, padding=1, 
                              bias=False)
        self.bn1 = nn.BatchNorm2d(out_channels)
        self.relu = nn.ReLU(inplace=True)
        
        self.conv2 = nn.Conv2d(out_channels, out_channels,
                              kernel_size=3, stride=1, padding=1,
                              bias=False)
        self.bn2 = nn.BatchNorm2d(out_channels)
        
        # Skip connection (identity or projection)
        self.skip = nn.Sequential()
        if stride != 1 or in_channels != out_channels:
            self.skip = nn.Sequential(
                nn.Conv2d(in_channels, out_channels, 
                         kernel_size=1, stride=stride, bias=False),
                nn.BatchNorm2d(out_channels)
            )
    
    def forward(self, x):
        identity = x
        
        # Main path
        out = self.conv1(x)
        out = self.bn1(out)
        out = self.relu(out)
        
        out = self.conv2(out)
        out = self.bn2(out)
        
        # Skip connection
        identity = self.skip(identity)
        
        # Add residual
        out += identity
        out = self.relu(out)
        
        return out

# Test
block = BasicBlock(64, 64)
x = torch.randn(4, 64, 32, 32)
y = block(x)
print(y.shape)  # torch.Size([4, 64, 32, 32])

# Test with dimension change
block_downsample = BasicBlock(64, 128, stride=2)
y2 = block_downsample(x)
print(y2.shape)  # torch.Size([4, 128, 16, 16])
\end{lstlisting}

\begin{pytorchtip}[Why bias=False?]
When using batch normalization immediately after convolution, the bias is redundant (BatchNorm has its own bias term). Setting \texttt{bias=False} saves parameters and computation without affecting performance.
\end{pytorchtip}

\clearpage
\subsubsection{Pre-Activation Residual Block}

\begin{lstlisting}
class PreActBlock(nn.Module):
    """Pre-activation residual block (improved version)."""
    
    def __init__(self, in_channels, out_channels, stride=1):
        super().__init__()
        
        # Pre-activation
        self.bn1 = nn.BatchNorm2d(in_channels)
        self.relu = nn.ReLU(inplace=True)
        self.conv1 = nn.Conv2d(in_channels, out_channels,
                              kernel_size=3, stride=stride, padding=1,
                              bias=False)
        
        self.bn2 = nn.BatchNorm2d(out_channels)
        self.conv2 = nn.Conv2d(out_channels, out_channels,
                              kernel_size=3, stride=1, padding=1,
                              bias=False)
        
        # Skip connection
        self.skip = nn.Sequential()
        if stride != 1 or in_channels != out_channels:
            self.skip = nn.Sequential(
                nn.Conv2d(in_channels, out_channels,
                         kernel_size=1, stride=stride, bias=False)
            )
    
    def forward(self, x):
        # Pre-activation
        out = self.bn1(x)
        out = self.relu(out)
        
        # Save for skip connection (after activation)
        identity = self.skip(out)
        
        # First conv
        out = self.conv1(out)
        
        # Second pre-activation + conv
        out = self.bn2(out)
        out = self.relu(out)
        out = self.conv2(out)
        
        # Add residual (no activation after addition!)
        out += identity
        
        return out
\end{lstlisting}

\textbf{Key difference:} In pre-activation, the final addition has no activation. The next block will apply activation first.

\subsubsection{Bottleneck Block}

For deeper networks (ResNet-50+), use bottleneck blocks to reduce parameters:

\begin{lstlisting}
class BottleneckBlock(nn.Module):
    """
    Bottleneck block: 1x1 -> 3x3 -> 1x1
    Reduces parameters by using 1x1 convolutions to reduce/expand channels.
    """
    expansion = 4  # Output channels = input channels * 4
    
    def __init__(self, in_channels, out_channels, stride=1):
        super().__init__()
        
        # Bottleneck: reduce channels
        self.conv1 = nn.Conv2d(in_channels, out_channels, 
                              kernel_size=1, bias=False)
        self.bn1 = nn.BatchNorm2d(out_channels)
        
        # Main 3x3 conv
        self.conv2 = nn.Conv2d(out_channels, out_channels,
                              kernel_size=3, stride=stride, 
                              padding=1, bias=False)
        self.bn2 = nn.BatchNorm2d(out_channels)
        
        # Expand channels
        self.conv3 = nn.Conv2d(out_channels, out_channels * self.expansion,
                              kernel_size=1, bias=False)
        self.bn3 = nn.BatchNorm2d(out_channels * self.expansion)
        
        self.relu = nn.ReLU(inplace=True)
        
        # Skip connection
        self.skip = nn.Sequential()
        if stride != 1 or in_channels != out_channels * self.expansion:
            self.skip = nn.Sequential(
                nn.Conv2d(in_channels, out_channels * self.expansion,
                         kernel_size=1, stride=stride, bias=False),
                nn.BatchNorm2d(out_channels * self.expansion)
            )
    
    def forward(self, x):
        identity = x
        
        out = self.conv1(x)
        out = self.bn1(out)
        out = self.relu(out)
        
        out = self.conv2(out)
        out = self.bn2(out)
        out = self.relu(out)
        
        out = self.conv3(out)
        out = self.bn3(out)
        
        identity = self.skip(identity)
        out += identity
        out = self.relu(out)
        
        return out

# Compare parameters
basic = BasicBlock(256, 256)
bottleneck = BottleneckBlock(256, 64)  # Output: 64*4 = 256

basic_params = sum(p.numel() for p in basic.parameters())
bottleneck_params = sum(p.numel() for p in bottleneck.parameters())

print(f"Basic block: {basic_params:,} parameters")
print(f"Bottleneck: {bottleneck_params:,} parameters")
# Basic block: ~590,000 parameters
# Bottleneck: ~70,000 parameters (much fewer!)
\end{lstlisting}

\clearpage
\subsubsection{Building a Complete ResNet}

\begin{lstlisting}
class ResNet(nn.Module):
    """ResNet architecture for CIFAR-10."""
    
    def __init__(self, block, num_blocks, num_classes=10):
        """
        Args:
            block: BasicBlock or BottleneckBlock
            num_blocks: List of number of blocks per layer
            num_classes: Number of output classes
        """
        super().__init__()
        self.in_channels = 64
        
        # Initial convolution
        self.conv1 = nn.Conv2d(3, 64, kernel_size=3, 
                              stride=1, padding=1, bias=False)
        self.bn1 = nn.BatchNorm2d(64)
        self.relu = nn.ReLU(inplace=True)
        
        # Residual layers
        self.layer1 = self._make_layer(block, 64, num_blocks[0], stride=1)
        self.layer2 = self._make_layer(block, 128, num_blocks[1], stride=2)
        self.layer3 = self._make_layer(block, 256, num_blocks[2], stride=2)
        self.layer4 = self._make_layer(block, 512, num_blocks[3], stride=2)
        
        # Global pooling and classifier
        self.avgpool = nn.AdaptiveAvgPool2d((1, 1))
        self.fc = nn.Linear(512 * block.expansion, num_classes)
    
    def _make_layer(self, block, out_channels, num_blocks, stride):
        """Create a layer with multiple residual blocks."""
        strides = [stride] + [1] * (num_blocks - 1)
        layers = []
        
        for stride in strides:
            layers.append(block(self.in_channels, out_channels, stride))
            self.in_channels = out_channels * block.expansion
        
        return nn.Sequential(*layers)
    
    def forward(self, x):
        out = self.conv1(x)
        out = self.bn1(out)
        out = self.relu(out)
        
        out = self.layer1(out)
        out = self.layer2(out)
        out = self.layer3(out)
        out = self.layer4(out)
        
        out = self.avgpool(out)
        out = out.view(out.size(0), -1)
        out = self.fc(out)
        
        return out

# ResNet-18
def ResNet18(num_classes=10):
    return ResNet(BasicBlock, [2, 2, 2, 2], num_classes)

# ResNet-34
def ResNet34(num_classes=10):
    return ResNet(BasicBlock, [3, 4, 6, 3], num_classes)

# ResNet-50 (uses bottleneck)
def ResNet50(num_classes=10):
    # Note: BottleneckBlock needs expansion=4
    BottleneckBlock.expansion = 4
    return ResNet(BottleneckBlock, [3, 4, 6, 3], num_classes)

# Test
model = ResNet18()
x = torch.randn(4, 3, 32, 32)
y = model(x)
print(y.shape)  # torch.Size([4, 10])

# Count parameters
total_params = sum(p.numel() for p in model.parameters())
print(f"ResNet-18: {total_params:,} parameters")
\end{lstlisting}

\begin{pytorchtip}[ResNet Naming]
ResNet-N refers to the total number of weight layers:

\textbf{ResNet-18:} 1 initial conv + 4 layers × 2 blocks × 2 convs + 1 FC = 18 layers

\textbf{ResNet-34:} 1 + (3+4+6+3) × 2 + 1 = 34 layers

\textbf{ResNet-50:} 1 + (3+4+6+3) × 3 + 1 = 50 layers (bottleneck has 3 convs)

Deeper = more capacity but slower training/inference.
\end{pytorchtip}

\clearpage
\subsubsection{Dense Connections (DenseNet)}

DenseNet takes skip connections further: \textbf{every layer connects to every other layer}.

\begin{lstlisting}
class DenseBlock(nn.Module):
    """Dense block where each layer receives all previous layers as input."""
    
    def __init__(self, in_channels, growth_rate, num_layers):
        """
        Args:
            in_channels: Initial number of channels
            growth_rate: Number of channels added per layer
            num_layers: Number of layers in the dense block
        """
        super().__init__()
        
        self.layers = nn.ModuleList()
        for i in range(num_layers):
            self.layers.append(
                self._make_dense_layer(in_channels + i * growth_rate, 
                                      growth_rate)
            )
    
    def _make_dense_layer(self, in_channels, growth_rate):
        return nn.Sequential(
            nn.BatchNorm2d(in_channels),
            nn.ReLU(inplace=True),
            nn.Conv2d(in_channels, growth_rate, kernel_size=3, 
                     padding=1, bias=False)
        )
    
    def forward(self, x):
        features = [x]
        for layer in self.layers:
            # Concatenate all previous feature maps
            new_features = layer(torch.cat(features, dim=1))
            features.append(new_features)
        
        # Return concatenation of all features
        return torch.cat(features, dim=1)

# Test
block = DenseBlock(in_channels=64, growth_rate=32, num_layers=4)
x = torch.randn(2, 64, 16, 16)
y = block(x)
print(y.shape)  # torch.Size([2, 192, 16, 16])
# 192 = 64 (input) + 4 * 32 (4 layers, each adds 32 channels)
\end{lstlisting}

\textbf{DenseNet characteristics:}
\begin{itemize}
    \item \textbf{Feature reuse:} All layers have access to all previous features
    \item \textbf{Fewer parameters:} Smaller growth\_rate than ResNet channel counts
    \item \textbf{Memory intensive:} Must store all intermediate features
    \item \textbf{Better gradient flow:} Even better than ResNet
\end{itemize}

\textbf{When to use DenseNet vs ResNet:}
\begin{itemize}
    \item \textbf{DenseNet:} When parameter efficiency is critical, smaller datasets
    \item \textbf{ResNet:} When memory is limited, faster training/inference, more standard
\end{itemize}

\subsubsection{Skip Connections in Practice}

\textbf{General principles for adding skip connections:}

\begin{enumerate}
    \item \textbf{Add skip every 2-3 layers:} Balance between gradient flow and complexity
    \item \textbf{Match dimensions:} Use projection when channels or spatial size change
    \item \textbf{Identity when possible:} Projection-free shortcuts are better
    \item \textbf{Add before final activation:} Preserves identity mapping
    \item \textbf{Don't overdo it:} Too many skip connections can hurt performance
\end{enumerate}

\textbf{Example: Adding skip connections to existing MLP:}

\begin{lstlisting}
class MLPWithSkipConnections(nn.Module):
    """MLP with residual connections."""
    
    def __init__(self, input_dim, hidden_dim, output_dim, num_layers=4):
        super().__init__()
        
        self.input_layer = nn.Linear(input_dim, hidden_dim)
        
        # Hidden layers with skip connections
        self.hidden_layers = nn.ModuleList()
        for _ in range(num_layers):
            self.hidden_layers.append(
                nn.Sequential(
                    nn.Linear(hidden_dim, hidden_dim),
                    nn.ReLU(),
                    nn.Linear(hidden_dim, hidden_dim)
                )
            )
        
        self.output_layer = nn.Linear(hidden_dim, output_dim)
        self.relu = nn.ReLU()
    
    def forward(self, x):
        x = self.relu(self.input_layer(x))
        
        # Apply each hidden layer with skip connection
        for layer in self.hidden_layers:
            residual = x
            x = layer(x)
            x = x + residual  # Skip connection
            x = self.relu(x)
        
        x = self.output_layer(x)
        return x
\end{lstlisting}

\clearpage
% =============================================
% SECTION 8: RESNET - PART 2 (Exercises & Debugging)
% =============================================

\subsection{Implementation: Debugging ResNets}

\subsubsection{Verifying Skip Connections}

\begin{lstlisting}
def verify_skip_connections(model, input_shape=(1, 3, 32, 32)):
    """Verify that skip connections are working."""
    x = torch.randn(*input_shape, requires_grad=True)
    
    # Forward pass
    output = model(x)
    loss = output.sum()
    
    # Backward pass
    loss.backward()
    
    # Check if gradients reach input
    if x.grad is not None:
        print(f"✓ Gradients flow to input: {x.grad.abs().mean().item():.6f}")
    else:
        print("✗ No gradients at input!")
    
    # Check gradient magnitudes throughout network
    print("\nGradient magnitudes by layer:")
    for name, param in model.named_parameters():
        if param.grad is not None:
            grad_mean = param.grad.abs().mean().item()
            grad_max = param.grad.abs().max().item()
            print(f"{name:30s}: mean={grad_mean:.6f}, max={grad_max:.6f}")

# Usage
# model = ResNet18()
# verify_skip_connections(model)
\end{lstlisting}

\subsubsection{Comparing With and Without Skip Connections}

\begin{lstlisting}
class PlainCNN(nn.Module):
    """Same architecture as ResNet but without skip connections."""
    
    def __init__(self, num_classes=10):
        super().__init__()
        
        self.conv1 = nn.Conv2d(3, 64, 3, padding=1, bias=False)
        self.bn1 = nn.BatchNorm2d(64)
        
        # Same structure as ResNet-18 but no skip connections
        self.layers = nn.Sequential(
            # Layer 1
            nn.Conv2d(64, 64, 3, padding=1, bias=False),
            nn.BatchNorm2d(64),
            nn.ReLU(),
            nn.Conv2d(64, 64, 3, padding=1, bias=False),
            nn.BatchNorm2d(64),
            nn.ReLU(),
            
            # Layer 2 (with stride=2 downsampling)
            nn.Conv2d(64, 128, 3, stride=2, padding=1, bias=False),
            nn.BatchNorm2d(128),
            nn.ReLU(),
            nn.Conv2d(128, 128, 3, padding=1, bias=False),
            nn.BatchNorm2d(128),
            nn.ReLU(),
            
            # Continue pattern...
        )
        
        self.avgpool = nn.AdaptiveAvgPool2d((1, 1))
        self.fc = nn.Linear(512, num_classes)
    
    def forward(self, x):
        x = nn.ReLU()(self.bn1(self.conv1(x)))
        x = self.layers(x)
        x = self.avgpool(x)
        x = x.view(x.size(0), -1)
        return self.fc(x)

# Compare training curves
# plain_model = PlainCNN()
# resnet_model = ResNet18()
# Train both and plot losses - ResNet should converge faster and better
\end{lstlisting}

\subsection{Exercises}

\begin{exercise}[8.1: Basic Residual Block - $\bigstar\bigstar$]
\textbf{Goal:} Implement and understand a basic residual block.

\begin{enumerate}
    \item Implement a \texttt{BasicBlock} with two 3×3 convolutions
    \item Include skip connection with projection for dimension changes
    \item Test with different input/output channels:
    \begin{itemize}
        \item Same channels (64 → 64)
        \item Different channels (64 → 128)
        \item With downsampling (stride=2)
    \end{itemize}
    \item Verify output shapes match expected dimensions
\end{enumerate}

\textbf{Starter code:}
\begin{lstlisting}
class MyBasicBlock(nn.Module):
    def __init__(self, in_channels, out_channels, stride=1):
        super().__init__()
        # Your code here
    
    def forward(self, x):
        # Your code here
        pass

# Test cases
block1 = MyBasicBlock(64, 64, stride=1)
block2 = MyBasicBlock(64, 128, stride=2)

x = torch.randn(4, 64, 32, 32)
print(block1(x).shape)  # Should be [4, 64, 32, 32]
print(block2(x).shape)  # Should be [4, 128, 16, 16]
\end{lstlisting}
\end{exercise}

\begin{exercise}[8.2: Plain vs Residual Comparison - $\bigstar\bigstar\bigstar$]
\textbf{Goal:} Empirically verify that skip connections help.

\begin{enumerate}
    \item Build two identical networks (10 layers each):
    \begin{itemize}
        \item One \textbf{without} skip connections (plain CNN)
        \item One \textbf{with} skip connections (ResNet-style)
    \end{itemize}
    \item Train both on CIFAR-10 or MNIST
    \item Track and plot:
    \begin{itemize}
        \item Training loss
        \item Validation accuracy
        \item Gradient magnitudes in early layers
    \end{itemize}
    \item Observe: ResNet should train faster and achieve better accuracy
\end{enumerate}

\textbf{Questions to answer:}
\begin{itemize}
    \item How much faster does ResNet converge?
    \item What's the final accuracy difference?
    \item How do gradient magnitudes compare?
\end{itemize}
\end{exercise}

\begin{exercise}[8.3: Pre-Activation vs Post-Activation - $\bigstar\bigstar\bigstar$]
\textbf{Goal:} Compare residual block variants.

\begin{enumerate}
    \item Implement both post-activation and pre-activation blocks
    \item Build identical ResNets with each type
    \item Train on the same dataset
    \item Compare:
    \begin{itemize}
        \item Training stability (loss curves)
        \item Final accuracy
        \item Training time
    \end{itemize}
\end{enumerate}

\textbf{Hypothesis:} Pre-activation should be slightly better, especially for deeper networks.

\textbf{Extension:} Try with different depths (18, 34, 50 layers) and see if the gap widens.
\end{exercise}

\begin{exercise}[8.4: Building ResNet-18 - $\bigstar\bigstar\bigstar$]
\textbf{Goal:} Build a complete ResNet from scratch.

\begin{enumerate}
    \item Implement ResNet-18 architecture:
    \begin{itemize}
        \item Initial 7×7 conv (or 3×3 for CIFAR)
        \item 4 residual layers with [2, 2, 2, 2] blocks
        \item Channels: 64 → 128 → 256 → 512
        \item Global average pooling + FC
    \end{itemize}
    \item Train on CIFAR-10
    \item Achieve >90\% test accuracy
    \item Save the model and load it for inference
\end{enumerate}

\textbf{Bonus:} Visualize learned features in first convolutional layer.
\end{exercise}

\begin{exercise}[8.5: Bottleneck Block - $\bigstar\bigstar\bigstar\bigstar$]
\textbf{Goal:} Implement parameter-efficient bottleneck blocks.

\begin{enumerate}
    \item Implement \texttt{BottleneckBlock}: 1×1 → 3×3 → 1×1
    \item Compare parameter count with \texttt{BasicBlock} for same input/output
    \item Build ResNet-50 using bottleneck blocks
    \item Train and compare with ResNet-18:
    \begin{itemize}
        \item Accuracy (ResNet-50 should be better)
        \item Training time (ResNet-50 slower)
        \item Memory usage
    \end{itemize}
\end{enumerate}

\textbf{Challenge:} How deep can you go before hitting memory limits? Try ResNet-101.
\end{exercise}

\begin{exercise}[8.6: Skip Connections in U-Net - $\bigstar\bigstar\bigstar\bigstar$]
\textbf{Goal:} Apply skip connections to encoder-decoder architectures.

\begin{enumerate}
    \item Take the U-Net from CNN section (Exercise 7.5)
    \item Add residual blocks in encoder and decoder paths
    \item Train on image denoising task
    \item Compare with original U-Net:
    \begin{itemize}
        \item Denoising quality (PSNR/SSIM)
        \item Training stability
        \item Convergence speed
    \end{itemize}
\end{enumerate}

\textbf{Architecture:}
\begin{itemize}
    \item Replace conv blocks with residual blocks
    \item Keep skip connections between encoder/decoder
    \item Use both residual blocks AND U-Net skip connections
\end{itemize}
\end{exercise}

\begin{exercise}[8.7: Dense Block Implementation - $\bigstar\bigstar\bigstar\bigstar$]
\textbf{Goal:} Implement and understand DenseNet.

\begin{enumerate}
    \item Implement a \texttt{DenseBlock} where each layer receives all previous layers
    \item Implement a transition layer (1×1 conv + pooling) to reduce channels
    \item Build a small DenseNet:
    \begin{itemize}
        \item 3 dense blocks with growth\_rate=12
        \item Transition layers between blocks
        \item Final global pooling + FC
    \end{itemize}
    \item Compare with ResNet of similar depth:
    \begin{itemize}
        \item Parameter count (DenseNet should be fewer)
        \item Memory usage (DenseNet higher during training)
        \item Accuracy
    \end{itemize}
\end{enumerate}

\textbf{Hint:} Use \texttt{torch.cat()} to concatenate feature maps along channel dimension.
\end{exercise}

\clearpage
\subsection{Key Takeaways}

\textbf{The fundamental problem:}
\begin{itemize}
    \item Very deep networks suffer from degradation (not just overfitting)
    \item Vanishing gradients make early layers hard to train
    \item Optimization becomes difficult, not just capacity
\end{itemize}

\textbf{How skip connections solve it:}
\begin{itemize}
    \item Create gradient highways: $\frac{\partial \mathcal{L}}{\partial \mathbf{x}}$ always has $+1$ term
    \item Learn residuals instead of direct mappings (easier optimization)
    \item Enable training of 100+ layer networks
    \item Identity mapping is easy to learn (set weights to zero)
\end{itemize}

\textbf{Residual block variants:}
\begin{itemize}
    \item \textbf{Post-activation:} Conv → BN → ReLU, then add (original)
    \item \textbf{Pre-activation:} BN → ReLU → Conv, then add (improved)
    \item \textbf{Bottleneck:} 1×1 → 3×3 → 1×1 (for deeper networks, fewer parameters)
    \item \textbf{Dense:} Every layer connects to all previous layers (extreme skip connections)
\end{itemize}

\textbf{Implementation guidelines:}
\begin{itemize}
    \item Use projection shortcuts when dimensions change
    \item Set \texttt{bias=False} when using BatchNorm
    \item Add skip connections every 2-3 layers
    \item Pre-activation is better for very deep networks (>100 layers)
    \item Use bottleneck blocks for ResNet-50+
\end{itemize}

\textbf{When to use skip connections:}
\begin{itemize}
    \item \textbf{Always use for deep networks (>20 layers)}
    \item \textbf{Vision tasks:} ResNet is standard backbone
    \item \textbf{U-Net style architectures:} Encoder-decoder with skip connections
    \item \textbf{Any architecture where gradient flow is critical}
    \item \textbf{When training is unstable or slow}
\end{itemize}

\textbf{Architecture choices:}
\begin{table}[h]
\centering
\begin{tabular}{llll}
\toprule
\textbf{Model} & \textbf{Layers} & \textbf{Use Case} & \textbf{Parameters} \\
\midrule
ResNet-18 & 18 & General purpose, fast & 11M \\
ResNet-34 & 34 & More capacity & 21M \\
ResNet-50 & 50 & High accuracy & 25M \\
ResNet-101 & 101 & Maximum accuracy & 44M \\
DenseNet-121 & 121 & Parameter efficient & 8M \\
\bottomrule
\end{tabular}
\caption{Popular ResNet/DenseNet architectures}
\end{table}

\textbf{Impact on deep learning:}
\begin{itemize}
    \item Enabled training of 100+ layer networks
    \item Now standard in almost all vision models
    \item Extended to other domains (Transformers use similar ideas)
    \item Fundamental principle: make it easy for network to learn identity
\end{itemize}

\textbf{Debugging tips:}
\begin{itemize}
    \item Verify gradients flow to early layers
    \item Compare with plain network (without skip connections)
    \item Check gradient magnitudes throughout network
    \item Visualize activations in residual blocks
    \item If training fails, check skip connection implementation
\end{itemize}

\textbf{Common mistakes:}
\begin{enumerate}
    \item Forgetting projection when dimensions change
    \item Applying activation after addition in pre-activation blocks
    \item Not using BatchNorm (skip connections alone aren't enough)
    \item Making skip connection too complex (keep it identity when possible)
    \item Adding too many skip connections (can hurt performance)
\end{enumerate}

\begin{pytorchtip}[Rule of Thumb]
\textbf{For networks >20 layers:} Always use residual connections

\textbf{For networks >50 layers:} Use bottleneck blocks

\textbf{For networks >100 layers:} Use pre-activation blocks

\textbf{When in doubt:} Use ResNet as your starting point and modify from there.
\end{pytorchtip}

\clearpage