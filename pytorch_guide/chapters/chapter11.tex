% =============================================
% SECTION 11: ATTENTION & TRANSFORMERS
% =============================================

\section{Attention \& Transformers}

\subsection{Introduction: Beyond RNNs}

RNNs revolutionized sequence modeling, but they have fundamental limitations:

\textbf{Problems with RNNs:}
\begin{enumerate}
    \item \textbf{Sequential processing:} Can't parallelize (slow!)
    \item \textbf{Long-range dependencies:} Even LSTM struggles beyond ~100 steps
    \item \textbf{Information bottleneck:} All info must pass through hidden state
    \item \textbf{Vanishing gradients:} Still an issue for very long sequences
\end{enumerate}

\textbf{The Attention mechanism} (2014) and \textbf{Transformers} (2017) solved these problems.

\textbf{Key innovation:} Instead of compressing everything into a fixed-size hidden state, let the model \textbf{attend to} any part of the input directly.

\textbf{Impact:}
\begin{itemize}
    \item GPT, BERT, ChatGPT: All based on Transformers
    \item Transformers now dominate NLP, vision (ViT), multimodal (CLIP), and more
    \item Enabled training on massive datasets (parallelization)
    \item State-of-the-art on virtually all sequence tasks
\end{itemize}

\subsection{Theory: The Attention Mechanism}

\subsubsection{Intuition: What is Attention?}

Imagine reading a document and answering a question. You don't re-read the entire document—you \textbf{focus} on relevant parts.

\textbf{Example:}
\begin{itemize}
    \item Document: "The cat sat on the mat. The dog played in the garden."
    \item Query: "Where is the cat?"
    \item You attend to: "The cat sat on the mat" (relevant)
    \item You ignore: "The dog played in the garden" (irrelevant)
\end{itemize}

\textbf{Attention does this automatically:}
\begin{itemize}
    \item Given a query, compute relevance scores for all inputs
    \item Higher scores = more attention
    \item Output is weighted sum of values, weighted by attention scores
\end{itemize}

\subsubsection{Mathematical Formulation}

\textbf{Attention has three components:}
\begin{itemize}
    \item \textbf{Query (Q):} What we're looking for
    \item \textbf{Keys (K):} What each position represents
    \item \textbf{Values (V):} The actual content at each position
\end{itemize}

\textbf{Step 1: Compute attention scores}
\[
\text{scores} = QK^T
\]

Each element $(i, j)$ is how relevant key $j$ is to query $i$.

\textbf{Step 2: Normalize with softmax}
\[
\text{attention\_weights} = \text{softmax}(\text{scores})
\]

Converts scores to probabilities (sum to 1).

\textbf{Step 3: Weighted sum of values}
\[
\text{output} = \text{attention\_weights} \cdot V
\]

\textbf{Complete formula (scaled dot-product attention):}
\[
\text{Attention}(Q, K, V) = \text{softmax}\left(\frac{QK^T}{\sqrt{d_k}}\right)V
\]

where $d_k$ is the dimension of keys (scaling prevents large dot products).

\begin{theorybox}[Why Scaling by $\sqrt{d_k}$?]
Without scaling, dot products can become very large when $d_k$ is large:
\begin{itemize}
    \item Large dot products → softmax saturates → tiny gradients
    \item Scaling by $\sqrt{d_k}$ keeps values in reasonable range
    \item Ensures stable gradients
\end{itemize}

Example: If $d_k = 512$, dot products could be $\sim 512$ without scaling. After scaling: $\sim \sqrt{512} \approx 22.6$.
\end{theorybox}

\clearpage
\subsubsection{Self-Attention vs Cross-Attention}

\textbf{Self-Attention:} Attention within same sequence

Query, Key, Value all come from the same source.

\textbf{Use case:} Understanding relationships between words in a sentence.

\textbf{Example:}
\begin{itemize}
    \item Sentence: "The animal didn't cross the street because it was too tired"
    \item "it" attends to "animal" (not "street")
\end{itemize}

\textbf{Cross-Attention:} Attention between two sequences

Query from one sequence, Keys and Values from another.

\textbf{Use case:} Machine translation, encoder-decoder models.

\textbf{Example:}
\begin{itemize}
    \item English (Keys/Values): "The cat is black"
    \item French decoder (Query): "Le chat est..."
    \item Decoder attends to relevant English words
\end{itemize}

\subsubsection{Multi-Head Attention}

Instead of one attention mechanism, use multiple in parallel:

\[
\text{MultiHead}(Q, K, V) = \text{Concat}(\text{head}_1, \ldots, \text{head}_h)W^O
\]

where each head is:
\[
\text{head}_i = \text{Attention}(QW^Q_i, KW^K_i, VW^V_i)
\]

\textbf{Why multiple heads?}
\begin{itemize}
    \item Different heads can focus on different aspects
    \item Head 1: Syntax (subject-verb agreement)
    \item Head 2: Semantics (meaning relationships)
    \item Head 3: Position (nearby words)
    \item Richer representations than single attention
\end{itemize}

\textbf{Typical values:}
\begin{itemize}
    \item $h = 8$ or $h = 16$ heads
    \item $d_{model} = 512$ (total dimension)
    \item $d_k = d_v = d_{model} / h = 64$ per head
\end{itemize}

\subsubsection{Positional Encoding}

\textbf{Problem:} Attention has no notion of order!

The sentence "cat chased dog" and "dog chased cat" look identical to attention (same words, same attention scores).

\textbf{Solution:} Add positional information to input embeddings.

\textbf{Sinusoidal positional encoding:}
\[
PE_{(pos, 2i)} = \sin\left(\frac{pos}{10000^{2i/d_{model}}}\right)
\]
\[
PE_{(pos, 2i+1)} = \cos\left(\frac{pos}{10000^{2i/d_{model}}}\right)
\]

where $pos$ is position, $i$ is dimension index.

\textbf{Properties:}
\begin{itemize}
    \item Different frequency for each dimension
    \item Can extrapolate to longer sequences than seen during training
    \item Model can learn to attend by relative position
\end{itemize}

\textbf{Alternative:} Learned positional embeddings (fixed maximum length).

\clearpage
\subsection{Theory: The Transformer Architecture}

\subsubsection{Complete Architecture}

Transformer consists of:
\begin{itemize}
    \item \textbf{Encoder:} Processes input sequence
    \item \textbf{Decoder:} Generates output sequence
\end{itemize}

\textbf{Encoder Block (repeated $N$ times):}
\begin{enumerate}
    \item Multi-head self-attention
    \item Add \& Layer Norm (residual connection)
    \item Feed-forward network (two linear layers with ReLU)
    \item Add \& Layer Norm (residual connection)
\end{enumerate}

\textbf{Decoder Block (repeated $N$ times):}
\begin{enumerate}
    \item Masked multi-head self-attention (can't see future)
    \item Add \& Layer Norm
    \item Multi-head cross-attention (attend to encoder output)
    \item Add \& Layer Norm
    \item Feed-forward network
    \item Add \& Layer Norm
\end{enumerate}

\textbf{Diagram:}
\begin{verbatim}
Input → Embedding → + Positional Encoding
                     ↓
                [Encoder Block] x N
                     ↓
                Encoder Output
                     ↓
Target → Embedding → + Positional Encoding
                     ↓
                [Decoder Block] x N
                     ↓
                Linear → Softmax → Output
\end{verbatim}

\subsubsection{Why Transformers Work So Well}

\textbf{1. Parallelization}

RNN: Must process step-by-step (slow)

Transformer: All positions processed in parallel (fast!)

Training on long sequences: 10-100× faster than RNNs.

\textbf{2. Long-range dependencies}

RNN: Information must flow through many steps (degrades)

Transformer: Direct connection between any two positions (one attention operation)

\textbf{3. Scalability}

Transformers scale beautifully with:
\begin{itemize}
    \item More data (billions of tokens)
    \item More parameters (billions to trillions)
    \item More compute (hundreds of GPUs)
\end{itemize}

\textbf{4. Transfer learning}

Pre-train once on massive data → Fine-tune for specific tasks.

Examples: BERT, GPT, T5.

\textbf{5. Inductive biases}

Minimal assumptions about data structure:
\begin{itemize}
    \item CNNs assume local structure (good for images)
    \item RNNs assume sequential processing (good for sequences)
    \item Transformers learn structure from data (flexible!)
\end{itemize}

\clearpage
\subsection{Implementation: Building Attention from Scratch}

\subsubsection{Scaled Dot-Product Attention}

\begin{lstlisting}
import torch
import torch.nn as nn
import torch.nn.functional as F
import math

def scaled_dot_product_attention(query, key, value, mask=None):
    """
    Scaled dot-product attention.
    
    Args:
        query: (batch, seq_len_q, d_k)
        key: (batch, seq_len_k, d_k)
        value: (batch, seq_len_k, d_v)
        mask: Optional mask (batch, seq_len_q, seq_len_k)
    
    Returns:
        output: (batch, seq_len_q, d_v)
        attention_weights: (batch, seq_len_q, seq_len_k)
    """
    d_k = query.size(-1)
    
    # Compute attention scores: Q @ K^T
    scores = torch.matmul(query, key.transpose(-2, -1))  # (batch, seq_len_q, seq_len_k)
    
    # Scale by sqrt(d_k)
    scores = scores / math.sqrt(d_k)
    
    # Apply mask if provided (set masked positions to -inf)
    if mask is not None:
        scores = scores.masked_fill(mask == 0, -1e9)
    
    # Softmax to get attention weights
    attention_weights = F.softmax(scores, dim=-1)  # (batch, seq_len_q, seq_len_k)
    
    # Weighted sum of values
    output = torch.matmul(attention_weights, value)  # (batch, seq_len_q, d_v)
    
    return output, attention_weights

# Test
batch_size = 2
seq_len = 10
d_k = 64

Q = torch.randn(batch_size, seq_len, d_k)
K = torch.randn(batch_size, seq_len, d_k)
V = torch.randn(batch_size, seq_len, d_k)

output, weights = scaled_dot_product_attention(Q, K, V)
print(f"Output shape: {output.shape}")  # (2, 10, 64)
print(f"Attention weights shape: {weights.shape}")  # (2, 10, 10)
print(f"Weights sum to 1: {weights.sum(dim=-1)}")  # All ones
\end{lstlisting}

\subsubsection{Multi-Head Attention}

\begin{lstlisting}
class MultiHeadAttention(nn.Module):
    """Multi-head attention mechanism."""
    
    def __init__(self, d_model, num_heads, dropout=0.1):
        """
        Args:
            d_model: Model dimension (e.g., 512)
            num_heads: Number of attention heads (e.g., 8)
            dropout: Dropout probability
        """
        super().__init__()
        
        assert d_model % num_heads == 0, "d_model must be divisible by num_heads"
        
        self.d_model = d_model
        self.num_heads = num_heads
        self.d_k = d_model // num_heads  # Dimension per head
        
        # Linear projections for Q, K, V
        self.W_q = nn.Linear(d_model, d_model)
        self.W_k = nn.Linear(d_model, d_model)
        self.W_v = nn.Linear(d_model, d_model)
        
        # Output projection
        self.W_o = nn.Linear(d_model, d_model)
        
        self.dropout = nn.Dropout(dropout)
    
    def split_heads(self, x):
        """
        Split last dimension into (num_heads, d_k).
        
        Input: (batch, seq_len, d_model)
        Output: (batch, num_heads, seq_len, d_k)
        """
        batch_size, seq_len, d_model = x.size()
        
        # Reshape to (batch, seq_len, num_heads, d_k)
        x = x.view(batch_size, seq_len, self.num_heads, self.d_k)
        
        # Transpose to (batch, num_heads, seq_len, d_k)
        return x.transpose(1, 2)
    
    def combine_heads(self, x):
        """
        Combine heads back.
        
        Input: (batch, num_heads, seq_len, d_k)
        Output: (batch, seq_len, d_model)
        """
        batch_size, num_heads, seq_len, d_k = x.size()
        
        # Transpose to (batch, seq_len, num_heads, d_k)
        x = x.transpose(1, 2)
        
        # Reshape to (batch, seq_len, d_model)
        return x.contiguous().view(batch_size, seq_len, self.d_model)
    
    def forward(self, query, key, value, mask=None):
        """
        Args:
            query: (batch, seq_len_q, d_model)
            key: (batch, seq_len_k, d_model)
            value: (batch, seq_len_k, d_model)
            mask: Optional (batch, seq_len_q, seq_len_k)
        """
        batch_size = query.size(0)
        
        # Linear projections
        Q = self.W_q(query)  # (batch, seq_len_q, d_model)
        K = self.W_k(key)    # (batch, seq_len_k, d_model)
        V = self.W_v(value)  # (batch, seq_len_k, d_model)
        
        # Split into multiple heads
        Q = self.split_heads(Q)  # (batch, num_heads, seq_len_q, d_k)
        K = self.split_heads(K)  # (batch, num_heads, seq_len_k, d_k)
        V = self.split_heads(V)  # (batch, num_heads, seq_len_k, d_k)
        
        # Adjust mask for multiple heads
        if mask is not None:
            # Add head dimension: (batch, 1, seq_len_q, seq_len_k)
            mask = mask.unsqueeze(1)
        
        # Scaled dot-product attention
        d_k = Q.size(-1)
        scores = torch.matmul(Q, K.transpose(-2, -1)) / math.sqrt(d_k)
        
        if mask is not None:
            scores = scores.masked_fill(mask == 0, -1e9)
        
        attention_weights = F.softmax(scores, dim=-1)
        attention_weights = self.dropout(attention_weights)
        
        # Apply attention to values
        attention_output = torch.matmul(attention_weights, V)
        # (batch, num_heads, seq_len_q, d_k)
        
        # Combine heads
        attention_output = self.combine_heads(attention_output)
        # (batch, seq_len_q, d_model)
        
        # Final linear projection
        output = self.W_o(attention_output)
        
        return output, attention_weights

# Test
mha = MultiHeadAttention(d_model=512, num_heads=8)
x = torch.randn(2, 10, 512)  # (batch, seq_len, d_model)
output, weights = mha(x, x, x)  # Self-attention

print(f"Output shape: {output.shape}")  # (2, 10, 512)
print(f"Attention weights shape: {weights.shape}")  # (2, 8, 10, 10)
\end{lstlisting}

\clearpage
\subsubsection{Positional Encoding}

\begin{lstlisting}
class PositionalEncoding(nn.Module):
    """Sinusoidal positional encoding."""
    
    def __init__(self, d_model, max_len=5000, dropout=0.1):
        super().__init__()
        self.dropout = nn.Dropout(dropout)
        
        # Create positional encoding matrix
        pe = torch.zeros(max_len, d_model)
        position = torch.arange(0, max_len, dtype=torch.float).unsqueeze(1)
        
        # Compute div_term for scaling
        div_term = torch.exp(
            torch.arange(0, d_model, 2).float() * 
            (-math.log(10000.0) / d_model)
        )
        
        # Apply sine to even indices
        pe[:, 0::2] = torch.sin(position * div_term)
        
        # Apply cosine to odd indices
        pe[:, 1::2] = torch.cos(position * div_term)
        
        # Add batch dimension: (1, max_len, d_model)
        pe = pe.unsqueeze(0)
        
        # Register as buffer (not a parameter, but part of state)
        self.register_buffer('pe', pe)
    
    def forward(self, x):
        """
        Args:
            x: (batch, seq_len, d_model)
        """
        # Add positional encoding
        x = x + self.pe[:, :x.size(1), :]
        return self.dropout(x)

# Test
pe = PositionalEncoding(d_model=512, max_len=100)
x = torch.randn(2, 50, 512)  # (batch, seq_len, d_model)
x_with_pos = pe(x)
print(x_with_pos.shape)  # (2, 50, 512)

# Visualize positional encoding
import matplotlib.pyplot as plt
plt.figure(figsize=(12, 4))
plt.imshow(pe.pe[0, :100, :].detach().numpy(), aspect='auto', cmap='viridis')
plt.xlabel('Dimension')
plt.ylabel('Position')
plt.colorbar()
plt.title('Positional Encoding')
plt.show()
\end{lstlisting}

\clearpage
\subsubsection{Feed-Forward Network}

\begin{lstlisting}
class PositionWiseFeedForward(nn.Module):
    """Position-wise feed-forward network."""
    
    def __init__(self, d_model, d_ff, dropout=0.1):
        """
        Args:
            d_model: Input/output dimension (e.g., 512)
            d_ff: Hidden dimension (typically 4 * d_model = 2048)
            dropout: Dropout probability
        """
        super().__init__()
        
        self.fc1 = nn.Linear(d_model, d_ff)
        self.fc2 = nn.Linear(d_ff, d_model)
        self.dropout = nn.Dropout(dropout)
    
    def forward(self, x):
        """
        Args:
            x: (batch, seq_len, d_model)
        """
        # First linear + ReLU
        x = F.relu(self.fc1(x))
        x = self.dropout(x)
        
        # Second linear
        x = self.fc2(x)
        
        return x

# Test
ffn = PositionWiseFeedForward(d_model=512, d_ff=2048)
x = torch.randn(2, 10, 512)
output = ffn(x)
print(output.shape)  # (2, 10, 512)
\end{lstlisting}

\subsubsection{Complete Transformer Encoder Block}

\begin{lstlisting}
class TransformerEncoderBlock(nn.Module):
    """Single Transformer encoder block."""
    
    def __init__(self, d_model, num_heads, d_ff, dropout=0.1):
        super().__init__()
        
        # Multi-head attention
        self.attention = MultiHeadAttention(d_model, num_heads, dropout)
        
        # Feed-forward network
        self.ffn = PositionWiseFeedForward(d_model, d_ff, dropout)
        
        # Layer normalization
        self.norm1 = nn.LayerNorm(d_model)
        self.norm2 = nn.LayerNorm(d_model)
        
        # Dropout
        self.dropout1 = nn.Dropout(dropout)
        self.dropout2 = nn.Dropout(dropout)
    
    def forward(self, x, mask=None):
        """
        Args:
            x: (batch, seq_len, d_model)
            mask: Optional attention mask
        """
        # Multi-head self-attention with residual and layer norm
        attn_output, _ = self.attention(x, x, x, mask)
        x = x + self.dropout1(attn_output)  # Residual connection
        x = self.norm1(x)  # Layer normalization
        
        # Feed-forward with residual and layer norm
        ffn_output = self.ffn(x)
        x = x + self.dropout2(ffn_output)  # Residual connection
        x = self.norm2(x)  # Layer normalization
        
        return x

# Test
encoder_block = TransformerEncoderBlock(d_model=512, num_heads=8, d_ff=2048)
x = torch.randn(2, 10, 512)
output = encoder_block(x)
print(output.shape)  # (2, 10, 512)
\end{lstlisting}

\begin{pytorchtip}[Layer Norm Placement: Pre-Norm vs Post-Norm]
\textbf{Original Transformer (post-norm):}
\begin{lstlisting}
x = norm(x + sublayer(x))
\end{lstlisting}

\textbf{Modern practice (pre-norm):}
\begin{lstlisting}
x = x + sublayer(norm(x))
\end{lstlisting}

Pre-norm is now preferred:
\begin{itemize}
    \item More stable training
    \item Can train deeper networks
    \item Better gradient flow
\end{itemize}

The code above uses post-norm (original paper), but many modern implementations use pre-norm.
\end{pytorchtip}

\clearpage
% =============================================
% SECTION 11: TRANSFORMERS - PART 2 (PyTorch Built-ins & Exercises)
% =============================================

\subsubsection{Using PyTorch's Built-in Attention}

\begin{lstlisting}
# PyTorch provides nn.MultiheadAttention
attention = nn.MultiheadAttention(
    embed_dim=512,
    num_heads=8,
    dropout=0.1,
    batch_first=True  # Important! Use batch_first=True
)

# Self-attention
x = torch.randn(2, 10, 512)  # (batch, seq_len, embed_dim)
attn_output, attn_weights = attention(x, x, x)

print(attn_output.shape)  # (2, 10, 512)
print(attn_weights.shape)  # (2, 10, 10) if need_weights=True

# Cross-attention (e.g., encoder-decoder)
encoder_output = torch.randn(2, 20, 512)
decoder_input = torch.randn(2, 10, 512)

attn_output, _ = attention(
    query=decoder_input,
    key=encoder_output,
    value=encoder_output
)
print(attn_output.shape)  # (2, 10, 512)
\end{lstlisting}

\begin{warningbox}[batch\_first in MultiheadAttention]
PyTorch's \texttt{nn.MultiheadAttention} defaults to \texttt{batch\_first=False}, expecting shape (seq\_len, batch, embed\_dim).

\textbf{Always use batch\_first=True} for consistency:
\begin{lstlisting}
# CONFUSING (default):
attention = nn.MultiheadAttention(embed_dim=512, num_heads=8)
x = torch.randn(10, 2, 512)  # (seq_len, batch, embed_dim)

# CLEAR (recommended):
attention = nn.MultiheadAttention(embed_dim=512, num_heads=8, 
                                 batch_first=True)
x = torch.randn(2, 10, 512)  # (batch, seq_len, embed_dim)
\end{lstlisting}
\end{warningbox}

\subsubsection{Masking for Causal Attention}

For autoregressive models (e.g., language generation), prevent attending to future positions:

\begin{lstlisting}
def create_causal_mask(seq_len):
    """
    Create causal mask to prevent attending to future positions.
    
    Returns:
        mask: (seq_len, seq_len) with 1s on and below diagonal
    """
    mask = torch.tril(torch.ones(seq_len, seq_len))
    return mask

# Example
mask = create_causal_mask(5)
print(mask)
"""
tensor([[1., 0., 0., 0., 0.],
        [1., 1., 0., 0., 0.],
        [1., 1., 1., 0., 0.],
        [1., 1., 1., 1., 0.],
        [1., 1., 1., 1., 1.]])
"""

# Use with attention
seq_len = 10
causal_mask = create_causal_mask(seq_len)

# PyTorch MultiheadAttention expects mask where True = ignore
# So we invert: 0 = attend, 1 = ignore
attn_mask = (1 - causal_mask).bool()

x = torch.randn(2, seq_len, 512)
attention = nn.MultiheadAttention(embed_dim=512, num_heads=8, 
                                 batch_first=True)
output, _ = attention(x, x, x, attn_mask=attn_mask)
\end{lstlisting}

\subsubsection{Complete Transformer for Classification}

\begin{lstlisting}
class TransformerClassifier(nn.Module):
    """Transformer for sequence classification."""
    
    def __init__(self, vocab_size, d_model, num_heads, num_layers, 
                 num_classes, d_ff, max_len=512, dropout=0.1):
        super().__init__()
        
        # Token embedding
        self.embedding = nn.Embedding(vocab_size, d_model)
        
        # Positional encoding
        self.pos_encoding = PositionalEncoding(d_model, max_len, dropout)
        
        # Transformer encoder layers
        encoder_layer = nn.TransformerEncoderLayer(
            d_model=d_model,
            nhead=num_heads,
            dim_feedforward=d_ff,
            dropout=dropout,
            batch_first=True
        )
        
        self.transformer_encoder = nn.TransformerEncoder(
            encoder_layer,
            num_layers=num_layers
        )
        
        # Classification head
        self.fc = nn.Linear(d_model, num_classes)
        
        self.d_model = d_model
    
    def forward(self, x, mask=None):
        """
        Args:
            x: (batch, seq_len) - token indices
            mask: Optional padding mask (batch, seq_len)
        """
        # Embedding and scaling
        x = self.embedding(x) * math.sqrt(self.d_model)
        
        # Add positional encoding
        x = self.pos_encoding(x)
        
        # Transformer encoding
        x = self.transformer_encoder(x, src_key_padding_mask=mask)
        
        # Global average pooling
        x = x.mean(dim=1)  # (batch, d_model)
        
        # Classification
        logits = self.fc(x)
        
        return logits

# Test
model = TransformerClassifier(
    vocab_size=10000,
    d_model=512,
    num_heads=8,
    num_layers=6,
    num_classes=3,
    d_ff=2048,
    dropout=0.1
)

# Random token sequences
x = torch.randint(0, 10000, (32, 50))  # (batch, seq_len)
output = model(x)
print(output.shape)  # (32, 3)

# Count parameters
total_params = sum(p.numel() for p in model.parameters())
print(f"Total parameters: {total_params:,}")
\end{lstlisting}

\clearpage
\subsubsection{Transformer for Sequence-to-Sequence}

\begin{lstlisting}
class TransformerSeq2Seq(nn.Module):
    """Complete Transformer with encoder and decoder."""
    
    def __init__(self, src_vocab_size, tgt_vocab_size, d_model=512, 
                 num_heads=8, num_layers=6, d_ff=2048, dropout=0.1):
        super().__init__()
        
        # Source embeddings
        self.src_embedding = nn.Embedding(src_vocab_size, d_model)
        self.src_pos_encoding = PositionalEncoding(d_model, dropout=dropout)
        
        # Target embeddings
        self.tgt_embedding = nn.Embedding(tgt_vocab_size, d_model)
        self.tgt_pos_encoding = PositionalEncoding(d_model, dropout=dropout)
        
        # Transformer
        self.transformer = nn.Transformer(
            d_model=d_model,
            nhead=num_heads,
            num_encoder_layers=num_layers,
            num_decoder_layers=num_layers,
            dim_feedforward=d_ff,
            dropout=dropout,
            batch_first=True
        )
        
        # Output projection
        self.fc_out = nn.Linear(d_model, tgt_vocab_size)
        
        self.d_model = d_model
    
    def forward(self, src, tgt, src_mask=None, tgt_mask=None, 
                src_padding_mask=None, tgt_padding_mask=None):
        """
        Args:
            src: (batch, src_seq_len) - source token indices
            tgt: (batch, tgt_seq_len) - target token indices
            src_mask: Optional source attention mask
            tgt_mask: Causal mask for target (prevent seeing future)
            src_padding_mask: Padding mask for source
            tgt_padding_mask: Padding mask for target
        """
        # Embed and add positional encoding
        src = self.src_embedding(src) * math.sqrt(self.d_model)
        src = self.src_pos_encoding(src)
        
        tgt = self.tgt_embedding(tgt) * math.sqrt(self.d_model)
        tgt = self.tgt_pos_encoding(tgt)
        
        # Transformer
        output = self.transformer(
            src, tgt,
            src_mask=src_mask,
            tgt_mask=tgt_mask,
            src_key_padding_mask=src_padding_mask,
            tgt_key_padding_mask=tgt_padding_mask
        )
        
        # Project to vocabulary
        logits = self.fc_out(output)
        
        return logits
    
    def generate(self, src, max_len, start_token, end_token):
        """
        Generate sequence autoregressively.
        
        Args:
            src: (batch, src_seq_len)
            max_len: Maximum generation length
            start_token: Token to start generation
            end_token: Token to stop generation
        """
        self.eval()
        batch_size = src.size(0)
        device = src.device
        
        # Encode source
        src = self.src_embedding(src) * math.sqrt(self.d_model)
        src = self.src_pos_encoding(src)
        memory = self.transformer.encoder(src)
        
        # Start with start_token
        tgt = torch.full((batch_size, 1), start_token, 
                        dtype=torch.long, device=device)
        
        for _ in range(max_len):
            # Create causal mask
            tgt_mask = nn.Transformer.generate_square_subsequent_mask(
                tgt.size(1)
            ).to(device)
            
            # Decode
            tgt_embedded = self.tgt_embedding(tgt) * math.sqrt(self.d_model)
            tgt_embedded = self.tgt_pos_encoding(tgt_embedded)
            
            output = self.transformer.decoder(tgt_embedded, memory, 
                                             tgt_mask=tgt_mask)
            
            # Project to vocabulary
            logits = self.fc_out(output[:, -1, :])  # Last position
            
            # Greedy decoding (take argmax)
            next_token = logits.argmax(dim=-1, keepdim=True)
            
            # Append to sequence
            tgt = torch.cat([tgt, next_token], dim=1)
            
            # Stop if all sequences generated end_token
            if (next_token == end_token).all():
                break
        
        return tgt

# Test
model = TransformerSeq2Seq(
    src_vocab_size=10000,
    tgt_vocab_size=10000,
    d_model=512,
    num_heads=8,
    num_layers=6
)

src = torch.randint(0, 10000, (2, 20))  # Source sequences
tgt = torch.randint(0, 10000, (2, 15))  # Target sequences

# Training forward pass
output = model(src, tgt[:, :-1])  # Shift target by 1
print(output.shape)  # (2, 14, 10000)

# Generation
generated = model.generate(src, max_len=20, start_token=1, end_token=2)
print(generated.shape)  # (2, <=20)
\end{lstlisting}

\clearpage
\subsection{Exercises}

\begin{exercise}[11.1: Attention from Scratch - $\bigstar\bigstar$]
\textbf{Goal:} Implement and understand attention mechanism.

\begin{enumerate}
    \item Implement scaled dot-product attention from scratch (without using the provided code)
    \item Test on random Q, K, V matrices
    \item Verify:
    \begin{itemize}
        \item Attention weights sum to 1
        \item Output shape is correct
        \item Scaling by $\sqrt{d_k}$ stabilizes scores
    \end{itemize}
    \item Visualize attention weights as a heatmap
\end{enumerate}

\textbf{Visualization:}
\begin{lstlisting}
import matplotlib.pyplot as plt

# Compute attention
output, weights = scaled_dot_product_attention(Q, K, V)

# Visualize attention weights (first sample in batch)
plt.figure(figsize=(8, 6))
plt.imshow(weights[0].detach().numpy(), cmap='viridis')
plt.colorbar()
plt.xlabel('Key Position')
plt.ylabel('Query Position')
plt.title('Attention Weights')
plt.show()
\end{lstlisting}
\end{exercise}

\begin{exercise}[11.2: Multi-Head Attention - $\bigstar\bigstar\bigstar$]
\textbf{Goal:} Implement multi-head attention.

\begin{enumerate}
    \item Complete the \texttt{MultiHeadAttention} class
    \item Test with different numbers of heads (1, 2, 4, 8)
    \item Verify outputs are identical when using 1 head vs your scaled dot-product attention
    \item Visualize attention patterns from different heads
    \item Compare parameter counts: single-head vs multi-head
\end{enumerate}

\textbf{Questions:}
\begin{itemize}
    \item Do different heads learn different attention patterns?
    \item How does performance change with number of heads?
\end{itemize}
\end{exercise}

\begin{exercise}[11.3: Positional Encoding Analysis - $\bigstar\bigstar\bigstar$]
\textbf{Goal:} Understand positional encoding.

\begin{enumerate}
    \item Implement sinusoidal positional encoding
    \item Visualize the encoding for positions 0-100
    \item Experiment:
    \begin{itemize}
        \item Train a Transformer with positional encoding
        \item Train without positional encoding
        \item Use learned positional embeddings instead
    \end{itemize}
    \item Compare:
    \begin{itemize}
        \item Final accuracy
        \item Does model learn order information without PE?
        \item Which PE works best?
    \end{itemize}
\end{enumerate}

\textbf{Task:} Sequence classification (e.g., sentiment analysis)
\end{exercise}

\begin{exercise}[11.4: Complete Transformer Encoder - $\bigstar\bigstar\bigstar\bigstar$]
\textbf{Goal:} Build a full Transformer encoder for classification.

\begin{enumerate}
    \item Implement complete Transformer encoder:
    \begin{itemize}
        \item Multi-head attention
        \item Position-wise FFN
        \item Layer normalization
        \item Residual connections
        \item Positional encoding
    \end{itemize}
    \item Train on text classification task (e.g., IMDB sentiment)
    \item Use vocabulary size = 10,000 (most common words)
    \item Hyperparameters:
    \begin{itemize}
        \item $d_{model} = 256$
        \item num\_heads = 8
        \item num\_layers = 4
        \item $d_{ff} = 1024$
    \end{itemize}
    \item Achieve >85\% accuracy
    \item Visualize attention weights on example sentences
\end{enumerate}

\textbf{Dataset preparation:}
\begin{lstlisting}
from torchtext.datasets import IMDB
from torchtext.data.utils import get_tokenizer
from collections import Counter

tokenizer = get_tokenizer('basic_english')

# Build vocabulary from training data
def build_vocab(data_iter, tokenizer, max_size=10000):
    counter = Counter()
    for label, text in data_iter:
        counter.update(tokenizer(text))
    
    vocab = {word: i+2 for i, (word, _) in 
             enumerate(counter.most_common(max_size))}
    vocab['<pad>'] = 0
    vocab['<unk>'] = 1
    
    return vocab

# Encode text
def encode_text(text, vocab, tokenizer, max_len=256):
    tokens = tokenizer(text)[:max_len]
    indices = [vocab.get(token, vocab['<unk>']) for token in tokens]
    
    # Pad to max_len
    indices += [vocab['<pad>']] * (max_len - len(indices))
    
    return torch.LongTensor(indices)
\end{lstlisting}
\end{exercise}

\begin{exercise}[11.5: Causal Attention for Language Modeling - $\bigstar\bigstar\bigstar\bigstar$]
\textbf{Goal:} Implement autoregressive generation with causal masking.

\begin{enumerate}
    \item Build a decoder-only Transformer (like GPT)
    \item Use causal masking (can't attend to future tokens)
    \item Train on character-level language modeling:
    \begin{itemize}
        \item Dataset: Shakespeare text or similar
        \item Task: Predict next character
    \end{itemize}
    \item Implement generation:
    \begin{itemize}
        \item Greedy decoding (argmax)
        \item Top-k sampling
        \item Temperature sampling
    \end{itemize}
    \item Generate text samples and evaluate quality
\end{enumerate}

\textbf{Generation strategies:}
\begin{lstlisting}
def generate_text(model, start_text, max_len=100, temperature=1.0, 
                 top_k=None):
    """Generate text autoregressively."""
    model.eval()
    
    # Encode start text
    tokens = encode(start_text)
    
    with torch.no_grad():
        for _ in range(max_len):
            # Forward pass
            logits = model(tokens)
            
            # Get logits for last position
            logits = logits[-1] / temperature
            
            # Top-k filtering
            if top_k is not None:
                values, _ = torch.topk(logits, top_k)
                logits[logits < values[-1]] = -float('inf')
            
            # Sample from distribution
            probs = F.softmax(logits, dim=-1)
            next_token = torch.multinomial(probs, 1)
            
            # Append to sequence
            tokens = torch.cat([tokens, next_token])
    
    return decode(tokens)
\end{lstlisting}
\end{exercise}

\begin{exercise}[11.6: Attention Visualization - $\bigstar\bigstar\bigstar\bigstar$]
\textbf{Goal:} Understand what Transformers learn.

\begin{enumerate}
    \item Train a Transformer on sequence task
    \item Extract attention weights from different layers and heads
    \item Visualize:
    \begin{itemize}
        \item Which heads focus on local context?
        \item Which heads capture long-range dependencies?
        \item How do patterns differ across layers?
    \end{itemize}
    \item Create attention heatmaps for example sentences
    \item Analyze:
    \begin{itemize}
        \item Do lower layers learn syntax?
        \item Do higher layers learn semantics?
    \end{itemize}
\end{enumerate}

\textbf{Visualization code:}
\begin{lstlisting}
def visualize_attention_heads(model, sentence, tokenizer):
    """Visualize attention patterns from all heads."""
    
    # Encode sentence
    tokens = encode(sentence, tokenizer)
    
    # Forward pass with hooks to capture attention
    attentions = []
    
    def hook_fn(module, input, output):
        # output[1] is attention weights
        attentions.append(output[1].detach())
    
    hooks = []
    for layer in model.transformer_encoder.layers:
        hook = layer.self_attn.register_forward_hook(hook_fn)
        hooks.append(hook)
    
    # Forward
    _ = model(tokens.unsqueeze(0))
    
    # Remove hooks
    for hook in hooks:
        hook.remove()
    
    # Plot attention from each layer and head
    num_layers = len(attentions)
    num_heads = attentions[0].size(1)
    
    fig, axes = plt.subplots(num_layers, num_heads, 
                            figsize=(num_heads*3, num_layers*3))
    
    for layer in range(num_layers):
        for head in range(num_heads):
            ax = axes[layer, head]
            attn = attentions[layer][0, head].cpu().numpy()
            
            im = ax.imshow(attn, cmap='viridis')
            ax.set_title(f'L{layer+1} H{head+1}')
            ax.set_xlabel('Key')
            ax.set_ylabel('Query')
    
    plt.tight_layout()
    plt.show()

# Example
visualize_attention_heads(model, "The cat sat on the mat", tokenizer)
\end{lstlisting}
\end{exercise}

\clearpage
\subsection{Key Takeaways}

\textbf{Why Transformers dominate:}
\begin{itemize}
    \item \textbf{Parallelization:} All positions processed simultaneously (10-100× faster than RNNs)
    \item \textbf{Long-range dependencies:} Direct connections between any positions
    \item \textbf{Scalability:} Works with billions of parameters and tokens
    \item \textbf{Flexibility:} Minimal inductive biases, learns from data
\end{itemize}

\textbf{Core components:}
\begin{itemize}
    \item \textbf{Scaled dot-product attention:} $\text{Attention}(Q,K,V) = \text{softmax}(QK^T/\sqrt{d_k})V$
    \item \textbf{Multi-head attention:} Multiple attention mechanisms in parallel
    \item \textbf{Positional encoding:} Inject position information (no inherent order)
    \item \textbf{Feed-forward networks:} Position-wise transformation
    \item \textbf{Residual connections + Layer norm:} Stable training
\end{itemize}

\textbf{Architecture patterns:}
\begin{itemize}
    \item \textbf{Encoder-only:} BERT (classification, understanding)
    \item \textbf{Decoder-only:} GPT (generation, autoregressive)
    \item \textbf{Encoder-decoder:} T5, BART (translation, summarization)
\end{itemize}

\textbf{Implementation tips:}
\begin{itemize}
    \item Always use \texttt{batch\_first=True} in PyTorch
    \item Scale attention scores by $\sqrt{d_k}$ (stability)
    \item Use causal masking for autoregressive tasks
    \item Pre-norm is more stable than post-norm for deep networks
    \item Typical hyperparameters: $d_{model}=512$, $h=8$, $d_{ff}=2048$
\end{itemize}

\textbf{Training considerations:}
\begin{itemize}
    \item Transformers need lots of data (millions of examples)
    \item Use warmup learning rate schedule
    \item Gradient clipping less critical than RNNs (but still useful)
    \item Dropout on attention weights (0.1) and residual connections
    \item Label smoothing helps (0.1)
\end{itemize}

\textbf{When to use Transformers:}
\begin{itemize}
    \item NLP tasks (now standard)
    \item Long sequences (>100 tokens)
    \item When you have lots of data
    \item When parallelization matters
    \item Vision tasks (Vision Transformer - ViT)
\end{itemize}

\textbf{Limitations:}
\begin{itemize}
    \item \textbf{Quadratic complexity:} $O(n^2)$ in sequence length (memory and compute)
    \item \textbf{Data hungry:} Requires large datasets for good performance
    \item \textbf{Memory intensive:} Attention matrices can be huge
    \item \textbf{Position limit:} Most models have fixed maximum sequence length
\end{itemize}

\textbf{Solutions to limitations:}
\begin{itemize}
    \item Sparse attention (only attend to subset of positions)
    \item Linear attention approximations (e.g., Linformer)
    \item Local attention windows (e.g., Longformer)
    \item Efficient Transformers (e.g., Reformer, Performer)
\end{itemize}

\textbf{Common mistakes:}
\begin{itemize}
    \item Forgetting positional encoding (model can't distinguish positions)
    \item Wrong mask shape or type
    \item Not using causal mask for generation
    \item Forgetting to scale attention scores
    \item Using post-norm for very deep networks (use pre-norm)
\end{itemize}

\textbf{Modern developments:}
\begin{itemize}
    \item \textbf{Vision Transformers (ViT):} Transformers for images
    \item \textbf{BERT:} Bidirectional encoder (masked language modeling)
    \item \textbf{GPT:} Decoder-only (autoregressive generation)
    \item \textbf{T5:} Unified text-to-text framework
    \item \textbf{LLaMA, ChatGPT:} Large language models at scale
\end{itemize}

\clearpage