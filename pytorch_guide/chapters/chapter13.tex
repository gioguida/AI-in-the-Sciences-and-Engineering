% =============================================
% PART III: PRACTICAL SKILLS
% =============================================

\part{Practical Skills}

% =============================================
% SECTION 13: DEBUGGING & BEST PRACTICES
% =============================================

\section{Debugging \& Best Practices}

\subsection{Introduction: Why Debugging Matters}

Deep learning is full of silent failures:
\begin{itemize}
    \item Model trains but doesn't learn (loss stuck at baseline)
    \item Loss is NaN after a few iterations
    \item Training loss decreases but validation doesn't
    \item Model works on toy data but fails on real data
    \item Gradients vanish or explode
\end{itemize}

Unlike traditional programming where bugs crash immediately, deep learning bugs often manifest as \textbf{poor performance}. You need systematic debugging strategies.

\subsection{Common Bugs and How to Fix Them}

\subsubsection{Bug 1: Loss is NaN}

\textbf{Symptoms:}
\begin{lstlisting}
Epoch 1: loss = 2.345
Epoch 2: loss = 1.876
Epoch 3: loss = nan
\end{lstlisting}

\textbf{Causes and fixes:}

\textbf{1. Learning rate too high}
\begin{lstlisting}
# Check: Does loss explode before NaN?
# Fix: Reduce learning rate
optimizer = torch.optim.Adam(model.parameters(), lr=1e-5)  # Try lower
\end{lstlisting}

\textbf{2. Numerical instability in loss}
\begin{lstlisting}
# BAD: Manual log can cause NaN
loss = -torch.log(predictions)  # NaN if predictions = 0

# GOOD: Use stable implementations
loss = F.cross_entropy(logits, targets)  # Numerically stable
# or
loss = F.binary_cross_entropy_with_logits(logits, targets)
\end{lstlisting}

\textbf{3. Exploding gradients (RNNs)}
\begin{lstlisting}
# Fix: Add gradient clipping
torch.nn.utils.clip_grad_norm_(model.parameters(), max_norm=1.0)
\end{lstlisting}

\textbf{4. Data contains NaN or Inf}
\begin{lstlisting}
# Check data
print(torch.isnan(data).any())
print(torch.isinf(data).any())

# Fix: Clean data or add checks
data = torch.nan_to_num(data, nan=0.0, posinf=1e6, neginf=-1e6)
\end{lstlisting}

\textbf{Detection code:}
\begin{lstlisting}
def check_for_nan(model, data, target):
    """Debug NaN issues."""
    # Check inputs
    print(f"Data has NaN: {torch.isnan(data).any()}")
    print(f"Data has Inf: {torch.isinf(data).any()}")
    
    # Forward pass
    output = model(data)
    print(f"Output has NaN: {torch.isnan(output).any()}")
    
    # Backward pass
    loss = criterion(output, target)
    print(f"Loss has NaN: {torch.isnan(loss).any()}")
    loss.backward()
    
    # Check gradients
    for name, param in model.named_parameters():
        if param.grad is not None:
            if torch.isnan(param.grad).any():
                print(f"NaN gradient in {name}")
            if torch.isinf(param.grad).any():
                print(f"Inf gradient in {name}")
\end{lstlisting}

\clearpage
\subsubsection{Bug 2: Loss Not Decreasing}

\textbf{Symptoms:}
\begin{lstlisting}
Epoch 1: loss = 2.345
Epoch 2: loss = 2.342
Epoch 3: loss = 2.340
...
Epoch 100: loss = 2.300  # Barely moving
\end{lstlisting}

\textbf{Causes and fixes:}

\textbf{1. Learning rate too low}
\begin{lstlisting}
# Fix: Increase learning rate
optimizer = torch.optim.Adam(model.parameters(), lr=1e-2)  # Try higher

# Or use learning rate finder
def find_lr(model, train_loader, init_lr=1e-8, final_lr=10):
    """Find optimal learning rate."""
    lrs = []
    losses = []
    
    optimizer = torch.optim.SGD(model.parameters(), lr=init_lr)
    lr_scheduler = torch.optim.lr_scheduler.ExponentialLR(
        optimizer, gamma=(final_lr/init_lr)**(1/100)
    )
    
    for batch in train_loader:
        optimizer.zero_grad()
        loss = compute_loss(model, batch)
        loss.backward()
        optimizer.step()
        
        lrs.append(optimizer.param_groups[0]['lr'])
        losses.append(loss.item())
        
        lr_scheduler.step()
        
        if len(lrs) >= 100:
            break
    
    # Plot
    import matplotlib.pyplot as plt
    plt.plot(lrs, losses)
    plt.xscale('log')
    plt.xlabel('Learning Rate')
    plt.ylabel('Loss')
    plt.title('Learning Rate Finder')
    plt.show()
    
    # Optimal LR is usually where loss decreases fastest
    # (steepest negative gradient)
\end{lstlisting}

\textbf{2. Wrong loss function}
\begin{lstlisting}
# BAD: Using MSE for classification
criterion = nn.MSELoss()  # Wrong for discrete labels!

# GOOD: Use appropriate loss
criterion = nn.CrossEntropyLoss()  # For classification
\end{lstlisting}

\textbf{3. Forgot to zero gradients}
\begin{lstlisting}
# BAD: Gradients accumulate
for batch in dataloader:
    loss = criterion(model(x), y)
    loss.backward()
    optimizer.step()  # FORGOT optimizer.zero_grad()!

# GOOD:
for batch in dataloader:
    optimizer.zero_grad()  # Always zero first!
    loss = criterion(model(x), y)
    loss.backward()
    optimizer.step()
\end{lstlisting}

\textbf{4. Data not normalized}
\begin{lstlisting}
# Check data range
print(f"Data min: {data.min()}, max: {data.max()}")

# Fix: Normalize
mean = data.mean()
std = data.std()
data = (data - mean) / (std + 1e-8)
\end{lstlisting}

\textbf{5. Dead ReLUs or vanishing gradients}
\begin{lstlisting}
# Check activation distributions
def check_activations(model, data):
    activations = {}
    
    def hook(name):
        def hook_fn(module, input, output):
            activations[name] = output.detach()
        return hook_fn
    
    # Register hooks
    for name, module in model.named_modules():
        if isinstance(module, nn.ReLU):
            module.register_forward_hook(hook(name))
    
    model(data)
    
    # Check what percentage is zero
    for name, act in activations.items():
        zero_pct = (act == 0).float().mean() * 100
        print(f"{name}: {zero_pct:.1f}% zeros")
        
        if zero_pct > 50:
            print(f"  WARNING: More than 50% dead neurons!")

# Fix: Use Leaky ReLU or adjust initialization
\end{lstlisting}

\clearpage
\subsubsection{Bug 3: Training Loss Decreases, Validation Doesn't}

\textbf{Symptoms:} Overfitting

\textbf{Causes and fixes:}

\textbf{1. Model too complex for data}
\begin{lstlisting}
# Fix: Reduce model size
model = SmallNet()  # Fewer layers, fewer parameters

# Or increase data
# Or add regularization (see below)
\end{lstlisting}

\textbf{2. No regularization}
\begin{lstlisting}
# Add dropout
model = nn.Sequential(
    nn.Linear(100, 128),
    nn.ReLU(),
    nn.Dropout(0.5),  # Add dropout
    nn.Linear(128, 10)
)

# Add weight decay
optimizer = torch.optim.Adam(model.parameters(), lr=1e-3, 
                            weight_decay=1e-4)  # L2 regularization

# Add data augmentation
transform = transforms.Compose([
    transforms.RandomHorizontalFlip(),
    transforms.RandomCrop(32, padding=4),
    transforms.ToTensor()
])
\end{lstlisting}

\textbf{3. Training too long}
\begin{lstlisting}
# Implement early stopping
class EarlyStopping:
    def __init__(self, patience=10, min_delta=0):
        self.patience = patience
        self.min_delta = min_delta
        self.counter = 0
        self.best_loss = None
        self.should_stop = False
    
    def __call__(self, val_loss):
        if self.best_loss is None:
            self.best_loss = val_loss
        elif val_loss > self.best_loss - self.min_delta:
            self.counter += 1
            if self.counter >= self.patience:
                self.should_stop = True
        else:
            self.best_loss = val_loss
            self.counter = 0

# Usage
early_stopping = EarlyStopping(patience=10)

for epoch in range(max_epochs):
    train_loss = train_epoch()
    val_loss = validate()
    
    early_stopping(val_loss)
    if early_stopping.should_stop:
        print(f"Early stopping at epoch {epoch}")
        break
\end{lstlisting}

\subsubsection{Bug 4: Model Works on Small Data, Fails on Full Dataset}

\textbf{Causes and fixes:}

\textbf{1. Batch size too small}
\begin{lstlisting}
# Batch norm unstable with small batches
# Fix: Increase batch size or use Layer Norm

# Or use Gradient Accumulation
accumulation_steps = 4

for i, batch in enumerate(dataloader):
    loss = criterion(model(x), y)
    loss = loss / accumulation_steps  # Scale loss
    loss.backward()
    
    if (i + 1) % accumulation_steps == 0:
        optimizer.step()
        optimizer.zero_grad()
\end{lstlisting}

\textbf{2. Data distribution shift}
\begin{lstlisting}
# Check: Are train and val distributions similar?
def check_distribution(train_data, val_data):
    print(f"Train mean: {train_data.mean()}, std: {train_data.std()}")
    print(f"Val mean: {val_data.mean()}, std: {val_data.std()}")
    
    # Visualize
    import matplotlib.pyplot as plt
    plt.hist(train_data.flatten(), bins=50, alpha=0.5, label='Train')
    plt.hist(val_data.flatten(), bins=50, alpha=0.5, label='Val')
    plt.legend()
    plt.show()

# Fix: Normalize using training statistics
mean = train_data.mean()
std = train_data.std()

train_data = (train_data - mean) / std
val_data = (val_data - mean) / std  # Use training stats!
\end{lstlisting}

\clearpage
\subsection{Debugging Strategies}

\subsubsection{Start Simple, Add Complexity}

\textbf{Step 1: Overfit on one batch}

If you can't overfit one batch, something is fundamentally wrong.

\begin{lstlisting}
# Get one batch
x, y = next(iter(train_loader))

# Try to overfit it
for epoch in range(1000):
    optimizer.zero_grad()
    output = model(x)
    loss = criterion(output, y)
    loss.backward()
    optimizer.step()
    
    if epoch % 100 == 0:
        print(f"Epoch {epoch}: Loss = {loss.item():.4f}")

# Should reach near-zero loss
# If not, check:
# - Model has enough capacity
# - Loss function is correct
# - Learning rate is reasonable
\end{lstlisting}

\textbf{Step 2: Overfit on small dataset}

Try 100 samples. Should still overfit easily.

\textbf{Step 3: Add validation}

Once overfitting works, add validation and regularization.

\textbf{Step 4: Scale to full dataset}

Now train on full data with proper regularization.

\subsubsection{Check Shapes at Every Step}

\begin{lstlisting}
def debug_shapes(model, x):
    """Print shapes through the network."""
    print(f"Input: {x.shape}")
    
    for i, (name, module) in enumerate(model.named_children()):
        x = module(x)
        print(f"{i}. {name}: {x.shape}")
    
    return x

# Usage
x = torch.randn(4, 3, 32, 32)
output = debug_shapes(model, x)
"""
Input: torch.Size([4, 3, 32, 32])
0. conv1: torch.Size([4, 64, 32, 32])
1. relu: torch.Size([4, 64, 32, 32])
2. pool: torch.Size([4, 64, 16, 16])
...
"""
\end{lstlisting}

\subsubsection{Visualize Gradients}

\begin{lstlisting}
def plot_grad_flow(named_parameters):
    """
    Plots gradient flow through the network.
    Helps identify vanishing/exploding gradients.
    """
    ave_grads = []
    max_grads = []
    layers = []
    
    for n, p in named_parameters:
        if p.requires_grad and p.grad is not None:
            layers.append(n)
            ave_grads.append(p.grad.abs().mean().item())
            max_grads.append(p.grad.abs().max().item())
    
    import matplotlib.pyplot as plt
    plt.figure(figsize=(12, 6))
    plt.bar(range(len(ave_grads)), ave_grads, alpha=0.5, label='mean')
    plt.bar(range(len(max_grads)), max_grads, alpha=0.5, label='max')
    plt.hlines(0, 0, len(ave_grads), linewidth=2, color='k')
    plt.xticks(range(len(layers)), layers, rotation='vertical')
    plt.xlabel('Layers')
    plt.ylabel('Gradient magnitude')
    plt.legend()
    plt.title('Gradient Flow')
    plt.grid(True)
    plt.tight_layout()
    plt.show()

# Usage after loss.backward()
plot_grad_flow(model.named_parameters())
\end{lstlisting}

\subsubsection{Use Hooks for Debugging}

\begin{lstlisting}
# Forward hook: inspect activations
def forward_hook(module, input, output):
    print(f"Forward: {output.shape}, mean={output.mean():.4f}")

# Backward hook: inspect gradients
def backward_hook(module, grad_input, grad_output):
    print(f"Backward: grad mean={grad_output[0].mean():.4f}")

# Register hooks
for name, module in model.named_modules():
    if isinstance(module, nn.Linear):
        module.register_forward_hook(forward_hook)
        module.register_backward_hook(backward_hook)

# Now run forward and backward
output = model(x)
loss = criterion(output, y)
loss.backward()

# Hooks will print debug info
\end{lstlisting}

\clearpage
\subsection{Best Practices for Development}

\subsubsection{Code Organization}

\textbf{Project structure:}
\begin{verbatim}
project/
├── data/
│   ├── __init__.py
│   ├── dataset.py        # Dataset classes
│   └── transforms.py     # Data augmentation
├── models/
│   ├── __init__.py
│   ├── resnet.py         # Model architectures
│   └── utils.py          # Model utilities
├── utils/
│   ├── __init__.py
│   ├── train.py          # Training utilities
│   ├── eval.py           # Evaluation utilities
│   └── metrics.py        # Custom metrics
├── configs/
│   └── config.yaml       # Hyperparameters
├── train.py              # Main training script
├── evaluate.py           # Evaluation script
└── requirements.txt      # Dependencies
\end{verbatim}

\textbf{Separate concerns:}
\begin{lstlisting}
# models/resnet.py
class ResNet(nn.Module):
    """Model definition only."""
    def __init__(self, ...):
        # Architecture
    
    def forward(self, x):
        # Forward pass
        return x

# utils/train.py
def train_epoch(model, loader, optimizer, criterion, device):
    """Training logic."""
    model.train()
    total_loss = 0
    
    for batch in loader:
        # Training step
        ...
    
    return total_loss / len(loader)

# train.py
def main():
    # Configuration
    # Data loading
    # Model creation
    # Training loop
    pass

if __name__ == "__main__":
    main()
\end{lstlisting}

\subsubsection{Configuration Management}

\begin{lstlisting}
# config.yaml
model:
  type: resnet18
  num_classes: 10

training:
  batch_size: 128
  epochs: 100
  learning_rate: 0.001
  weight_decay: 0.0001

data:
  root: ./data
  num_workers: 4
  augmentation: true

# Load config
import yaml

with open('config.yaml', 'r') as f:
    config = yaml.safe_load(f)

# Use config
batch_size = config['training']['batch_size']
lr = config['training']['learning_rate']
\end{lstlisting}

\subsubsection{Experiment Tracking}

\begin{lstlisting}
# Simple logging
import logging

logging.basicConfig(
    level=logging.INFO,
    format='%(asctime)s - %(levelname)s - %(message)s',
    handlers=[
        logging.FileHandler('training.log'),
        logging.StreamHandler()
    ]
)

logger = logging.getLogger(__name__)

# Use in training
logger.info(f"Epoch {epoch}: Train Loss = {train_loss:.4f}")

# TensorBoard (better)
from torch.utils.tensorboard import SummaryWriter

writer = SummaryWriter('runs/experiment_1')

for epoch in range(num_epochs):
    train_loss = train_epoch()
    val_loss = validate()
    
    writer.add_scalar('Loss/train', train_loss, epoch)
    writer.add_scalar('Loss/val', val_loss, epoch)
    writer.add_scalar('LR', optimizer.param_groups[0]['lr'], epoch)

writer.close()

# View: tensorboard --logdir=runs
\end{lstlisting}

\clearpage
\subsubsection{Reproducibility}

\begin{lstlisting}
import random
import numpy as np
import torch

def set_seed(seed=42):
    """Set all random seeds for reproducibility."""
    random.seed(seed)
    np.random.seed(seed)
    torch.manual_seed(seed)
    torch.cuda.manual_seed(seed)
    torch.cuda.manual_seed_all(seed)
    
    # Make cudnn deterministic
    torch.backends.cudnn.deterministic = True
    torch.backends.cudnn.benchmark = False

# Call at start of training
set_seed(42)

# Note: This makes training slower but reproducible
# For production: set benchmark=True for speed
\end{lstlisting}

\subsubsection{Checkpointing}

\begin{lstlisting}
def save_checkpoint(model, optimizer, epoch, loss, path):
    """Save model checkpoint."""
    torch.save({
        'epoch': epoch,
        'model_state_dict': model.state_dict(),
        'optimizer_state_dict': optimizer.state_dict(),
        'loss': loss,
    }, path)

def load_checkpoint(model, optimizer, path):
    """Load model checkpoint."""
    checkpoint = torch.load(path)
    model.load_state_dict(checkpoint['model_state_dict'])
    optimizer.load_state_dict(checkpoint['optimizer_state_dict'])
    epoch = checkpoint['epoch']
    loss = checkpoint['loss']
    return epoch, loss

# Save best model
best_loss = float('inf')

for epoch in range(num_epochs):
    train_loss = train_epoch()
    val_loss = validate()
    
    # Save if best
    if val_loss < best_loss:
        best_loss = val_loss
        save_checkpoint(model, optimizer, epoch, val_loss, 
                       'best_model.pth')
    
    # Save periodic checkpoints
    if epoch % 10 == 0:
        save_checkpoint(model, optimizer, epoch, val_loss,
                       f'checkpoint_epoch_{epoch}.pth')
\end{lstlisting}

\subsection{Debugging Checklist}

\textbf{When starting a new project:}
\begin{enumerate}
    \item ☐ Can model overfit one batch?
    \item ☐ Can model overfit small dataset (100 samples)?
    \item ☐ Are shapes correct at every layer?
    \item ☐ Is loss function appropriate for task?
    \item ☐ Is learning rate reasonable? (Try LR finder)
    \item ☐ Is data normalized?
    \item ☐ Are you using the right optimizer?
    \item ☐ Did you set model.train() / model.eval()?
\end{enumerate}

\textbf{When loss is NaN:}
\begin{enumerate}
    \item ☐ Check for NaN in data
    \item ☐ Reduce learning rate
    \item ☐ Add gradient clipping
    \item ☐ Use stable loss functions
    \item ☐ Check for division by zero
\end{enumerate}

\textbf{When loss doesn't decrease:}
\begin{enumerate}
    \item ☐ Try higher learning rate
    \item ☐ Check if you're zeroing gradients
    \item ☐ Verify loss function is correct
    \item ☐ Check data normalization
    \item ☐ Verify model has enough capacity
    \item ☐ Check for dead ReLUs
\end{enumerate}

\textbf{When overfitting:}
\begin{enumerate}
    \item ☐ Add dropout
    \item ☐ Add weight decay
    \item ☐ Use data augmentation
    \item ☐ Reduce model size
    \item ☐ Get more data
    \item ☐ Use early stopping
\end{enumerate}

\clearpage
\subsection{Key Takeaways}

\textbf{Systematic debugging:}
\begin{itemize}
    \item Start simple (overfit one batch)
    \item Add complexity gradually
    \item Check shapes at every step
    \item Visualize gradients and activations
    \item Use hooks for inspection
\end{itemize}

\textbf{Common bugs:}
\begin{itemize}
    \item NaN loss: LR too high, numerical instability, exploding gradients
    \item Loss not decreasing: LR too low, wrong loss, forgot zero\_grad, bad data
    \item Overfitting: Model too complex, no regularization, training too long
    \item Works on toy data, fails on real: Batch size, distribution shift
\end{itemize}

\textbf{Best practices:}
\begin{itemize}
    \item Separate code into modules (data, models, training)
    \item Use configuration files
    \item Track experiments (TensorBoard)
    \item Set random seeds for reproducibility
    \item Save checkpoints regularly
    \item Log everything
\end{itemize}

\textbf{Development workflow:}
\begin{enumerate}
    \item Start with simple baseline
    \item Verify can overfit small data
    \item Add regularization
    \item Scale to full dataset
    \item Tune hyperparameters
    \item Validate on held-out test set
\end{enumerate}

\textbf{Tools for debugging:}
\begin{itemize}
    \item Print statements (shapes, values, statistics)
    \item Hooks (forward and backward)
    \item TensorBoard (visualize training)
    \item Gradient visualization
    \item Learning rate finder
    \item Activation distribution checks
\end{itemize}

\textbf{Remember:}
\begin{itemize}
    \item Deep learning bugs are often silent (poor performance, not crashes)
    \item Always start with simplest possible setup
    \item Check your data first (garbage in = garbage out)
    \item One change at a time (isolate what works)
    \item Keep good records (what did you try, what worked)
\end{itemize}

\clearpage