% =============================================
% SECTION 2: AUTOGRAD - AUTOMATIC DIFFERENTIATION
% =============================================

\section{Autograd: Automatic Differentiation}

\subsection{Introduction: Why Autograd Matters}

Deep learning is fundamentally an optimization problem. We have a model with parameters (weights), a loss function that measures how wrong our predictions are, and we want to adjust the parameters to minimize the loss.

The key insight: we use \textbf{gradient descent}. We compute the gradient of the loss with respect to each parameter, then update the parameters in the direction that decreases the loss:

\[
\theta_{\text{new}} = \theta_{\text{old}} - \eta \nabla_\theta \mathcal{L}
\]

For a neural network with millions of parameters, computing these gradients by hand is impossible. PyTorch's \texttt{autograd} system does this automatically through the chain rule.

\subsection{Theory: Computational Graphs and Backpropagation}

\subsubsection{The Computational Graph}

Every operation you perform on a tensor creates a \textbf{computational graph}. This graph tracks how each tensor was computed from other tensors.

\textbf{Example:} Simple computation

\begin{lstlisting}
import torch

x = torch.tensor(2.0, requires_grad=True)
y = torch.tensor(3.0, requires_grad=True)
z = x * y + x**2  # z = xy + x^2
\end{lstlisting}

This creates a graph:

\begin{center}
\begin{verbatim}
    x (leaf)    y (leaf)
      |           |
      +-----+-----+
      |     |     |
     x^2    *     |
       \   /      |
         +  ------+
         |
         z
\end{verbatim}
\end{center}

\textbf{Key Concepts:}

\begin{itemize}
    \item \textbf{Leaf tensors:} Input tensors created by the user with \texttt{requires\_grad=True}
    \item \textbf{Non-leaf tensors:} Result of operations on other tensors
    \item \textbf{grad\_fn:} Each non-leaf tensor remembers the operation that created it
\end{itemize}

\subsubsection{Backpropagation: The Chain Rule}

When you call \texttt{z.backward()}, PyTorch:
\begin{enumerate}
    \item Traverses the graph backwards from \texttt{z}
    \item Applies the chain rule at each operation
    \item Accumulates gradients in the \texttt{.grad} attribute of leaf tensors
\end{enumerate}

For our example:
\[
\frac{\partial z}{\partial x} = \frac{\partial}{\partial x}(xy + x^2) = y + 2x
\]
\[
\frac{\partial z}{\partial y} = \frac{\partial}{\partial y}(xy + x^2) = x
\]

With $x=2, y=3$: $\frac{\partial z}{\partial x} = 7$, $\frac{\partial z}{\partial y} = 2$

\begin{theorybox}[Why "Automatic"?]
You never write the derivative formulas! PyTorch knows the derivative of every operation (add, multiply, sin, exp, matrix multiply, etc.) and chains them automatically. This is the magic of autograd.
\end{theorybox}

\subsubsection{Dynamic Computation Graphs}

PyTorch uses \textbf{dynamic computation graphs} (define-by-run):
\begin{itemize}
    \item The graph is built \textit{during} the forward pass
    \item You can use Python control flow (if statements, loops)
    \item The graph can be different each time you run the code
    \item After \texttt{.backward()}, the graph is \textit{destroyed} (by default)
\end{itemize}

This is different from TensorFlow 1.x (static graphs) and makes PyTorch more flexible for research.

\subsection{Implementation: Using Autograd in Practice}

\subsubsection{Basic Gradient Computation}

\begin{lstlisting}
import torch

# Create tensors with gradient tracking
x = torch.tensor(2.0, requires_grad=True)
y = torch.tensor(3.0, requires_grad=True)

# Forward pass: compute z
z = x**2 + 2*x*y + y**2

print(z)  # tensor(25., grad_fn=<AddBackward0>)
print(z.grad_fn)  # Shows the operation that created z

# Backward pass: compute gradients
z.backward()

# Gradients are now in .grad
print(x.grad)  # dz/dx = 2x + 2y = 10
print(y.grad)  # dz/dy = 2x + 2y = 10
\end{lstlisting}

\subsubsection{Leaf vs Non-Leaf Tensors}

\begin{lstlisting}
x = torch.tensor(2.0, requires_grad=True)
y = x * 2

print(x.is_leaf)  # True - created by user
print(y.is_leaf)  # False - result of operation

# Only leaf tensors store gradients by default
z = y.sum()
z.backward()

print(x.grad)  # Available! x is a leaf
print(y.grad)  # None! Non-leaf gradients not retained
\end{lstlisting}

To retain gradients for non-leaf tensors:

\begin{lstlisting}
x = torch.tensor([1.0, 2.0, 3.0], requires_grad=True)
y = x * 2
y.retain_grad()  # Tell PyTorch to keep this gradient

z = y.sum()
z.backward()

print(x.grad)  # tensor([2., 2., 2.])
print(y.grad)  # tensor([1., 1., 1.]) - now available!
\end{lstlisting}

\clearpage
\subsubsection{Gradient Accumulation and Zeroing}

\begin{warningbox}[Gradients Accumulate!]
Calling \texttt{.backward()} multiple times \textbf{adds} to existing gradients. You must zero them manually between iterations.
\end{warningbox}

\begin{lstlisting}
x = torch.tensor(2.0, requires_grad=True)

# First backward pass
y = x**2
y.backward()
print(x.grad)  # tensor(4.)

# Second backward pass without zeroing
y = x**2
y.backward()
print(x.grad)  # tensor(8.) - DOUBLED!

# Correct approach: zero gradients first
x.grad.zero_()
y = x**2
y.backward()
print(x.grad)  # tensor(4.) - correct again
\end{lstlisting}

Why does this matter? In training loops, we compute gradients for each batch and update weights. Between batches, we must zero gradients:

\begin{lstlisting}
# Training loop pattern (simplified)
for batch in dataloader:
    # Zero gradients from previous batch
    optimizer.zero_grad()  # Or: model.zero_grad()
    
    # Forward pass
    output = model(batch)
    loss = criterion(output, target)
    
    # Backward pass
    loss.backward()
    
    # Update weights
    optimizer.step()
\end{lstlisting}

\subsubsection{Vector-Jacobian Products}

\texttt{.backward()} can only be called on scalar outputs (or you must provide a gradient argument):

\begin{lstlisting}
# Scalar output - works
x = torch.tensor([1.0, 2.0, 3.0], requires_grad=True)
y = x.sum()  # Scalar
y.backward()
print(x.grad)  # Works!

# Vector output - need gradient argument
x = torch.tensor([1.0, 2.0, 3.0], requires_grad=True)
y = x * 2  # Vector: [2, 4, 6]
# y.backward()  # ERROR! Output is not scalar

# Provide gradient (weights for each output)
gradient = torch.tensor([1.0, 1.0, 1.0])
y.backward(gradient)
print(x.grad)  # tensor([2., 2., 2.])
\end{lstlisting}

\textbf{What's happening?} When output is a vector $\mathbf{y}$, we compute:
\[
\text{x.grad} = \frac{\partial \mathbf{y}}{\partial \mathbf{x}}^T \cdot \text{gradient}
\]

This is a \textbf{vector-Jacobian product}. For scalar loss, gradient is implicitly 1.

\subsubsection{Disabling Gradient Tracking}

Sometimes you don't need gradients (inference, debugging, certain computations):

\begin{lstlisting}
x = torch.tensor([1.0, 2.0], requires_grad=True)

# Method 1: torch.no_grad() context
with torch.no_grad():
    y = x * 2
    z = y.sum()
print(z.requires_grad)  # False
# z.backward()  # ERROR - can't backprop

# Method 2: detach() - creates a view without gradients
y = x * 2
y_detached = y.detach()
print(y.requires_grad)         # True
print(y_detached.requires_grad) # False

# Method 3: requires_grad_() - in-place toggle
x = torch.tensor([1.0, 2.0], requires_grad=True)
x.requires_grad_(False)
print(x.requires_grad)  # False
\end{lstlisting}

\begin{pytorchtip}[When to Disable Gradients]
\begin{itemize}
    \item \textbf{Inference:} Use \texttt{torch.no\_grad()} or \texttt{torch.inference\_mode()} for evaluation
    \item \textbf{Freezing layers:} Set \texttt{param.requires\_grad = False} for parameters you don't want to train
    \item \textbf{Saving memory:} Gradient tracking uses extra memory
    \item \textbf{Performance:} Operations are faster without gradient tracking
\end{itemize}
\end{pytorchtip}

\clearpage
\subsubsection{In-Place Operations}

\begin{warningbox}[In-Place Operations Break Gradients]
In-place operations (those ending with \texttt{\_}) can cause errors in backpropagation.
\end{warningbox}

\begin{lstlisting}
# BAD: In-place operation on tensor used in computation
x = torch.tensor([1.0, 2.0], requires_grad=True)
y = x * 2
x.add_(1)  # Modifies x in-place
# y.backward(torch.ones_like(y))  # ERROR! x was modified

# GOOD: Don't modify tensors that are part of the graph
x = torch.tensor([1.0, 2.0], requires_grad=True)
y = x * 2
z = y.sum()
z.backward()  # Works!
\end{lstlisting}

However, in-place operations are fine for parameters during optimization (outside the graph):

\begin{lstlisting}
# This is OK - happens after .backward()
with torch.no_grad():
    x.add_(-learning_rate * x.grad)
\end{lstlisting}

\subsubsection{Retaining Graphs}

By default, the computation graph is freed after \texttt{.backward()}:

\begin{lstlisting}
x = torch.tensor(2.0, requires_grad=True)
y = x**2
y.backward()
# y.backward()  # ERROR! Graph was freed

# To keep the graph:
x = torch.tensor(2.0, requires_grad=True)
y = x**2
y.backward(retain_graph=True)
y.backward()  # OK now! But remember to zero gradients if needed
\end{lstlisting}

When is this useful? Multiple backward passes (rare), or when you need to call \texttt{.backward()} multiple times in a single forward pass.

\subsubsection{Higher-Order Derivatives}

PyTorch can compute gradients of gradients:

\begin{lstlisting}
# First derivative
x = torch.tensor(2.0, requires_grad=True)
y = x**3  # y = x^3

dy_dx = torch.autograd.grad(y, x, create_graph=True)[0]
print(dy_dx)  # 3*x^2 = 12

# Second derivative
d2y_dx2 = torch.autograd.grad(dy_dx, x)[0]
print(d2y_dx2)  # 6*x = 12
\end{lstlisting}

\textbf{Note:} We use \texttt{torch.autograd.grad()} instead of \texttt{.backward()} here because it gives us the gradient as a tensor (useful for higher-order derivatives).

\begin{pytorchtip}[torch.autograd.grad vs .backward()]
\begin{itemize}
    \item \texttt{.backward()}: Accumulates gradients in \texttt{.grad} attribute. Standard for training.
    \item \texttt{torch.autograd.grad()}: Returns gradients as tensors. Useful for:
    \begin{itemize}
        \item Higher-order derivatives
        \item Multiple gradient computations
        \item When you need gradient as a value, not accumulated
    \end{itemize}
\end{itemize}
\end{pytorchtip}

\subsection{Implementation: Common Patterns and Debugging}

\subsubsection{Checking Gradients}

When implementing custom operations or debugging, verify gradients numerically:

\begin{lstlisting}
import torch

def numerical_gradient(f, x, eps=1e-5):
    """Compute gradient numerically using finite differences."""
    grad = torch.zeros_like(x)
    for i in range(x.numel()):
        x_plus = x.clone()
        x_plus.flatten()[i] += eps
        x_minus = x.clone()
        x_minus.flatten()[i] -= eps
        grad.flatten()[i] = (f(x_plus) - f(x_minus)) / (2 * eps)
    return grad

# Test it
x = torch.tensor([2.0, 3.0], requires_grad=True)
f = lambda t: (t**2).sum()

# Analytical gradient
y = f(x)
y.backward()
analytical_grad = x.grad

# Numerical gradient
numerical_grad = numerical_gradient(f, x)

print("Analytical:", analytical_grad)  # [4., 6.]
print("Numerical:", numerical_grad)    # [4., 6.] (approximately)
print("Close?", torch.allclose(analytical_grad, numerical_grad))
\end{lstlisting}

PyTorch also provides \texttt{torch.autograd.gradcheck}:

\begin{lstlisting}
from torch.autograd import gradcheck

x = torch.randn(3, 4, requires_grad=True, dtype=torch.float64)
func = lambda t: (t**2).sum()

# Check if gradients are correct
test = gradcheck(func, x, eps=1e-6)
print(f"Gradient check passed: {test}")
\end{lstlisting}

\subsubsection{Gradient Flow Debugging}

When training doesn't work, check if gradients are flowing:

\begin{lstlisting}
import torch.nn as nn

# Simple model
model = nn.Sequential(
    nn.Linear(10, 50),
    nn.ReLU(),
    nn.Linear(50, 1)
)

# Forward pass
x = torch.randn(32, 10)
y = model(x)
loss = y.mean()

# Backward pass
loss.backward()

# Check gradients
for name, param in model.named_parameters():
    if param.grad is not None:
        print(f"{name}: grad mean = {param.grad.abs().mean():.6f}")
    else:
        print(f"{name}: NO GRADIENT!")
\end{lstlisting}

Common issues:
\begin{itemize}
    \item \textbf{All zeros:} Dead ReLUs, wrong loss function
    \item \textbf{None:} Part of model disconnected from loss
    \item \textbf{Very large:} Exploding gradients (lower learning rate, add gradient clipping)
    \item \textbf{Very small:} Vanishing gradients (different architecture, normalization)
\end{itemize}

\clearpage
\subsubsection{Custom Autograd Functions}

For operations PyTorch doesn't support, or to override gradients:

\begin{lstlisting}
class MySquare(torch.autograd.Function):
    @staticmethod
    def forward(ctx, input):
        """
        Forward pass: compute output and save what's needed for backward.
        ctx is a context object for saving information.
        """
        ctx.save_for_backward(input)
        return input ** 2
    
    @staticmethod
    def backward(ctx, grad_output):
        """
        Backward pass: compute gradient of loss w.r.t. input.
        grad_output is the gradient of loss w.r.t. output.
        Return gradient w.r.t. each input (input only here).
        """
        input, = ctx.saved_tensors
        grad_input = grad_output * 2 * input
        return grad_input

# Use it
x = torch.tensor([1.0, 2.0, 3.0], requires_grad=True)
y = MySquare.apply(x)  # Note: .apply() not ()
z = y.sum()
z.backward()

print(x.grad)  # [2., 4., 6.] = 2*x
\end{lstlisting}

We'll use this for the exercises!

\subsection{Exercises}

\begin{exercise}[2.1: Basic Gradient Computation - $\bigstar\bigstar$]
\textbf{Goal:} Get comfortable with \texttt{.backward()} and \texttt{.grad}.

\begin{enumerate}
    \item Create tensors $a=3$, $b=4$ with \texttt{requires\_grad=True}
    \item Compute $c = a^2 + b^2$
    \item Call \texttt{c.backward()} and verify:
    \begin{itemize}
        \item \texttt{a.grad} = $2a = 6$
        \item \texttt{b.grad} = $2b = 8$
    \end{itemize}
    \item Now compute $d = 2a + 3b$ and call \texttt{d.backward()}
    \item What are \texttt{a.grad} and \texttt{b.grad} now? Why?
\end{enumerate}

\textbf{Hint:} Remember that gradients accumulate!
\end{exercise}

\begin{exercise}[2.2: Matrix Gradients - $\bigstar\bigstar$]
\textbf{Goal:} Understand gradients for matrix operations.

\begin{enumerate}
    \item Create matrix $W$ of shape (3, 4) with random values, \texttt{requires\_grad=True}
    \item Create input $x$ of shape (4,) with random values (no gradients needed)
    \item Compute $y = Wx$ (matrix-vector product)
    \item Compute $L = ||y||^2$ (sum of squares)
    \item Call \texttt{L.backward()} and check the shape of \texttt{W.grad}
    \item Verify it has the same shape as $W$
\end{enumerate}

\textbf{Starter code:}
\begin{lstlisting}
import torch
torch.manual_seed(42)

W = torch.randn(3, 4, requires_grad=True)
x = torch.randn(4)

# Your code here
\end{lstlisting}
\end{exercise}

\begin{exercise}[2.3: Gradient Accumulation - $\bigstar\bigstar\bigstar$]
\textbf{Goal:} Master gradient zeroing in training loops.

Implement a mini training loop (without a real model):
\begin{enumerate}
    \item Create a parameter \texttt{w} initialized to 5.0
    \item For 10 iterations:
    \begin{itemize}
        \item Compute $\text{loss} = (w - 2)^2$ (we want w to reach 2)
        \item Compute gradient
        \item Update: $w = w - 0.1 \cdot \text{grad}$
        \item \textbf{Zero the gradient}
    \end{itemize}
    \item Print the final value of \texttt{w} (should be close to 2)
\end{enumerate}

\textbf{Hint:} Use \texttt{with torch.no\_grad():} for the update step.
\end{exercise}

\begin{exercise}[2.4: Gradient Detaching - $\bigstar\bigstar\bigstar$]
\textbf{Goal:} Understand when and why to detach gradients.

\begin{enumerate}
    \item Create $x = [1, 2, 3]$ with gradients
    \item Compute $y = x^2$
    \item Detach $y$ to create $y_{\text{detach}}$
    \item Compute $z_1 = y.sum()$ and $z_2 = y_{\text{detach}}.sum()$
    \item Try to call \texttt{z1.backward()} - what happens?
    \item Try to call \texttt{z2.backward()} - what happens?
    \item Why would detaching be useful in training?
\end{enumerate}
\end{exercise}

\begin{exercise}[2.5: Higher-Order Derivatives - $\bigstar\bigstar\bigstar\bigstar$]
\textbf{Goal:} Compute second derivatives (useful for some scientific ML methods).

\begin{enumerate}
    \item Create $x = 2.0$ with gradients
    \item Compute $y = \sin(x)$
    \item Use \texttt{torch.autograd.grad()} to compute $\frac{dy}{dx}$ with \texttt{create\_graph=True}
    \item Compute the second derivative $\frac{d^2y}{dx^2}$
    \item Verify analytically: $y=\sin(x) \Rightarrow y'=\cos(x) \Rightarrow y''=-\sin(x)$
    \item At $x=2$: $y''=-\sin(2) \approx -0.909$
\end{enumerate}

\textbf{Starter code:}
\begin{lstlisting}
import torch

x = torch.tensor(2.0, requires_grad=True)
y = torch.sin(x)

# First derivative
dy_dx = torch.autograd.grad(y, x, create_graph=True)[0]

# Your code for second derivative
\end{lstlisting}
\end{exercise}

\begin{exercise}[2.6: Custom Autograd Function - $\bigstar\bigstar\bigstar\bigstar$]
\textbf{Goal:} Implement a custom operation with manual gradient.

Implement a custom \textbf{sigmoid} function:
\begin{enumerate}
    \item Forward: $\sigma(x) = \frac{1}{1 + e^{-x}}$
    \item Backward: $\frac{d\sigma}{dx} = \sigma(x)(1 - \sigma(x))$
    \item Inherit from \texttt{torch.autograd.Function}
    \item Implement \texttt{forward()} and \texttt{backward()} static methods
    \item Test it on $x = [0, 1, 2]$
    \item Verify gradient using \texttt{torch.autograd.gradcheck}
\end{enumerate}

\textbf{Hint:} In \texttt{backward()}, you'll need the output of sigmoid. You can either:
\begin{itemize}
    \item Save the output in \texttt{forward} using \texttt{ctx.save\_for\_backward}
    \item Or recompute it from the saved input
\end{itemize}

\textbf{Starter code:}
\begin{lstlisting}
class MySigmoid(torch.autograd.Function):
    @staticmethod
    def forward(ctx, input):
        # Your code here
        pass
    
    @staticmethod
    def backward(ctx, grad_output):
        # Your code here
        pass

# Test
x = torch.tensor([0.0, 1.0, 2.0], requires_grad=True)
y = MySigmoid.apply(x)
\end{lstlisting}
\end{exercise}

\clearpage