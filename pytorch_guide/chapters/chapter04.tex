% =============================================
% SECTION 4: THE TRAINING LOOP & OPTIMIZERS
% =============================================

\section{The Training Loop \& Optimizers}

\subsection{Introduction: The Training Process}

We now have all the pieces:
\begin{itemize}
    \item \textbf{Tensors:} For data representation
    \item \textbf{Autograd:} For computing gradients
    \item \textbf{Modules:} For building models
\end{itemize}

Now we need to \textbf{train} the model—adjust its parameters to minimize a loss function. This involves:
\begin{enumerate}
    \item Forward pass: Compute predictions
    \item Loss computation: Measure how wrong we are
    \item Backward pass: Compute gradients
    \item Parameter update: Move in the direction that reduces loss
    \item Repeat!
\end{enumerate}

\subsection{Theory: Gradient Descent and Optimization}

\subsubsection{The Optimization Problem}

Training is an optimization problem. Given:
\begin{itemize}
    \item Model $f_\theta$ with parameters $\theta$
    \item Dataset $\{(x_i, y_i)\}_{i=1}^N$
    \item Loss function $\mathcal{L}$
\end{itemize}

We want to find:
\[
\theta^* = \arg\min_\theta \frac{1}{N} \sum_{i=1}^N \mathcal{L}(f_\theta(x_i), y_i)
\]

\subsubsection{Gradient Descent}

The basic update rule:
\[
\theta_{t+1} = \theta_t - \eta \nabla_\theta \mathcal{L}
\]

where $\eta$ is the \textbf{learning rate}.

\textbf{Variants:}
\begin{itemize}
    \item \textbf{Batch Gradient Descent:} Use entire dataset (slow, memory intensive)
    \item \textbf{Stochastic Gradient Descent (SGD):} Use one sample (fast, noisy)
    \item \textbf{Mini-Batch SGD:} Use a batch of samples (sweet spot)
\end{itemize}

In practice, we always use mini-batch SGD (typically 32-256 samples per batch).

\subsubsection{Learning Rate}

The learning rate $\eta$ is the most important hyperparameter:

\begin{itemize}
    \item \textbf{Too large:} Training diverges (loss increases)
    \item \textbf{Too small:} Training is very slow
    \item \textbf{Just right:} Steady convergence
\end{itemize}

Typical ranges: $10^{-4}$ to $10^{-2}$ for Adam, $10^{-3}$ to $10^{-1}$ for SGD.

\begin{theorybox}[Momentum and Adaptive Learning Rates]
Modern optimizers go beyond basic gradient descent:
\begin{itemize}
    \item \textbf{Momentum:} Accumulate gradients over time (helps escape local minima)
    \item \textbf{Adaptive rates:} Different learning rate for each parameter (Adam, RMSprop)
\end{itemize}
These techniques make training more stable and faster.
\end{theorybox}

\subsection{Implementation: The Standard Training Loop}

\subsubsection{Complete Training Example}

Here's the canonical PyTorch training loop:

\begin{lstlisting}
import torch
import torch.nn as nn
import torch.optim as optim

# 1. CREATE MODEL
model = nn.Sequential(
    nn.Linear(10, 50),
    nn.ReLU(),
    nn.Linear(50, 1)
)

# 2. DEFINE LOSS AND OPTIMIZER
criterion = nn.MSELoss()
optimizer = optim.Adam(model.parameters(), lr=0.001)

# 3. CREATE DUMMY DATA (replace with real DataLoader)
X_train = torch.randn(1000, 10)
y_train = torch.randn(1000, 1)

# 4. TRAINING LOOP
num_epochs = 100
batch_size = 32

for epoch in range(num_epochs):
    # Shuffle data (simplified; use DataLoader in practice)
    indices = torch.randperm(len(X_train))
    
    epoch_loss = 0.0
    for i in range(0, len(X_train), batch_size):
        # Get batch
        batch_indices = indices[i:i+batch_size]
        X_batch = X_train[batch_indices]
        y_batch = y_train[batch_indices]
        
        # FORWARD PASS
        predictions = model(X_batch)
        loss = criterion(predictions, y_batch)
        
        # BACKWARD PASS
        optimizer.zero_grad()  # Clear old gradients
        loss.backward()        # Compute new gradients
        
        # UPDATE PARAMETERS
        optimizer.step()
        
        epoch_loss += loss.item()
    
    # Print progress
    if (epoch + 1) % 10 == 0:
        avg_loss = epoch_loss / (len(X_train) / batch_size)
        print(f'Epoch [{epoch+1}/{num_epochs}], Loss: {avg_loss:.4f}')
\end{lstlisting}

\begin{pytorchtip}[The Five Steps of Training]
Every training loop has these steps:
\begin{enumerate}
    \item \texttt{optimizer.zero\_grad()}: Clear old gradients
    \item \texttt{output = model(input)}: Forward pass
    \item \texttt{loss = criterion(output, target)}: Compute loss
    \item \texttt{loss.backward()}: Compute gradients
    \item \texttt{optimizer.step()}: Update parameters
\end{enumerate}
Always in this order!
\end{pytorchtip}

\clearpage
\subsubsection{Loss Functions}

Different tasks require different loss functions:

\textbf{Regression (continuous output):}

\begin{lstlisting}
# Mean Squared Error (L2 loss)
criterion = nn.MSELoss()

# Mean Absolute Error (L1 loss, more robust to outliers)
criterion = nn.L1Loss()

# Smooth L1 (Huber loss, combines L1 and L2)
criterion = nn.SmoothL1Loss()
\end{lstlisting}

\textbf{Binary Classification (0 or 1):}

\begin{lstlisting}
# Binary Cross Entropy (output must be in [0,1], use sigmoid)
criterion = nn.BCELoss()

# BCE with built-in sigmoid (more numerically stable)
criterion = nn.BCEWithLogitsLoss()  # Preferred!

# Example
model = nn.Sequential(nn.Linear(10, 1))  # No sigmoid in model
criterion = nn.BCEWithLogitsLoss()
logits = model(x)  # Raw outputs
loss = criterion(logits, targets)  # Applies sigmoid internally
\end{lstlisting}

\textbf{Multi-Class Classification (one of K classes):}

\begin{lstlisting}
# Cross Entropy Loss (combines LogSoftmax + NLLLoss)
criterion = nn.CrossEntropyLoss()

# Example
model = nn.Sequential(nn.Linear(10, 5))  # 5 classes, no softmax!
criterion = nn.CrossEntropyLoss()

logits = model(x)  # Shape: (batch, 5), raw scores
targets = torch.tensor([0, 2, 1, 4, 3])  # Class indices

loss = criterion(logits, targets)  # Applies softmax internally
\end{lstlisting}

\begin{warningbox}[Don't Apply Softmax Before CrossEntropyLoss]
\texttt{nn.CrossEntropyLoss} expects \textbf{raw logits}, not probabilities. It applies softmax internally for numerical stability. Applying softmax yourself will give wrong results!

\textbf{WRONG:}
\begin{lstlisting}
output = torch.softmax(model(x), dim=1)
loss = nn.CrossEntropyLoss()(output, targets)  # INCORRECT!
\end{lstlisting}

\textbf{CORRECT:}
\begin{lstlisting}
output = model(x)  # Raw logits
loss = nn.CrossEntropyLoss()(output, targets)  # Correct!
\end{lstlisting}
\end{warningbox}

\textbf{Multi-Label Classification (multiple classes can be active):}

\begin{lstlisting}
# Use BCE with Logits for each class independently
criterion = nn.BCEWithLogitsLoss()

model = nn.Sequential(nn.Linear(10, 5))  # 5 binary outputs
logits = model(x)  # Shape: (batch, 5)
targets = torch.tensor([[1, 0, 1, 0, 0],   # Multiple 1s allowed
                        [0, 1, 1, 1, 0]])  # Shape: (batch, 5)

loss = criterion(logits, targets.float())
\end{lstlisting}

\subsubsection{Optimizers}

PyTorch provides many optimizers. Here are the most common:

\textbf{Stochastic Gradient Descent (SGD):}

\begin{lstlisting}
# Basic SGD
optimizer = optim.SGD(model.parameters(), lr=0.01)

# SGD with momentum (recommended)
optimizer = optim.SGD(model.parameters(), lr=0.01, momentum=0.9)

# SGD with momentum and weight decay (L2 regularization)
optimizer = optim.SGD(model.parameters(), lr=0.01, 
                      momentum=0.9, weight_decay=1e-4)
\end{lstlisting}

\textbf{Adam (Adaptive Moment Estimation):}

\begin{lstlisting}
# Adam: adaptive learning rates per parameter
optimizer = optim.Adam(model.parameters(), lr=0.001)

# Adam with weight decay
optimizer = optim.Adam(model.parameters(), lr=0.001, weight_decay=1e-4)
\end{lstlisting}

\textbf{AdamW (Adam with decoupled weight decay):}

\begin{lstlisting}
# AdamW: better weight decay than Adam
optimizer = optim.AdamW(model.parameters(), lr=0.001, weight_decay=0.01)
\end{lstlisting}

\begin{pytorchtip}[Which Optimizer to Use?]
\textbf{For most tasks:} Start with \texttt{Adam} or \texttt{AdamW} with lr=0.001

\textbf{Use Adam/AdamW when:}
\begin{itemize}
    \item You want fast initial progress
    \item You're prototyping
    \item You have many hyperparameters (adaptive rates help)
\end{itemize}

\textbf{Use SGD with momentum when:}
\begin{itemize}
    \item You need the absolute best final performance
    \item You're training CNNs for vision tasks
    \item You can afford to tune learning rate carefully
\end{itemize}

\textbf{General rule:} Adam for fast development, SGD for final tuning.
\end{pytorchtip}

\clearpage
\subsubsection{Learning Rate Schedules}

Instead of fixed learning rate, reduce it during training:

\begin{lstlisting}
import torch.optim.lr_scheduler as lr_scheduler

optimizer = optim.Adam(model.parameters(), lr=0.001)

# Step decay: reduce LR every step_size epochs
scheduler = lr_scheduler.StepLR(optimizer, step_size=30, gamma=0.1)

# Exponential decay
scheduler = lr_scheduler.ExponentialLR(optimizer, gamma=0.95)

# Reduce on plateau: reduce when loss stops improving
scheduler = lr_scheduler.ReduceLROnPlateau(optimizer, mode='min', 
                                           factor=0.5, patience=10)

# Cosine annealing: smooth decay to zero
scheduler = lr_scheduler.CosineAnnealingLR(optimizer, T_max=100)

# Usage in training loop
for epoch in range(num_epochs):
    train_one_epoch()
    
    # Update learning rate
    scheduler.step()  # For most schedulers
    
    # For ReduceLROnPlateau, pass validation loss
    # scheduler.step(val_loss)
    
    # Check current LR
    current_lr = optimizer.param_groups[0]['lr']
    print(f"Current LR: {current_lr}")
\end{lstlisting}

\textbf{When to use schedules:}
\begin{itemize}
    \item Long training runs (>100 epochs)
    \item When loss plateaus
    \item Fine-tuning pretrained models
\end{itemize}

\textbf{When not to use:}
\begin{itemize}
    \item Short training (<50 epochs)
    \item Already using Adam with good convergence
    \item Adds complexity without clear benefit
\end{itemize}

\subsubsection{Gradient Clipping}

Prevent exploding gradients by limiting their magnitude:

\begin{lstlisting}
# Training loop with gradient clipping
for epoch in range(num_epochs):
    for batch in dataloader:
        optimizer.zero_grad()
        
        output = model(batch)
        loss = criterion(output, targets)
        loss.backward()
        
        # Clip gradients by norm
        torch.nn.utils.clip_grad_norm_(model.parameters(), max_norm=1.0)
        
        optimizer.step()
\end{lstlisting}

\textbf{Two methods:}

\begin{lstlisting}
# Clip by norm (preferred): scale down if norm > max_norm
torch.nn.utils.clip_grad_norm_(model.parameters(), max_norm=1.0)

# Clip by value: clamp each gradient to [-clip_value, clip_value]
torch.nn.utils.clip_grad_value_(model.parameters(), clip_value=0.5)
\end{lstlisting}

\begin{pytorchtip}[When to Use Gradient Clipping]
\textbf{Always use for:}
\begin{itemize}
    \item RNNs and LSTMs (prone to exploding gradients)
    \item Transformers
    \item Any sequence model
\end{itemize}

\textbf{Often useful for:}
\begin{itemize}
    \item Deep networks (>50 layers)
    \item Training instability
\end{itemize}

\textbf{Typical values:} max\_norm=1.0 to 5.0
\end{pytorchtip}

\clearpage
\subsubsection{Validation Loop}

Always evaluate on held-out data:

\begin{lstlisting}
def train_epoch(model, dataloader, criterion, optimizer, device):
    model.train()  # Set to training mode
    total_loss = 0.0
    
    for batch_idx, (data, target) in enumerate(dataloader):
        data, target = data.to(device), target.to(device)
        
        optimizer.zero_grad()
        output = model(data)
        loss = criterion(output, target)
        loss.backward()
        optimizer.step()
        
        total_loss += loss.item()
    
    return total_loss / len(dataloader)

def validate(model, dataloader, criterion, device):
    model.eval()  # Set to evaluation mode
    total_loss = 0.0
    correct = 0
    total = 0
    
    with torch.no_grad():  # Disable gradient computation
        for data, target in dataloader:
            data, target = data.to(device), target.to(device)
            
            output = model(data)
            loss = criterion(output, target)
            total_loss += loss.item()
            
            # For classification: compute accuracy
            _, predicted = torch.max(output, 1)
            total += target.size(0)
            correct += (predicted == target).sum().item()
    
    avg_loss = total_loss / len(dataloader)
    accuracy = 100 * correct / total
    return avg_loss, accuracy

# Main training loop
for epoch in range(num_epochs):
    train_loss = train_epoch(model, train_loader, criterion, 
                            optimizer, device)
    val_loss, val_acc = validate(model, val_loader, criterion, device)
    
    print(f'Epoch {epoch+1}: Train Loss={train_loss:.4f}, '
          f'Val Loss={val_loss:.4f}, Val Acc={val_acc:.2f}%')
\end{lstlisting}

\begin{warningbox}[Don't Forget model.eval() and torch.no\_grad()]
When validating:
\begin{itemize}
    \item \texttt{model.eval()}: Disables dropout, uses running stats for batch norm
    \item \texttt{torch.no\_grad()}: Saves memory and computation
\end{itemize}

Forgetting these can lead to:
\begin{itemize}
    \item Incorrect validation metrics
    \item Out of memory errors
    \item Slower evaluation
\end{itemize}
\end{warningbox}

\subsubsection{Early Stopping}

Stop training when validation loss stops improving:

\begin{lstlisting}
class EarlyStopping:
    def __init__(self, patience=10, min_delta=0):
        """
        patience: how many epochs to wait after last improvement
        min_delta: minimum change to qualify as improvement
        """
        self.patience = patience
        self.min_delta = min_delta
        self.counter = 0
        self.best_loss = None
        self.early_stop = False
    
    def __call__(self, val_loss):
        if self.best_loss is None:
            self.best_loss = val_loss
        elif val_loss > self.best_loss - self.min_delta:
            self.counter += 1
            if self.counter >= self.patience:
                self.early_stop = True
        else:
            self.best_loss = val_loss
            self.counter = 0

# Usage
early_stopping = EarlyStopping(patience=10)

for epoch in range(num_epochs):
    train_loss = train_epoch(...)
    val_loss, val_acc = validate(...)
    
    early_stopping(val_loss)
    if early_stopping.early_stop:
        print(f"Early stopping at epoch {epoch+1}")
        break
\end{lstlisting}

\clearpage
\subsection{Implementation: Debugging Training}

\subsubsection{When Training Goes Wrong}

\textbf{Problem: Loss is NaN}

\begin{lstlisting}
# Check for NaN
if torch.isnan(loss):
    print("NaN detected!")
    print(f"Input contains NaN: {torch.isnan(data).any()}")
    print(f"Output contains NaN: {torch.isnan(output).any()}")
    
    # Check gradients
    for name, param in model.named_parameters():
        if param.grad is not None:
            print(f"{name} grad: {param.grad.abs().max()}")
\end{lstlisting}

\textbf{Common causes:}
\begin{itemize}
    \item Learning rate too high
    \item Numerical instability (use \texttt{BCEWithLogitsLoss}, not \texttt{BCELoss})
    \item Division by zero
    \item Log of zero or negative number
\end{itemize}

\textbf{Problem: Loss doesn't decrease}

\begin{lstlisting}
# Sanity checks
print("Model output range:", output.min().item(), output.max().item())
print("Target range:", target.min().item(), target.max().item())

# Check if gradients are flowing
for name, param in model.named_parameters():
    if param.grad is not None:
        print(f"{name}: grad mean = {param.grad.abs().mean():.6f}")
\end{lstlisting}

\textbf{Common causes:}
\begin{itemize}
    \item Learning rate too low
    \item Wrong loss function
    \item Forgot \texttt{optimizer.zero\_grad()}
    \item Model architecture issue (dead ReLUs, wrong dimensions)
    \item Data not normalized
\end{itemize}

\textbf{Problem: Loss explodes}

\begin{itemize}
    \item Learning rate too high (reduce by 10x)
    \item Use gradient clipping
    \item Check for large batch effects (try smaller batches)
\end{itemize}

\textbf{Problem: Overfitting (train loss << val loss)}

\begin{lstlisting}
# Add regularization
model = nn.Sequential(
    nn.Linear(10, 50),
    nn.ReLU(),
    nn.Dropout(0.5),  # Add dropout
    nn.Linear(50, 1)
)

# Add weight decay
optimizer = optim.Adam(model.parameters(), lr=0.001, weight_decay=1e-4)

# Use data augmentation (for images)
# Reduce model size
# Get more data
\end{lstlisting}

\textbf{Problem: Underfitting (both losses high)}

\begin{itemize}
    \item Increase model capacity (more layers, more neurons)
    \item Train longer
    \item Increase learning rate
    \item Remove regularization
    \item Check if task is actually learnable
\end{itemize}

\subsubsection{Monitoring Training}

\begin{lstlisting}
# Track metrics
train_losses = []
val_losses = []

for epoch in range(num_epochs):
    train_loss = train_epoch(...)
    val_loss, val_acc = validate(...)
    
    train_losses.append(train_loss)
    val_losses.append(val_loss)
    
    # Plot periodically
    if epoch % 10 == 0:
        import matplotlib.pyplot as plt
        plt.plot(train_losses, label='Train')
        plt.plot(val_losses, label='Val')
        plt.xlabel('Epoch')
        plt.ylabel('Loss')
        plt.legend()
        plt.show()
\end{lstlisting}

\clearpage
\subsection{Exercises}

\begin{exercise}[4.1: Basic Training Loop - $\bigstar\bigstar$]
\textbf{Goal:} Implement a complete training loop from scratch.

Task: Fit a neural network to $y = \sin(x)$ for $x \in [0, 2\pi]$

\begin{enumerate}
    \item Generate 1000 training samples: $x$ uniform in $[0, 2\pi]$, $y = \sin(x)$ + small noise
    \item Create a 3-layer MLP: 1 → 50 → 50 → 1 with ReLU
    \item Use MSE loss and Adam optimizer
    \item Train for 100 epochs, batch size 32
    \item Print loss every 10 epochs
    \item Plot predictions vs true function
\end{enumerate}

\textbf{Starter code:}
\begin{lstlisting}
import torch
import torch.nn as nn
import torch.optim as optim
import numpy as np

# Generate data
x = torch.linspace(0, 2*np.pi, 1000).reshape(-1, 1)
y = torch.sin(x) + 0.1 * torch.randn_like(x)

# Your code here
\end{lstlisting}
\end{exercise}

\begin{exercise}[4.2: Loss Function Comparison - $\bigstar\bigstar\bigstar$]
\textbf{Goal:} Understand different loss functions.

Train the same model with different losses:
\begin{enumerate}
    \item MSE Loss (L2)
    \item L1 Loss
    \item Smooth L1 Loss
\end{enumerate}

Add outliers to the data (10\% of points with large noise) and compare:
\begin{itemize}
    \item Final training loss
    \item Robustness to outliers
    \item Convergence speed
\end{itemize}

Which loss is most robust to outliers?
\end{exercise}

\begin{exercise}[4.3: Optimizer Comparison - $\bigstar\bigstar\bigstar$]
\textbf{Goal:} Compare optimizer behavior.

Train the same model with:
\begin{enumerate}
    \item SGD (lr=0.01)
    \item SGD with momentum (lr=0.01, momentum=0.9)
    \item Adam (lr=0.001)
    \item AdamW (lr=0.001)
\end{enumerate}

Plot all loss curves on the same graph. Observations:
\begin{itemize}
    \item Which converges fastest initially?
    \item Which achieves lowest final loss?
    \item How does momentum help SGD?
\end{itemize}
\end{exercise}

\begin{exercise}[4.4: Learning Rate Experiments - $\bigstar\bigstar\bigstar$]
\textbf{Goal:} Understand learning rate impact.

Train the same model with different learning rates:
\begin{itemize}
    \item $10^{-4}$, $10^{-3}$, $10^{-2}$, $10^{-1}$, $1.0$
\end{itemize}

For each:
\begin{enumerate}
    \item Track loss over epochs
    \item Identify: too small, just right, too large
    \item Observe divergence (NaN loss) for very large LR
\end{enumerate}

\textbf{Extra:} Implement a learning rate finder (sweep from small to large, stop when loss diverges).
\end{exercise}

\begin{exercise}[4.5: Overfitting Detection - $\bigstar\bigstar\bigstar\bigstar$]
\textbf{Goal:} Identify and prevent overfitting.

\begin{enumerate}
    \item Generate a small dataset (100 samples)
    \item Split into train (80) and validation (20)
    \item Train a large model (can easily overfit)
    \item Track both train and validation loss
    \item Identify when overfitting starts (losses diverge)
    \item Add regularization:
    \begin{itemize}
        \item Dropout
        \item Weight decay
        \item Early stopping
    \end{itemize}
    \item Compare final validation performance
\end{enumerate}
\end{exercise}

\begin{exercise}[4.6: Complete Training Pipeline - $\bigstar\bigstar\bigstar\bigstar$]
\textbf{Goal:} Build a production-ready training pipeline.

Implement a complete pipeline with:
\begin{enumerate}
    \item Data splitting (train/val/test)
    \item Training loop with progress bars
    \item Validation after each epoch
    \item Model checkpointing (save best model)
    \item Early stopping
    \item Learning rate scheduling
    \item Tensorboard logging (optional)
    \item Final test set evaluation
\end{enumerate}

Test on a real dataset (e.g., sklearn's make\_classification).

\textbf{Hint:} Use \texttt{tqdm} for progress bars:
\begin{lstlisting}
from tqdm import tqdm

for epoch in tqdm(range(num_epochs)):
    # training code
\end{lstlisting}
\end{exercise}

\begin{exercise}[4.7: Debugging Challenge - $\bigstar\bigstar\bigstar\bigstar$]
\textbf{Goal:} Practice debugging common training issues.

Given buggy training code (deliberately broken):

\begin{lstlisting}
# This code has 5 bugs - find and fix them!
model = nn.Sequential(nn.Linear(10, 1), nn.Softmax())
criterion = nn.CrossEntropyLoss()
optimizer = optim.Adam(model.parameters(), lr=10.0)

for epoch in range(100):
    output = model(X_train)
    loss = criterion(output, y_train)
    loss.backward()
    optimizer.step()
    print(f"Epoch {epoch}, Loss: {loss.item()}")
\end{lstlisting}

Find and fix:
\begin{enumerate}
    \item Wrong activation before CrossEntropyLoss
    \item Missing \texttt{optimizer.zero\_grad()}
    \item Learning rate too high
    \item No train/eval modes
    \item No validation set
\end{enumerate}
\end{exercise}

\clearpage