% =============================================
% SECTION 10: RECURRENT NEURAL NETWORKS (RNNs)
% =============================================

\section{Recurrent Neural Networks (RNNs)}

\subsection{Introduction: Processing Sequential Data}

So far, we've seen networks that process fixed-size inputs. But many real-world problems involve \textbf{sequences}:
\begin{itemize}
    \item Time series (stock prices, sensor data, climate data)
    \item Text (words, characters, sentences)
    \item Audio (speech, music)
    \item Video (frames over time)
    \item Biological sequences (DNA, proteins)
\end{itemize}

\textbf{Challenges with sequences:}
\begin{enumerate}
    \item Variable length (sentences have different lengths)
    \item Temporal dependencies (past affects future)
    \item Need to maintain state (remember what happened before)
\end{enumerate}

\textbf{Why not use MLPs or CNNs?}

\textbf{MLPs:}
\begin{itemize}
    \item Fixed input size (can't handle variable-length sequences)
    \item No notion of order (position 1 vs position 100 treated same)
    \item Can't share parameters across time (learn same pattern at different positions)
\end{itemize}

\textbf{CNNs:}
\begin{itemize}
    \item Can handle sequences with 1D convolutions
    \item Limited temporal context (receptive field)
    \item No true memory of long-term dependencies
\end{itemize}

\textbf{RNNs solve these problems} by maintaining a \textbf{hidden state} that gets updated at each time step.

\subsection{Theory: How RNNs Work}

\subsubsection{The Basic RNN}

An RNN processes a sequence one element at a time, maintaining a hidden state:

\textbf{At each time step $t$:}
\[
h_t = \tanh(W_{hh} h_{t-1} + W_{xh} x_t + b_h)
\]
\[
y_t = W_{hy} h_t + b_y
\]

where:
\begin{itemize}
    \item $x_t$: Input at time $t$
    \item $h_t$: Hidden state at time $t$ (memory)
    \item $y_t$: Output at time $t$
    \item $W_{hh}$: Hidden-to-hidden weights (memory transformation)
    \item $W_{xh}$: Input-to-hidden weights
    \item $W_{hy}$: Hidden-to-output weights
\end{itemize}

\textbf{Key idea:} The hidden state $h_t$ summarizes information from all previous time steps.

\begin{theorybox}[The Hidden State]
The hidden state $h_t$ is the RNN's \textbf{memory}:
\begin{itemize}
    \item $h_0$: Initial memory (usually zeros)
    \item $h_1 = f(x_1, h_0)$: Combines first input with initial state
    \item $h_2 = f(x_2, h_1)$: Combines second input with previous state
    \item $h_3 = f(x_3, h_2)$: And so on...
\end{itemize}

By the end: $h_T$ contains information about the entire sequence $x_1, x_2, \ldots, x_T$.
\end{theorybox}

\textbf{Unrolled view:}

\begin{verbatim}
x1 → [RNN] → h1 → [RNN] → h2 → [RNN] → h3 → ...
      ↓            ↓            ↓
      y1           y2           y3
\end{verbatim}

All boxes share the \textbf{same weights} (parameter sharing across time).

\clearpage
\subsubsection{Parameter Sharing}

\textbf{Why parameter sharing matters:}

For a sequence of length 100:
\begin{itemize}
    \item MLP: Would need 100 separate weight matrices (millions of parameters)
    \item RNN: Uses same weights at each step (thousands of parameters)
\end{itemize}

The network learns patterns that work at \textbf{any position} in the sequence.

\subsubsection{Sequence Modeling Patterns}

\textbf{1. One-to-Many (sequence generation):}
\begin{verbatim}
x → [RNN] → [RNN] → [RNN] → ...
             ↓       ↓       ↓
             y1      y2      y3
Example: Image captioning (image → sentence)
\end{verbatim}

\textbf{2. Many-to-One (sequence classification):}
\begin{verbatim}
x1 → [RNN] → x2 → [RNN] → x3 → [RNN]
                                 ↓
                                 y
Example: Sentiment analysis (sentence → positive/negative)
\end{verbatim}

\textbf{3. Many-to-Many (same length):}
\begin{verbatim}
x1 → [RNN] → x2 → [RNN] → x3 → [RNN]
     ↓           ↓           ↓
     y1          y2          y3
Example: Part-of-speech tagging (word → tag)
\end{verbatim}

\textbf{4. Many-to-Many (different length, encoder-decoder):}
\begin{verbatim}
Encoder:  x1 → [RNN] → x2 → [RNN] → h
Decoder:  h → [RNN] → [RNN] → [RNN]
               ↓       ↓       ↓
               y1      y2      y3
Example: Machine translation (English → French)
\end{verbatim}

\subsubsection{The Vanishing Gradient Problem}

\textbf{Problem:} Vanilla RNNs struggle with long sequences.

When backpropagating through time, gradients are multiplied by $W_{hh}$ at each step:

\[
\frac{\partial h_t}{\partial h_0} = \prod_{i=1}^{t} W_{hh} \cdot \text{diag}(\tanh'(h_i))
\]

If $|W_{hh}| < 1$: Gradients vanish exponentially (network can't learn long-term dependencies)

If $|W_{hh}| > 1$: Gradients explode (training becomes unstable)

\textbf{In practice:} Vanilla RNNs can only remember ~10-20 steps back.

\textbf{Solution:} LSTM and GRU architectures (next sections).

\clearpage
\subsection{Theory: Long Short-Term Memory (LSTM)}

LSTMs solve the vanishing gradient problem through a clever gating mechanism.

\subsubsection{LSTM Architecture}

LSTM has two states:
\begin{itemize}
    \item $h_t$: Hidden state (short-term memory)
    \item $c_t$: Cell state (long-term memory)
\end{itemize}

And three gates:
\begin{itemize}
    \item $f_t$: Forget gate (what to remove from memory)
    \item $i_t$: Input gate (what new information to add)
    \item $o_t$: Output gate (what to output)
\end{itemize}

\textbf{LSTM Equations:}

\textbf{1. Forget gate (what to forget):}
\[
f_t = \sigma(W_f [h_{t-1}, x_t] + b_f)
\]

\textbf{2. Input gate (what to add):}
\[
i_t = \sigma(W_i [h_{t-1}, x_t] + b_i)
\]
\[
\tilde{c}_t = \tanh(W_c [h_{t-1}, x_t] + b_c) \quad \text{(candidate values)}
\]

\textbf{3. Update cell state:}
\[
c_t = f_t \odot c_{t-1} + i_t \odot \tilde{c}_t
\]

\textbf{4. Output gate (what to output):}
\[
o_t = \sigma(W_o [h_{t-1}, x_t] + b_o)
\]
\[
h_t = o_t \odot \tanh(c_t)
\]

where $\odot$ is element-wise multiplication, $\sigma$ is sigmoid, $[a, b]$ is concatenation.

\begin{theorybox}[LSTM Intuition]
\textbf{Cell state ($c_t$):} The "memory highway"
\begin{itemize}
    \item Runs through the entire sequence
    \item Modified only by additions and element-wise multiplications
    \item Gradients flow easily (no repeated matrix multiplications!)
\end{itemize}

\textbf{Forget gate ($f_t$):} "Should I forget the past?"
\begin{itemize}
    \item Values near 0: Forget most of $c_{t-1}$
    \item Values near 1: Keep most of $c_{t-1}$
\end{itemize}

\textbf{Input gate ($i_t$):} "Should I add this new information?"
\begin{itemize}
    \item Controls how much of $\tilde{c}_t$ to add to memory
\end{itemize}

\textbf{Output gate ($o_t$):} "How much of memory should I expose?"
\begin{itemize}
    \item Controls how much of $c_t$ becomes $h_t$
\end{itemize}
\end{theorybox}

\textbf{Why LSTM works:}

The cell state $c_t$ provides a path where gradients can flow without repeated matrix multiplications (just element-wise operations). This solves vanishing gradients!

\clearpage
\subsection{Theory: Gated Recurrent Unit (GRU)}

GRU is a simplified version of LSTM with fewer parameters.

\subsubsection{GRU Architecture}

GRU combines forget and input gates into a single \textbf{update gate} and merges cell state with hidden state.

\textbf{GRU Equations:}

\textbf{1. Update gate (how much to update):}
\[
z_t = \sigma(W_z [h_{t-1}, x_t] + b_z)
\]

\textbf{2. Reset gate (how much past to forget):}
\[
r_t = \sigma(W_r [h_{t-1}, x_t] + b_r)
\]

\textbf{3. Candidate hidden state:}
\[
\tilde{h}_t = \tanh(W_h [r_t \odot h_{t-1}, x_t] + b_h)
\]

\textbf{4. Update hidden state:}
\[
h_t = (1 - z_t) \odot h_{t-1} + z_t \odot \tilde{h}_t
\]

\textbf{Intuition:}
\begin{itemize}
    \item $z_t \approx 0$: Keep old state $h_{t-1}$ (don't update)
    \item $z_t \approx 1$: Use new candidate $\tilde{h}_t$ (full update)
    \item $z_t \approx 0.5$: Blend old and new
\end{itemize}

\textbf{LSTM vs GRU:}

\begin{table}[h]
\centering
\begin{tabular}{lll}
\toprule
\textbf{Aspect} & \textbf{LSTM} & \textbf{GRU} \\
\midrule
Gates & 3 (forget, input, output) & 2 (update, reset) \\
States & 2 ($h_t$, $c_t$) & 1 ($h_t$) \\
Parameters & More & Fewer (faster) \\
Performance & Slightly better & Nearly as good \\
Training speed & Slower & Faster \\
Memory & More & Less \\
\bottomrule
\end{tabular}
\caption{LSTM vs GRU comparison}
\end{table}

\textbf{When to use each:}
\begin{itemize}
    \item \textbf{LSTM:} Default choice, especially for complex tasks
    \item \textbf{GRU:} When speed matters or dataset is smaller
    \item \textbf{Rule of thumb:} Try both, see which works better
\end{itemize}

\subsection{Implementation: RNNs in PyTorch}

\subsubsection{Vanilla RNN}

\begin{lstlisting}
import torch
import torch.nn as nn

# Simple RNN cell
rnn_cell = nn.RNNCell(input_size=10, hidden_size=20)

# Input: (batch, input_size)
# Hidden: (batch, hidden_size)
x = torch.randn(32, 10)  # Batch of 32
h = torch.zeros(32, 20)  # Initial hidden state

# One step
h_next = rnn_cell(x, h)
print(h_next.shape)  # torch.Size([32, 20])

# Full RNN layer (processes entire sequence)
rnn = nn.RNN(input_size=10, hidden_size=20, num_layers=1, batch_first=True)

# Input: (batch, seq_len, input_size)
x = torch.randn(32, 100, 10)  # 32 sequences, length 100
h0 = torch.zeros(1, 32, 20)   # (num_layers, batch, hidden_size)

# Forward pass
output, h_n = rnn(x, h0)
# output: (32, 100, 20) - outputs at each time step
# h_n: (1, 32, 20) - final hidden state

print(output.shape, h_n.shape)
\end{lstlisting}

\begin{pytorchtip}[batch\_first Parameter]
PyTorch RNNs default to shape (seq\_len, batch, features), but this is confusing!

Always use \texttt{batch\_first=True} for shape (batch, seq\_len, features):
\begin{lstlisting}
rnn = nn.RNN(..., batch_first=True)  # Recommended
# Input: (batch, seq_len, features)
# Output: (batch, seq_len, hidden_size)
\end{lstlisting}
\end{pytorchtip}

\clearpage
\subsubsection{LSTM}

\begin{lstlisting}
# LSTM cell
lstm_cell = nn.LSTMCell(input_size=10, hidden_size=20)

x = torch.randn(32, 10)
h = torch.zeros(32, 20)
c = torch.zeros(32, 20)  # Cell state

# One step
h_next, c_next = lstm_cell(x, (h, c))  # Note: tuple of (h, c)

# Full LSTM layer
lstm = nn.LSTM(input_size=10, hidden_size=20, num_layers=2, 
               batch_first=True, dropout=0.2)

x = torch.randn(32, 100, 10)
h0 = torch.zeros(2, 32, 20)  # (num_layers, batch, hidden)
c0 = torch.zeros(2, 32, 20)  # (num_layers, batch, hidden)

# Forward pass
output, (h_n, c_n) = lstm(x, (h0, c0))
# output: (32, 100, 20) - outputs at each time step
# h_n: (2, 32, 20) - final hidden state for each layer
# c_n: (2, 32, 20) - final cell state for each layer

print(output.shape, h_n.shape, c_n.shape)
\end{lstlisting}

\begin{warningbox}[LSTM Hidden State is a Tuple!]
LSTM returns \texttt{(h, c)} as a tuple, not just \texttt{h}:

\begin{lstlisting}
# WRONG:
output, hidden = lstm(x)
h_n = hidden  # This is a tuple, not a tensor!

# CORRECT:
output, (h_n, c_n) = lstm(x, (h0, c0))
# or
output, hidden = lstm(x, (h0, c0))
h_n, c_n = hidden
\end{lstlisting}
\end{warningbox}

\subsubsection{GRU}

\begin{lstlisting}
# GRU (similar to LSTM, but only one hidden state)
gru = nn.GRU(input_size=10, hidden_size=20, num_layers=2, 
             batch_first=True, dropout=0.2)

x = torch.randn(32, 100, 10)
h0 = torch.zeros(2, 32, 20)

# Forward pass
output, h_n = gru(x, h0)  # Only h, no cell state
# output: (32, 100, 20)
# h_n: (2, 32, 20)

print(output.shape, h_n.shape)
\end{lstlisting}

\subsubsection{Building a Sequence Classifier}

\begin{lstlisting}
class SequenceClassifier(nn.Module):
    """Many-to-one: sequence → single label."""
    
    def __init__(self, input_size, hidden_size, num_classes, num_layers=2):
        super().__init__()
        
        self.lstm = nn.LSTM(
            input_size=input_size,
            hidden_size=hidden_size,
            num_layers=num_layers,
            batch_first=True,
            dropout=0.2 if num_layers > 1 else 0
        )
        
        self.fc = nn.Linear(hidden_size, num_classes)
    
    def forward(self, x):
        # x: (batch, seq_len, input_size)
        
        # LSTM forward
        output, (h_n, c_n) = self.lstm(x)
        # output: (batch, seq_len, hidden_size)
        # h_n: (num_layers, batch, hidden_size)
        
        # Use last time step's output
        last_output = output[:, -1, :]  # (batch, hidden_size)
        
        # Or use final hidden state from last layer
        # last_hidden = h_n[-1]  # (batch, hidden_size)
        
        # Classify
        logits = self.fc(last_output)
        return logits

# Test
model = SequenceClassifier(input_size=10, hidden_size=64, num_classes=3)
x = torch.randn(32, 100, 10)  # 32 sequences, length 100
output = model(x)
print(output.shape)  # torch.Size([32, 3])
\end{lstlisting}

\clearpage
\subsubsection{Sequence-to-Sequence (Many-to-Many)}

\begin{lstlisting}
class SequenceLabeler(nn.Module):
    """Many-to-many: each input → each output (same length)."""
    
    def __init__(self, input_size, hidden_size, num_classes):
        super().__init__()
        
        self.lstm = nn.LSTM(
            input_size=input_size,
            hidden_size=hidden_size,
            num_layers=2,
            batch_first=True,
            dropout=0.2
        )
        
        # Apply classifier at each time step
        self.fc = nn.Linear(hidden_size, num_classes)
    
    def forward(self, x):
        # x: (batch, seq_len, input_size)
        
        output, _ = self.lstm(x)
        # output: (batch, seq_len, hidden_size)
        
        # Apply classifier to all time steps
        logits = self.fc(output)
        # logits: (batch, seq_len, num_classes)
        
        return logits

# Test
model = SequenceLabeler(input_size=10, hidden_size=64, num_classes=5)
x = torch.randn(32, 100, 10)
output = model(x)
print(output.shape)  # torch.Size([32, 100, 5])
\end{lstlisting}

\subsubsection{Bidirectional RNNs}

Process sequence in both directions (forward and backward):

\begin{lstlisting}
class BidirectionalRNN(nn.Module):
    """Bidirectional LSTM for sequence classification."""
    
    def __init__(self, input_size, hidden_size, num_classes):
        super().__init__()
        
        self.lstm = nn.LSTM(
            input_size=input_size,
            hidden_size=hidden_size,
            num_layers=2,
            batch_first=True,
            dropout=0.2,
            bidirectional=True  # Process both directions
        )
        
        # Hidden size is doubled (forward + backward)
        self.fc = nn.Linear(hidden_size * 2, num_classes)
    
    def forward(self, x):
        # x: (batch, seq_len, input_size)
        
        output, _ = self.lstm(x)
        # output: (batch, seq_len, hidden_size * 2)
        
        # Use last time step (or all for seq-to-seq)
        last_output = output[:, -1, :]
        
        logits = self.fc(last_output)
        return logits

# Test
model = BidirectionalRNN(input_size=10, hidden_size=64, num_classes=3)
x = torch.randn(32, 100, 10)
output = model(x)
print(output.shape)  # torch.Size([32, 3])
\end{lstlisting}

\textbf{When to use bidirectional:}
\begin{itemize}
    \item Text classification (can see future context)
    \item Part-of-speech tagging
    \item Named entity recognition
    \item Any task where you have the full sequence before prediction
\end{itemize}

\textbf{Don't use bidirectional for:}
\begin{itemize}
    \item Time series prediction (can't see future!)
    \item Language generation (need causal/autoregressive)
    \item Online/streaming applications
\end{itemize}

\clearpage
\subsubsection{Handling Variable-Length Sequences}

\begin{lstlisting}
from torch.nn.utils.rnn import pack_padded_sequence, pad_packed_sequence

class VariableLengthRNN(nn.Module):
    """Efficiently handle sequences of different lengths."""
    
    def __init__(self, input_size, hidden_size, num_classes):
        super().__init__()
        self.lstm = nn.LSTM(input_size, hidden_size, batch_first=True)
        self.fc = nn.Linear(hidden_size, num_classes)
    
    def forward(self, x, lengths):
        """
        Args:
            x: Padded sequences (batch, max_len, input_size)
            lengths: Actual lengths of each sequence
        """
        # Pack sequences (skip padding in computation)
        packed = pack_padded_sequence(x, lengths, batch_first=True, 
                                     enforce_sorted=False)
        
        # LSTM forward on packed sequence
        packed_output, (h_n, c_n) = self.lstm(packed)
        
        # Unpack
        output, _ = pad_packed_sequence(packed_output, batch_first=True)
        
        # Use final hidden state
        logits = self.fc(h_n[-1])
        
        return logits

# Example usage
# Sequences of different lengths: [50, 30, 40]
lengths = torch.tensor([50, 30, 40])
max_len = lengths.max()

# Pad to max length
x = torch.randn(3, max_len, 10)  # Some values are padding

model = VariableLengthRNN(10, 64, 3)
output = model(x, lengths)
print(output.shape)  # torch.Size([3, 3])
\end{lstlisting}

\textbf{Why pack sequences?}
\begin{itemize}
    \item Skip computation on padding (faster)
    \item More accurate (padding doesn't affect hidden state)
    \item More memory efficient
\end{itemize}

\clearpage
% =============================================
% SECTION 10: RNNs - PART 2 (Advanced Topics & Exercises)
% =============================================

\subsubsection{Gradient Clipping (Essential for RNNs)}

RNNs are prone to exploding gradients. Gradient clipping is \textbf{essential}.

\begin{lstlisting}
# Method 1: Clip by norm (recommended)
torch.nn.utils.clip_grad_norm_(model.parameters(), max_norm=5.0)

# Method 2: Clip by value
torch.nn.utils.clip_grad_value_(model.parameters(), clip_value=0.5)

# In training loop
for epoch in range(num_epochs):
    for batch in dataloader:
        optimizer.zero_grad()
        
        output = model(batch)
        loss = criterion(output, target)
        loss.backward()
        
        # Clip gradients AFTER backward, BEFORE step
        torch.nn.utils.clip_grad_norm_(model.parameters(), max_norm=5.0)
        
        optimizer.step()
\end{lstlisting}

\begin{warningbox}[Always Use Gradient Clipping for RNNs]
Without gradient clipping:
\begin{itemize}
    \item Loss may suddenly spike to NaN
    \item Training becomes unstable
    \item Network may never converge
\end{itemize}

\textbf{Typical values:}
\begin{itemize}
    \item \texttt{max\_norm=1.0}: Conservative (very stable)
    \item \texttt{max\_norm=5.0}: Standard (good default)
    \item \texttt{max\_norm=10.0}: Aggressive (faster but less stable)
\end{itemize}
\end{warningbox}

\subsubsection{Training Tips for RNNs}

\textbf{1. Learning rate}
\begin{lstlisting}
# RNNs typically need lower learning rates than CNNs
optimizer = torch.optim.Adam(model.parameters(), lr=0.001)  # Good start

# Use learning rate scheduling
scheduler = torch.optim.lr_scheduler.ReduceLROnPlateau(
    optimizer, mode='min', factor=0.5, patience=5
)
\end{lstlisting}

\textbf{2. Weight initialization}
\begin{lstlisting}
def init_lstm_weights(lstm):
    """Initialize LSTM weights for better convergence."""
    for name, param in lstm.named_parameters():
        if 'weight_ih' in name:
            # Input-hidden weights: Xavier
            nn.init.xavier_uniform_(param)
        elif 'weight_hh' in name:
            # Hidden-hidden weights: Orthogonal (better for RNNs)
            nn.init.orthogonal_(param)
        elif 'bias' in name:
            # Biases: zeros, except forget gate bias = 1
            nn.init.zeros_(param)
            # Set forget gate bias to 1 (helps gradient flow)
            n = param.size(0)
            param.data[n//4:n//2].fill_(1.0)

# Apply to model
for name, module in model.named_modules():
    if isinstance(module, nn.LSTM):
        init_lstm_weights(module)
\end{lstlisting}

\textbf{3. Dropout placement}
\begin{lstlisting}
# Dropout between layers (built-in)
lstm = nn.LSTM(input_size, hidden_size, num_layers=3, dropout=0.2)
# This applies dropout between layers, not within recurrent connections

# For dropout on inputs/outputs:
class RNNWithDropout(nn.Module):
    def __init__(self, input_size, hidden_size, num_classes):
        super().__init__()
        self.input_dropout = nn.Dropout(0.2)
        self.lstm = nn.LSTM(input_size, hidden_size, num_layers=2)
        self.output_dropout = nn.Dropout(0.5)
        self.fc = nn.Linear(hidden_size, num_classes)
    
    def forward(self, x):
        x = self.input_dropout(x)
        output, _ = self.lstm(x)
        output = self.output_dropout(output[:, -1, :])
        return self.fc(output)
\end{lstlisting}

\textbf{4. Batch size considerations}

RNNs benefit from larger batch sizes:
\begin{itemize}
    \item Batch size 32-128 typically good
    \item Too small (<16): Noisy gradients, unstable training
    \item Too large (>256): May hurt generalization
\end{itemize}

\clearpage
\subsubsection{Common RNN Architectures}

\textbf{Stacked RNNs (Deep RNNs):}
\begin{lstlisting}
class DeepLSTM(nn.Module):
    """Multiple LSTM layers stacked."""
    
    def __init__(self, input_size, hidden_size, num_layers, num_classes):
        super().__init__()
        
        # num_layers > 1 creates stacked LSTM
        self.lstm = nn.LSTM(
            input_size=input_size,
            hidden_size=hidden_size,
            num_layers=num_layers,
            batch_first=True,
            dropout=0.3 if num_layers > 1 else 0  # Dropout between layers
        )
        
        self.fc = nn.Linear(hidden_size, num_classes)
    
    def forward(self, x):
        output, (h_n, c_n) = self.lstm(x)
        # h_n[-1]: hidden state from the last layer
        return self.fc(h_n[-1])

# Typical depths: 2-4 layers
# Deeper than 4 often doesn't help much
model = DeepLSTM(input_size=10, hidden_size=128, num_layers=3, 
                 num_classes=10)
\end{lstlisting}

\textbf{Encoder-Decoder (Seq2Seq):}
\begin{lstlisting}
class Seq2Seq(nn.Module):
    """Encoder-decoder for sequence-to-sequence tasks."""
    
    def __init__(self, input_size, hidden_size, output_size):
        super().__init__()
        
        # Encoder: processes input sequence
        self.encoder = nn.LSTM(input_size, hidden_size, batch_first=True)
        
        # Decoder: generates output sequence
        self.decoder = nn.LSTM(output_size, hidden_size, batch_first=True)
        
        # Output projection
        self.fc = nn.Linear(hidden_size, output_size)
    
    def forward(self, src, tgt, teacher_forcing_ratio=0.5):
        """
        Args:
            src: Source sequence (batch, src_len, input_size)
            tgt: Target sequence (batch, tgt_len, output_size)
            teacher_forcing_ratio: Probability of using true target vs prediction
        """
        batch_size = src.size(0)
        tgt_len = tgt.size(1)
        output_size = tgt.size(2)
        
        # Encode source
        _, (h_n, c_n) = self.encoder(src)
        
        # Initialize decoder hidden state with encoder final state
        decoder_hidden = (h_n, c_n)
        
        # First decoder input: start token (zeros)
        decoder_input = torch.zeros(batch_size, 1, output_size).to(src.device)
        
        outputs = []
        
        # Generate sequence step by step
        for t in range(tgt_len):
            # Decode one step
            decoder_output, decoder_hidden = self.decoder(
                decoder_input, decoder_hidden
            )
            
            # Project to output space
            prediction = self.fc(decoder_output)
            outputs.append(prediction)
            
            # Teacher forcing: use true target or prediction?
            use_teacher_forcing = torch.rand(1).item() < teacher_forcing_ratio
            
            if use_teacher_forcing:
                decoder_input = tgt[:, t:t+1, :]  # Use true target
            else:
                decoder_input = prediction  # Use prediction
        
        # Stack all outputs
        outputs = torch.cat(outputs, dim=1)
        return outputs

# Test
model = Seq2Seq(input_size=10, hidden_size=64, output_size=10)
src = torch.randn(32, 20, 10)  # 32 sequences, length 20
tgt = torch.randn(32, 15, 10)  # Target length 15
output = model(src, tgt)
print(output.shape)  # torch.Size([32, 15, 10])
\end{lstlisting}

\clearpage
\subsubsection{Time Series Forecasting Example}

\begin{lstlisting}
class TimeSeriesForecaster(nn.Module):
    """Predict future values from past observations."""
    
    def __init__(self, input_size=1, hidden_size=64, num_layers=2, 
                 forecast_horizon=10):
        super().__init__()
        
        self.hidden_size = hidden_size
        self.num_layers = num_layers
        self.forecast_horizon = forecast_horizon
        
        # Encoder LSTM
        self.lstm = nn.LSTM(
            input_size=input_size,
            hidden_size=hidden_size,
            num_layers=num_layers,
            batch_first=True,
            dropout=0.2 if num_layers > 1 else 0
        )
        
        # Decoder: predict next step
        self.fc = nn.Linear(hidden_size, input_size)
    
    def forward(self, x, future_steps=0):
        """
        Args:
            x: Past observations (batch, seq_len, input_size)
            future_steps: How many steps to predict into future
        
        Returns:
            predictions: (batch, seq_len + future_steps, input_size)
        """
        batch_size = x.size(0)
        
        # Process historical data
        output, (h, c) = self.lstm(x)
        
        # Predict on historical data
        predictions = self.fc(output)
        
        # If we need to predict future steps
        if future_steps > 0:
            future_preds = []
            
            # Use last prediction as next input
            last_pred = predictions[:, -1:, :]
            
            for _ in range(future_steps):
                # Predict next step
                output, (h, c) = self.lstm(last_pred, (h, c))
                pred = self.fc(output)
                future_preds.append(pred)
                
                # Use prediction as next input (autoregressive)
                last_pred = pred
            
            # Concatenate all predictions
            future_preds = torch.cat(future_preds, dim=1)
            predictions = torch.cat([predictions, future_preds], dim=1)
        
        return predictions

# Example usage
model = TimeSeriesForecaster(input_size=1, hidden_size=64)

# Historical data: 100 time steps
x = torch.randn(32, 100, 1)

# Predict on historical + 20 future steps
predictions = model(x, future_steps=20)
print(predictions.shape)  # torch.Size([32, 120, 1])
\end{lstlisting}

\subsection{Exercises}

\begin{exercise}[10.1: Simple Sequence Classification - $\bigstar\bigstar$]
\textbf{Goal:} Build your first RNN.

\begin{enumerate}
    \item Generate synthetic sequences: sine waves (class 0) and cosine waves (class 1)
    \item Each sequence: 50 time steps
    \item Build an LSTM classifier
    \item Train for 20 epochs
    \item Achieve >95\% accuracy
\end{enumerate}

\textbf{Starter code:}
\begin{lstlisting}
import numpy as np

def generate_sine_data(n_samples=1000, seq_len=50):
    """Generate sine wave sequences."""
    X = []
    y = []
    for _ in range(n_samples):
        # Random frequency and phase
        freq = np.random.uniform(0.5, 2.0)
        phase = np.random.uniform(0, 2*np.pi)
        t = np.linspace(0, 4*np.pi, seq_len)
        
        if np.random.rand() < 0.5:
            # Sine wave
            seq = np.sin(freq * t + phase)
            label = 0
        else:
            # Cosine wave
            seq = np.cos(freq * t + phase)
            label = 1
        
        X.append(seq)
        y.append(label)
    
    return torch.FloatTensor(X).unsqueeze(-1), torch.LongTensor(y)

# Your LSTM classifier here
\end{lstlisting}
\end{exercise}

\begin{exercise}[10.2: LSTM vs GRU Comparison - $\bigstar\bigstar\bigstar$]
\textbf{Goal:} Compare LSTM and GRU empirically.

\begin{enumerate}
    \item Generate a moderately complex sequence task
    \item Train identical architectures with:
    \begin{itemize}
        \item LSTM (2 layers, 128 hidden)
        \item GRU (2 layers, 128 hidden)
        \item Vanilla RNN (2 layers, 128 hidden)
    \end{itemize}
    \item Compare:
    \begin{itemize}
        \item Training speed (time per epoch)
        \item Convergence speed (epochs to 90\% accuracy)
        \item Final accuracy
        \item Number of parameters
    \end{itemize}
\end{enumerate}

\textbf{Expected results:}
\begin{itemize}
    \item LSTM and GRU should perform similarly
    \item GRU should be faster (fewer parameters)
    \item Vanilla RNN should struggle (vanishing gradients)
\end{itemize}
\end{exercise}

\begin{exercise}[10.3: Sequence-to-Sequence Labeling - $\bigstar\bigstar\bigstar$]
\textbf{Goal:} Many-to-many prediction.

\begin{enumerate}
    \item Task: Given a noisy sine wave, denoise it at each time step
    \item Input: sine + Gaussian noise
    \item Target: clean sine wave
    \item Build a bidirectional LSTM
    \item Output prediction at each time step
    \item Visualize: noisy input vs clean output
\end{enumerate}

\textbf{Evaluation:}
\begin{itemize}
    \item Mean squared error on test set
    \item Visual inspection of denoised signals
\end{itemize}
\end{exercise}

\begin{exercise}[10.4: Variable-Length Sequences - $\bigstar\bigstar\bigstar$]
\textbf{Goal:} Handle sequences of different lengths efficiently.

\begin{enumerate}
    \item Generate sequences with random lengths (20-100)
    \item Implement proper padding and packing:
    \begin{itemize}
        \item Pad sequences to same length
        \item Use \texttt{pack\_padded\_sequence}
        \item Use \texttt{pad\_packed\_sequence}
    \end{itemize}
    \item Train two models:
    \begin{itemize}
        \item Without packing (processes padding)
        \item With packing (skips padding)
    \end{itemize}
    \item Compare:
    \begin{itemize}
        \item Training time
        \item Accuracy
        \item Verify packing gives same results
    \end{itemize}
\end{enumerate}
\end{exercise}

\begin{exercise}[10.5: Time Series Forecasting - $\bigstar\bigstar\bigstar\bigstar$]
\textbf{Goal:} Predict future values.

\begin{enumerate}
    \item Use real time series data (or generate complex synthetic data)
    \item Sliding window approach:
    \begin{itemize}
        \item Input: past 50 steps
        \item Target: next 10 steps
    \end{itemize}
    \item Build forecasting model with LSTM
    \item Implement autoregressive prediction (use predictions as inputs)
    \item Evaluate:
    \begin{itemize}
        \item One-step-ahead prediction
        \item Multi-step-ahead prediction
        \item Compare with baseline (simple moving average)
    \end{itemize}
    \item Visualize predictions vs ground truth
\end{enumerate}

\textbf{Metrics:}
\begin{lstlisting}
def calculate_metrics(y_true, y_pred):
    """Calculate forecasting metrics."""
    mse = torch.mean((y_true - y_pred) ** 2)
    mae = torch.mean(torch.abs(y_true - y_pred))
    
    # MAPE (Mean Absolute Percentage Error)
    mape = torch.mean(torch.abs((y_true - y_pred) / (y_true + 1e-8))) * 100
    
    return {
        'MSE': mse.item(),
        'MAE': mae.item(),
        'MAPE': mape.item()
    }
\end{lstlisting}
\end{exercise}

\begin{exercise}[10.6: Gradient Clipping Investigation - $\bigstar\bigstar\bigstar\bigstar$]
\textbf{Goal:} Understand why gradient clipping is essential.

\begin{enumerate}
    \item Train an RNN on a long sequence task (100+ steps)
    \item Train multiple times with different settings:
    \begin{itemize}
        \item No gradient clipping
        \item \texttt{max\_norm=1.0}
        \item \texttt{max\_norm=5.0}
        \item \texttt{max\_norm=10.0}
    \end{itemize}
    \item Monitor and plot:
    \begin{itemize}
        \item Gradient norms over time
        \item Loss curves
        \item Percentage of batches where gradients are clipped
    \end{itemize}
    \item Observe:
    \begin{itemize}
        \item Without clipping: loss spikes, NaN values
        \item With clipping: stable training
    \end{itemize}
\end{enumerate}

\textbf{Visualization code:}
\begin{lstlisting}
def track_gradient_norms(model):
    """Track gradient norms during training."""
    total_norm = 0
    for p in model.parameters():
        if p.grad is not None:
            param_norm = p.grad.data.norm(2)
            total_norm += param_norm.item() ** 2
    total_norm = total_norm ** 0.5
    return total_norm

# In training loop
gradient_norms = []
for epoch in range(num_epochs):
    for batch in dataloader:
        optimizer.zero_grad()
        loss = ...
        loss.backward()
        
        # Track before clipping
        norm_before = track_gradient_norms(model)
        gradient_norms.append(norm_before)
        
        torch.nn.utils.clip_grad_norm_(model.parameters(), max_norm=5.0)
        optimizer.step()

# Plot
plt.plot(gradient_norms)
plt.axhline(y=5.0, color='r', linestyle='--', label='Clip threshold')
plt.xlabel('Step')
plt.ylabel('Gradient Norm')
plt.legend()
plt.show()
\end{lstlisting}
\end{exercise}

\clearpage
\subsection{Key Takeaways}

\textbf{When to use RNNs:}
\begin{itemize}
    \item Sequential data with temporal dependencies
    \item Variable-length sequences
    \item Time series forecasting
    \item Text processing (though Transformers now dominate)
    \item When order matters
\end{itemize}

\textbf{Architecture choices:}
\begin{itemize}
    \item \textbf{Vanilla RNN:} Almost never use (vanishing gradients)
    \item \textbf{LSTM:} Default choice, especially for complex tasks
    \item \textbf{GRU:} When speed matters or dataset is smaller
    \item \textbf{Bidirectional:} When you have full sequence (not streaming)
    \item \textbf{Stacked (2-4 layers):} For complex patterns
\end{itemize}

\textbf{Essential training practices:}
\begin{itemize}
    \item \textbf{Always use gradient clipping} (max\_norm=5.0 is good default)
    \item Initialize forget gate bias to 1 (helps gradient flow)
    \item Use orthogonal initialization for hidden-to-hidden weights
    \item Lower learning rates than CNNs (0.001 typical)
    \item Larger batch sizes (32-128)
    \item Pack sequences when lengths vary (efficiency)
\end{itemize}

\textbf{Common patterns:}
\begin{itemize}
    \item \textbf{Many-to-one:} Classification (use last hidden state)
    \item \textbf{Many-to-many (same length):} Sequence labeling (output at each step)
    \item \textbf{Many-to-many (different length):} Seq2seq with encoder-decoder
    \item \textbf{One-to-many:} Generation (feed output back as input)
\end{itemize}

\textbf{Why LSTMs work better than vanilla RNNs:}
\begin{itemize}
    \item Cell state provides gradient highway (no repeated matrix multiplications)
    \item Gates control information flow (selective memory)
    \item Can learn dependencies over 100+ steps
    \item Forget gate prevents exploding activations
\end{itemize}

\textbf{Limitations of RNNs:}
\begin{itemize}
    \item Sequential processing (slow, can't parallelize)
    \item Still struggle with very long sequences (>1000 steps)
    \item Transformers now preferred for many NLP tasks
    \item Require careful tuning (gradient clipping, learning rate)
\end{itemize}

\textbf{Practical tips:}
\begin{itemize}
    \item Start with 2-layer LSTM, hidden size = 128 or 256
    \item Use \texttt{batch\_first=True} always
    \item Remember LSTM returns \texttt{(h, c)} tuple
    \item Dropout between layers (0.2-0.3), higher on outputs (0.5)
    \item Monitor gradient norms (should be <10)
    \item Use early stopping (RNNs prone to overfitting)
\end{itemize}

\textbf{Debugging checklist:}
\begin{itemize}
    \item Loss explodes? → Add/increase gradient clipping
    \item Loss stuck? → Check learning rate, try lower
    \item Poor performance? → Try LSTM instead of GRU, add layers
    \item Slow training? → Use packing for variable lengths
    \item Different train/test? → Check dropout mode
\end{itemize}

\textbf{Modern alternatives:}
\begin{itemize}
    \item \textbf{Transformers:} Better for most NLP tasks (parallelizable)
    \item \textbf{Temporal CNNs:} Good for some time series (efficient)
    \item \textbf{State space models:} Emerging alternative (S4, Mamba)
\end{itemize}

However, RNNs still useful for:
\begin{itemize}
    \item Online/streaming applications (Transformers need full context)
    \item Very long sequences where Transformers too expensive
    \item Small datasets (fewer parameters than Transformers)
    \item Scientific time series (physics-informed RNNs)
\end{itemize}

\clearpage