\documentclass[11pt, a4paper]{article}

% =============================================
% ===    ENCODING AND LANGUAGE SETUP      ===
% =============================================
\usepackage[utf8]{inputenc}
\usepackage[T1]{fontenc}
\usepackage{lmodern}            % A clean, modern font
\usepackage[english]{babel}

% =============================================
% ===         PAGE GEOMETRY               ===
% =============================================
\usepackage{geometry}
\geometry{
    a4paper,
    margin=1.2in,
    bindingoffset=0.5cm,      % Extra space for binding
    headheight=15pt
}

% =============================================
% ===     MATHEMATICAL TYPESETTING        ===
% =============================================
\usepackage{amsmath, amssymb, amsthm, mathtools}
\usepackage{bm}                 % Bold math symbols
\usepackage{dsfont}             % For indicator functions \mathds{1}

% Math operators for neural networks
\DeclareMathOperator*{\argmax}{arg\,max}
\DeclareMathOperator*{\argmin}{arg\,min}
\DeclareMathOperator{\softmax}{softmax}
\DeclareMathOperator{\sigmoid}{sigmoid}
\DeclareMathOperator{\ReLU}{ReLU}
\DeclareMathOperator{\GELU}{GELU}
\DeclareMathOperator{\Tr}{Tr}
\DeclareMathOperator{\diag}{diag}
\DeclareMathOperator{\conv}{conv}
\DeclareMathOperator{\Attention}{Attention}

% Common abbreviations
\newcommand{\E}{\mathbb{E}}      % Expectation
\newcommand{\R}{\mathbb{R}}      % Real numbers
\newcommand{\N}{\mathbb{N}}      % Natural numbers
\newcommand{\Z}{\mathbb{Z}}      % Integers
\newcommand{\C}{\mathbb{C}}      % Complex numbers
\newcommand{\Prob}{\mathbb{P}}   % Probability

% Vector and matrix notation
\newcommand{\vect}[1]{\bm{#1}}
\newcommand{\mat}[1]{\bm{#1}}
\newcommand{\transpose}{^\top}
\newcommand{\inv}{^{-1}}

% Derivatives and gradients
\newcommand{\dd}{\mathrm{d}}
\newcommand{\dderiv}[2]{\frac{\dd #1}{\dd #2}}
\newcommand{\pderiv}[2]{\frac{\partial #1}{\partial #2}}
\newcommand{\grad}{\nabla}
\newcommand{\divergence}{\nabla \cdot}
\newcommand{\laplacian}{\nabla^2}

% Loss functions and norms
\newcommand{\loss}{\mathcal{L}}
\newcommand{\norm}[1]{\left\| #1 \right\|}
\newcommand{\abs}[1]{\left| #1 \right|}
\newcommand{\inner}[2]{\left\langle #1, #2 \right\rangle}

% Neural network notation
\newcommand{\weight}{\mat{W}}
\newcommand{\bias}{\vect{b}}
\newcommand{\activation}{\sigma}
\newcommand{\layer}[1]{\vect{h}^{(#1)}}

% =============================================
% ===     GRAPHICS AND FIGURES            ===
% =============================================
\usepackage{graphicx}
\usepackage{float}              % For [H] placement
\usepackage{subcaption}         % For subfigures
\usepackage{wrapfig}            % Text wrapping around figures
\usepackage{tikz}               % For drawing diagrams
\usetikzlibrary{arrows.meta, positioning, shapes.geometric, calc, decorations.pathreplacing}

% Figure path (if using external figures)
% \graphicspath{{./figures/}}

% =============================================
% ===          COLOR DEFINITIONS          ===
% =============================================
\usepackage[dvipsnames, svgnames, table]{xcolor}

% Code highlighting colors
\definecolor{codegray}{gray}{0.95}
\definecolor{codepurple}{rgb}{0.58, 0, 0.82}
\definecolor{codeblue}{rgb}{0, 0, 0.6}
\definecolor{codegreen}{rgb}{0.1, 0.5, 0.1}
\definecolor{codeorange}{rgb}{0.9, 0.4, 0}
\definecolor{codered}{rgb}{0.6, 0, 0}

% Box colors for different environments
\definecolor{noteblue}{RGB}{230, 240, 255}
\definecolor{defgreen}{RGB}{235, 255, 235}
\definecolor{pytorchorange}{RGB}{255, 245, 230}
\definecolor{exercisered}{RGB}{255, 240, 240}
\definecolor{warningyellow}{RGB}{255, 250, 205}
\definecolor{trickpurple}{RGB}{245, 235, 255}
\definecolor{applicationcyan}{RGB}{230, 250, 250}

% =============================================
% ===         HYPERLINKS AND TOC          ===
% =============================================
\usepackage{hyperref}
\hypersetup{
    colorlinks=true,
    linkcolor=blue!60!black,
    citecolor=green!50!black,
    urlcolor=blue!60!black,
    pdftitle={Deep Learning Architectures for Scientific Machine Learning},
    pdfauthor={Professor Vectorex Gradiens},
    pdfsubject={Deep Learning, PyTorch, Scientific Computing},
    pdfkeywords={PyTorch, Deep Learning, Scientific ML, Neural Networks},
    bookmarksnumbered=true,
    bookmarksopen=true,
    pdfpagemode=UseOutlines
}

% =============================================
% ===     TYPOGRAPHY AND FORMATTING       ===
% =============================================
\usepackage{microtype}          % Subtle typographical improvements
\usepackage{parskip}            % Cleaner paragraph breaks (no indent)
\usepackage{enumitem}           % Better control over lists
\setlist{nosep, topsep=2pt, partopsep=0pt, parsep=0pt, itemsep=2pt}
\usepackage{booktabs}           % Better tables
\usepackage{array}              % Enhanced table features
\usepackage{multirow}           % Multirow cells in tables
\usepackage{caption}            % Better captions
\captionsetup{
    font=small,
    labelfont=bf,
    format=hang,
    justification=justified
}

% Line spacing
\usepackage{setspace}
\setstretch{0.9}                 % Compact line spacing

% Headers and footers
\usepackage{fancyhdr}
\pagestyle{fancy}
\fancyhf{}
\fancyhead[LE]{\leftmark}
\fancyhead[RO]{\rightmark}
\fancyfoot[C]{\thepage}
\renewcommand{\headrulewidth}{0.4pt}
\renewcommand{\footrulewidth}{0pt}

% Section styling (article doesn't have chapters)
\usepackage{titlesec}
\titleformat{\section}
    {\normalfont\Large\bfseries\color{blue!60!black}}
    {\thesection}{1em}{}
\titleformat{\subsection}
    {\normalfont\large\bfseries\color{blue!50!black}}
    {\thesubsection}{1em}{}

% =============================================
% ===         CODE LISTINGS               ===
% =============================================
\usepackage{listings}

% Python style
\lstdefinestyle{pythonstyle}{
    language=Python,
    backgroundcolor=\color{codegray},
    commentstyle=\color{codegreen}\itshape,
    keywordstyle=\color{codeblue}\bfseries,
    numberstyle=\tiny\color{gray},
    stringstyle=\color{codepurple},
    basicstyle=\ttfamily\small,
    breakatwhitespace=false,
    breaklines=true,
    captionpos=b,
    keepspaces=true,
    numbers=left,
    numbersep=5pt,
    showspaces=false,
    showstringspaces=false,
    showtabs=false,
    tabsize=2,
    frame=tb,
    framerule=0.5pt,
    xleftmargin=2em,
    framexleftmargin=1.5em,
    morekeywords={self, True, False, None, with, as},
    % PyTorch-specific keywords
    emphstyle=\color{codeorange}\bfseries,
    emph={torch, nn, functional, optim, autograd, cuda, Tensor, Module, Parameter, DataLoader, Dataset}
}
\lstset{style=pythonstyle}

% Inline code style
\newcommand{\code}[1]{\texttt{\color{codeblue}#1}}
\newcommand{\pythoninline}[1]{\lstinline[language=Python, basicstyle=\ttfamily\small]{#1}}

% =============================================
% ===       COLORED BOXES (TCOLORBOX)     ===
% =============================================
\usepackage[most]{tcolorbox}

% Professor's Note Box
\newtcolorbox{profnote}[1][Professor's Note]{
    colback=noteblue,
    colframe=Blue!60!Black,
    fonttitle=\bfseries,
    title={#1},
    enhanced,
    attach boxed title to top left={yshift=-2mm, xshift=5mm},
    boxed title style={colback=Blue!60!Black, colframe=Blue!60!Black}
}

% Definition Box
\newtcolorbox{definition}[1][Definition]{
    colback=defgreen,
    colframe=Green!50!Black,
    fonttitle=\bfseries,
    title={#1},
    enhanced,
    attach boxed title to top left={yshift=-2mm, xshift=5mm},
    boxed title style={colback=Green!50!Black, colframe=Green!50!Black}
}

% PyTorch Command/Idiom Box
\newtcolorbox{pytorchcmd}[1][PyTorch Idiom]{
    colback=pytorchorange,
    colframe=Orange!80!Black,
    fonttitle=\bfseries,
    title={#1},
    enhanced,
    attach boxed title to top left={yshift=-2mm, xshift=5mm},
    boxed title style={colback=Orange!80!Black, colframe=Orange!80!Black}
}

% Exercise Box
\newtcolorbox{exercise}[1][Exercise]{
    colback=exercisered,
    colframe=Red!70!Black,
    fonttitle=\bfseries,
    title={#1},
    enhanced,
    attach boxed title to top left={yshift=-2mm, xshift=5mm},
    boxed title style={colback=Red!70!Black, colframe=Red!70!Black},
    breakable
}

% Warning Box
\newtcolorbox{warning}[1][Warning]{
    colback=warningyellow,
    colframe=orange!80!black,
    fonttitle=\bfseries,
    title={#1},
    enhanced,
    attach boxed title to top left={yshift=-2mm, xshift=5mm},
    boxed title style={colback=orange!80!black, colframe=orange!80!black}
}

% Expert Trick Box
\newtcolorbox{experttrick}[1][Expert Trick]{
    colback=trickpurple,
    colframe=purple!70!black,
    fonttitle=\bfseries,
    title={#1},
    enhanced,
    attach boxed title to top left={yshift=-2mm, xshift=5mm},
    boxed title style={colback=purple!70!black, colframe=purple!70!black}
}

% Scientific Application Box
\newtcolorbox{application}[1][Scientific Application]{
    colback=applicationcyan,
    colframe=cyan!60!black,
    fonttitle=\bfseries,
    title={#1},
    enhanced,
    attach boxed title to top left={yshift=-2mm, xshift=5mm},
    boxed title style={colback=cyan!60!black, colframe=cyan!60!black},
    breakable
}

% =============================================
% ===       THEOREM ENVIRONMENTS          ===
% =============================================
\theoremstyle{definition}
\newtheorem{theorem}{Theorem}[section]
\newtheorem{lemma}[theorem]{Lemma}
\newtheorem{proposition}[theorem]{Proposition}
\newtheorem{corollary}[theorem]{Corollary}

\theoremstyle{remark}
\newtheorem{remark}{Remark}[section]
\newtheorem{example}{Example}[section]

% =============================================
% ===       ALGORITHM ENVIRONMENT         ===
% =============================================
\usepackage[ruled, vlined, linesnumbered]{algorithm2e}
\SetKwInput{KwInput}{Input}
\SetKwInput{KwOutput}{Output}
\SetKwInput{KwParameters}{Parameters}
\SetKwComment{Comment}{/* }{ */}

% =============================================
% ===     BIBLIOGRAPHY SETUP              ===
% =============================================
\usepackage[
    backend=biber,
    style=numeric,
    sorting=nyt,
    maxbibnames=99
]{biblatex}
% \addbibresource{references.bib}  % Uncomment and add your .bib file

% =============================================
% ===          INDEX SETUP                ===
% =============================================
\usepackage{makeidx}
\makeindex

% =============================================
% ===       CUSTOM COMMANDS               ===
% =============================================

% Difficulty stars for exercises
% \newcommand{\difficulty}[1]{%
%     \ifnum#1=1 $\bigstar$\fi%
%     \ifnum#1=2 $\bigstar\bigstar$\fi%
%     \ifnum#1=3 $\bigstar\bigstar\bigstar$\fi%
%     \ifnum#1=4 $\bigstar\bigstar\bigstar\bigstar$\fi%
%     \ifnum#1=5 $\bigstar\bigstar\bigstar\bigstar\bigstar$\fi%
% }

% Emphasize key concepts
\newcommand{\concept}[1]{\textbf{\color{blue!70!black}#1}}
\newcommand{\pytorch}[1]{\texttt{\color{orange!80!black}#1}}

% Matrix dimensions helper
\newcommand{\matdim}[1]{\text{\scriptsize(#1)}}

% Common neural network layers notation
\newcommand{\Linear}[2]{\text{Linear}(#1 \to #2)}
\newcommand{\Conv}[3]{\text{Conv2d}(#1, #2, k=#3)}
\newcommand{\BatchNorm}[1]{\text{BatchNorm}(#1)}
\newcommand{\LayerNorm}[1]{\text{LayerNorm}(#1)}

% Computational complexity
\newcommand{\bigO}{\mathcal{O}}
\newcommand{\complexity}[1]{\bigO(#1)}

% =============================================
% ===       DOCUMENT METADATA             ===
% =============================================
\title{
    \Huge\bfseries
    Deep Learning Architectures for \\
    Scientific Machine Learning \\[0.5cm]
    \large A Comprehensive PyTorch Guide
}
\author{\large Professor Vectorex Gradiens}
\date{\today \\ \vspace{0.5cm} \small Version 1.0}

% =============================================
% ===       PDF METADATA                  ===
% =============================================
\usepackage{datetime}
\newdateformat{mydate}{\THEDAY\ \monthname[\THEMONTH] \THEYEAR}

% =============================================
% ===    ADDITIONAL USEFUL PACKAGES       ===
% =============================================
\usepackage{lipsum}             % For placeholder text (remove in final)
\usepackage{todonotes}          % For TODO notes (remove in final)

% =============================================
% ===          DOCUMENT BEGINS            ===
% =============================================
\begin{document}

% =============================================
% ===          FRONT MATTER               ===
% =============================================

% --- Title Page ---
\maketitle
\thispagestyle{empty}

\clearpage

% --- Copyright and Disclaimer ---
\thispagestyle{empty}
\vspace*{\fill}
\begin{center}
\textbf{Disclaimer}
\end{center}

\textbf{Important Notice:} Apart from this paragraph, every single word in this document was written with the assistance of Large Language Models. This document is intended as an aid for learning PyTorch, in particular for implementing deep learning architectures presented in the course "AI in the Sciences and Engineering" held at ETH Zürich.

The content is extremely likely to contain errors, so use it with care, knowing that this isn't reviewed or approved by domain experts. Always verify critical implementations and consult primary sources for production use.

\textit{Giovanni Guidarini, December 2025}

\vspace{1cm}

\textbf{License:} This work is shared for educational purposes. PyTorch is developed by Meta AI and is licensed under the BSD-style license. All code examples are provided as-is without warranty.

\vspace{1cm}

\textbf{Acknowledgments:} This textbook builds upon the collective wisdom of the PyTorch community, scientific machine learning researchers, and educators worldwide. Special thanks to the developers of PyTorch, the authors of foundational papers in deep learning, and students whose questions shaped this material.

\vspace*{\fill}
\clearpage

% --- Dedication (optional) ---
% \thispagestyle{empty}
% \vspace*{\fill}
% \begin{center}
% \textit{To all scientists and engineers venturing into the realm of deep learning—\\
% may you find clarity in complexity and beauty in gradients.}
% \end{center}
% \vspace*{\fill}
\clearpage

% --- Preface ---
\section*{Preface}
\addcontentsline{toc}{section}{Preface}

My dear student,

Welcome. It is a distinct pleasure to guide you on this journey. You are already embedded in the world of scientific computing, a field of precision, rigor, and computational elegance. You will find that deep learning is not so different. At its heart, it is a powerful form of high-dimensional function approximation, built on the familiar foundations of linear algebra and calculus. The "magic" is simply that we have found a way to make these function approximators (\textit{neural networks}) trainable on a massive scale.

Your lack of PyTorch experience is not a hindrance; it is an opportunity. You arrive with no bad habits. We will build your knowledge from the ground up, with the same care one takes in formulating a proof or designing a robust simulation. PyTorch is our "language" for this. It is expressive, powerful, and, once you are fluent, a genuine joy to use.

This textbook is comprehensive, spanning from the fundamental \pytorch{torch.Tensor} to state-of-the-art architectures like Transformers and Neural Operators. Each chapter builds systematically, combining rigorous theory with practical implementation. You will not merely learn to use PyTorch—you will understand \textit{why} it works the way it does, and you will develop the intuition to design your own architectures for scientific problems.

\textbf{How to Use This Book:}
\begin{itemize}
    \item \textbf{Theory sections} provide mathematical foundations and intuition. Do not skip these—they are the bedrock upon which everything rests.
    \item \textbf{Implementation sections} show you the "PyTorch way"—idioms, best practices, and common pitfalls. Type out the code yourself; reading is not enough.
    \item \textbf{Exercises} range from elementary to research-level. Attempt all of them. Struggle is where learning happens.
    \item \textbf{Colored boxes} highlight key insights:
    \begin{itemize}
        \item Professor's Notes for expert intuition
        \item PyTorch Idioms for best practices
        \item Warnings for common mistakes
        \item Expert Tricks for advanced techniques
        \item Scientific Applications showing real-world use
    \end{itemize}
\end{itemize}

Treat this not just as a technical exercise, but as an art. The code we write, like the \LaTeX{} that renders this page, can and should be clean, precise, and beautiful.

Let us begin.

\vspace{1cm}
\begin{flushright}
Professor Vectorex Gradiens \\
\textit{ETH Zürich} \\
December 2025
\end{flushright}

\clearpage

% --- Table of Contents ---
\tableofcontents
\clearpage

% --- List of Figures (optional) ---
% \listoffigures
% \clearpage

% --- List of Tables (optional) ---
% \listoftables
% \clearpage

% =============================================
% ===          MAIN MATTER                ===
% =============================================

% =============================================
% === PART I: FOUNDATIONS                 ===
% =============================================
\part{Foundations of PyTorch}
\label{part:foundations}

% Chapter 1 will begin here
% =============================================
% SECTION 1: THE TORCH.TENSOR
% =============================================

\section{The \texttt{torch.Tensor}}

\subsection{Introduction: Why Tensors?}

Everything in PyTorch—from your input data (images, time series, simulation meshes) to the parameters of your neural network (weights and biases)—is represented as a \textbf{tensor}.

If you've used NumPy, you already know the core idea: a tensor is essentially a multi-dimensional array (like \texttt{ndarray}). So why use PyTorch instead of NumPy?

\textbf{Two critical reasons:}

\begin{enumerate}
    \item \textbf{GPU Acceleration:} PyTorch tensors can be moved to GPUs with a single command, enabling massive parallelization. For deep learning and large-scale scientific computing, this isn't optional—it's essential.
    
    \item \textbf{Automatic Differentiation:} PyTorch can automatically compute gradients of any output with respect to any input. This \texttt{autograd} system is the engine that makes deep learning possible. We'll explore this in detail in Section 2.
\end{enumerate}

\subsection{Theory: Tensor Fundamentals}

\begin{definition}[Tensor]
A \textbf{tensor} is a multi-dimensional array of numerical values. It generalizes:
\begin{itemize}
    \item \textbf{Scalar} (0D tensor): a single number
    \item \textbf{Vector} (1D tensor): an array of numbers
    \item \textbf{Matrix} (2D tensor): a 2D grid of numbers
    \item \textbf{Higher-dimensional tensors}: 3D, 4D, ... arrays
\end{itemize}
\end{definition}

\textbf{Key Properties:}

\begin{itemize}
    \item \textbf{Shape:} The dimensions of the tensor, e.g., \pythoninline{(3, 4, 5)} means 3×4×5
    \item \textbf{Dtype:} The data type (e.g., \texttt{float32}, \texttt{int64}, \texttt{bool})
    \item \textbf{Device:} Where the tensor lives (\texttt{cpu} or \texttt{cuda:0}, \texttt{cuda:1}, etc.)
    \item \textbf{Requires\_grad:} Whether to track operations for automatic differentiation
\end{itemize}

\textbf{Common Tensor Shapes in Deep Learning:}

\begin{table}[h]
\centering
\begin{tabular}{ll}
\toprule
\textbf{Shape} & \textbf{Typical Use} \\
\midrule
\texttt{(N,)} & Vector of N values \\
\texttt{(N, D)} & Batch of N samples, D features each \\
\texttt{(N, C, H, W)} & Batch of N images: C channels, H×W pixels \\
\texttt{(N, L, D)} & Batch of N sequences, length L, D features \\
\texttt{(N, C, D, H, W)} & Batch of N 3D volumes \\
\bottomrule
\end{tabular}
\caption{Common tensor shapes. N=batch size, C=channels, H/W=height/width, L=sequence length, D=feature dimension}
\end{table}

\subsection{Implementation: Creating and Manipulating Tensors}

\subsubsection{Creating Tensors}

\begin{lstlisting}
import torch

# From Python lists
x = torch.tensor([1, 2, 3])
print(x)  # tensor([1, 2, 3])

# Specify dtype
x = torch.tensor([1.0, 2.0, 3.0], dtype=torch.float32)
print(x.dtype)  # torch.float32

# Common initialization functions
zeros = torch.zeros(2, 3)           # 2x3 matrix of zeros
ones = torch.ones(2, 3)             # 2x3 matrix of ones
random = torch.randn(2, 3)          # Random normal distribution
uniform = torch.rand(2, 3)          # Uniform distribution [0, 1)
arange = torch.arange(0, 10, 2)     # [0, 2, 4, 6, 8]
linspace = torch.linspace(0, 1, 5)  # [0.0, 0.25, 0.5, 0.75, 1.0]

# Like existing tensor (same shape/dtype)
x = torch.ones(3, 4)
y = torch.zeros_like(x)  # Same shape as x
z = torch.randn_like(x)  # Same shape, random values
\end{lstlisting}

\begin{pytorchtip}[Default Dtype]
PyTorch's default floating point dtype is \texttt{torch.float32}. For most deep learning, this is fine. You can change it globally with:
\begin{lstlisting}
torch.set_default_dtype(torch.float64)  # More precision
\end{lstlisting}
However, \texttt{float32} is usually the sweet spot: enough precision, less memory, faster on GPUs.
\end{pytorchtip}

\clearpage
\subsubsection{Tensor Properties}

\begin{lstlisting}
x = torch.randn(2, 3, 4)

print(x.shape)        # torch.Size([2, 3, 4])
print(x.size())       # Same as .shape
print(x.ndim)         # 3 (number of dimensions)
print(x.numel())      # 24 (total number of elements)
print(x.dtype)        # torch.float32
print(x.device)       # cpu

# Check if tensor requires gradients
print(x.requires_grad)  # False by default
\end{lstlisting}

\subsubsection{Indexing and Slicing}

PyTorch indexing works like NumPy:

\begin{lstlisting}
x = torch.arange(12).reshape(3, 4)
# tensor([[ 0,  1,  2,  3],
#         [ 4,  5,  6,  7],
#         [ 8,  9, 10, 11]])

# Basic indexing
print(x[0])       # First row: tensor([0, 1, 2, 3])
print(x[:, 0])    # First column: tensor([0, 4, 8])
print(x[1, 2])    # Element at (1, 2): tensor(6)

# Slicing
print(x[:2, :2])  # Top-left 2x2:
# tensor([[0, 1],
#         [4, 5]])

# Advanced indexing
indices = torch.tensor([0, 2])
print(x[indices])  # Rows 0 and 2

# Boolean masking
mask = x > 5
print(x[mask])    # tensor([ 6,  7,  8,  9, 10, 11])
\end{lstlisting}

\begin{warningbox}[Views vs Copies]
Slicing creates a \textbf{view}, not a copy! Modifying a view changes the original:
\begin{lstlisting}
x = torch.ones(3, 3)
y = x[0]      # View of first row
y[0] = 999    # Modifies x!
print(x)      # First element is now 999

# To create a copy:
y = x[0].clone()
\end{lstlisting}
\end{warningbox}

\subsubsection{Shape Manipulation}

\begin{lstlisting}
x = torch.arange(12)

# Reshape (must have same total elements)
y = x.reshape(3, 4)
z = x.view(2, 6)      # Similar to reshape
w = x.reshape(3, -1)  # -1 means "infer this dimension" -> (3, 4)

# Transpose
x = torch.randn(2, 3)
y = x.T               # Shape: (3, 2)
z = x.transpose(0, 1) # Same as .T

# Permute (generalized transpose)
x = torch.randn(2, 3, 4)
y = x.permute(2, 0, 1)  # Shape: (4, 2, 3)

# Flatten
x = torch.randn(2, 3, 4)
y = x.flatten()          # Shape: (24,)
z = x.flatten(1)         # Flatten from dim 1: (2, 12)

# Squeeze and unsqueeze (remove/add dimensions of size 1)
x = torch.randn(2, 1, 3, 1)
y = x.squeeze()          # Shape: (2, 3)
z = x.squeeze(1)         # Remove dim 1: (2, 3, 1)

x = torch.randn(2, 3)
y = x.unsqueeze(0)       # Shape: (1, 2, 3)
z = x.unsqueeze(-1)      # Shape: (2, 3, 1)
\end{lstlisting}

\begin{pytorchtip}[Understanding Dimensions]
\begin{itemize}
    \item Dimension indices start at 0
    \item Negative indices count from the end: -1 is the last dimension
    \item \pythoninline{unsqueeze(1)} adds a dimension at position 1
    \item \pythoninline{squeeze()} removes ALL dimensions of size 1
\end{itemize}
\end{pytorchtip}

\clearpage
\subsubsection{Broadcasting}

Broadcasting is one of the most powerful—and confusing—features. It allows operations between tensors of different shapes.

\textbf{Broadcasting Rules:}
\begin{enumerate}
    \item If tensors have different number of dimensions, prepend 1s to the smaller one
    \item Dimensions are compatible if they're equal or one of them is 1
    \item The result shape is the maximum along each dimension
\end{enumerate}

\begin{lstlisting}
# Example 1: Vector + Scalar
x = torch.tensor([1, 2, 3])  # Shape: (3,)
y = 10                        # Shape: () - scalar
z = x + y                     # Shape: (3,) - broadcasts scalar to (3,)
# Result: tensor([11, 12, 13])

# Example 2: Matrix + Vector (row-wise)
x = torch.ones(3, 4)          # Shape: (3, 4)
y = torch.tensor([1, 2, 3, 4])  # Shape: (4,)
z = x + y                     # y broadcasts to (1, 4) then (3, 4)
# Each row of x has y added to it

# Example 3: Matrix + Column Vector
x = torch.ones(3, 4)          # Shape: (3, 4)
y = torch.tensor([[1], [2], [3]])  # Shape: (3, 1)
z = x + y                     # y broadcasts to (3, 4)
# Each column of x has corresponding y value added

# Example 4: Batch operations
x = torch.randn(10, 3, 4)     # 10 matrices of size 3x4
y = torch.randn(3, 4)         # Single matrix
z = x + y                     # y broadcasts to (10, 3, 4)
# Same matrix y added to all 10 matrices in x
\end{lstlisting}

\begin{warningbox}[Common Broadcasting Mistake]
\begin{lstlisting}
# Trying to add vectors of different sizes
x = torch.ones(3)     # Shape: (3,)
y = torch.ones(4)     # Shape: (4,)
z = x + y             # ERROR! Shapes don't match

# Need to explicitly reshape:
x = x.unsqueeze(1)    # Shape: (3, 1)
y = y.unsqueeze(0)    # Shape: (1, 4)
z = x + y             # Shape: (3, 4) - outer sum
\end{lstlisting}
\end{warningbox}

\subsubsection{Mathematical Operations}

\begin{lstlisting}
x = torch.tensor([1.0, 2.0, 3.0])
y = torch.tensor([4.0, 5.0, 6.0])

# Element-wise operations
z = x + y          # Addition
z = x - y          # Subtraction
z = x * y          # Multiplication
z = x / y          # Division
z = x ** 2         # Power
z = torch.sqrt(x)  # Square root
z = torch.exp(x)   # Exponential
z = torch.log(x)   # Natural log

# In-place operations (modify the tensor)
x.add_(5)          # Add 5 to x in-place
x.mul_(2)          # Multiply x by 2 in-place

# Matrix operations
A = torch.randn(3, 4)
B = torch.randn(4, 5)
C = torch.mm(A, B)           # Matrix multiply: (3, 5)
C = torch.matmul(A, B)       # Same, but works for batches too
C = A @ B                    # Python 3.5+ operator

# Batch matrix multiply
A = torch.randn(10, 3, 4)    # 10 matrices of size 3x4
B = torch.randn(10, 4, 5)    # 10 matrices of size 4x5
C = torch.bmm(A, B)          # Shape: (10, 3, 5)

# Reduction operations
x = torch.randn(3, 4)
mean = x.mean()              # Scalar: mean of all elements
std = x.std()                # Standard deviation
sum_all = x.sum()            # Sum of all elements
sum_dim0 = x.sum(dim=0)      # Sum along dimension 0: (4,)
sum_dim1 = x.sum(dim=1)      # Sum along dimension 1: (3,)
max_val, max_idx = x.max(dim=1)  # Max value and index per row
\end{lstlisting}

\begin{pytorchtip}[Dimension Conventions]
When reducing along a dimension:
\begin{itemize}
    \item \pythoninline{dim=0}: Operate across rows (result has fewer rows)
    \item \pythoninline{dim=1}: Operate across columns (result has fewer columns)
    \item \pythoninline{dim=-1}: Last dimension (most common for sequences)
    \item \pythoninline{keepdim=True}: Keep the dimension (size 1) in result
\end{itemize}
\end{pytorchtip}

\clearpage
\subsubsection{Moving Tensors Between Devices}

\begin{lstlisting}
# Check if CUDA is available
if torch.cuda.is_available():
    device = torch.device("cuda")
    print(f"Using GPU: {torch.cuda.get_device_name(0)}")
else:
    device = torch.device("cpu")
    print("Using CPU")

# Create tensor on CPU
x = torch.randn(1000, 1000)

# Move to GPU
x_gpu = x.to(device)
# Or:
x_gpu = x.cuda()  # Shorthand for .to('cuda')

# Move back to CPU
x_cpu = x_gpu.to('cpu')
# Or:
x_cpu = x_gpu.cpu()

# Operations must be on same device
y_gpu = torch.randn(1000, 1000, device=device)  # Create directly on GPU
z = x_gpu + y_gpu  # OK: both on GPU

# This would ERROR:
# z = x_cpu + y_gpu  # ERROR! Different devices
\end{lstlisting}

\begin{warningbox}[Device Mismatch]
One of the most common errors:
\begin{lstlisting}
RuntimeError: Expected all tensors to be on the same device
\end{lstlisting}
Always ensure tensors are on the same device before operations! Use \pythoninline{.to(device)} consistently.
\end{warningbox}

\subsection{Exercises}

\begin{exercise}[1.1: Tensor Creation - $\bigstar\bigstar$]
\textbf{Goal:} Get comfortable creating tensors in different ways.

\begin{enumerate}
    \item Create a tensor of shape (5, 3) filled with random values from a standard normal distribution
    \item Create a tensor of shape (4, 4) filled with values from 0 to 15 (use \texttt{arange} and \texttt{reshape})
    \item Create a tensor of shape (10,) with evenly spaced values from 0 to 1
    \item Create a (3, 3) identity matrix (hint: \texttt{torch.eye})
\end{enumerate}

\textbf{Starter code:}
\begin{lstlisting}
import torch

# Your code here
x1 = ...
x2 = ...
x3 = ...
x4 = ...

print(x1.shape, x2.shape, x3.shape, x4.shape)
\end{lstlisting}
\end{exercise}

\begin{exercise}[1.2: Shape Manipulation - $\bigstar\bigstar$]
\textbf{Goal:} Master reshaping and dimension manipulation.

Given \pythoninline{x = torch.arange(24)}:
\begin{enumerate}
    \item Reshape x to (2, 3, 4)
    \item Transpose the last two dimensions to get (2, 4, 3)
    \item Flatten the result to 1D
    \item Create a view with an extra dimension at position 0 to get (1, 24)
\end{enumerate}

\textbf{Hint:} Use \texttt{reshape}, \texttt{transpose}, \texttt{flatten}, \texttt{unsqueeze}.
\end{exercise}

\begin{exercise}[1.3: Broadcasting Mastery - $\bigstar\bigstar\bigstar$]
\textbf{Goal:} Understand broadcasting and avoid common mistakes.

\begin{enumerate}
    \item Create matrix A of shape (5, 1) with values [1, 2, 3, 4, 5]
    \item Create vector b of shape (3,) with values [10, 20, 30]
    \item Compute C = A + b using broadcasting. What is the shape of C?
    \item Compute the outer product of two vectors u (size 4) and v (size 3) without loops (result should be 4×3)
    \item Given batch of images (16, 3, 32, 32) and per-channel mean (3,), subtract the mean from each image
\end{enumerate}

\textbf{Debugging tip:} If broadcasting fails, print the shapes! Use \pythoninline{print(A.shape, b.shape)}.
\end{exercise}

\begin{exercise}[1.4: Advanced Indexing - $\bigstar\bigstar\bigstar$]
\textbf{Goal:} Learn advanced indexing techniques.

Given \pythoninline{x = torch.randn(100, 5)}:
\begin{enumerate}
    \item Select all rows where the first column is positive
    \item Get the maximum value in each row and its index
    \item Create a new tensor with only the even-indexed rows (0, 2, 4, ...)
    \item Replace all negative values with 0 (use boolean indexing)
\end{enumerate}

\textbf{Hint:} Use boolean masks: \pythoninline{mask = x[:, 0] > 0}, then \pythoninline{x[mask]}.
\end{exercise}

\begin{exercise}[1.5: Implementing Distance Matrix - $\bigstar\bigstar\bigstar\bigstar$]
\textbf{Goal:} Use broadcasting for efficient computation.

Given a set of points \pythoninline{X} of shape (N, D) where N=100 points in D=2 dimensions:
\begin{enumerate}
    \item Compute the pairwise Euclidean distance matrix (N×N) \textbf{without loops}
    \item The (i, j) entry should be $\|x_i - x_j\|_2$
    \item Verify the diagonal is all zeros
    \item Find the nearest neighbor for each point (excluding itself)
\end{enumerate}

\textbf{Hint:} Expand dimensions: \pythoninline{X_i = X.unsqueeze(1)} (shape N, 1, D) and \pythoninline{X_j = X.unsqueeze(0)} (shape 1, N, D). Then compute \pythoninline{(X_i - X_j)**2}.
\end{exercise}

\begin{exercise}[1.6: GPU Operations - $\bigstar\bigstar\bigstar$]
\textbf{Goal:} Practice moving tensors between devices.

\begin{enumerate}
    \item Check if CUDA is available on your system
    \item Create a large tensor (1000×1000) on CPU
    \item Move it to GPU (if available) and time a matrix multiplication
    \item Compare the time with the same operation on CPU
    \item Handle the case where GPU is not available gracefully
\end{enumerate}

\textbf{Starter code:}
\begin{lstlisting}
import torch
import time

device = torch.device("cuda" if torch.cuda.is_available() else "cpu")
print(f"Using device: {device}")

# Your code here
\end{lstlisting}
\end{exercise}

\clearpage
% =============================================
% SECTION 2: AUTOGRAD - AUTOMATIC DIFFERENTIATION
% =============================================

\section{Autograd: Automatic Differentiation}

\subsection{Introduction: Why Autograd Matters}

Deep learning is fundamentally an optimization problem. We have a model with parameters (weights), a loss function that measures how wrong our predictions are, and we want to adjust the parameters to minimize the loss.

The key insight: we use \textbf{gradient descent}. We compute the gradient of the loss with respect to each parameter, then update the parameters in the direction that decreases the loss:

\[
\theta_{\text{new}} = \theta_{\text{old}} - \eta \nabla_\theta \mathcal{L}
\]

For a neural network with millions of parameters, computing these gradients by hand is impossible. PyTorch's \texttt{autograd} system does this automatically through the chain rule.

\subsection{Theory: Computational Graphs and Backpropagation}

\subsubsection{The Computational Graph}

Every operation you perform on a tensor creates a \textbf{computational graph}. This graph tracks how each tensor was computed from other tensors.

\textbf{Example:} Simple computation

\begin{lstlisting}
import torch

x = torch.tensor(2.0, requires_grad=True)
y = torch.tensor(3.0, requires_grad=True)
z = x * y + x**2  # z = xy + x^2
\end{lstlisting}

This creates a graph:

\begin{center}
\begin{verbatim}
    x (leaf)    y (leaf)
      |           |
      +-----+-----+
      |     |     |
     x^2    *     |
       \   /      |
         +  ------+
         |
         z
\end{verbatim}
\end{center}

\textbf{Key Concepts:}

\begin{itemize}
    \item \textbf{Leaf tensors:} Input tensors created by the user with \texttt{requires\_grad=True}
    \item \textbf{Non-leaf tensors:} Result of operations on other tensors
    \item \textbf{grad\_fn:} Each non-leaf tensor remembers the operation that created it
\end{itemize}

\subsubsection{Backpropagation: The Chain Rule}

When you call \texttt{z.backward()}, PyTorch:
\begin{enumerate}
    \item Traverses the graph backwards from \texttt{z}
    \item Applies the chain rule at each operation
    \item Accumulates gradients in the \texttt{.grad} attribute of leaf tensors
\end{enumerate}

For our example:
\[
\frac{\partial z}{\partial x} = \frac{\partial}{\partial x}(xy + x^2) = y + 2x
\]
\[
\frac{\partial z}{\partial y} = \frac{\partial}{\partial y}(xy + x^2) = x
\]

With $x=2, y=3$: $\frac{\partial z}{\partial x} = 7$, $\frac{\partial z}{\partial y} = 2$

\begin{theorybox}[Why "Automatic"?]
You never write the derivative formulas! PyTorch knows the derivative of every operation (add, multiply, sin, exp, matrix multiply, etc.) and chains them automatically. This is the magic of autograd.
\end{theorybox}

\subsubsection{Dynamic Computation Graphs}

PyTorch uses \textbf{dynamic computation graphs} (define-by-run):
\begin{itemize}
    \item The graph is built \textit{during} the forward pass
    \item You can use Python control flow (if statements, loops)
    \item The graph can be different each time you run the code
    \item After \texttt{.backward()}, the graph is \textit{destroyed} (by default)
\end{itemize}

This is different from TensorFlow 1.x (static graphs) and makes PyTorch more flexible for research.

\subsection{Implementation: Using Autograd in Practice}

\subsubsection{Basic Gradient Computation}

\begin{lstlisting}
import torch

# Create tensors with gradient tracking
x = torch.tensor(2.0, requires_grad=True)
y = torch.tensor(3.0, requires_grad=True)

# Forward pass: compute z
z = x**2 + 2*x*y + y**2

print(z)  # tensor(25., grad_fn=<AddBackward0>)
print(z.grad_fn)  # Shows the operation that created z

# Backward pass: compute gradients
z.backward()

# Gradients are now in .grad
print(x.grad)  # dz/dx = 2x + 2y = 10
print(y.grad)  # dz/dy = 2x + 2y = 10
\end{lstlisting}

\subsubsection{Leaf vs Non-Leaf Tensors}

\begin{lstlisting}
x = torch.tensor(2.0, requires_grad=True)
y = x * 2

print(x.is_leaf)  # True - created by user
print(y.is_leaf)  # False - result of operation

# Only leaf tensors store gradients by default
z = y.sum()
z.backward()

print(x.grad)  # Available! x is a leaf
print(y.grad)  # None! Non-leaf gradients not retained
\end{lstlisting}

To retain gradients for non-leaf tensors:

\begin{lstlisting}
x = torch.tensor([1.0, 2.0, 3.0], requires_grad=True)
y = x * 2
y.retain_grad()  # Tell PyTorch to keep this gradient

z = y.sum()
z.backward()

print(x.grad)  # tensor([2., 2., 2.])
print(y.grad)  # tensor([1., 1., 1.]) - now available!
\end{lstlisting}

\clearpage
\subsubsection{Gradient Accumulation and Zeroing}

\begin{warningbox}[Gradients Accumulate!]
Calling \texttt{.backward()} multiple times \textbf{adds} to existing gradients. You must zero them manually between iterations.
\end{warningbox}

\begin{lstlisting}
x = torch.tensor(2.0, requires_grad=True)

# First backward pass
y = x**2
y.backward()
print(x.grad)  # tensor(4.)

# Second backward pass without zeroing
y = x**2
y.backward()
print(x.grad)  # tensor(8.) - DOUBLED!

# Correct approach: zero gradients first
x.grad.zero_()
y = x**2
y.backward()
print(x.grad)  # tensor(4.) - correct again
\end{lstlisting}

Why does this matter? In training loops, we compute gradients for each batch and update weights. Between batches, we must zero gradients:

\begin{lstlisting}
# Training loop pattern (simplified)
for batch in dataloader:
    # Zero gradients from previous batch
    optimizer.zero_grad()  # Or: model.zero_grad()
    
    # Forward pass
    output = model(batch)
    loss = criterion(output, target)
    
    # Backward pass
    loss.backward()
    
    # Update weights
    optimizer.step()
\end{lstlisting}

\subsubsection{Vector-Jacobian Products}

\texttt{.backward()} can only be called on scalar outputs (or you must provide a gradient argument):

\begin{lstlisting}
# Scalar output - works
x = torch.tensor([1.0, 2.0, 3.0], requires_grad=True)
y = x.sum()  # Scalar
y.backward()
print(x.grad)  # Works!

# Vector output - need gradient argument
x = torch.tensor([1.0, 2.0, 3.0], requires_grad=True)
y = x * 2  # Vector: [2, 4, 6]
# y.backward()  # ERROR! Output is not scalar

# Provide gradient (weights for each output)
gradient = torch.tensor([1.0, 1.0, 1.0])
y.backward(gradient)
print(x.grad)  # tensor([2., 2., 2.])
\end{lstlisting}

\textbf{What's happening?} When output is a vector $\mathbf{y}$, we compute:
\[
\text{x.grad} = \frac{\partial \mathbf{y}}{\partial \mathbf{x}}^T \cdot \text{gradient}
\]

This is a \textbf{vector-Jacobian product}. For scalar loss, gradient is implicitly 1.

\subsubsection{Disabling Gradient Tracking}

Sometimes you don't need gradients (inference, debugging, certain computations):

\begin{lstlisting}
x = torch.tensor([1.0, 2.0], requires_grad=True)

# Method 1: torch.no_grad() context
with torch.no_grad():
    y = x * 2
    z = y.sum()
print(z.requires_grad)  # False
# z.backward()  # ERROR - can't backprop

# Method 2: detach() - creates a view without gradients
y = x * 2
y_detached = y.detach()
print(y.requires_grad)         # True
print(y_detached.requires_grad) # False

# Method 3: requires_grad_() - in-place toggle
x = torch.tensor([1.0, 2.0], requires_grad=True)
x.requires_grad_(False)
print(x.requires_grad)  # False
\end{lstlisting}

\begin{pytorchtip}[When to Disable Gradients]
\begin{itemize}
    \item \textbf{Inference:} Use \texttt{torch.no\_grad()} or \texttt{torch.inference\_mode()} for evaluation
    \item \textbf{Freezing layers:} Set \texttt{param.requires\_grad = False} for parameters you don't want to train
    \item \textbf{Saving memory:} Gradient tracking uses extra memory
    \item \textbf{Performance:} Operations are faster without gradient tracking
\end{itemize}
\end{pytorchtip}

\clearpage
\subsubsection{In-Place Operations}

\begin{warningbox}[In-Place Operations Break Gradients]
In-place operations (those ending with \texttt{\_}) can cause errors in backpropagation.
\end{warningbox}

\begin{lstlisting}
# BAD: In-place operation on tensor used in computation
x = torch.tensor([1.0, 2.0], requires_grad=True)
y = x * 2
x.add_(1)  # Modifies x in-place
# y.backward(torch.ones_like(y))  # ERROR! x was modified

# GOOD: Don't modify tensors that are part of the graph
x = torch.tensor([1.0, 2.0], requires_grad=True)
y = x * 2
z = y.sum()
z.backward()  # Works!
\end{lstlisting}

However, in-place operations are fine for parameters during optimization (outside the graph):

\begin{lstlisting}
# This is OK - happens after .backward()
with torch.no_grad():
    x.add_(-learning_rate * x.grad)
\end{lstlisting}

\subsubsection{Retaining Graphs}

By default, the computation graph is freed after \texttt{.backward()}:

\begin{lstlisting}
x = torch.tensor(2.0, requires_grad=True)
y = x**2
y.backward()
# y.backward()  # ERROR! Graph was freed

# To keep the graph:
x = torch.tensor(2.0, requires_grad=True)
y = x**2
y.backward(retain_graph=True)
y.backward()  # OK now! But remember to zero gradients if needed
\end{lstlisting}

When is this useful? Multiple backward passes (rare), or when you need to call \texttt{.backward()} multiple times in a single forward pass.

\subsubsection{Higher-Order Derivatives}

PyTorch can compute gradients of gradients:

\begin{lstlisting}
# First derivative
x = torch.tensor(2.0, requires_grad=True)
y = x**3  # y = x^3

dy_dx = torch.autograd.grad(y, x, create_graph=True)[0]
print(dy_dx)  # 3*x^2 = 12

# Second derivative
d2y_dx2 = torch.autograd.grad(dy_dx, x)[0]
print(d2y_dx2)  # 6*x = 12
\end{lstlisting}

\textbf{Note:} We use \texttt{torch.autograd.grad()} instead of \texttt{.backward()} here because it gives us the gradient as a tensor (useful for higher-order derivatives).

\begin{pytorchtip}[torch.autograd.grad vs .backward()]
\begin{itemize}
    \item \texttt{.backward()}: Accumulates gradients in \texttt{.grad} attribute. Standard for training.
    \item \texttt{torch.autograd.grad()}: Returns gradients as tensors. Useful for:
    \begin{itemize}
        \item Higher-order derivatives
        \item Multiple gradient computations
        \item When you need gradient as a value, not accumulated
    \end{itemize}
\end{itemize}
\end{pytorchtip}

\subsection{Implementation: Common Patterns and Debugging}

\subsubsection{Checking Gradients}

When implementing custom operations or debugging, verify gradients numerically:

\begin{lstlisting}
import torch

def numerical_gradient(f, x, eps=1e-5):
    """Compute gradient numerically using finite differences."""
    grad = torch.zeros_like(x)
    for i in range(x.numel()):
        x_plus = x.clone()
        x_plus.flatten()[i] += eps
        x_minus = x.clone()
        x_minus.flatten()[i] -= eps
        grad.flatten()[i] = (f(x_plus) - f(x_minus)) / (2 * eps)
    return grad

# Test it
x = torch.tensor([2.0, 3.0], requires_grad=True)
f = lambda t: (t**2).sum()

# Analytical gradient
y = f(x)
y.backward()
analytical_grad = x.grad

# Numerical gradient
numerical_grad = numerical_gradient(f, x)

print("Analytical:", analytical_grad)  # [4., 6.]
print("Numerical:", numerical_grad)    # [4., 6.] (approximately)
print("Close?", torch.allclose(analytical_grad, numerical_grad))
\end{lstlisting}

PyTorch also provides \texttt{torch.autograd.gradcheck}:

\begin{lstlisting}
from torch.autograd import gradcheck

x = torch.randn(3, 4, requires_grad=True, dtype=torch.float64)
func = lambda t: (t**2).sum()

# Check if gradients are correct
test = gradcheck(func, x, eps=1e-6)
print(f"Gradient check passed: {test}")
\end{lstlisting}

\subsubsection{Gradient Flow Debugging}

When training doesn't work, check if gradients are flowing:

\begin{lstlisting}
import torch.nn as nn

# Simple model
model = nn.Sequential(
    nn.Linear(10, 50),
    nn.ReLU(),
    nn.Linear(50, 1)
)

# Forward pass
x = torch.randn(32, 10)
y = model(x)
loss = y.mean()

# Backward pass
loss.backward()

# Check gradients
for name, param in model.named_parameters():
    if param.grad is not None:
        print(f"{name}: grad mean = {param.grad.abs().mean():.6f}")
    else:
        print(f"{name}: NO GRADIENT!")
\end{lstlisting}

Common issues:
\begin{itemize}
    \item \textbf{All zeros:} Dead ReLUs, wrong loss function
    \item \textbf{None:} Part of model disconnected from loss
    \item \textbf{Very large:} Exploding gradients (lower learning rate, add gradient clipping)
    \item \textbf{Very small:} Vanishing gradients (different architecture, normalization)
\end{itemize}

\clearpage
\subsubsection{Custom Autograd Functions}

For operations PyTorch doesn't support, or to override gradients:

\begin{lstlisting}
class MySquare(torch.autograd.Function):
    @staticmethod
    def forward(ctx, input):
        """
        Forward pass: compute output and save what's needed for backward.
        ctx is a context object for saving information.
        """
        ctx.save_for_backward(input)
        return input ** 2
    
    @staticmethod
    def backward(ctx, grad_output):
        """
        Backward pass: compute gradient of loss w.r.t. input.
        grad_output is the gradient of loss w.r.t. output.
        Return gradient w.r.t. each input (input only here).
        """
        input, = ctx.saved_tensors
        grad_input = grad_output * 2 * input
        return grad_input

# Use it
x = torch.tensor([1.0, 2.0, 3.0], requires_grad=True)
y = MySquare.apply(x)  # Note: .apply() not ()
z = y.sum()
z.backward()

print(x.grad)  # [2., 4., 6.] = 2*x
\end{lstlisting}

We'll use this for the exercises!

\subsection{Exercises}

\begin{exercise}[2.1: Basic Gradient Computation - $\bigstar\bigstar$]
\textbf{Goal:} Get comfortable with \texttt{.backward()} and \texttt{.grad}.

\begin{enumerate}
    \item Create tensors $a=3$, $b=4$ with \texttt{requires\_grad=True}
    \item Compute $c = a^2 + b^2$
    \item Call \texttt{c.backward()} and verify:
    \begin{itemize}
        \item \texttt{a.grad} = $2a = 6$
        \item \texttt{b.grad} = $2b = 8$
    \end{itemize}
    \item Now compute $d = 2a + 3b$ and call \texttt{d.backward()}
    \item What are \texttt{a.grad} and \texttt{b.grad} now? Why?
\end{enumerate}

\textbf{Hint:} Remember that gradients accumulate!
\end{exercise}

\begin{exercise}[2.2: Matrix Gradients - $\bigstar\bigstar$]
\textbf{Goal:} Understand gradients for matrix operations.

\begin{enumerate}
    \item Create matrix $W$ of shape (3, 4) with random values, \texttt{requires\_grad=True}
    \item Create input $x$ of shape (4,) with random values (no gradients needed)
    \item Compute $y = Wx$ (matrix-vector product)
    \item Compute $L = ||y||^2$ (sum of squares)
    \item Call \texttt{L.backward()} and check the shape of \texttt{W.grad}
    \item Verify it has the same shape as $W$
\end{enumerate}

\textbf{Starter code:}
\begin{lstlisting}
import torch
torch.manual_seed(42)

W = torch.randn(3, 4, requires_grad=True)
x = torch.randn(4)

# Your code here
\end{lstlisting}
\end{exercise}

\begin{exercise}[2.3: Gradient Accumulation - $\bigstar\bigstar\bigstar$]
\textbf{Goal:} Master gradient zeroing in training loops.

Implement a mini training loop (without a real model):
\begin{enumerate}
    \item Create a parameter \texttt{w} initialized to 5.0
    \item For 10 iterations:
    \begin{itemize}
        \item Compute $\text{loss} = (w - 2)^2$ (we want w to reach 2)
        \item Compute gradient
        \item Update: $w = w - 0.1 \cdot \text{grad}$
        \item \textbf{Zero the gradient}
    \end{itemize}
    \item Print the final value of \texttt{w} (should be close to 2)
\end{enumerate}

\textbf{Hint:} Use \texttt{with torch.no\_grad():} for the update step.
\end{exercise}

\begin{exercise}[2.4: Gradient Detaching - $\bigstar\bigstar\bigstar$]
\textbf{Goal:} Understand when and why to detach gradients.

\begin{enumerate}
    \item Create $x = [1, 2, 3]$ with gradients
    \item Compute $y = x^2$
    \item Detach $y$ to create $y_{\text{detach}}$
    \item Compute $z_1 = y.sum()$ and $z_2 = y_{\text{detach}}.sum()$
    \item Try to call \texttt{z1.backward()} - what happens?
    \item Try to call \texttt{z2.backward()} - what happens?
    \item Why would detaching be useful in training?
\end{enumerate}
\end{exercise}

\begin{exercise}[2.5: Higher-Order Derivatives - $\bigstar\bigstar\bigstar\bigstar$]
\textbf{Goal:} Compute second derivatives (useful for some scientific ML methods).

\begin{enumerate}
    \item Create $x = 2.0$ with gradients
    \item Compute $y = \sin(x)$
    \item Use \texttt{torch.autograd.grad()} to compute $\frac{dy}{dx}$ with \texttt{create\_graph=True}
    \item Compute the second derivative $\frac{d^2y}{dx^2}$
    \item Verify analytically: $y=\sin(x) \Rightarrow y'=\cos(x) \Rightarrow y''=-\sin(x)$
    \item At $x=2$: $y''=-\sin(2) \approx -0.909$
\end{enumerate}

\textbf{Starter code:}
\begin{lstlisting}
import torch

x = torch.tensor(2.0, requires_grad=True)
y = torch.sin(x)

# First derivative
dy_dx = torch.autograd.grad(y, x, create_graph=True)[0]

# Your code for second derivative
\end{lstlisting}
\end{exercise}

\begin{exercise}[2.6: Custom Autograd Function - $\bigstar\bigstar\bigstar\bigstar$]
\textbf{Goal:} Implement a custom operation with manual gradient.

Implement a custom \textbf{sigmoid} function:
\begin{enumerate}
    \item Forward: $\sigma(x) = \frac{1}{1 + e^{-x}}$
    \item Backward: $\frac{d\sigma}{dx} = \sigma(x)(1 - \sigma(x))$
    \item Inherit from \texttt{torch.autograd.Function}
    \item Implement \texttt{forward()} and \texttt{backward()} static methods
    \item Test it on $x = [0, 1, 2]$
    \item Verify gradient using \texttt{torch.autograd.gradcheck}
\end{enumerate}

\textbf{Hint:} In \texttt{backward()}, you'll need the output of sigmoid. You can either:
\begin{itemize}
    \item Save the output in \texttt{forward} using \texttt{ctx.save\_for\_backward}
    \item Or recompute it from the saved input
\end{itemize}

\textbf{Starter code:}
\begin{lstlisting}
class MySigmoid(torch.autograd.Function):
    @staticmethod
    def forward(ctx, input):
        # Your code here
        pass
    
    @staticmethod
    def backward(ctx, grad_output):
        # Your code here
        pass

# Test
x = torch.tensor([0.0, 1.0, 2.0], requires_grad=True)
y = MySigmoid.apply(x)
\end{lstlisting}
\end{exercise}

\clearpage
% =============================================
% SECTION 3: BUILDING MODELS WITH NN.MODULE
% =============================================

\section{Building Models with \texttt{nn.Module}}

\subsection{Introduction: Why \texttt{nn.Module}?}

In Section 2, we manually created tensors with \texttt{requires\_grad=True} and computed gradients. While this works for simple cases, real neural networks have hundreds of layers and millions of parameters. We need a way to organize this complexity.

\texttt{nn.Module} is PyTorch's base class for all neural network components. It provides:

\begin{itemize}
    \item \textbf{Automatic parameter management:} Tracks all learnable parameters
    \item \textbf{GPU/CPU transfer:} Move entire models between devices with one command
    \item \textbf{Training/eval modes:} Control behavior of layers like dropout and batch norm
    \item \textbf{State management:} Save and load model weights
    \item \textbf{Composability:} Build complex models from simple modules
\end{itemize}

Every neural network layer (\texttt{nn.Linear}, \texttt{nn.Conv2d}, etc.) and every model you build inherits from \texttt{nn.Module}.

\subsection{Theory: The Module Hierarchy}

A \texttt{nn.Module} is a container that can hold:
\begin{itemize}
    \item \textbf{Parameters:} Learnable tensors (weights, biases)
    \item \textbf{Buffers:} Non-learnable tensors (running statistics, constants)
    \item \textbf{Sub-modules:} Other \texttt{nn.Module} objects
\end{itemize}

\textbf{Example hierarchy:}

\begin{verbatim}
MyModel (nn.Module)
├── layer1 (nn.Linear)
│   ├── weight (Parameter)
│   └── bias (Parameter)
├── layer2 (nn.Linear)
│   ├── weight (Parameter)
│   └── bias (Parameter)
└── activation (nn.ReLU)
\end{verbatim}

When you call \texttt{model.parameters()}, PyTorch automatically finds all parameters in the entire tree.

\begin{theorybox}[Key Principle: The Forward Method]
Every \texttt{nn.Module} must implement a \texttt{forward()} method that defines the computation. You never call \texttt{forward()} directly—instead, you call the module as a function: \texttt{output = model(input)}. This triggers PyTorch's hooks and tracking mechanisms.
\end{theorybox}

\subsection{Implementation: Creating Your First Module}

\subsubsection{A Simple Linear Model}

Let's build a simple linear regression model from scratch:

\begin{lstlisting}
import torch
import torch.nn as nn

class LinearRegression(nn.Module):
    def __init__(self, input_dim, output_dim):
        """
        Initialize the module.
        Always call super().__init__() first!
        """
        super().__init__()  # Initialize parent class
        
        # Define learnable parameters
        self.linear = nn.Linear(input_dim, output_dim)
    
    def forward(self, x):
        """
        Define the forward pass.
        x: input tensor of shape (batch_size, input_dim)
        returns: output tensor of shape (batch_size, output_dim)
        """
        return self.linear(x)

# Create the model
model = LinearRegression(input_dim=10, output_dim=1)

# Use the model
x = torch.randn(32, 10)  # Batch of 32 samples
y = model(x)             # Call the model (invokes forward())
print(y.shape)           # torch.Size([32, 1])
\end{lstlisting}

\begin{pytorchtip}[Always Call super().\_\_init\_\_()]
Forgetting \texttt{super().\_\_init\_\_()} is a common mistake. Without it, PyTorch can't track parameters properly. Always put it as the first line of \texttt{\_\_init\_\_()}.
\end{pytorchtip}

\subsubsection{Accessing Parameters}

\begin{lstlisting}
model = LinearRegression(10, 1)

# Get all parameters
for name, param in model.named_parameters():
    print(f"{name}: {param.shape}")
# Output:
# linear.weight: torch.Size([1, 10])
# linear.bias: torch.Size([1])

# Count parameters
total_params = sum(p.numel() for p in model.parameters())
print(f"Total parameters: {total_params}")  # 11 (10 weights + 1 bias)

# Access specific parameters
print(model.linear.weight)  # The weight matrix
print(model.linear.bias)    # The bias vector
\end{lstlisting}

\subsubsection{Using \texttt{nn.Sequential}}

For simple feed-forward architectures, \texttt{nn.Sequential} is convenient:

\begin{lstlisting}
# Method 1: Pass modules as arguments
model = nn.Sequential(
    nn.Linear(10, 50),
    nn.ReLU(),
    nn.Linear(50, 20),
    nn.ReLU(),
    nn.Linear(20, 1)
)

# Method 2: Use OrderedDict for named layers
from collections import OrderedDict

model = nn.Sequential(OrderedDict([
    ('fc1', nn.Linear(10, 50)),
    ('relu1', nn.ReLU()),
    ('fc2', nn.Linear(50, 20)),
    ('relu2', nn.ReLU()),
    ('output', nn.Linear(20, 1))
]))

# Use it
x = torch.randn(32, 10)
y = model(x)

# Access layers
print(model[0])  # First layer (fc1)
print(model.fc1)  # Same, if using OrderedDict
\end{lstlisting}

\textbf{When to use Sequential:}
\begin{itemize}
    \item Simple feed-forward architecture
    \item No branching or skip connections
    \item No custom logic in forward pass
\end{itemize}

\textbf{When to use custom Module:}
\begin{itemize}
    \item Complex architectures (ResNets, U-Nets)
    \item Multiple inputs/outputs
    \item Custom forward logic (if statements, loops)
    \item Need to access intermediate values
\end{itemize}

\clearpage
\subsubsection{Custom Modules with Multiple Layers}

\begin{lstlisting}
class MLP(nn.Module):
    """Multi-Layer Perceptron with configurable hidden layers."""
    
    def __init__(self, input_dim, hidden_dims, output_dim):
        """
        Args:
            input_dim: Input feature dimension
            hidden_dims: List of hidden layer dimensions
            output_dim: Output dimension
        """
        super().__init__()
        
        # Build layers dynamically
        layers = []
        prev_dim = input_dim
        
        for hidden_dim in hidden_dims:
            layers.append(nn.Linear(prev_dim, hidden_dim))
            layers.append(nn.ReLU())
            prev_dim = hidden_dim
        
        # Output layer (no activation)
        layers.append(nn.Linear(prev_dim, output_dim))
        
        # Store as Sequential
        self.network = nn.Sequential(*layers)
    
    def forward(self, x):
        return self.network(x)

# Create a 3-layer network
model = MLP(input_dim=10, hidden_dims=[50, 30], output_dim=1)

# The architecture is: 10 -> 50 -> 30 -> 1
# With ReLU after first two layers
\end{lstlisting}

\subsubsection{Parameters vs Buffers}

\textbf{Parameters:} Learnable tensors (updated by optimizer)

\textbf{Buffers:} Non-learnable tensors (not updated by optimizer, but saved with model)

\begin{lstlisting}
class MyModule(nn.Module):
    def __init__(self):
        super().__init__()
        
        # Learnable parameter
        self.weight = nn.Parameter(torch.randn(10, 10))
        
        # Non-learnable buffer (e.g., running statistics)
        self.register_buffer('running_mean', torch.zeros(10))
    
    def forward(self, x):
        # Use both in computation
        return x @ self.weight + self.running_mean

model = MyModule()

# Parameters (will be updated by optimizer)
print(len(list(model.parameters())))  # 1 (weight only)

# Buffers (won't be updated by optimizer)
print(len(list(model.buffers())))     # 1 (running_mean)

# Both are saved in state_dict
print(model.state_dict().keys())
# dict_keys(['weight', 'running_mean'])
\end{lstlisting}

\textbf{When to use buffers:}
\begin{itemize}
    \item Running statistics in BatchNorm
    \item Fixed embeddings
    \item Constants that should be saved with the model
    \item Anything that needs to move with the model to GPU but shouldn't be trained
\end{itemize}

\subsubsection{Training vs Evaluation Mode}

Some layers behave differently during training and evaluation:

\begin{lstlisting}
model = nn.Sequential(
    nn.Linear(10, 50),
    nn.ReLU(),
    nn.Dropout(0.5),  # Randomly drops 50% of values during training
    nn.Linear(50, 1)
)

# Training mode (default)
model.train()
x = torch.randn(5, 10)
print(model(x))  # Dropout is active

# Evaluation mode
model.eval()
print(model(x))  # Dropout is disabled

# Check current mode
print(model.training)  # False when in eval mode
\end{lstlisting}

\begin{warningbox}[Always Set Mode Correctly]
Forgetting to call \texttt{model.eval()} during inference can lead to incorrect results (dropout still active, batch norm using batch statistics instead of running statistics). Always:
\begin{lstlisting}
model.train()  # Before training
model.eval()   # Before evaluation/inference
\end{lstlisting}
\end{warningbox}

\clearpage
\subsubsection{Moving Models to GPU}

\begin{lstlisting}
device = torch.device('cuda' if torch.cuda.is_available() else 'cpu')

# Create model
model = MLP(10, [50, 30], 1)

# Move to GPU
model = model.to(device)

# All parameters and buffers are now on GPU
print(next(model.parameters()).device)  # cuda:0

# Input must also be on GPU
x = torch.randn(32, 10, device=device)
y = model(x)  # Works! Both model and input on same device

# Moving back to CPU
model = model.to('cpu')
\end{lstlisting}

\begin{pytorchtip}[Device Management Pattern]
Common pattern for device-agnostic code:
\begin{lstlisting}
device = torch.device('cuda' if torch.cuda.is_available() else 'cpu')
model = model.to(device)

for batch in dataloader:
    x, y = batch
    x, y = x.to(device), y.to(device)
    output = model(x)
    # ... rest of training loop
\end{lstlisting}
\end{pytorchtip}

\subsubsection{Saving and Loading Models}

\begin{lstlisting}
# Save model weights
model = MLP(10, [50, 30], 1)
torch.save(model.state_dict(), 'model_weights.pth')

# Load weights into new model
new_model = MLP(10, [50, 30], 1)  # Must have same architecture!
new_model.load_state_dict(torch.load('model_weights.pth'))

# Save entire model (architecture + weights)
torch.save(model, 'entire_model.pth')
loaded_model = torch.load('entire_model.pth')

# For inference, set to eval mode
loaded_model.eval()
\end{lstlisting}

\textbf{Best practice:} Save \texttt{state\_dict()} rather than entire model. This is more flexible and portable.

\begin{lstlisting}
# Better: Save with additional info
checkpoint = {
    'model_state_dict': model.state_dict(),
    'optimizer_state_dict': optimizer.state_dict(),
    'epoch': epoch,
    'loss': loss,
}
torch.save(checkpoint, 'checkpoint.pth')

# Load
checkpoint = torch.load('checkpoint.pth')
model.load_state_dict(checkpoint['model_state_dict'])
optimizer.load_state_dict(checkpoint['optimizer_state_dict'])
\end{lstlisting}

\subsubsection{Parameter Initialization}

PyTorch layers have default initialization, but you often want custom initialization:

\begin{lstlisting}
import torch.nn.init as init

class InitializedMLP(nn.Module):
    def __init__(self, input_dim, hidden_dim, output_dim):
        super().__init__()
        self.fc1 = nn.Linear(input_dim, hidden_dim)
        self.fc2 = nn.Linear(hidden_dim, output_dim)
        self.relu = nn.ReLU()
        
        # Custom initialization
        self._initialize_weights()
    
    def _initialize_weights(self):
        # Xavier initialization for fc1
        init.xavier_uniform_(self.fc1.weight)
        init.zeros_(self.fc1.bias)
        
        # He initialization for fc2 (better for ReLU)
        init.kaiming_normal_(self.fc2.weight, nonlinearity='relu')
        init.zeros_(self.fc2.bias)
    
    def forward(self, x):
        x = self.relu(self.fc1(x))
        x = self.fc2(x)
        return x
\end{lstlisting}

\textbf{Common initialization strategies:}

\begin{table}[h]
\centering
\begin{tabular}{lll}
\toprule
\textbf{Method} & \textbf{Use Case} & \textbf{Function} \\
\midrule
Xavier/Glorot & Sigmoid/Tanh & \texttt{init.xavier\_uniform\_} \\
He/Kaiming & ReLU & \texttt{init.kaiming\_normal\_} \\
Zeros & Biases & \texttt{init.zeros\_} \\
Constant & Set to specific value & \texttt{init.constant\_} \\
Normal & Custom std dev & \texttt{init.normal\_} \\
Orthogonal & RNNs & \texttt{init.orthogonal\_} \\
\bottomrule
\end{tabular}
\caption{Common weight initialization methods}
\end{table}

\begin{pytorchtip}[Why Initialization Matters]
Poor initialization can cause:
\begin{itemize}
    \item \textbf{Vanishing gradients:} Weights too small, gradients die out
    \item \textbf{Exploding gradients:} Weights too large, gradients explode
    \item \textbf{Slow convergence:} Network takes forever to learn
    \item \textbf{Dead neurons:} ReLUs output zero and never recover
\end{itemize}
Use Xavier for sigmoid/tanh activations, He for ReLU activations.
\end{pytorchtip}

\clearpage
\subsubsection{Module Hooks for Debugging}

Hooks let you inspect intermediate values during forward/backward pass:

\begin{lstlisting}
model = nn.Sequential(
    nn.Linear(10, 50),
    nn.ReLU(),
    nn.Linear(50, 1)
)

# Forward hook: called after forward pass
def forward_hook(module, input, output):
    print(f"Forward: {output.shape}")

# Register hook on second layer
handle = model[2].register_forward_hook(forward_hook)

# Forward pass
x = torch.randn(5, 10)
y = model(x)  # Prints: Forward: torch.Size([5, 1])

# Remove hook when done
handle.remove()

# Backward hook: called during backward pass
def backward_hook(module, grad_input, grad_output):
    print(f"Grad output shape: {grad_output[0].shape}")

handle = model[2].register_backward_hook(backward_hook)
y = model(x)
y.sum().backward()  # Prints grad shape

handle.remove()
\end{lstlisting}

\textbf{Use cases for hooks:}
\begin{itemize}
    \item Debugging shape mismatches
    \item Visualizing activations
    \item Gradient flow analysis
    \item Feature extraction from intermediate layers
\end{itemize}

\subsection{Implementation: Practical Patterns}

\subsubsection{Freezing Layers}

Sometimes you want to freeze part of the model (transfer learning):

\begin{lstlisting}
# Freeze all parameters
for param in model.parameters():
    param.requires_grad = False

# Freeze specific layer
for param in model.layer1.parameters():
    param.requires_grad = False

# Only train the last layer
model = nn.Sequential(
    nn.Linear(10, 50),
    nn.ReLU(),
    nn.Linear(50, 1)
)

# Freeze first two layers
for param in model[:2].parameters():
    param.requires_grad = False

# Now only model[2].parameters() will be updated
optimizer = torch.optim.Adam(
    filter(lambda p: p.requires_grad, model.parameters()),
    lr=0.001
)
\end{lstlisting}

\subsubsection{Multiple Inputs/Outputs}

\begin{lstlisting}
class MultiInputModel(nn.Module):
    def __init__(self):
        super().__init__()
        self.branch1 = nn.Linear(10, 50)
        self.branch2 = nn.Linear(5, 50)
        self.combine = nn.Linear(100, 1)
    
    def forward(self, x1, x2):
        """
        x1: shape (batch, 10)
        x2: shape (batch, 5)
        """
        out1 = self.branch1(x1)
        out2 = self.branch2(x2)
        combined = torch.cat([out1, out2], dim=1)  # (batch, 100)
        return self.combine(combined)

model = MultiInputModel()
x1 = torch.randn(32, 10)
x2 = torch.randn(32, 5)
y = model(x1, x2)
\end{lstlisting}

\subsubsection{Custom Layer with Parameters}

\begin{lstlisting}
class ScaledLinear(nn.Module):
    """Linear layer with learnable scaling factor."""
    
    def __init__(self, in_features, out_features):
        super().__init__()
        self.linear = nn.Linear(in_features, out_features)
        # Custom learnable parameter
        self.scale = nn.Parameter(torch.ones(1))
    
    def forward(self, x):
        return self.scale * self.linear(x)

layer = ScaledLinear(10, 5)

# Parameters include linear weights/bias AND scale
for name, param in layer.named_parameters():
    print(f"{name}: {param.shape}")
# linear.weight: torch.Size([5, 10])
# linear.bias: torch.Size([5])
# scale: torch.Size([1])
\end{lstlisting}

\clearpage
\subsection{Exercises}

\begin{exercise}[3.1: Simple Module - $\bigstar\bigstar$]
\textbf{Goal:} Create your first custom module.

Implement a \texttt{TwoLayerNet} module:
\begin{enumerate}
    \item Two linear layers: input\_dim → hidden\_dim → output\_dim
    \item ReLU activation between them
    \item No activation after the output layer
    \item Test with input\_dim=20, hidden\_dim=50, output\_dim=10
    \item Verify the output shape for a batch of 16 samples
\end{enumerate}

\textbf{Starter code:}
\begin{lstlisting}
import torch
import torch.nn as nn

class TwoLayerNet(nn.Module):
    def __init__(self, input_dim, hidden_dim, output_dim):
        super().__init__()
        # Your code here
    
    def forward(self, x):
        # Your code here
        pass

# Test
model = TwoLayerNet(20, 50, 10)
x = torch.randn(16, 20)
y = model(x)
print(y.shape)  # Should be torch.Size([16, 10])
\end{lstlisting}
\end{exercise}

\begin{exercise}[3.2: Parameter Counting - $\bigstar\bigstar$]
\textbf{Goal:} Understand model size.

\begin{enumerate}
    \item Create a model: 100 → 500 → 200 → 50 → 10 (all linear, ReLU between)
    \item Count total parameters programmatically
    \item Calculate manually and verify:
    \begin{itemize}
        \item Layer 1: $100 \times 500 + 500 = 50{,}500$
        \item Layer 2: $500 \times 200 + 200 = 100{,}200$
        \item ... (continue)
    \end{itemize}
    \item Print the number of parameters in each layer
\end{enumerate}

\textbf{Hint:} Use \texttt{param.numel()} to count elements in each parameter.
\end{exercise}

\begin{exercise}[3.3: Sequential vs Custom - $\bigstar\bigstar\bigstar$]
\textbf{Goal:} Compare two approaches.

Build the same network two ways:
\begin{enumerate}
    \item Using \texttt{nn.Sequential}
    \item Using a custom \texttt{nn.Module} with explicit layers
    \item Both should have: 10 → 50 → 30 → 1 with ReLU activations
    \item Load the same random seed before creating each
    \item Verify they produce identical outputs for the same input
\end{enumerate}

\textbf{Challenge:} Can you implement a method to copy weights from one to the other?
\end{exercise}

\begin{exercise}[3.4: Custom Initialization - $\bigstar\bigstar\bigstar$]
\textbf{Goal:} Implement custom weight initialization.

\begin{enumerate}
    \item Create a 3-layer MLP
    \item Initialize all weights with He initialization
    \item Initialize all biases to zero
    \item Create another identical model with default initialization
    \item Train both on a simple task (e.g., fit $y = x^2$ for x in [-1, 1])
    \item Compare convergence speed (plot losses)
\end{enumerate}

\textbf{Hint:} Use \texttt{nn.init.kaiming\_normal\_()} for He initialization.
\end{exercise}

\begin{exercise}[3.5: Save and Load - $\bigstar\bigstar\bigstar$]
\textbf{Goal:} Practice model persistence.

\begin{enumerate}
    \item Create a model and train it for a few steps (any simple task)
    \item Save the state dict
    \item Create a new model instance
    \item Load the weights
    \item Verify both models produce identical outputs
    \item Try saving/loading with \texttt{torch.save(model, ...)} and compare
\end{enumerate}
\end{exercise}

\begin{exercise}[3.6: Multi-Input Network - $\bigstar\bigstar\bigstar\bigstar$]
\textbf{Goal:} Build a more complex architecture.

Implement a model that takes two inputs and combines them:
\begin{enumerate}
    \item Input 1: shape (batch, 20)
    \item Input 2: shape (batch, 10)
    \item Process each through separate 2-layer networks
    \item Concatenate the results
    \item Pass through a final layer to get output shape (batch, 1)
    \item Test with random inputs
\end{enumerate}

\textbf{Extra challenge:} Add skip connections (concatenate the original inputs with processed features before the final layer).
\end{exercise}

\begin{exercise}[3.7: Frozen Layers - $\bigstar\bigstar\bigstar\bigstar$]
\textbf{Goal:} Practice transfer learning patterns.

\begin{enumerate}
    \item Create a 4-layer network
    \item "Pretrain" it on a simple task
    \item Freeze the first two layers (\texttt{requires\_grad=False})
    \item "Fine-tune" on a different task (only last two layers train)
    \item Verify the first two layers' weights don't change
    \item Compare with training all layers from scratch
\end{enumerate}

\textbf{Task idea:} First task: classify points in quadrants, second task: classify by distance from origin.
\end{exercise}

\clearpage
% =============================================
% SECTION 4: THE TRAINING LOOP & OPTIMIZERS
% =============================================

\section{The Training Loop \& Optimizers}

\subsection{Introduction: The Training Process}

We now have all the pieces:
\begin{itemize}
    \item \textbf{Tensors:} For data representation
    \item \textbf{Autograd:} For computing gradients
    \item \textbf{Modules:} For building models
\end{itemize}

Now we need to \textbf{train} the model—adjust its parameters to minimize a loss function. This involves:
\begin{enumerate}
    \item Forward pass: Compute predictions
    \item Loss computation: Measure how wrong we are
    \item Backward pass: Compute gradients
    \item Parameter update: Move in the direction that reduces loss
    \item Repeat!
\end{enumerate}

\subsection{Theory: Gradient Descent and Optimization}

\subsubsection{The Optimization Problem}

Training is an optimization problem. Given:
\begin{itemize}
    \item Model $f_\theta$ with parameters $\theta$
    \item Dataset $\{(x_i, y_i)\}_{i=1}^N$
    \item Loss function $\mathcal{L}$
\end{itemize}

We want to find:
\[
\theta^* = \arg\min_\theta \frac{1}{N} \sum_{i=1}^N \mathcal{L}(f_\theta(x_i), y_i)
\]

\subsubsection{Gradient Descent}

The basic update rule:
\[
\theta_{t+1} = \theta_t - \eta \nabla_\theta \mathcal{L}
\]

where $\eta$ is the \textbf{learning rate}.

\textbf{Variants:}
\begin{itemize}
    \item \textbf{Batch Gradient Descent:} Use entire dataset (slow, memory intensive)
    \item \textbf{Stochastic Gradient Descent (SGD):} Use one sample (fast, noisy)
    \item \textbf{Mini-Batch SGD:} Use a batch of samples (sweet spot)
\end{itemize}

In practice, we always use mini-batch SGD (typically 32-256 samples per batch).

\subsubsection{Learning Rate}

The learning rate $\eta$ is the most important hyperparameter:

\begin{itemize}
    \item \textbf{Too large:} Training diverges (loss increases)
    \item \textbf{Too small:} Training is very slow
    \item \textbf{Just right:} Steady convergence
\end{itemize}

Typical ranges: $10^{-4}$ to $10^{-2}$ for Adam, $10^{-3}$ to $10^{-1}$ for SGD.

\begin{theorybox}[Momentum and Adaptive Learning Rates]
Modern optimizers go beyond basic gradient descent:
\begin{itemize}
    \item \textbf{Momentum:} Accumulate gradients over time (helps escape local minima)
    \item \textbf{Adaptive rates:} Different learning rate for each parameter (Adam, RMSprop)
\end{itemize}
These techniques make training more stable and faster.
\end{theorybox}

\subsection{Implementation: The Standard Training Loop}

\subsubsection{Complete Training Example}

Here's the canonical PyTorch training loop:

\begin{lstlisting}
import torch
import torch.nn as nn
import torch.optim as optim

# 1. CREATE MODEL
model = nn.Sequential(
    nn.Linear(10, 50),
    nn.ReLU(),
    nn.Linear(50, 1)
)

# 2. DEFINE LOSS AND OPTIMIZER
criterion = nn.MSELoss()
optimizer = optim.Adam(model.parameters(), lr=0.001)

# 3. CREATE DUMMY DATA (replace with real DataLoader)
X_train = torch.randn(1000, 10)
y_train = torch.randn(1000, 1)

# 4. TRAINING LOOP
num_epochs = 100
batch_size = 32

for epoch in range(num_epochs):
    # Shuffle data (simplified; use DataLoader in practice)
    indices = torch.randperm(len(X_train))
    
    epoch_loss = 0.0
    for i in range(0, len(X_train), batch_size):
        # Get batch
        batch_indices = indices[i:i+batch_size]
        X_batch = X_train[batch_indices]
        y_batch = y_train[batch_indices]
        
        # FORWARD PASS
        predictions = model(X_batch)
        loss = criterion(predictions, y_batch)
        
        # BACKWARD PASS
        optimizer.zero_grad()  # Clear old gradients
        loss.backward()        # Compute new gradients
        
        # UPDATE PARAMETERS
        optimizer.step()
        
        epoch_loss += loss.item()
    
    # Print progress
    if (epoch + 1) % 10 == 0:
        avg_loss = epoch_loss / (len(X_train) / batch_size)
        print(f'Epoch [{epoch+1}/{num_epochs}], Loss: {avg_loss:.4f}')
\end{lstlisting}

\begin{pytorchtip}[The Five Steps of Training]
Every training loop has these steps:
\begin{enumerate}
    \item \texttt{optimizer.zero\_grad()}: Clear old gradients
    \item \texttt{output = model(input)}: Forward pass
    \item \texttt{loss = criterion(output, target)}: Compute loss
    \item \texttt{loss.backward()}: Compute gradients
    \item \texttt{optimizer.step()}: Update parameters
\end{enumerate}
Always in this order!
\end{pytorchtip}

\clearpage
\subsubsection{Loss Functions}

Different tasks require different loss functions:

\textbf{Regression (continuous output):}

\begin{lstlisting}
# Mean Squared Error (L2 loss)
criterion = nn.MSELoss()

# Mean Absolute Error (L1 loss, more robust to outliers)
criterion = nn.L1Loss()

# Smooth L1 (Huber loss, combines L1 and L2)
criterion = nn.SmoothL1Loss()
\end{lstlisting}

\textbf{Binary Classification (0 or 1):}

\begin{lstlisting}
# Binary Cross Entropy (output must be in [0,1], use sigmoid)
criterion = nn.BCELoss()

# BCE with built-in sigmoid (more numerically stable)
criterion = nn.BCEWithLogitsLoss()  # Preferred!

# Example
model = nn.Sequential(nn.Linear(10, 1))  # No sigmoid in model
criterion = nn.BCEWithLogitsLoss()
logits = model(x)  # Raw outputs
loss = criterion(logits, targets)  # Applies sigmoid internally
\end{lstlisting}

\textbf{Multi-Class Classification (one of K classes):}

\begin{lstlisting}
# Cross Entropy Loss (combines LogSoftmax + NLLLoss)
criterion = nn.CrossEntropyLoss()

# Example
model = nn.Sequential(nn.Linear(10, 5))  # 5 classes, no softmax!
criterion = nn.CrossEntropyLoss()

logits = model(x)  # Shape: (batch, 5), raw scores
targets = torch.tensor([0, 2, 1, 4, 3])  # Class indices

loss = criterion(logits, targets)  # Applies softmax internally
\end{lstlisting}

\begin{warningbox}[Don't Apply Softmax Before CrossEntropyLoss]
\texttt{nn.CrossEntropyLoss} expects \textbf{raw logits}, not probabilities. It applies softmax internally for numerical stability. Applying softmax yourself will give wrong results!

\textbf{WRONG:}
\begin{lstlisting}
output = torch.softmax(model(x), dim=1)
loss = nn.CrossEntropyLoss()(output, targets)  # INCORRECT!
\end{lstlisting}

\textbf{CORRECT:}
\begin{lstlisting}
output = model(x)  # Raw logits
loss = nn.CrossEntropyLoss()(output, targets)  # Correct!
\end{lstlisting}
\end{warningbox}

\textbf{Multi-Label Classification (multiple classes can be active):}

\begin{lstlisting}
# Use BCE with Logits for each class independently
criterion = nn.BCEWithLogitsLoss()

model = nn.Sequential(nn.Linear(10, 5))  # 5 binary outputs
logits = model(x)  # Shape: (batch, 5)
targets = torch.tensor([[1, 0, 1, 0, 0],   # Multiple 1s allowed
                        [0, 1, 1, 1, 0]])  # Shape: (batch, 5)

loss = criterion(logits, targets.float())
\end{lstlisting}

\subsubsection{Optimizers}

PyTorch provides many optimizers. Here are the most common:

\textbf{Stochastic Gradient Descent (SGD):}

\begin{lstlisting}
# Basic SGD
optimizer = optim.SGD(model.parameters(), lr=0.01)

# SGD with momentum (recommended)
optimizer = optim.SGD(model.parameters(), lr=0.01, momentum=0.9)

# SGD with momentum and weight decay (L2 regularization)
optimizer = optim.SGD(model.parameters(), lr=0.01, 
                      momentum=0.9, weight_decay=1e-4)
\end{lstlisting}

\textbf{Adam (Adaptive Moment Estimation):}

\begin{lstlisting}
# Adam: adaptive learning rates per parameter
optimizer = optim.Adam(model.parameters(), lr=0.001)

# Adam with weight decay
optimizer = optim.Adam(model.parameters(), lr=0.001, weight_decay=1e-4)
\end{lstlisting}

\textbf{AdamW (Adam with decoupled weight decay):}

\begin{lstlisting}
# AdamW: better weight decay than Adam
optimizer = optim.AdamW(model.parameters(), lr=0.001, weight_decay=0.01)
\end{lstlisting}

\begin{pytorchtip}[Which Optimizer to Use?]
\textbf{For most tasks:} Start with \texttt{Adam} or \texttt{AdamW} with lr=0.001

\textbf{Use Adam/AdamW when:}
\begin{itemize}
    \item You want fast initial progress
    \item You're prototyping
    \item You have many hyperparameters (adaptive rates help)
\end{itemize}

\textbf{Use SGD with momentum when:}
\begin{itemize}
    \item You need the absolute best final performance
    \item You're training CNNs for vision tasks
    \item You can afford to tune learning rate carefully
\end{itemize}

\textbf{General rule:} Adam for fast development, SGD for final tuning.
\end{pytorchtip}

\clearpage
\subsubsection{Learning Rate Schedules}

Instead of fixed learning rate, reduce it during training:

\begin{lstlisting}
import torch.optim.lr_scheduler as lr_scheduler

optimizer = optim.Adam(model.parameters(), lr=0.001)

# Step decay: reduce LR every step_size epochs
scheduler = lr_scheduler.StepLR(optimizer, step_size=30, gamma=0.1)

# Exponential decay
scheduler = lr_scheduler.ExponentialLR(optimizer, gamma=0.95)

# Reduce on plateau: reduce when loss stops improving
scheduler = lr_scheduler.ReduceLROnPlateau(optimizer, mode='min', 
                                           factor=0.5, patience=10)

# Cosine annealing: smooth decay to zero
scheduler = lr_scheduler.CosineAnnealingLR(optimizer, T_max=100)

# Usage in training loop
for epoch in range(num_epochs):
    train_one_epoch()
    
    # Update learning rate
    scheduler.step()  # For most schedulers
    
    # For ReduceLROnPlateau, pass validation loss
    # scheduler.step(val_loss)
    
    # Check current LR
    current_lr = optimizer.param_groups[0]['lr']
    print(f"Current LR: {current_lr}")
\end{lstlisting}

\textbf{When to use schedules:}
\begin{itemize}
    \item Long training runs (>100 epochs)
    \item When loss plateaus
    \item Fine-tuning pretrained models
\end{itemize}

\textbf{When not to use:}
\begin{itemize}
    \item Short training (<50 epochs)
    \item Already using Adam with good convergence
    \item Adds complexity without clear benefit
\end{itemize}

\subsubsection{Gradient Clipping}

Prevent exploding gradients by limiting their magnitude:

\begin{lstlisting}
# Training loop with gradient clipping
for epoch in range(num_epochs):
    for batch in dataloader:
        optimizer.zero_grad()
        
        output = model(batch)
        loss = criterion(output, targets)
        loss.backward()
        
        # Clip gradients by norm
        torch.nn.utils.clip_grad_norm_(model.parameters(), max_norm=1.0)
        
        optimizer.step()
\end{lstlisting}

\textbf{Two methods:}

\begin{lstlisting}
# Clip by norm (preferred): scale down if norm > max_norm
torch.nn.utils.clip_grad_norm_(model.parameters(), max_norm=1.0)

# Clip by value: clamp each gradient to [-clip_value, clip_value]
torch.nn.utils.clip_grad_value_(model.parameters(), clip_value=0.5)
\end{lstlisting}

\begin{pytorchtip}[When to Use Gradient Clipping]
\textbf{Always use for:}
\begin{itemize}
    \item RNNs and LSTMs (prone to exploding gradients)
    \item Transformers
    \item Any sequence model
\end{itemize}

\textbf{Often useful for:}
\begin{itemize}
    \item Deep networks (>50 layers)
    \item Training instability
\end{itemize}

\textbf{Typical values:} max\_norm=1.0 to 5.0
\end{pytorchtip}

\clearpage
\subsubsection{Validation Loop}

Always evaluate on held-out data:

\begin{lstlisting}
def train_epoch(model, dataloader, criterion, optimizer, device):
    model.train()  # Set to training mode
    total_loss = 0.0
    
    for batch_idx, (data, target) in enumerate(dataloader):
        data, target = data.to(device), target.to(device)
        
        optimizer.zero_grad()
        output = model(data)
        loss = criterion(output, target)
        loss.backward()
        optimizer.step()
        
        total_loss += loss.item()
    
    return total_loss / len(dataloader)

def validate(model, dataloader, criterion, device):
    model.eval()  # Set to evaluation mode
    total_loss = 0.0
    correct = 0
    total = 0
    
    with torch.no_grad():  # Disable gradient computation
        for data, target in dataloader:
            data, target = data.to(device), target.to(device)
            
            output = model(data)
            loss = criterion(output, target)
            total_loss += loss.item()
            
            # For classification: compute accuracy
            _, predicted = torch.max(output, 1)
            total += target.size(0)
            correct += (predicted == target).sum().item()
    
    avg_loss = total_loss / len(dataloader)
    accuracy = 100 * correct / total
    return avg_loss, accuracy

# Main training loop
for epoch in range(num_epochs):
    train_loss = train_epoch(model, train_loader, criterion, 
                            optimizer, device)
    val_loss, val_acc = validate(model, val_loader, criterion, device)
    
    print(f'Epoch {epoch+1}: Train Loss={train_loss:.4f}, '
          f'Val Loss={val_loss:.4f}, Val Acc={val_acc:.2f}%')
\end{lstlisting}

\begin{warningbox}[Don't Forget model.eval() and torch.no\_grad()]
When validating:
\begin{itemize}
    \item \texttt{model.eval()}: Disables dropout, uses running stats for batch norm
    \item \texttt{torch.no\_grad()}: Saves memory and computation
\end{itemize}

Forgetting these can lead to:
\begin{itemize}
    \item Incorrect validation metrics
    \item Out of memory errors
    \item Slower evaluation
\end{itemize}
\end{warningbox}

\subsubsection{Early Stopping}

Stop training when validation loss stops improving:

\begin{lstlisting}
class EarlyStopping:
    def __init__(self, patience=10, min_delta=0):
        """
        patience: how many epochs to wait after last improvement
        min_delta: minimum change to qualify as improvement
        """
        self.patience = patience
        self.min_delta = min_delta
        self.counter = 0
        self.best_loss = None
        self.early_stop = False
    
    def __call__(self, val_loss):
        if self.best_loss is None:
            self.best_loss = val_loss
        elif val_loss > self.best_loss - self.min_delta:
            self.counter += 1
            if self.counter >= self.patience:
                self.early_stop = True
        else:
            self.best_loss = val_loss
            self.counter = 0

# Usage
early_stopping = EarlyStopping(patience=10)

for epoch in range(num_epochs):
    train_loss = train_epoch(...)
    val_loss, val_acc = validate(...)
    
    early_stopping(val_loss)
    if early_stopping.early_stop:
        print(f"Early stopping at epoch {epoch+1}")
        break
\end{lstlisting}

\clearpage
\subsection{Implementation: Debugging Training}

\subsubsection{When Training Goes Wrong}

\textbf{Problem: Loss is NaN}

\begin{lstlisting}
# Check for NaN
if torch.isnan(loss):
    print("NaN detected!")
    print(f"Input contains NaN: {torch.isnan(data).any()}")
    print(f"Output contains NaN: {torch.isnan(output).any()}")
    
    # Check gradients
    for name, param in model.named_parameters():
        if param.grad is not None:
            print(f"{name} grad: {param.grad.abs().max()}")
\end{lstlisting}

\textbf{Common causes:}
\begin{itemize}
    \item Learning rate too high
    \item Numerical instability (use \texttt{BCEWithLogitsLoss}, not \texttt{BCELoss})
    \item Division by zero
    \item Log of zero or negative number
\end{itemize}

\textbf{Problem: Loss doesn't decrease}

\begin{lstlisting}
# Sanity checks
print("Model output range:", output.min().item(), output.max().item())
print("Target range:", target.min().item(), target.max().item())

# Check if gradients are flowing
for name, param in model.named_parameters():
    if param.grad is not None:
        print(f"{name}: grad mean = {param.grad.abs().mean():.6f}")
\end{lstlisting}

\textbf{Common causes:}
\begin{itemize}
    \item Learning rate too low
    \item Wrong loss function
    \item Forgot \texttt{optimizer.zero\_grad()}
    \item Model architecture issue (dead ReLUs, wrong dimensions)
    \item Data not normalized
\end{itemize}

\textbf{Problem: Loss explodes}

\begin{itemize}
    \item Learning rate too high (reduce by 10x)
    \item Use gradient clipping
    \item Check for large batch effects (try smaller batches)
\end{itemize}

\textbf{Problem: Overfitting (train loss << val loss)}

\begin{lstlisting}
# Add regularization
model = nn.Sequential(
    nn.Linear(10, 50),
    nn.ReLU(),
    nn.Dropout(0.5),  # Add dropout
    nn.Linear(50, 1)
)

# Add weight decay
optimizer = optim.Adam(model.parameters(), lr=0.001, weight_decay=1e-4)

# Use data augmentation (for images)
# Reduce model size
# Get more data
\end{lstlisting}

\textbf{Problem: Underfitting (both losses high)}

\begin{itemize}
    \item Increase model capacity (more layers, more neurons)
    \item Train longer
    \item Increase learning rate
    \item Remove regularization
    \item Check if task is actually learnable
\end{itemize}

\subsubsection{Monitoring Training}

\begin{lstlisting}
# Track metrics
train_losses = []
val_losses = []

for epoch in range(num_epochs):
    train_loss = train_epoch(...)
    val_loss, val_acc = validate(...)
    
    train_losses.append(train_loss)
    val_losses.append(val_loss)
    
    # Plot periodically
    if epoch % 10 == 0:
        import matplotlib.pyplot as plt
        plt.plot(train_losses, label='Train')
        plt.plot(val_losses, label='Val')
        plt.xlabel('Epoch')
        plt.ylabel('Loss')
        plt.legend()
        plt.show()
\end{lstlisting}

\clearpage
\subsection{Exercises}

\begin{exercise}[4.1: Basic Training Loop - $\bigstar\bigstar$]
\textbf{Goal:} Implement a complete training loop from scratch.

Task: Fit a neural network to $y = \sin(x)$ for $x \in [0, 2\pi]$

\begin{enumerate}
    \item Generate 1000 training samples: $x$ uniform in $[0, 2\pi]$, $y = \sin(x)$ + small noise
    \item Create a 3-layer MLP: 1 → 50 → 50 → 1 with ReLU
    \item Use MSE loss and Adam optimizer
    \item Train for 100 epochs, batch size 32
    \item Print loss every 10 epochs
    \item Plot predictions vs true function
\end{enumerate}

\textbf{Starter code:}
\begin{lstlisting}
import torch
import torch.nn as nn
import torch.optim as optim
import numpy as np

# Generate data
x = torch.linspace(0, 2*np.pi, 1000).reshape(-1, 1)
y = torch.sin(x) + 0.1 * torch.randn_like(x)

# Your code here
\end{lstlisting}
\end{exercise}

\begin{exercise}[4.2: Loss Function Comparison - $\bigstar\bigstar\bigstar$]
\textbf{Goal:} Understand different loss functions.

Train the same model with different losses:
\begin{enumerate}
    \item MSE Loss (L2)
    \item L1 Loss
    \item Smooth L1 Loss
\end{enumerate}

Add outliers to the data (10\% of points with large noise) and compare:
\begin{itemize}
    \item Final training loss
    \item Robustness to outliers
    \item Convergence speed
\end{itemize}

Which loss is most robust to outliers?
\end{exercise}

\begin{exercise}[4.3: Optimizer Comparison - $\bigstar\bigstar\bigstar$]
\textbf{Goal:} Compare optimizer behavior.

Train the same model with:
\begin{enumerate}
    \item SGD (lr=0.01)
    \item SGD with momentum (lr=0.01, momentum=0.9)
    \item Adam (lr=0.001)
    \item AdamW (lr=0.001)
\end{enumerate}

Plot all loss curves on the same graph. Observations:
\begin{itemize}
    \item Which converges fastest initially?
    \item Which achieves lowest final loss?
    \item How does momentum help SGD?
\end{itemize}
\end{exercise}

\begin{exercise}[4.4: Learning Rate Experiments - $\bigstar\bigstar\bigstar$]
\textbf{Goal:} Understand learning rate impact.

Train the same model with different learning rates:
\begin{itemize}
    \item $10^{-4}$, $10^{-3}$, $10^{-2}$, $10^{-1}$, $1.0$
\end{itemize}

For each:
\begin{enumerate}
    \item Track loss over epochs
    \item Identify: too small, just right, too large
    \item Observe divergence (NaN loss) for very large LR
\end{enumerate}

\textbf{Extra:} Implement a learning rate finder (sweep from small to large, stop when loss diverges).
\end{exercise}

\begin{exercise}[4.5: Overfitting Detection - $\bigstar\bigstar\bigstar\bigstar$]
\textbf{Goal:} Identify and prevent overfitting.

\begin{enumerate}
    \item Generate a small dataset (100 samples)
    \item Split into train (80) and validation (20)
    \item Train a large model (can easily overfit)
    \item Track both train and validation loss
    \item Identify when overfitting starts (losses diverge)
    \item Add regularization:
    \begin{itemize}
        \item Dropout
        \item Weight decay
        \item Early stopping
    \end{itemize}
    \item Compare final validation performance
\end{enumerate}
\end{exercise}

\begin{exercise}[4.6: Complete Training Pipeline - $\bigstar\bigstar\bigstar\bigstar$]
\textbf{Goal:} Build a production-ready training pipeline.

Implement a complete pipeline with:
\begin{enumerate}
    \item Data splitting (train/val/test)
    \item Training loop with progress bars
    \item Validation after each epoch
    \item Model checkpointing (save best model)
    \item Early stopping
    \item Learning rate scheduling
    \item Tensorboard logging (optional)
    \item Final test set evaluation
\end{enumerate}

Test on a real dataset (e.g., sklearn's make\_classification).

\textbf{Hint:} Use \texttt{tqdm} for progress bars:
\begin{lstlisting}
from tqdm import tqdm

for epoch in tqdm(range(num_epochs)):
    # training code
\end{lstlisting}
\end{exercise}

\begin{exercise}[4.7: Debugging Challenge - $\bigstar\bigstar\bigstar\bigstar$]
\textbf{Goal:} Practice debugging common training issues.

Given buggy training code (deliberately broken):

\begin{lstlisting}
# This code has 5 bugs - find and fix them!
model = nn.Sequential(nn.Linear(10, 1), nn.Softmax())
criterion = nn.CrossEntropyLoss()
optimizer = optim.Adam(model.parameters(), lr=10.0)

for epoch in range(100):
    output = model(X_train)
    loss = criterion(output, y_train)
    loss.backward()
    optimizer.step()
    print(f"Epoch {epoch}, Loss: {loss.item()}")
\end{lstlisting}

Find and fix:
\begin{enumerate}
    \item Wrong activation before CrossEntropyLoss
    \item Missing \texttt{optimizer.zero\_grad()}
    \item Learning rate too high
    \item No train/eval modes
    \item No validation set
\end{enumerate}
\end{exercise}

\clearpage
% =============================================
% SECTION 5: DATA LOADING & PREPROCESSING
% =============================================

\section{Data Loading \& Preprocessing}

\subsection{Introduction: Why Proper Data Loading Matters}

So far, we've used simple tensors for data. In practice, you'll have:
\begin{itemize}
    \item Large datasets that don't fit in memory
    \item Complex preprocessing pipelines
    \item Data on disk (files, databases)
    \item Need for efficient batching and shuffling
    \item Different data types (images, time series, point clouds, meshes)
\end{itemize}

PyTorch provides \texttt{Dataset} and \texttt{DataLoader} to handle this elegantly:
\begin{itemize}
    \item \textbf{Dataset:} Defines how to access individual samples
    \item \textbf{DataLoader:} Handles batching, shuffling, parallel loading
\end{itemize}

\subsection{Theory: Dataset and DataLoader Architecture}

\begin{theorybox}[The Data Pipeline]
\begin{verbatim}
Raw Data → Dataset → DataLoader → Model
           (access)  (batch, shuffle, parallel)
\end{verbatim}

\textbf{Dataset} answers: "How do I get sample $i$?"

\textbf{DataLoader} answers: "How do I create batches efficiently?"
\end{theorybox}

\textbf{Dataset Requirements:}

A valid dataset must implement:
\begin{enumerate}
    \item \texttt{\_\_len\_\_()}: Return total number of samples
    \item \texttt{\_\_getitem\_\_(idx)}: Return sample at index \texttt{idx}
\end{enumerate}

\textbf{DataLoader Benefits:}
\begin{itemize}
    \item Automatic batching
    \item Shuffling for training
    \item Parallel data loading (multiple workers)
    \item Automatic memory pinning for GPU transfer
    \item Handling variable-length sequences
\end{itemize}

\subsection{Implementation: Basic Usage}

\subsubsection{Using Built-in Datasets}

PyTorch provides common datasets in \texttt{torchvision.datasets}:

\begin{lstlisting}
import torch
from torch.utils.data import DataLoader
import torchvision
import torchvision.transforms as transforms

# Load MNIST dataset
transform = transforms.ToTensor()  # Convert PIL Image to tensor

train_dataset = torchvision.datasets.MNIST(
    root='./data',
    train=True,
    download=True,
    transform=transform
)

test_dataset = torchvision.datasets.MNIST(
    root='./data',
    train=False,
    download=True,
    transform=transform
)

# Create DataLoaders
train_loader = DataLoader(
    train_dataset,
    batch_size=64,
    shuffle=True,      # Shuffle training data
    num_workers=2,     # Parallel loading
    pin_memory=True    # Faster GPU transfer
)

test_loader = DataLoader(
    test_dataset,
    batch_size=64,
    shuffle=False,     # Don't shuffle test data
    num_workers=2
)

# Iterate through batches
for batch_idx, (images, labels) in enumerate(train_loader):
    # images: (64, 1, 28, 28) - batch of 64 grayscale 28x28 images
    # labels: (64,) - batch of 64 labels
    print(f"Batch {batch_idx}: {images.shape}, {labels.shape}")
    break
\end{lstlisting}

\begin{pytorchtip}[DataLoader Parameters]
Key parameters:
\begin{itemize}
    \item \textbf{batch\_size:} Number of samples per batch (32-256 typically)
    \item \textbf{shuffle:} Randomize order (True for training, False for evaluation)
    \item \textbf{num\_workers:} Parallel processes for loading (0=main process, 2-4 for speed)
    \item \textbf{pin\_memory:} Faster CPU→GPU transfer (True if using GPU)
    \item \textbf{drop\_last:} Drop incomplete last batch (useful for batch norm)
\end{itemize}
\end{pytorchtip}

\clearpage
\subsubsection{Creating Custom Datasets}

For your own data, create a custom \texttt{Dataset}:

\begin{lstlisting}
from torch.utils.data import Dataset

class SimpleDataset(Dataset):
    """Example: Dataset from numpy arrays."""
    
    def __init__(self, X, y):
        """
        Args:
            X: numpy array or tensor of features
            y: numpy array or tensor of labels
        """
        self.X = torch.FloatTensor(X) if not isinstance(X, torch.Tensor) else X
        self.y = torch.FloatTensor(y) if not isinstance(y, torch.Tensor) else y
    
    def __len__(self):
        """Return total number of samples."""
        return len(self.X)
    
    def __getitem__(self, idx):
        """Return sample at index idx."""
        return self.X[idx], self.y[idx]

# Usage
import numpy as np

X = np.random.randn(1000, 10)
y = np.random.randn(1000, 1)

dataset = SimpleDataset(X, y)
loader = DataLoader(dataset, batch_size=32, shuffle=True)

for batch_X, batch_y in loader:
    print(batch_X.shape, batch_y.shape)
    break
\end{lstlisting}

\subsubsection{Dataset from Files}

More realistic: loading data from disk:

\begin{lstlisting}
import os
import pandas as pd
from PIL import Image

class ImageDataset(Dataset):
    """Load images from a directory with CSV labels."""
    
    def __init__(self, csv_file, img_dir, transform=None):
        """
        Args:
            csv_file: Path to CSV with columns [filename, label]
            img_dir: Directory with images
            transform: Optional transform to apply
        """
        self.data = pd.read_csv(csv_file)
        self.img_dir = img_dir
        self.transform = transform
    
    def __len__(self):
        return len(self.data)
    
    def __getitem__(self, idx):
        # Get image filename and label
        img_name = self.data.iloc[idx, 0]
        label = self.data.iloc[idx, 1]
        
        # Load image
        img_path = os.path.join(self.img_dir, img_name)
        image = Image.open(img_path).convert('RGB')
        
        # Apply transforms
        if self.transform:
            image = self.transform(image)
        
        return image, label

# Usage with transforms
transform = transforms.Compose([
    transforms.Resize((224, 224)),
    transforms.ToTensor(),
    transforms.Normalize(mean=[0.485, 0.456, 0.406],
                        std=[0.229, 0.224, 0.225])
])

dataset = ImageDataset('labels.csv', 'images/', transform=transform)
loader = DataLoader(dataset, batch_size=32, shuffle=True)
\end{lstlisting}

\subsubsection{Dataset for Time Series}

For scientific computing, time series are common:

\begin{lstlisting}
class TimeSeriesDataset(Dataset):
    """
    Create windows from a long time series.
    Useful for forecasting, where we predict future from past.
    """
    
    def __init__(self, data, window_size, forecast_horizon):
        """
        Args:
            data: 1D array of time series values
            window_size: Length of input sequence
            forecast_horizon: How many steps ahead to predict
        """
        self.data = torch.FloatTensor(data)
        self.window_size = window_size
        self.forecast_horizon = forecast_horizon
    
    def __len__(self):
        # Number of valid windows
        return len(self.data) - self.window_size - self.forecast_horizon + 1
    
    def __getitem__(self, idx):
        # Input: window_size points starting at idx
        x = self.data[idx : idx + self.window_size]
        
        # Target: forecast_horizon points after the input
        y = self.data[idx + self.window_size : 
                     idx + self.window_size + self.forecast_horizon]
        
        return x, y

# Example: predict next 10 steps from past 50
data = np.sin(np.linspace(0, 100, 1000))
dataset = TimeSeriesDataset(data, window_size=50, forecast_horizon=10)

print(f"Dataset size: {len(dataset)}")
x, y = dataset[0]
print(f"Input shape: {x.shape}, Target shape: {y.shape}")
# Input shape: torch.Size([50]), Target shape: torch.Size([10])

loader = DataLoader(dataset, batch_size=32, shuffle=False)
# Note: Don't shuffle time series if order matters!
\end{lstlisting}

\clearpage
\subsubsection{Dataset for Point Clouds}

For 3D data (molecular structures, point clouds):

\begin{lstlisting}
class PointCloudDataset(Dataset):
    """Dataset for 3D point clouds."""
    
    def __init__(self, data_dir, num_points=1024):
        """
        Args:
            data_dir: Directory with .npy files, each containing Nx3 points
            num_points: Sample this many points from each cloud
        """
        self.files = [os.path.join(data_dir, f) 
                     for f in os.listdir(data_dir) if f.endswith('.npy')]
        self.num_points = num_points
    
    def __len__(self):
        return len(self.files)
    
    def __getitem__(self, idx):
        # Load point cloud (N x 3)
        points = np.load(self.files[idx])
        
        # Sample fixed number of points
        if len(points) > self.num_points:
            # Random sampling
            indices = np.random.choice(len(points), self.num_points, 
                                      replace=False)
            points = points[indices]
        elif len(points) < self.num_points:
            # Pad with zeros
            padding = np.zeros((self.num_points - len(points), 3))
            points = np.vstack([points, padding])
        
        return torch.FloatTensor(points)

# Usage
# dataset = PointCloudDataset('point_clouds/')
# loader = DataLoader(dataset, batch_size=8, shuffle=True)
# for batch in loader:
#     print(batch.shape)  # (8, 1024, 3)
\end{lstlisting}

\subsubsection{Data Transforms}

Transforms process data on-the-fly:

\begin{lstlisting}
from torchvision import transforms

# Common image transforms
transform = transforms.Compose([
    # Resize to fixed size
    transforms.Resize((224, 224)),
    
    # Data augmentation (for training)
    transforms.RandomHorizontalFlip(p=0.5),
    transforms.RandomRotation(degrees=15),
    transforms.ColorJitter(brightness=0.2, contrast=0.2),
    
    # Convert to tensor
    transforms.ToTensor(),  # Converts to [0, 1] and (C, H, W)
    
    # Normalize (important!)
    transforms.Normalize(mean=[0.485, 0.456, 0.406],  # ImageNet stats
                        std=[0.229, 0.224, 0.225])
])

# For evaluation/test: no data augmentation
test_transform = transforms.Compose([
    transforms.Resize((224, 224)),
    transforms.ToTensor(),
    transforms.Normalize(mean=[0.485, 0.456, 0.406],
                        std=[0.229, 0.224, 0.225])
])

# Custom transform as a callable
class AddGaussianNoise:
    """Add Gaussian noise to tensor."""
    def __init__(self, mean=0., std=0.1):
        self.mean = mean
        self.std = std
    
    def __call__(self, tensor):
        return tensor + torch.randn(tensor.size()) * self.std + self.mean

# Use in Compose
transform = transforms.Compose([
    transforms.ToTensor(),
    AddGaussianNoise(std=0.05)
])
\end{lstlisting}

\begin{pytorchtip}[When to Apply Transforms]
Transforms can be applied in two places:

\textbf{In Dataset (\_\_getitem\_\_):}
\begin{itemize}
    \item Pro: Each sample transformed on-the-fly
    \item Pro: Can use data augmentation
    \item Con: Adds computation during training
\end{itemize}

\textbf{Pre-computed and saved:}
\begin{itemize}
    \item Pro: Faster training (no transform overhead)
    \item Con: More disk space
    \item Con: Can't use random augmentation
\end{itemize}

For training with augmentation: use Dataset transforms.
For inference: consider pre-computing.
\end{pytorchtip}

\clearpage
\subsubsection{Normalization Strategies}

Normalization is crucial for training stability:

\begin{lstlisting}
# Method 1: Standardization (zero mean, unit variance)
def compute_mean_std(dataset):
    """Compute mean and std for normalization."""
    loader = DataLoader(dataset, batch_size=64, shuffle=False)
    
    mean = 0.0
    std = 0.0
    total_samples = 0
    
    for data, _ in loader:
        batch_samples = data.size(0)
        data = data.view(batch_samples, data.size(1), -1)
        mean += data.mean(2).sum(0)
        std += data.std(2).sum(0)
        total_samples += batch_samples
    
    mean /= total_samples
    std /= total_samples
    
    return mean, std

# Usage
# mean, std = compute_mean_std(train_dataset)
# transform = transforms.Normalize(mean=mean, std=std)

# Method 2: Min-Max scaling to [0, 1] or [-1, 1]
class MinMaxScaler:
    def __init__(self, data):
        self.min = data.min()
        self.max = data.max()
    
    def transform(self, data):
        return (data - self.min) / (self.max - self.min)
    
    def inverse_transform(self, data):
        return data * (self.max - self.min) + self.min

# Method 3: Per-feature normalization (for tabular data)
class StandardScaler:
    def __init__(self, data):
        """
        data: (N, D) tensor
        """
        self.mean = data.mean(dim=0)
        self.std = data.std(dim=0)
    
    def transform(self, data):
        return (data - self.mean) / (self.std + 1e-8)
    
    def inverse_transform(self, data):
        return data * self.std + self.mean

# Usage
X_train = torch.randn(1000, 10)
scaler = StandardScaler(X_train)
X_train_normalized = scaler.transform(X_train)
X_test_normalized = scaler.transform(X_test)  # Use training statistics!
\end{lstlisting}

\begin{warningbox}[Always Use Training Statistics]
When normalizing:
\begin{enumerate}
    \item Compute mean/std from \textbf{training data only}
    \item Apply the same transformation to validation and test data
    \item \textbf{Never} compute statistics from test data!
\end{enumerate}

Why? Using test statistics is a form of data leakage and gives overly optimistic results.
\end{warningbox}

\subsubsection{Handling Variable-Length Sequences}

For sequences of different lengths (text, time series with variable duration):

\begin{lstlisting}
from torch.nn.utils.rnn import pad_sequence, pack_padded_sequence

def collate_fn(batch):
    """
    Custom collate function for variable-length sequences.
    
    batch: List of (sequence, label) tuples
    sequences: List of tensors with different lengths
    """
    sequences, labels = zip(*batch)
    
    # Get lengths
    lengths = torch.LongTensor([len(seq) for seq in sequences])
    
    # Pad sequences to same length
    sequences_padded = pad_sequence(sequences, batch_first=True, 
                                    padding_value=0)
    
    labels = torch.LongTensor(labels)
    
    return sequences_padded, labels, lengths

# Example dataset with variable lengths
class VariableLengthDataset(Dataset):
    def __init__(self, num_samples=100):
        self.data = []
        for _ in range(num_samples):
            # Random length between 10 and 50
            length = torch.randint(10, 50, (1,)).item()
            seq = torch.randn(length, 5)  # 5 features
            label = torch.randint(0, 2, (1,)).item()
            self.data.append((seq, label))
    
    def __len__(self):
        return len(self.data)
    
    def __getitem__(self, idx):
        return self.data[idx]

# Use with custom collate_fn
dataset = VariableLengthDataset()
loader = DataLoader(dataset, batch_size=8, collate_fn=collate_fn)

for sequences, labels, lengths in loader:
    print(f"Padded sequences: {sequences.shape}")  # (8, max_len, 5)
    print(f"Lengths: {lengths}")  # (8,)
    break
\end{lstlisting}

\clearpage
\subsubsection{Train/Validation/Test Split}

Proper data splitting:

\begin{lstlisting}
from torch.utils.data import random_split

# Method 1: random_split
dataset = SimpleDataset(X, y)
train_size = int(0.7 * len(dataset))
val_size = int(0.15 * len(dataset))
test_size = len(dataset) - train_size - val_size

train_dataset, val_dataset, test_dataset = random_split(
    dataset, [train_size, val_size, test_size]
)

# Method 2: Using indices (more control)
from torch.utils.data import Subset
import numpy as np

indices = np.arange(len(dataset))
np.random.shuffle(indices)

train_idx = indices[:train_size]
val_idx = indices[train_size:train_size+val_size]
test_idx = indices[train_size+val_size:]

train_dataset = Subset(dataset, train_idx)
val_dataset = Subset(dataset, val_idx)
test_dataset = Subset(dataset, test_idx)

# Method 3: sklearn for stratified split (classification)
from sklearn.model_selection import train_test_split

X = np.random.randn(1000, 10)
y = np.random.randint(0, 3, 1000)  # 3 classes

X_train, X_temp, y_train, y_temp = train_test_split(
    X, y, test_size=0.3, stratify=y, random_state=42
)
X_val, X_test, y_val, y_test = train_test_split(
    X_temp, y_temp, test_size=0.5, stratify=y_temp, random_state=42
)

train_dataset = SimpleDataset(X_train, y_train)
val_dataset = SimpleDataset(X_val, y_val)
test_dataset = SimpleDataset(X_test, y_test)
\end{lstlisting}

\subsubsection{Efficient Data Loading}

For large datasets that don't fit in RAM:

\begin{lstlisting}
class LazyLoadDataset(Dataset):
    """Load data on-demand, don't store in memory."""
    
    def __init__(self, file_list):
        """
        file_list: List of file paths
        """
        self.file_list = file_list
    
    def __len__(self):
        return len(self.file_list)
    
    def __getitem__(self, idx):
        # Load from disk only when needed
        data = np.load(self.file_list[idx])
        return torch.FloatTensor(data)

# For even larger datasets: memory-mapped arrays
class MemMapDataset(Dataset):
    """Use memory-mapped arrays for huge datasets."""
    
    def __init__(self, memmap_file, shape, dtype):
        """
        memmap_file: Path to .npy file
        shape: Shape of the data array
        dtype: Data type
        """
        self.data = np.memmap(memmap_file, dtype=dtype, 
                             mode='r', shape=shape)
    
    def __len__(self):
        return self.data.shape[0]
    
    def __getitem__(self, idx):
        # Read only the needed slice from disk
        return torch.FloatTensor(self.data[idx])
\end{lstlisting}

\begin{pytorchtip}[Optimizing DataLoader Performance]
If data loading is slow:

\begin{enumerate}
    \item \textbf{Increase num\_workers:} 2-4 usually optimal
    \item \textbf{Use pin\_memory=True:} If using GPU
    \item \textbf{Prefetch:} DataLoader automatically prefetches next batch
    \item \textbf{Reduce transforms:} Pre-compute expensive transforms
    \item \textbf{Use SSD:} Faster disk I/O
    \item \textbf{Check CPU bottleneck:} If GPU is idle, data loading is the issue
\end{enumerate}

Warning: Too many workers can hurt performance (overhead).
\end{pytorchtip}

\clearpage
\subsection{Exercises}

\begin{exercise}[5.1: Simple Dataset - $\bigstar\bigstar$]
\textbf{Goal:} Create your first custom dataset.

\begin{enumerate}
    \item Create a synthetic dataset: $y = 2x_1 + 3x_2 + \epsilon$
    \item Generate 1000 samples with $x_1, x_2 \sim \mathcal{N}(0, 1)$ and noise $\epsilon \sim \mathcal{N}(0, 0.1)$
    \item Implement a custom \texttt{Dataset} class
    \item Create a \texttt{DataLoader} with batch\_size=32
    \item Iterate through one epoch and print batch shapes
\end{enumerate}
\end{exercise}

\begin{exercise}[5.2: Train/Val/Test Split - $\bigstar\bigstar$]
\textbf{Goal:} Practice proper data splitting.

\begin{enumerate}
    \item Generate a classification dataset (1000 samples, 3 classes)
    \item Split into train (70\%), validation (15\%), test (15\%)
    \item Ensure the split is stratified (balanced classes in each split)
    \item Create separate DataLoaders for each split
    \item Verify class distributions are similar
\end{enumerate}

\textbf{Hint:} Use \texttt{sklearn.model\_selection.train\_test\_split} with \texttt{stratify}.
\end{exercise}

\begin{exercise}[5.3: Time Series Dataset - $\bigstar\bigstar\bigstar$]
\textbf{Goal:} Handle sequential data.

\begin{enumerate}
    \item Generate a noisy sine wave with 10,000 points
    \item Create a dataset that returns windows of size 50 to predict the next 10 points
    \item Implement the \texttt{TimeSeriesDataset} class
    \item Train a simple MLP to predict future values
    \item Visualize predictions vs ground truth
\end{enumerate}

\textbf{Challenge:} Try with a real time series (stock prices, weather data).
\end{exercise}

\begin{exercise}[5.4: Data Normalization - $\bigstar\bigstar\bigstar$]
\textbf{Goal:} Understand normalization impact.

\begin{enumerate}
    \item Create a dataset with features on different scales:
    \begin{itemize}
        \item Feature 1: range [0, 1]
        \item Feature 2: range [0, 1000]
        \item Feature 3: range [-100, 100]
    \end{itemize}
    \item Train a model \textbf{without} normalization
    \item Train a model \textbf{with} standardization
    \item Compare:
    \begin{itemize}
        \item Training speed (epochs to converge)
        \item Final loss
        \item Gradient magnitudes
    \end{itemize}
\end{enumerate}
\end{exercise}

\begin{exercise}[5.5: Data Augmentation - $\bigstar\bigstar\bigstar$]
\textbf{Goal:} Implement custom transforms.

\begin{enumerate}
    \item Create a simple image dataset (or use MNIST)
    \item Implement custom transforms:
    \begin{itemize}
        \item Add Gaussian noise
        \item Random scaling
        \item Random shifts
    \end{itemize}
    \item Train models with and without augmentation
    \item Compare validation performance
    \item Visualize augmented samples
\end{enumerate}
\end{exercise}

\begin{exercise}[5.6: Variable-Length Sequences - $\bigstar\bigstar\bigstar\bigstar$]
\textbf{Goal:} Handle sequences of different lengths.

\begin{enumerate}
    \item Create a dataset of sequences with random lengths (10-100)
    \item Implement a custom \texttt{collate\_fn} to pad sequences
    \item Create a DataLoader using this collate function
    \item Train an RNN that handles the padded sequences
    \item Use \texttt{pack\_padded\_sequence} for efficiency (we'll learn this properly in the RNN section)
\end{enumerate}

\textbf{Hint:} The collate function should return sequences, lengths, and labels.
\end{exercise}

\begin{exercise}[5.7: Complete Data Pipeline - $\bigstar\bigstar\bigstar\bigstar$]
\textbf{Goal:} Build a production-ready data pipeline.

Create a complete pipeline with:
\begin{enumerate}
    \item Custom dataset loading from CSV files
    \item Train/val/test split with stratification
    \item Different transforms for train (with augmentation) and test
    \item Normalization using training statistics
    \item Efficient DataLoaders with multiple workers
    \item Save and load normalization parameters
    \item Handle class imbalance with weighted sampling (optional)
\end{enumerate}

Test on a real dataset (e.g., UCI datasets, Kaggle).
\end{exercise}

\clearpage
% =============================================
% PART II: DEEP LEARNING ARCHITECTURES
% =============================================

\part{Deep Learning Architectures}

% =============================================
% SECTION 6: FULLY CONNECTED NETWORKS (MLPs)
% =============================================

\section{Fully Connected Networks (MLPs)}

\subsection{Introduction: The Foundation of Deep Learning}

Multi-Layer Perceptrons (MLPs), also called fully connected networks or feedforward networks, are the simplest and most fundamental deep learning architecture. Every neuron in one layer connects to every neuron in the next layer—hence "fully connected."

Despite their simplicity, MLPs are:
\begin{itemize}
    \item \textbf{Universal approximators:} Can approximate any continuous function
    \item \textbf{Building blocks:} Components of more complex architectures
    \item \textbf{Highly effective:} For tabular data, function approximation, and many scientific computing tasks
\end{itemize}

\textbf{When to use MLPs:}
\begin{itemize}
    \item Tabular data (features don't have spatial/temporal structure)
    \item Function approximation in scientific computing
    \item As components in larger architectures
    \item When you have relatively small input dimensions (<1000s of features)
\end{itemize}

\textbf{When NOT to use MLPs:}
\begin{itemize}
    \item Images (use CNNs—exploit spatial structure)
    \item Sequences (use RNNs/Transformers—exploit temporal structure)
    \item Very high-dimensional inputs (too many parameters)
\end{itemize}

\subsection{Theory: Understanding MLPs}

\subsubsection{Architecture}

An MLP consists of:
\begin{itemize}
    \item \textbf{Input layer:} Receives features $\mathbf{x} \in \mathbb{R}^{d_{in}}$
    \item \textbf{Hidden layers:} Transform representations
    \item \textbf{Output layer:} Produces predictions $\hat{\mathbf{y}} \in \mathbb{R}^{d_{out}}$
\end{itemize}

\textbf{Single hidden layer MLP:}
\[
\mathbf{h} = \sigma(\mathbf{W}_1 \mathbf{x} + \mathbf{b}_1)
\]
\[
\hat{\mathbf{y}} = \mathbf{W}_2 \mathbf{h} + \mathbf{b}_2
\]

where:
\begin{itemize}
    \item $\mathbf{W}_1 \in \mathbb{R}^{h \times d_{in}}$: First layer weights
    \item $\mathbf{b}_1 \in \mathbb{R}^h$: First layer biases
    \item $\sigma$: Activation function (nonlinearity)
    \item $\mathbf{h} \in \mathbb{R}^h$: Hidden representation
    \item $\mathbf{W}_2, \mathbf{b}_2$: Second layer parameters
\end{itemize}

\begin{theorybox}[Universal Approximation Theorem]
A single-hidden-layer MLP with enough neurons can approximate any continuous function on a compact domain to arbitrary accuracy.

\textbf{Key insight:} Width matters! But in practice, depth is often more efficient than width.

\textbf{Intuition:} Think of hidden neurons as "basis functions." With enough of them, you can represent complex functions as combinations of simple ones.
\end{theorybox}

\subsubsection{Depth vs Width}

\textbf{Wide networks} (few layers, many neurons per layer):
\begin{itemize}
    \item Pro: Easier to optimize (fewer layers = less gradient propagation)
    \item Pro: Universal approximation holds
    \item Con: Need exponentially many neurons for complex functions
    \item Con: Don't learn hierarchical features
\end{itemize}

\textbf{Deep networks} (many layers, moderate neurons per layer):
\begin{itemize}
    \item Pro: More parameter efficient (exponentially fewer parameters)
    \item Pro: Learn hierarchical representations (low-level → high-level features)
    \item Pro: Better generalization (implicit regularization)
    \item Con: Harder to optimize (vanishing/exploding gradients)
    \item Con: Requires careful initialization and normalization
\end{itemize}

\textbf{Rule of thumb:}
\begin{itemize}
    \item Start with 2-3 hidden layers
    \item Width: 1-2× input dimension for first layer, then decrease
    \item Increase depth if data is complex and you have lots of samples
    \item Use modern techniques (batch norm, skip connections) for very deep networks
\end{itemize}

\clearpage
\subsubsection{Activation Functions}

Activation functions introduce nonlinearity. Without them, multiple layers would collapse to a single linear transformation.

\textbf{ReLU (Rectified Linear Unit):}
\[
\text{ReLU}(x) = \max(0, x)
\]

\begin{itemize}
    \item \textbf{Pros:} Fast to compute, sparse activations, no vanishing gradient for $x>0$
    \item \textbf{Cons:} "Dead ReLUs" (neurons that always output 0)
    \item \textbf{Derivative:} $\frac{d}{dx}\text{ReLU}(x) = \begin{cases} 1 & x > 0 \\ 0 & x \leq 0 \end{cases}$
    \item \textbf{Use when:} Default choice for most tasks, especially deep networks
\end{itemize}

\textbf{Leaky ReLU:}
\[
\text{LeakyReLU}(x) = \max(\alpha x, x) \quad \text{(typically } \alpha=0.01\text{)}
\]

\begin{itemize}
    \item \textbf{Pros:} Fixes dead ReLU problem (always has gradient)
    \item \textbf{Cons:} Extra hyperparameter $\alpha$
    \item \textbf{Use when:} ReLU causes many dead neurons
\end{itemize}

\textbf{ELU (Exponential Linear Unit):}
\[
\text{ELU}(x) = \begin{cases} x & x > 0 \\ \alpha(e^x - 1) & x \leq 0 \end{cases}
\]

\begin{itemize}
    \item \textbf{Pros:} Smooth, mean activations closer to zero
    \item \textbf{Cons:} Slower to compute (exponential)
    \item \textbf{Use when:} You want smoother gradients
\end{itemize}

\textbf{Tanh (Hyperbolic Tangent):}
\[
\text{tanh}(x) = \frac{e^x - e^{-x}}{e^x + e^{-x}}
\]

\begin{itemize}
    \item \textbf{Pros:} Zero-centered (outputs in $[-1, 1]$), smooth
    \item \textbf{Cons:} Vanishing gradient for large $|x|$
    \item \textbf{Use when:} Need zero-centered activations (RNNs, some scientific applications)
\end{itemize}

\textbf{Sigmoid:}
\[
\sigma(x) = \frac{1}{1 + e^{-x}}
\]

\begin{itemize}
    \item \textbf{Pros:} Outputs in $(0, 1)$ (interpretable as probabilities)
    \item \textbf{Cons:} Strong vanishing gradient, not zero-centered
    \item \textbf{Use when:} \textbf{Only in output layer} for binary classification
\end{itemize}

\textbf{GELU (Gaussian Error Linear Unit):}
\[
\text{GELU}(x) = x \cdot \Phi(x) \quad \text{where } \Phi \text{ is CDF of } \mathcal{N}(0,1)
\]

\begin{itemize}
    \item \textbf{Pros:} Smooth, used in state-of-the-art models (BERT, GPT)
    \item \textbf{Cons:} More expensive than ReLU
    \item \textbf{Use when:} Training Transformers or want best performance
\end{itemize}

\textbf{Swish (SiLU):}
\[
\text{Swish}(x) = x \cdot \sigma(x)
\]

\begin{itemize}
    \item \textbf{Pros:} Smooth, unbounded above, self-gated
    \item \textbf{Cons:} More expensive than ReLU
    \item \textbf{Use when:} Similar to GELU, modern architecture
\end{itemize}

\begin{pytorchtip}[Choosing Activation Functions]
\textbf{Default choice:} Use \textbf{ReLU} for hidden layers unless you have a reason not to.

\textbf{Use Leaky ReLU or ELU if:} You observe many dead neurons (check activation distributions)

\textbf{Use Tanh if:} You need zero-centered activations (RNNs, specific scientific applications)

\textbf{Use GELU or Swish if:} You're training a Transformer or want state-of-the-art performance and can afford the compute

\textbf{Never use Sigmoid in hidden layers:} Strong vanishing gradient makes training very difficult
\end{pytorchtip}

\clearpage
\subsubsection{Why Networks Fail to Train}

\textbf{1. Vanishing Gradients}

As gradients backpropagate through many layers, they can become exponentially small.

\textbf{Cause:} Chain rule multiplies many small derivatives ($<1$)

\textbf{Effect:} Early layers learn very slowly or not at all

\textbf{Solutions:}
\begin{itemize}
    \item Use ReLU instead of sigmoid/tanh (ReLU has gradient 1 for $x>0$)
    \item Proper initialization (Xavier, He)
    \item Batch normalization
    \item Skip connections (ResNets—we'll cover this later)
\end{itemize}

\textbf{2. Exploding Gradients}

Gradients become exponentially large, causing NaN or very large parameter updates.

\textbf{Cause:} Chain rule multiplies many large derivatives ($>1$)

\textbf{Effect:} Training diverges, loss becomes NaN

\textbf{Solutions:}
\begin{itemize}
    \item Gradient clipping
    \item Proper initialization
    \item Lower learning rate
    \item Batch normalization
\end{itemize}

\textbf{3. Dead ReLUs}

ReLU neurons that always output 0 (because input is always negative).

\textbf{Cause:} Large negative bias or poor initialization

\textbf{Effect:} Neurons become permanently inactive, reducing model capacity

\textbf{Detection:}
\begin{lstlisting}
# Check percentage of dead neurons
activations = model(x)
dead_percentage = (activations == 0).float().mean()
print(f"Dead neurons: {dead_percentage*100:.2f}%")
\end{lstlisting}

\textbf{Solutions:}
\begin{itemize}
    \item Use Leaky ReLU or ELU
    \item Proper initialization (He initialization)
    \item Lower learning rate
    \item Batch normalization
\end{itemize}

\textbf{4. Poor Initialization}

Starting with inappropriate parameter values.

\textbf{Effects:}
\begin{itemize}
    \item Too small: Vanishing activations/gradients
    \item Too large: Exploding activations/gradients
    \item All same: All neurons learn same features (symmetry problem)
\end{itemize}

\textbf{Solution:} Use proper initialization schemes (next section).

\subsection{Implementation: Building MLPs in PyTorch}

\subsubsection{Simple MLP with Sequential}

\begin{lstlisting}
import torch
import torch.nn as nn

# 3-layer MLP: input_dim -> 128 -> 64 -> output_dim
model = nn.Sequential(
    nn.Linear(10, 128),
    nn.ReLU(),
    nn.Linear(128, 64),
    nn.ReLU(),
    nn.Linear(64, 1)
)

# Test
x = torch.randn(32, 10)  # Batch of 32 samples
y = model(x)
print(y.shape)  # torch.Size([32, 1])
\end{lstlisting}

\subsubsection{Custom MLP Module}

\begin{lstlisting}
class MLP(nn.Module):
    """Flexible MLP with configurable architecture."""
    
    def __init__(self, input_dim, hidden_dims, output_dim, 
                 activation='relu', dropout=0.0):
        """
        Args:
            input_dim: Input feature dimension
            hidden_dims: List of hidden layer dimensions
            output_dim: Output dimension
            activation: Activation function ('relu', 'tanh', 'gelu', etc.)
            dropout: Dropout probability (0 = no dropout)
        """
        super().__init__()
        
        # Build layers
        layers = []
        prev_dim = input_dim
        
        for hidden_dim in hidden_dims:
            layers.append(nn.Linear(prev_dim, hidden_dim))
            
            # Activation
            if activation == 'relu':
                layers.append(nn.ReLU())
            elif activation == 'tanh':
                layers.append(nn.Tanh())
            elif activation == 'gelu':
                layers.append(nn.GELU())
            elif activation == 'leaky_relu':
                layers.append(nn.LeakyReLU(0.01))
            
            # Dropout (if specified)
            if dropout > 0:
                layers.append(nn.Dropout(dropout))
            
            prev_dim = hidden_dim
        
        # Output layer (no activation)
        layers.append(nn.Linear(prev_dim, output_dim))
        
        self.network = nn.Sequential(*layers)
    
    def forward(self, x):
        return self.network(x)

# Usage
model = MLP(input_dim=10, hidden_dims=[128, 64, 32], output_dim=1,
            activation='relu', dropout=0.2)

print(model)
"""
MLP(
  (network): Sequential(
    (0): Linear(in_features=10, out_features=128, bias=True)
    (1): ReLU()
    (2): Dropout(p=0.2, inplace=False)
    (3): Linear(in_features=128, out_features=64, bias=True)
    (4): ReLU()
    (5): Dropout(p=0.2, inplace=False)
    ...
  )
)
"""
\end{lstlisting}

\clearpage
\subsubsection{Weight Initialization}

\textbf{Why initialization matters:}

\begin{lstlisting}
# Bad initialization (too small)
model = nn.Linear(100, 100)
with torch.no_grad():
    model.weight.fill_(0.01)
    model.bias.fill_(0)

x = torch.randn(1, 100)
for i in range(10):
    x = torch.relu(model(x))
    print(f"Layer {i}: mean={x.mean():.4f}, std={x.std():.4f}")
# Output: Values shrink to near zero (vanishing)

# Bad initialization (too large)
model = nn.Linear(100, 100)
with torch.no_grad():
    model.weight.fill_(1.0)
    model.bias.fill_(0)

x = torch.randn(1, 100)
for i in range(10):
    x = torch.relu(model(x))
    print(f"Layer {i}: mean={x.mean():.4f}, std={x.std():.4f}")
# Output: Values explode (exploding)
\end{lstlisting}

\textbf{Xavier/Glorot Initialization (for Tanh/Sigmoid):}

Maintains variance across layers for linear activations.

\[
W \sim \mathcal{U}\left(-\sqrt{\frac{6}{n_{in} + n_{out}}}, \sqrt{\frac{6}{n_{in} + n_{out}}}\right)
\]

\begin{lstlisting}
import torch.nn.init as init

# Xavier uniform
init.xavier_uniform_(model.weight)

# Xavier normal
init.xavier_normal_(model.weight)
\end{lstlisting}

\textbf{He/Kaiming Initialization (for ReLU):}

Accounts for ReLU killing half the neurons (setting them to 0).

\[
W \sim \mathcal{N}\left(0, \sqrt{\frac{2}{n_{in}}}\right)
\]

\begin{lstlisting}
# He/Kaiming normal (preferred for ReLU)
init.kaiming_normal_(model.weight, mode='fan_in', nonlinearity='relu')

# He/Kaiming uniform
init.kaiming_uniform_(model.weight, nonlinearity='relu')
\end{lstlisting}

\textbf{Applying initialization to your model:}

\begin{lstlisting}
def init_weights(m):
    """Initialize weights for Linear layers."""
    if isinstance(m, nn.Linear):
        # He initialization for weights
        init.kaiming_normal_(m.weight, mode='fan_in', nonlinearity='relu')
        # Zero initialization for biases
        if m.bias is not None:
            init.constant_(m.bias, 0)

# Apply to model
model = MLP(10, [128, 64], 1)
model.apply(init_weights)

# Verify
for name, param in model.named_parameters():
    if 'weight' in name:
        print(f"{name}: mean={param.mean():.4f}, std={param.std():.4f}")
\end{lstlisting}

\begin{pytorchtip}[Initialization Guidelines]
\textbf{For ReLU networks:} Use He/Kaiming initialization
\begin{lstlisting}
init.kaiming_normal_(weight, nonlinearity='relu')
\end{lstlisting}

\textbf{For Tanh/Sigmoid networks:} Use Xavier/Glorot initialization
\begin{lstlisting}
init.xavier_normal_(weight)
\end{lstlisting}

\textbf{For biases:} Initialize to zero (or small constant)
\begin{lstlisting}
init.constant_(bias, 0)
\end{lstlisting}

\textbf{PyTorch defaults:} Most layers use reasonable defaults (Kaiming uniform for Linear). Explicit initialization is often unnecessary but can help for deep networks.
\end{pytorchtip}

\clearpage
\subsubsection{Batch Normalization}

Normalize activations within each mini-batch to stabilize training.

\textbf{How it works:}
\[
\hat{x} = \frac{x - \mu_B}{\sqrt{\sigma_B^2 + \epsilon}}
\]
\[
y = \gamma \hat{x} + \beta
\]

where $\mu_B$, $\sigma_B$ are batch mean/std, and $\gamma$, $\beta$ are learnable parameters.

\begin{lstlisting}
class MLPWithBatchNorm(nn.Module):
    def __init__(self, input_dim, hidden_dims, output_dim):
        super().__init__()
        
        layers = []
        prev_dim = input_dim
        
        for hidden_dim in hidden_dims:
            layers.append(nn.Linear(prev_dim, hidden_dim))
            layers.append(nn.BatchNorm1d(hidden_dim))  # Add BatchNorm
            layers.append(nn.ReLU())
            prev_dim = hidden_dim
        
        layers.append(nn.Linear(prev_dim, output_dim))
        self.network = nn.Sequential(*layers)
    
    def forward(self, x):
        return self.network(x)

# Usage
model = MLPWithBatchNorm(10, [128, 64], 1)
\end{lstlisting}

\textbf{Where to place BatchNorm?}

Two conventions:
\begin{enumerate}
    \item Linear → BatchNorm → Activation (more common)
    \item Linear → Activation → BatchNorm
\end{enumerate}

Both work; the first is more standard.

\textbf{Training vs Eval mode:}

\begin{lstlisting}
model.train()  # Use batch statistics
output = model(x)

model.eval()   # Use running statistics
with torch.no_grad():
    output = model(x)
\end{lstlisting}

\begin{warningbox}[BatchNorm Pitfalls]
\begin{enumerate}
    \item \textbf{Small batches:} BatchNorm works poorly with batch\_size < 8. Use LayerNorm instead.
    \item \textbf{Forgetting eval():} Model behaves differently in train/eval mode
    \item \textbf{Before or after activation:} Be consistent in your architecture
\end{enumerate}
\end{warningbox}

\subsubsection{Layer Normalization}

Normalize across features instead of batch:

\[
\hat{x} = \frac{x - \mu}{\sqrt{\sigma^2 + \epsilon}}
\]

where $\mu$, $\sigma$ are computed per sample across features.

\begin{lstlisting}
class MLPWithLayerNorm(nn.Module):
    def __init__(self, input_dim, hidden_dims, output_dim):
        super().__init__()
        
        layers = []
        prev_dim = input_dim
        
        for hidden_dim in hidden_dims:
            layers.append(nn.Linear(prev_dim, hidden_dim))
            layers.append(nn.LayerNorm(hidden_dim))  # LayerNorm
            layers.append(nn.ReLU())
            prev_dim = hidden_dim
        
        layers.append(nn.Linear(prev_dim, output_dim))
        self.network = nn.Sequential(*layers)
    
    def forward(self, x):
        return self.network(x)
\end{lstlisting}

\textbf{BatchNorm vs LayerNorm:}

\begin{table}[h]
\centering
\begin{tabular}{lll}
\toprule
\textbf{Aspect} & \textbf{BatchNorm} & \textbf{LayerNorm} \\
\midrule
Normalizes over & Batch dimension & Feature dimension \\
Requires batches & Yes (issues with batch=1) & No \\
Train/eval modes & Different & Same \\
Use in & CNNs, MLPs & Transformers, RNNs \\
Typical batch size & $\geq 8$ & Any (even 1) \\
\bottomrule
\end{tabular}
\caption{BatchNorm vs LayerNorm comparison}
\end{table}

\clearpage
\subsubsection{Dropout}

Randomly zero out neurons during training to prevent overfitting:

\begin{lstlisting}
class MLPWithDropout(nn.Module):
    def __init__(self, input_dim, hidden_dims, output_dim, dropout=0.5):
        super().__init__()
        
        layers = []
        prev_dim = input_dim
        
        for hidden_dim in hidden_dims:
            layers.append(nn.Linear(prev_dim, hidden_dim))
            layers.append(nn.ReLU())
            layers.append(nn.Dropout(dropout))  # Dropout after activation
            prev_dim = hidden_dim
        
        layers.append(nn.Linear(prev_dim, output_dim))
        self.network = nn.Sequential(*layers)
    
    def forward(self, x):
        return self.network(x)

# Usage
model = MLPWithDropout(10, [128, 64], 1, dropout=0.5)

# Training: dropout active
model.train()
output = model(x)  # Some neurons randomly zeroed

# Eval: dropout disabled
model.eval()
output = model(x)  # All neurons active, outputs scaled
\end{lstlisting}

\textbf{Dropout probabilities:}
\begin{itemize}
    \item Typical values: 0.2 - 0.5
    \item Higher dropout = more regularization (but too high hurts performance)
    \item Often use different dropout rates per layer (higher in later layers)
\end{itemize}

\textbf{Where to place dropout:}
\begin{itemize}
    \item After activation functions
    \item \textbf{Not} before the output layer (usually)
    \item Can place after input layer (input dropout, typically lower rate like 0.1)
\end{itemize}

\subsubsection{Complete MLP with All Techniques}

\begin{lstlisting}
class ModernMLP(nn.Module):
    """MLP with all modern techniques."""
    
    def __init__(self, input_dim, hidden_dims, output_dim,
                 activation='relu', dropout=0.0, use_batch_norm=True):
        super().__init__()
        
        self.input_bn = nn.BatchNorm1d(input_dim) if use_batch_norm else None
        
        layers = []
        prev_dim = input_dim
        
        for i, hidden_dim in enumerate(hidden_dims):
            # Linear layer
            layers.append(nn.Linear(prev_dim, hidden_dim))
            
            # Batch normalization
            if use_batch_norm:
                layers.append(nn.BatchNorm1d(hidden_dim))
            
            # Activation
            if activation == 'relu':
                layers.append(nn.ReLU())
            elif activation == 'gelu':
                layers.append(nn.GELU())
            elif activation == 'leaky_relu':
                layers.append(nn.LeakyReLU(0.01))
            
            # Dropout (if specified)
            if dropout > 0:
                layers.append(nn.Dropout(dropout))
            
            prev_dim = hidden_dim
        
        # Output layer
        layers.append(nn.Linear(prev_dim, output_dim))
        
        self.network = nn.Sequential(*layers)
        
        # Initialize weights
        self.apply(self._init_weights)
    
    def _init_weights(self, m):
        if isinstance(m, nn.Linear):
            init.kaiming_normal_(m.weight, nonlinearity='relu')
            if m.bias is not None:
                init.constant_(m.bias, 0)
    
    def forward(self, x):
        # Optional input batch norm
        if self.input_bn is not None:
            x = self.input_bn(x)
        return self.network(x)

# Usage: best practices
model = ModernMLP(
    input_dim=10,
    hidden_dims=[256, 128, 64],
    output_dim=1,
    activation='relu',
    dropout=0.3,
    use_batch_norm=True
)
\end{lstlisting}

\clearpage
% =============================================
% SECTION 6: MLPs - PART 2 (Debugging & Exercises)
% =============================================

\subsection{Implementation: Debugging MLPs}

\subsubsection{Checking for Dead ReLUs}

\begin{lstlisting}
def check_dead_neurons(model, dataloader, device):
    """Check percentage of dead ReLU neurons."""
    model.eval()
    
    # Hooks to capture activations
    activations = {}
    
    def hook_fn(name):
        def hook(module, input, output):
            if name not in activations:
                activations[name] = []
            activations[name].append((output == 0).float())
        return hook
    
    # Register hooks on ReLU layers
    hooks = []
    for name, module in model.named_modules():
        if isinstance(module, nn.ReLU):
            hooks.append(module.register_forward_hook(hook_fn(name)))
    
    # Forward pass on some data
    with torch.no_grad():
        for data, _ in dataloader:
            data = data.to(device)
            model(data)
            break  # Just one batch
    
    # Remove hooks
    for hook in hooks:
        hook.remove()
    
    # Compute dead neuron percentage
    for name, acts in activations.items():
        dead_pct = torch.cat(acts).mean().item() * 100
        print(f"{name}: {dead_pct:.2f}% neurons always zero")

# Usage
# check_dead_neurons(model, train_loader, device)
\end{lstlisting}

\subsubsection{Visualizing Activation Distributions}

\begin{lstlisting}
import matplotlib.pyplot as plt

def plot_activation_distributions(model, x):
    """Plot distribution of activations in each layer."""
    model.eval()
    
    activations = []
    
    def hook_fn(module, input, output):
        activations.append(output.detach().cpu().flatten())
    
    # Register hooks
    hooks = []
    for name, module in model.named_modules():
        if isinstance(module, (nn.Linear, nn.ReLU)):
            hooks.append(module.register_forward_hook(hook_fn))
    
    # Forward pass
    with torch.no_grad():
        model(x)
    
    # Remove hooks
    for hook in hooks:
        hook.remove()
    
    # Plot
    fig, axes = plt.subplots(len(activations), 1, 
                            figsize=(10, 3*len(activations)))
    for i, act in enumerate(activations):
        axes[i].hist(act.numpy(), bins=50)
        axes[i].set_title(f'Layer {i} activations')
        axes[i].set_xlabel('Value')
        axes[i].set_ylabel('Count')
    plt.tight_layout()
    plt.show()

# Usage
# x = torch.randn(100, 10)
# plot_activation_distributions(model, x)
\end{lstlisting}

\subsubsection{Gradient Flow Visualization}

\begin{lstlisting}
def plot_gradient_flow(model):
    """
    Plot gradient flow through network.
    Call after loss.backward() but before optimizer.step()
    """
    ave_grads = []
    max_grads = []
    layers = []
    
    for name, param in model.named_parameters():
        if param.requires_grad and param.grad is not None:
            layers.append(name)
            ave_grads.append(param.grad.abs().mean().item())
            max_grads.append(param.grad.abs().max().item())
    
    plt.figure(figsize=(12, 6))
    plt.bar(range(len(ave_grads)), ave_grads, alpha=0.5, label='mean')
    plt.bar(range(len(max_grads)), max_grads, alpha=0.5, label='max')
    plt.hlines(0, 0, len(ave_grads), linewidth=2, color='k')
    plt.xticks(range(len(layers)), layers, rotation='vertical')
    plt.xlim(-1, len(ave_grads))
    plt.xlabel('Layers')
    plt.ylabel('Gradient magnitude')
    plt.legend()
    plt.title('Gradient Flow')
    plt.grid(True)
    plt.tight_layout()
    plt.show()

# Usage in training loop
# output = model(x)
# loss = criterion(output, y)
# loss.backward()
# plot_gradient_flow(model)  # Before optimizer.step()
# optimizer.step()
\end{lstlisting}

\clearpage
\subsection{Exercises}

\begin{exercise}[6.1: Building Your First MLP - $\bigstar\bigstar$]
\textbf{Goal:} Create and train a basic MLP.

\begin{enumerate}
    \item Build an MLP: 20 → 64 → 32 → 1
    \item Use ReLU activations
    \item Train on synthetic data: $y = \sum_{i=1}^{20} x_i + \epsilon$
    \item Use 1000 samples, MSE loss, Adam optimizer
    \item Train for 100 epochs
    \item Plot training loss curve
\end{enumerate}

\textbf{Success criterion:} Final loss < 0.1

\textbf{Starter code:}
\begin{lstlisting}
import torch
import torch.nn as nn
import torch.optim as optim

# Generate data
X = torch.randn(1000, 20)
y = X.sum(dim=1, keepdim=True) + 0.1 * torch.randn(1000, 1)

# Your MLP here
model = ...

# Training loop here
\end{lstlisting}
\end{exercise}

\begin{exercise}[6.2: Activation Function Comparison - $\bigstar\bigstar\bigstar$]
\textbf{Goal:} Understand how different activations affect training.

Train identical MLPs (3 layers, 64 hidden units) with different activations:
\begin{enumerate}
    \item ReLU
    \item Leaky ReLU
    \item Tanh
    \item GELU
    \item Sigmoid (observe why this is bad for hidden layers!)
\end{enumerate}

For each:
\begin{itemize}
    \item Train on the same task (e.g., nonlinear function approximation)
    \item Track training loss over epochs
    \item Measure final test performance
    \item Plot all loss curves on the same graph
\end{itemize}

\textbf{Questions to answer:}
\begin{itemize}
    \item Which converges fastest?
    \item Which achieves best final performance?
    \item Why does sigmoid perform poorly?
\end{itemize}
\end{exercise}

\begin{exercise}[6.3: Initialization Impact - $\bigstar\bigstar\bigstar$]
\textbf{Goal:} See how initialization affects training.

Create the same MLP with different initializations:
\begin{enumerate}
    \item All zeros (model won't learn—why?)
    \item All ones (all neurons learn the same features)
    \item Random normal with $\sigma=0.01$ (too small)
    \item Random normal with $\sigma=1.0$ (too large)
    \item Xavier initialization
    \item He initialization (best for ReLU)
\end{enumerate}

For each:
\begin{itemize}
    \item Train for 50 epochs
    \item Track loss
    \item Check for NaN values
\end{itemize}

\textbf{Visualize:} Plot all loss curves. Which initializations fail? Which work best?
\end{exercise}

\begin{exercise}[6.4: Depth vs Width Experiment - $\bigstar\bigstar\bigstar$]
\textbf{Goal:} Compare deep vs wide networks.

Create networks with approximately the same number of parameters:
\begin{enumerate}
    \item \textbf{Shallow \& wide:} 2 layers, 256 neurons each
    \item \textbf{Deep \& narrow:} 8 layers, 64 neurons each
    \item \textbf{Medium:} 4 layers, 128 neurons each
\end{enumerate}

Train all three on a moderately complex task (e.g., 2D function approximation).

\textbf{Compare:}
\begin{itemize}
    \item Training speed (epochs to converge)
    \item Final test performance
    \item Generalization (train vs test loss gap)
\end{itemize}

\textbf{Challenge:} Add batch normalization to the deep network. Does it help?
\end{exercise}

\begin{exercise}[6.5: Regularization Techniques - $\bigstar\bigstar\bigstar\bigstar$]
\textbf{Goal:} Prevent overfitting with regularization.

\begin{enumerate}
    \item Create a small dataset (200 samples)
    \item Build a large MLP (can easily overfit): 10 → 128 → 128 → 64 → 1
    \item Train without regularization (observe overfitting)
    \item Add dropout (try 0.1, 0.3, 0.5)
    \item Add batch normalization
    \item Add weight decay
    \item Combine techniques
\end{enumerate}

For each variant:
\begin{itemize}
    \item Track train and validation loss
    \item Measure final test performance
    \item Plot train vs val loss curves
\end{itemize}

\textbf{Questions:}
\begin{itemize}
    \item Which technique helps most?
    \item What's the best combination?
    \item Can you achieve zero overfitting gap?
\end{itemize}
\end{exercise}

\begin{exercise}[6.6: Complex Function Approximation - $\bigstar\bigstar\bigstar\bigstar$]
\textbf{Goal:} Use MLP for scientific computing task.

Approximate a complex 2D function:
\[
f(x, y) = \sin(5x) \cos(5y) + 0.5\sin(10x) + 0.3\cos(10y)
\]

\begin{enumerate}
    \item Generate training data: sample $(x, y)$ uniformly in $[-1, 1]^2$
    \item Create MLP: 2 inputs → hidden layers → 1 output
    \item Train with MSE loss
    \item Visualize:
    \begin{itemize}
        \item True function as heatmap
        \item MLP prediction as heatmap
        \item Absolute error heatmap
    \end{itemize}
    \item Try different architectures (depth, width)
    \item Try different activations
\end{enumerate}

\textbf{Success criterion:} Mean absolute error < 0.05 on test set

\textbf{Starter code for visualization:}
\begin{lstlisting}
import numpy as np
import matplotlib.pyplot as plt

# Create grid
x = np.linspace(-1, 1, 100)
y = np.linspace(-1, 1, 100)
X, Y = np.meshgrid(x, y)

# True function
Z_true = np.sin(5*X)*np.cos(5*Y) + 0.5*np.sin(10*X) + 0.3*np.cos(10*Y)

# Model prediction
grid_points = torch.FloatTensor(np.stack([X.ravel(), Y.ravel()], axis=1))
with torch.no_grad():
    Z_pred = model(grid_points).numpy().reshape(X.shape)

# Plot
fig, axes = plt.subplots(1, 3, figsize=(15, 4))
axes[0].contourf(X, Y, Z_true, levels=20)
axes[0].set_title('True Function')
axes[1].contourf(X, Y, Z_pred, levels=20)
axes[1].set_title('MLP Prediction')
axes[2].contourf(X, Y, np.abs(Z_true - Z_pred), levels=20)
axes[2].set_title('Absolute Error')
plt.show()
\end{lstlisting}
\end{exercise}

\clearpage
\subsection{Key Takeaways}

\textbf{MLPs are versatile:}
\begin{itemize}
    \item Universal function approximators
    \item Foundation for understanding all neural networks
    \item Effective for tabular data and function approximation
\end{itemize}

\textbf{Design choices matter:}
\begin{itemize}
    \item \textbf{Activation:} ReLU is default, but consider alternatives for specific tasks
    \item \textbf{Depth vs Width:} Deeper networks more parameter-efficient but harder to train
    \item \textbf{Initialization:} He for ReLU, Xavier for Tanh, crucial for deep networks
    \item \textbf{Normalization:} BatchNorm stabilizes training, especially for deep networks
    \item \textbf{Regularization:} Dropout and weight decay prevent overfitting
\end{itemize}

\textbf{Common failure modes:}
\begin{itemize}
    \item Dead ReLUs → Use Leaky ReLU or proper initialization
    \item Vanishing gradients → Use ReLU, batch norm, or skip connections
    \item Exploding gradients → Use gradient clipping, lower LR, batch norm
    \item Overfitting → Add dropout, weight decay, or get more data
\end{itemize}

\textbf{Best practices for MLPs:}
\begin{enumerate}
    \item Start with 2-3 hidden layers
    \item Use ReLU activation
    \item Initialize with He initialization
    \item Add batch normalization for deep networks
    \item Use dropout (0.2-0.5) to prevent overfitting
    \item Monitor both train and validation loss
    \item Visualize activations and gradients when debugging
\end{enumerate}

\clearpage
% =============================================
% SECTION 7: CONVOLUTIONAL NEURAL NETWORKS
% =============================================

\section{Convolutional Neural Networks (CNNs)}

\subsection{Introduction: Why Convolutions?}

MLPs work well for tabular data, but they have serious limitations for spatial data like images:

\textbf{Problems with MLPs for images:}
\begin{itemize}
    \item \textbf{Too many parameters:} A 224×224 RGB image has 150,528 inputs. First hidden layer with 1000 neurons = 150M parameters!
    \item \textbf{No spatial structure:} Treats each pixel independently, ignoring that nearby pixels are related
    \item \textbf{Not translation invariant:} A cat in the top-left corner is different from a cat in the center
    \item \textbf{Can't generalize:} Learning a feature at one position doesn't help recognize it elsewhere
\end{itemize}

\textbf{Convolutions solve these problems through:}
\begin{enumerate}
    \item \textbf{Parameter sharing:} Same kernel applied everywhere (drastically fewer parameters)
    \item \textbf{Local connectivity:} Each neuron only looks at a small region (receptive field)
    \item \textbf{Translation invariance:} Features learned at one position work everywhere
    \item \textbf{Hierarchical learning:} Early layers detect edges, later layers detect complex patterns
\end{enumerate}

\textbf{When to use CNNs:}
\begin{itemize}
    \item Images (most common use case)
    \item Time series (1D convolutions)
    \item Volumetric data (3D medical scans, physical simulations)
    \item Spatial/temporal data with local correlations
\end{itemize}

\textbf{When NOT to use CNNs:}
\begin{itemize}
    \item Tabular data (no spatial structure)
    \item Very small spatial dimensions (convolutions overkill)
    \item When you need long-range dependencies only (use attention/Transformers)
\end{itemize}

\subsection{Theory: How Convolutions Work}

\subsubsection{The Convolution Operation}

A 2D convolution slides a \textbf{kernel} (or \textbf{filter}) over the input, computing dot products:

\[
\text{Output}[i, j] = \sum_{m=0}^{k-1} \sum_{n=0}^{k-1} \text{Input}[i+m, j+n] \cdot \text{Kernel}[m, n]
\]

\textbf{Example:} 3×3 kernel on 5×5 input

\begin{verbatim}
Input (5x5):              Kernel (3x3):         
1  2  3  4  5             1  0 -1
2  3  4  5  6      *      2  0 -2      
3  4  5  6  7             1  0 -1
4  5  6  7  8
5  6  7  8  9

Output (3x3):
-4  -4  -4
-4  -4  -4
-4  -4  -4
\end{verbatim}

This particular kernel is a vertical edge detector (Sobel filter).

\begin{theorybox}[Key Concepts]
\begin{itemize}
    \item \textbf{Kernel/Filter:} Small matrix of learnable weights (e.g., 3×3, 5×5)
    \item \textbf{Stride:} How many pixels to move the kernel each step (stride=1 → move 1 pixel)
    \item \textbf{Padding:} Add zeros around input to control output size
    \item \textbf{Channels:} Input depth (RGB image has 3 channels)
    \item \textbf{Feature maps:} Output of applying convolution (number = number of kernels)
\end{itemize}
\end{theorybox}

\subsubsection{Parameter Sharing}

A single 3×3 kernel has only 9 parameters, but is applied to every position in the image.

\textbf{MLP for 28×28 image:}
\begin{itemize}
    \item Input: 784 pixels
    \item Hidden layer: 128 neurons
    \item Parameters: $784 \times 128 = 100{,}352$
\end{itemize}

\textbf{CNN for same image:}
\begin{itemize}
    \item Input: 28×28×1
    \item Conv layer: 32 filters, 3×3 kernel
    \item Parameters: $3 \times 3 \times 1 \times 32 + 32 = 320$ (including biases)
\end{itemize}

\textbf{300× fewer parameters!}

\clearpage
\subsubsection{Receptive Fields}

The \textbf{receptive field} of a neuron is the region of the input it "sees."

\textbf{Example:} Two 3×3 conv layers

\begin{verbatim}
Layer 1: Each neuron sees 3×3 region
Layer 2: Each neuron sees 3×3 of Layer 1
         → Sees 5×5 region of original input!
\end{verbatim}

\textbf{Receptive field grows with depth:}
\begin{itemize}
    \item 1 layer (3×3 kernel): $3 \times 3 = 9$ pixels
    \item 2 layers: $5 \times 5 = 25$ pixels
    \item 3 layers: $7 \times 7 = 49$ pixels
    \item $n$ layers: $(2n+1) \times (2n+1)$ pixels (for 3×3 kernels, stride=1)
\end{itemize}

\textbf{Formula for receptive field:}
\[
r_n = r_{n-1} + (k - 1) \cdot \prod_{i=1}^{n-1} s_i
\]

where $k$ is kernel size, $s_i$ is stride at layer $i$.

\begin{pytorchtip}[Why Receptive Fields Matter]
For a network to classify an entire image, the final layer's receptive field must cover the whole image. This determines your network depth:

\textbf{Small images (32×32):} 5-10 layers sufficient

\textbf{Large images (224×224):} 15-20+ layers needed

Use pooling or larger strides to grow receptive field faster.
\end{pytorchtip}

\subsubsection{Output Size Calculation}

\textbf{Formula for output size:}
\[
\text{Output size} = \left\lfloor \frac{\text{Input size} + 2 \times \text{Padding} - \text{Kernel size}}{\text{Stride}} \right\rfloor + 1
\]

\textbf{Examples:}

\begin{table}[h]
\centering
\begin{tabular}{cccccc}
\toprule
\textbf{Input} & \textbf{Kernel} & \textbf{Stride} & \textbf{Padding} & \textbf{Output} & \textbf{Note} \\
\midrule
28 & 3 & 1 & 0 & 26 & Shrinks by 2 \\
28 & 3 & 1 & 1 & 28 & Same size \\
28 & 3 & 2 & 0 & 13 & Half size \\
28 & 5 & 1 & 2 & 28 & Same size \\
32 & 3 & 1 & 0 & 30 & Shrinks by 2 \\
\bottomrule
\end{tabular}
\caption{Output size examples}
\end{table}

\textbf{Common padding strategies:}
\begin{itemize}
    \item \textbf{Valid (no padding):} Output shrinks
    \item \textbf{Same:} Output size = input size (for stride=1)
    \item \textbf{Full:} Maximum padding (output grows)
\end{itemize}

\subsubsection{Pooling}

Pooling \textbf{downsamples} the spatial dimensions, making the network:
\begin{itemize}
    \item More computationally efficient
    \item More translation invariant
    \item With larger receptive fields
\end{itemize}

\textbf{Max Pooling:} Take maximum value in each window

\begin{verbatim}
Input (4x4):          Max Pool 2x2, stride=2:
1  2  3  4            
2  4  6  8      →     4   8
3  6  9  12           12  16
4  8  12 16
\end{verbatim}

\textbf{Average Pooling:} Take average value in each window

\begin{verbatim}
Input (4x4):          Avg Pool 2x2, stride=2:
1  2  3  4            
2  4  6  8      →     2.25  5.25
3  6  9  12           5.25  11.25
4  8  12 16
\end{verbatim}

\textbf{When to use each:}
\begin{itemize}
    \item \textbf{Max pooling:} Most common. Preserves strong features. Good for classification.
    \item \textbf{Average pooling:} Smoother. Good for final global pooling. Sometimes better for regression.
    \item \textbf{No pooling:} Use strided convolutions instead (more learnable, modern approach)
\end{itemize}

\clearpage
\subsection{Implementation: Building CNNs in PyTorch}

\subsubsection{Conv2d: The Core Layer}

\begin{lstlisting}
import torch
import torch.nn as nn

# Basic Conv2d
conv = nn.Conv2d(
    in_channels=3,      # Input channels (e.g., RGB = 3)
    out_channels=64,    # Number of filters (output channels)
    kernel_size=3,      # 3x3 kernel
    stride=1,           # Move 1 pixel at a time
    padding=1,          # Pad with 1 pixel of zeros (same size)
    bias=True           # Include bias term
)

# Input: (batch_size, channels, height, width)
x = torch.randn(8, 3, 32, 32)  # 8 RGB images, 32x32
output = conv(x)
print(output.shape)  # torch.Size([8, 64, 32, 32])
# 64 feature maps, same spatial size due to padding=1

# Number of parameters
params = conv.weight.numel() + conv.bias.numel()
print(f"Parameters: {params}")  # 3*3*3*64 + 64 = 1792
\end{lstlisting}

\textbf{Key parameters explained:}

\begin{itemize}
    \item \textbf{in\_channels:} Depth of input (3 for RGB, 1 for grayscale)
    \item \textbf{out\_channels:} Number of filters = depth of output
    \item \textbf{kernel\_size:} Can be int (3 → 3×3) or tuple (3, 5 → 3×5)
    \item \textbf{stride:} Step size. stride=2 halves spatial dimensions
    \item \textbf{padding:} 
        \begin{itemize}
            \item int: uniform padding
            \item tuple: (pad\_height, pad\_width)
            \item 'same': auto-pad to keep size (stride=1 only)
            \item 'valid': no padding
        \end{itemize}
    \item \textbf{dilation:} Spacing between kernel elements (1 = standard, >1 = dilated)
    \item \textbf{groups:} Divide channels into groups (1 = standard, in\_channels = depthwise)
\end{itemize}

\subsubsection{Shape Calculation Practice}

\begin{lstlisting}
# Practice calculating output shapes
x = torch.randn(1, 3, 64, 64)  # Single 64x64 RGB image

# Example 1: Standard convolution
conv1 = nn.Conv2d(3, 32, kernel_size=3, padding=1)
out1 = conv1(x)
print(out1.shape)  # [1, 32, 64, 64] - same size

# Example 2: Strided convolution (downsampling)
conv2 = nn.Conv2d(3, 32, kernel_size=3, stride=2, padding=1)
out2 = conv2(x)
print(out2.shape)  # [1, 32, 32, 32] - halved

# Example 3: No padding (shrinks)
conv3 = nn.Conv2d(3, 32, kernel_size=5, padding=0)
out3 = conv3(x)
print(out3.shape)  # [1, 32, 60, 60] - shrinks by 4

# Example 4: Large kernel
conv4 = nn.Conv2d(3, 32, kernel_size=7, padding=3)
out4 = conv4(x)
print(out4.shape)  # [1, 32, 64, 64] - same size

# Calculate expected output
def calc_output_size(input_size, kernel_size, stride, padding):
    return (input_size + 2*padding - kernel_size) // stride + 1

print(calc_output_size(64, 3, 1, 1))  # 64
print(calc_output_size(64, 3, 2, 1))  # 32
print(calc_output_size(64, 5, 1, 0))  # 60
\end{lstlisting}

\begin{warningbox}[Common Shape Mistakes]
\textbf{Mistake 1:} Forgetting to flatten before Linear layer
\begin{lstlisting}
# WRONG
output = conv_layers(x)  # Shape: (batch, channels, H, W)
output = fc(output)      # ERROR! Wrong shape

# CORRECT
output = conv_layers(x)
output = output.view(output.size(0), -1)  # Flatten
output = fc(output)  # OK
\end{lstlisting}

\textbf{Mistake 2:} Wrong channel order (PyTorch uses NCHW, not NHWC)
\begin{lstlisting}
# PyTorch expects: (batch, channels, height, width)
x = torch.randn(8, 32, 32, 3)  # WRONG order
x = x.permute(0, 3, 1, 2)      # Fix to (8, 3, 32, 32)
\end{lstlisting}
\end{warningbox}

\clearpage
\subsubsection{Building a Simple CNN}

\begin{lstlisting}
class SimpleCNN(nn.Module):
    """Simple CNN for MNIST (28x28 grayscale images)."""
    
    def __init__(self, num_classes=10):
        super().__init__()
        
        # Convolutional layers
        self.conv1 = nn.Conv2d(1, 32, kernel_size=3, padding=1)
        # Output: 32 x 28 x 28
        
        self.conv2 = nn.Conv2d(32, 64, kernel_size=3, padding=1)
        # Output: 64 x 28 x 28
        
        self.pool = nn.MaxPool2d(2, 2)  # 2x2 pooling
        # After pool: 64 x 14 x 14
        
        self.conv3 = nn.Conv2d(64, 128, kernel_size=3, padding=1)
        # Output: 128 x 14 x 14
        # After pool: 128 x 7 x 7
        
        # Fully connected layers
        self.fc1 = nn.Linear(128 * 7 * 7, 256)
        self.fc2 = nn.Linear(256, num_classes)
        
        self.relu = nn.ReLU()
        self.dropout = nn.Dropout(0.5)
    
    def forward(self, x):
        # Conv block 1
        x = self.relu(self.conv1(x))
        x = self.pool(x)
        
        # Conv block 2
        x = self.relu(self.conv2(x))
        x = self.pool(x)
        
        # Conv block 3
        x = self.relu(self.conv3(x))
        
        # Flatten
        x = x.view(x.size(0), -1)  # or x.flatten(1)
        
        # Fully connected
        x = self.relu(self.fc1(x))
        x = self.dropout(x)
        x = self.fc2(x)
        
        return x

# Test
model = SimpleCNN()
x = torch.randn(4, 1, 28, 28)  # 4 grayscale 28x28 images
output = model(x)
print(output.shape)  # torch.Size([4, 10])
\end{lstlisting}

\subsubsection{Modern CNN with Batch Normalization}

\begin{lstlisting}
class ModernCNN(nn.Module):
    """CNN with batch normalization and modern practices."""
    
    def __init__(self, num_classes=10):
        super().__init__()
        
        self.features = nn.Sequential(
            # Block 1: 1 -> 32
            nn.Conv2d(1, 32, 3, padding=1),
            nn.BatchNorm2d(32),
            nn.ReLU(),
            nn.Conv2d(32, 32, 3, padding=1),
            nn.BatchNorm2d(32),
            nn.ReLU(),
            nn.MaxPool2d(2, 2),  # 28x28 -> 14x14
            
            # Block 2: 32 -> 64
            nn.Conv2d(32, 64, 3, padding=1),
            nn.BatchNorm2d(64),
            nn.ReLU(),
            nn.Conv2d(64, 64, 3, padding=1),
            nn.BatchNorm2d(64),
            nn.ReLU(),
            nn.MaxPool2d(2, 2),  # 14x14 -> 7x7
            
            # Block 3: 64 -> 128
            nn.Conv2d(64, 128, 3, padding=1),
            nn.BatchNorm2d(128),
            nn.ReLU(),
            nn.Conv2d(128, 128, 3, padding=1),
            nn.BatchNorm2d(128),
            nn.ReLU(),
        )
        
        self.classifier = nn.Sequential(
            nn.AdaptiveAvgPool2d((1, 1)),  # Global average pooling
            nn.Flatten(),
            nn.Linear(128, 256),
            nn.ReLU(),
            nn.Dropout(0.5),
            nn.Linear(256, num_classes)
        )
    
    def forward(self, x):
        x = self.features(x)
        x = self.classifier(x)
        return x

# Test
model = ModernCNN()
x = torch.randn(4, 1, 28, 28)
output = model(x)
print(output.shape)  # torch.Size([4, 10])
\end{lstlisting}

\begin{pytorchtip}[AdaptiveAvgPool2d vs Flatten]
\textbf{Old approach:} Fixed spatial size before FC layer
\begin{lstlisting}
x = x.view(batch_size, -1)  # Must know spatial size
fc = nn.Linear(128 * 7 * 7, num_classes)
\end{lstlisting}

\textbf{Modern approach:} Global pooling adapts to any input size
\begin{lstlisting}
x = nn.AdaptiveAvgPool2d((1, 1))(x)  # Always outputs 1x1
x = x.flatten(1)  # (batch, 128, 1, 1) -> (batch, 128)
fc = nn.Linear(128, num_classes)
\end{lstlisting}

Benefit: Same network works for different input sizes (e.g., 28×28 or 32×32).
\end{pytorchtip}

\clearpage
\subsubsection{1D Convolutions for Time Series}

\begin{lstlisting}
class TimeSeriesCNN(nn.Module):
    """1D CNN for time series or sequence data."""
    
    def __init__(self, input_channels=1, num_classes=10):
        super().__init__()
        
        self.conv1 = nn.Conv1d(
            in_channels=input_channels,
            out_channels=64,
            kernel_size=7,  # Look at 7 time steps
            padding=3
        )
        
        self.conv2 = nn.Conv1d(64, 128, kernel_size=5, padding=2)
        self.conv3 = nn.Conv1d(128, 256, kernel_size=3, padding=1)
        
        self.pool = nn.MaxPool1d(2)  # Downsample by 2
        
        self.global_pool = nn.AdaptiveAvgPool1d(1)
        self.fc = nn.Linear(256, num_classes)
        
        self.relu = nn.ReLU()
    
    def forward(self, x):
        # x shape: (batch, channels, sequence_length)
        
        x = self.relu(self.conv1(x))
        x = self.pool(x)
        
        x = self.relu(self.conv2(x))
        x = self.pool(x)
        
        x = self.relu(self.conv3(x))
        
        # Global pooling
        x = self.global_pool(x)  # (batch, 256, 1)
        x = x.squeeze(-1)        # (batch, 256)
        
        x = self.fc(x)
        return x

# Test on time series
model = TimeSeriesCNN(input_channels=1)
x = torch.randn(32, 1, 100)  # 32 samples, 1 channel, 100 time steps
output = model(x)
print(output.shape)  # torch.Size([32, 10])
\end{lstlisting}

\textbf{When to use 1D convolutions:}
\begin{itemize}
    \item Time series data (sensor readings, stock prices)
    \item Audio signals (raw waveforms)
    \item Text (character or word level)
    \item Any 1D sequence with local correlations
\end{itemize}

\subsubsection{3D Convolutions for Volumetric Data}

\begin{lstlisting}
class VolumetricCNN(nn.Module):
    """3D CNN for volumetric data (medical scans, video, 3D simulations)."""
    
    def __init__(self, in_channels=1, num_classes=2):
        super().__init__()
        
        self.conv1 = nn.Conv3d(in_channels, 32, kernel_size=3, padding=1)
        self.conv2 = nn.Conv3d(32, 64, kernel_size=3, padding=1)
        self.conv3 = nn.Conv3d(64, 128, kernel_size=3, padding=1)
        
        self.pool = nn.MaxPool3d(2)  # Downsample all 3 dimensions
        
        self.global_pool = nn.AdaptiveAvgPool3d((1, 1, 1))
        self.fc = nn.Linear(128, num_classes)
        
        self.relu = nn.ReLU()
    
    def forward(self, x):
        # x shape: (batch, channels, depth, height, width)
        
        x = self.relu(self.conv1(x))
        x = self.pool(x)
        
        x = self.relu(self.conv2(x))
        x = self.pool(x)
        
        x = self.relu(self.conv3(x))
        
        x = self.global_pool(x)
        x = x.view(x.size(0), -1)
        
        x = self.fc(x)
        return x

# Test on 3D volume
model = VolumetricCNN()
x = torch.randn(4, 1, 32, 32, 32)  # 4 volumes, 32x32x32
output = model(x)
print(output.shape)  # torch.Size([4, 2])
\end{lstlisting}

\textbf{When to use 3D convolutions:}
\begin{itemize}
    \item Medical imaging (CT scans, MRI)
    \item Video (treat time as 3rd spatial dimension)
    \item 3D physical simulations
    \item Molecular structure analysis
\end{itemize}

\textbf{Warning:} 3D convolutions are very memory-intensive! Use smaller batch sizes.

\clearpage
% =============================================
% SECTION 7: CNNs - PART 2 (Advanced Topics & Exercises)
% =============================================

\subsubsection{Transposed Convolutions (Upsampling)}

Transposed convolutions (sometimes called "deconvolutions") \textbf{increase} spatial dimensions. Used in:
\begin{itemize}
    \item Autoencoders (decoder part)
    \item Generative models (GANs, VAEs)
    \item Semantic segmentation (U-Net)
    \item Super-resolution
\end{itemize}

\begin{lstlisting}
# Regular convolution (downsampling)
conv = nn.Conv2d(64, 128, kernel_size=3, stride=2, padding=1)
x = torch.randn(1, 64, 32, 32)
out = conv(x)
print(out.shape)  # torch.Size([1, 128, 16, 16]) - halved

# Transposed convolution (upsampling)
deconv = nn.ConvTranspose2d(128, 64, kernel_size=3, stride=2, padding=1,
                            output_padding=1)
out2 = deconv(out)
print(out2.shape)  # torch.Size([1, 64, 32, 32]) - doubled back
\end{lstlisting}

\textbf{Output size calculation:}
\[
\text{Output size} = (\text{Input size} - 1) \times \text{Stride} - 2 \times \text{Padding} + \text{Kernel size} + \text{Output padding}
\]

\textbf{Simple U-Net style autoencoder:}

\begin{lstlisting}
class SimpleAutoencoder(nn.Module):
    """Convolutional autoencoder for images."""
    
    def __init__(self):
        super().__init__()
        
        # Encoder (downsampling)
        self.encoder = nn.Sequential(
            nn.Conv2d(1, 32, 3, stride=2, padding=1),  # 28 -> 14
            nn.ReLU(),
            nn.Conv2d(32, 64, 3, stride=2, padding=1), # 14 -> 7
            nn.ReLU(),
            nn.Conv2d(64, 128, 3, stride=2, padding=1), # 7 -> 4 (rounds up)
            nn.ReLU(),
        )
        
        # Decoder (upsampling)
        self.decoder = nn.Sequential(
            nn.ConvTranspose2d(128, 64, 3, stride=2, padding=1, 
                              output_padding=1),  # 4 -> 7
            nn.ReLU(),
            nn.ConvTranspose2d(64, 32, 3, stride=2, padding=1, 
                              output_padding=1),  # 7 -> 14
            nn.ReLU(),
            nn.ConvTranspose2d(32, 1, 3, stride=2, padding=1, 
                              output_padding=1),  # 14 -> 28
            nn.Sigmoid()  # Output in [0, 1]
        )
    
    def forward(self, x):
        encoded = self.encoder(x)
        decoded = self.decoder(encoded)
        return decoded

# Test
model = SimpleAutoencoder()
x = torch.randn(4, 1, 28, 28)
reconstructed = model(x)
print(reconstructed.shape)  # torch.Size([4, 1, 28, 28])
\end{lstlisting}

\begin{warningbox}[Checkerboard Artifacts]
Transposed convolutions can create checkerboard artifacts (visible patterns in output).

\textbf{Solution:} Use \textbf{resize + convolution} instead:
\begin{lstlisting}
# Instead of ConvTranspose2d
nn.ConvTranspose2d(64, 32, kernel_size=4, stride=2, padding=1)

# Use this (better quality)
nn.Sequential(
    nn.Upsample(scale_factor=2, mode='bilinear', align_corners=False),
    nn.Conv2d(64, 32, kernel_size=3, padding=1)
)
\end{lstlisting}
\end{warningbox}

\subsubsection{Dilated Convolutions}

Dilated (atrous) convolutions increase receptive field without increasing parameters or losing resolution.

\begin{lstlisting}
# Standard 3x3 convolution
conv_standard = nn.Conv2d(64, 64, kernel_size=3, padding=1)
# Receptive field: 3x3

# Dilated 3x3 convolution (dilation=2)
conv_dilated = nn.Conv2d(64, 64, kernel_size=3, padding=2, dilation=2)
# Receptive field: 5x5 (with gaps)
# Same parameters as standard 3x3!

# Effective kernel size: k + (k-1)*(d-1)
# For k=3, d=2: 3 + (3-1)*(2-1) = 5
\end{lstlisting}

\textbf{When to use dilated convolutions:}
\begin{itemize}
    \item Semantic segmentation (maintain resolution while growing receptive field)
    \item Dense prediction tasks
    \item When you need large receptive field but want to keep resolution
\end{itemize}

\clearpage
\subsubsection{Common CNN Architectures}

\textbf{VGG-style (stacking blocks):}

\begin{lstlisting}
def vgg_block(in_channels, out_channels, num_convs):
    """VGG-style block: multiple convs + pool."""
    layers = []
    for _ in range(num_convs):
        layers.append(nn.Conv2d(in_channels, out_channels, 
                               kernel_size=3, padding=1))
        layers.append(nn.ReLU())
        in_channels = out_channels
    layers.append(nn.MaxPool2d(2, 2))
    return nn.Sequential(*layers)

class VGGStyleNet(nn.Module):
    def __init__(self, num_classes=10):
        super().__init__()
        
        self.features = nn.Sequential(
            vgg_block(3, 64, 2),    # 2 convs, 64 filters
            vgg_block(64, 128, 2),  # 2 convs, 128 filters
            vgg_block(128, 256, 3), # 3 convs, 256 filters
            vgg_block(256, 512, 3), # 3 convs, 512 filters
        )
        
        self.classifier = nn.Sequential(
            nn.AdaptiveAvgPool2d((1, 1)),
            nn.Flatten(),
            nn.Linear(512, 512),
            nn.ReLU(),
            nn.Dropout(0.5),
            nn.Linear(512, num_classes)
        )
    
    def forward(self, x):
        x = self.features(x)
        x = self.classifier(x)
        return x
\end{lstlisting}

\textbf{Key design patterns:}
\begin{enumerate}
    \item \textbf{Double channels after pooling:} 64 → 128 → 256 → 512
    \item \textbf{Multiple convs per block:} Learn richer features before downsampling
    \item \textbf{Small kernels (3×3):} More efficient than large kernels
    \item \textbf{Batch norm after conv:} Stabilizes training
    \item \textbf{Global pooling:} More flexible than fixed FC input size
\end{enumerate}

\subsection{Implementation: Debugging CNNs}

\subsubsection{Visualizing Feature Maps}

\begin{lstlisting}
def visualize_feature_maps(model, x, layer_num=0):
    """Visualize activations from a specific layer."""
    activation = {}
    
    def get_activation(name):
        def hook(model, input, output):
            activation[name] = output.detach()
        return hook
    
    # Register hook
    layer_name = f'features.{layer_num}'
    handle = dict(model.named_modules())[layer_name].register_forward_hook(
        get_activation(layer_name))
    
    # Forward pass
    model.eval()
    with torch.no_grad():
        _ = model(x)
    
    handle.remove()
    
    # Visualize
    act = activation[layer_name].squeeze()  # Remove batch dim
    
    import matplotlib.pyplot as plt
    fig, axes = plt.subplots(4, 8, figsize=(16, 8))
    for i, ax in enumerate(axes.flat):
        if i < act.shape[0]:
            ax.imshow(act[i].cpu(), cmap='viridis')
            ax.axis('off')
    plt.tight_layout()
    plt.show()

# Usage
# model = ModernCNN()
# x = torch.randn(1, 1, 28, 28)
# visualize_feature_maps(model, x, layer_num=0)
\end{lstlisting}

\subsubsection{Common CNN Debugging Issues}

\textbf{Problem: Output shape wrong}

\begin{lstlisting}
# Debug shape through network
def debug_shapes(model, input_shape):
    """Print shapes through the network."""
    x = torch.randn(*input_shape)
    print(f"Input: {x.shape}")
    
    for name, module in model.named_children():
        x = module(x)
        print(f"{name}: {x.shape}")

# Usage
model = SimpleCNN()
debug_shapes(model, (1, 1, 28, 28))
"""
Input: torch.Size([1, 1, 28, 28])
conv1: torch.Size([1, 32, 28, 28])
pool: torch.Size([1, 32, 14, 14])
...
"""
\end{lstlisting}

\textbf{Problem: Memory issues with 3D convolutions}

\begin{lstlisting}
# Use gradient checkpointing for large models
from torch.utils.checkpoint import checkpoint

class MemoryEfficientCNN(nn.Module):
    def __init__(self):
        super().__init__()
        self.conv1 = nn.Conv3d(1, 32, 3)
        self.conv2 = nn.Conv3d(32, 64, 3)
        # ... more layers
    
    def forward(self, x):
        # Checkpoint expensive layers
        x = checkpoint(self.conv1, x)
        x = checkpoint(self.conv2, x)
        return x
\end{lstlisting}

\textbf{Tips for reducing memory:}
\begin{itemize}
    \item Use smaller batch sizes
    \item Use mixed precision training (FP16)
    \item Reduce number of filters
    \item Use gradient checkpointing
    \item Process data in patches (for very large images)
\end{itemize}

\clearpage
\subsection{Exercises}

\begin{exercise}[7.1: Basic CNN for MNIST - $\bigstar\bigstar$]
\textbf{Goal:} Build and train your first CNN.

\begin{enumerate}
    \item Build a CNN for MNIST: 2 conv layers + 2 FC layers
    \item Use 32 and 64 filters, 3×3 kernels, ReLU, max pooling
    \item Train for 5 epochs
    \item Achieve >95\% test accuracy
    \item Visualize some predictions
\end{enumerate}

\textbf{Starter code:}
\begin{lstlisting}
from torchvision import datasets, transforms

# Load MNIST
transform = transforms.ToTensor()
train_dataset = datasets.MNIST('./data', train=True, 
                              download=True, transform=transform)
test_dataset = datasets.MNIST('./data', train=False, transform=transform)

# Your CNN here
class MyCNN(nn.Module):
    def __init__(self):
        super().__init__()
        # Your architecture
    
    def forward(self, x):
        # Your forward pass
        pass
\end{lstlisting}
\end{exercise}

\begin{exercise}[7.2: Shape Calculation Practice - $\bigstar\bigstar$]
\textbf{Goal:} Master output size calculations.

For a 32×32 input image, calculate output sizes for:
\begin{enumerate}
    \item Conv(3×3, stride=1, padding=1)
    \item Conv(5×5, stride=2, padding=2)
    \item Conv(3×3, stride=1, padding=0) followed by MaxPool(2×2)
    \item Three consecutive Conv(3×3, stride=1, padding=1) layers
    \item Conv(3×3, stride=1, padding=1, dilation=2)
\end{enumerate}

Then implement these layers and verify your calculations with actual tensors.
\end{exercise}

\begin{exercise}[7.3: 1D CNN for Time Series - $\bigstar\bigstar\bigstar$]
\textbf{Goal:} Apply convolutions to sequence data.

\begin{enumerate}
    \item Generate synthetic time series: $y = \sin(x) + 0.5\sin(3x) + \text{noise}$
    \item Create dataset: windows of 100 points predict next 10 points
    \item Build a 1D CNN: 3 conv layers with increasing filters
    \item Train and compare with a simple MLP baseline
    \item Visualize predictions on test data
\end{enumerate}

\textbf{Hint:} Use \texttt{nn.Conv1d} with kernel\_size=7 or 9.
\end{exercise}

\begin{exercise}[7.4: VGG-Style Network - $\bigstar\bigstar\bigstar$]
\textbf{Goal:} Build a deeper network with proper architecture.

\begin{enumerate}
    \item Implement a VGG-style network:
    \begin{itemize}
        \item Block 1: 2×Conv(64) + Pool
        \item Block 2: 2×Conv(128) + Pool
        \item Block 3: 3×Conv(256) + Pool
        \item Classifier: Global pooling + FC layers
    \end{itemize}
    \item Add batch normalization after each conv
    \item Train on CIFAR-10
    \item Achieve >70\% test accuracy
\end{enumerate}

\textbf{Challenge:} Add dropout and compare with/without it.
\end{exercise}

\begin{exercise}[7.5: Simple U-Net - $\bigstar\bigstar\bigstar\bigstar$]
\textbf{Goal:} Implement encoder-decoder architecture.

Build a simple U-Net for image denoising:
\begin{enumerate}
    \item Encoder: 3 conv blocks with downsampling
    \item Decoder: 3 transposed conv blocks with upsampling
    \item Add skip connections (concatenate encoder features with decoder)
    \item Train on MNIST with added Gaussian noise
    \item Visualize: noisy input → denoised output → clean target
\end{enumerate}

\textbf{Architecture:}
\begin{verbatim}
Encoder:        Decoder:
1 -> 32 -------> concat -> 32 -> 1
     |                      |
32 -> 64 -------> concat -> 64 -> 32
     |                      |
64 -> 128 -----> 128 -> 64
\end{verbatim}

\textbf{Starter code:}
\begin{lstlisting}
class SimpleUNet(nn.Module):
    def __init__(self):
        super().__init__()
        # Encoder
        self.enc1 = self.conv_block(1, 32)
        self.enc2 = self.conv_block(32, 64)
        self.enc3 = self.conv_block(64, 128)
        
        # Decoder
        self.dec3 = self.upconv_block(128, 64)
        self.dec2 = self.upconv_block(128, 32)  # 128 = 64 + 64 from skip
        self.dec1 = nn.Conv2d(64, 1, 1)  # 64 = 32 + 32 from skip
        
        self.pool = nn.MaxPool2d(2)
    
    def conv_block(self, in_ch, out_ch):
        return nn.Sequential(
            nn.Conv2d(in_ch, out_ch, 3, padding=1),
            nn.BatchNorm2d(out_ch),
            nn.ReLU()
        )
    
    def upconv_block(self, in_ch, out_ch):
        return nn.Sequential(
            nn.ConvTranspose2d(in_ch, out_ch, 2, stride=2),
            nn.BatchNorm2d(out_ch),
            nn.ReLU()
        )
    
    def forward(self, x):
        # Encoder with skip connections
        e1 = self.enc1(x)
        e2 = self.enc2(self.pool(e1))
        e3 = self.enc3(self.pool(e2))
        
        # Decoder with skip connections
        d3 = self.dec3(e3)
        d2 = self.dec2(torch.cat([d3, e2], dim=1))  # Skip connection
        d1 = self.dec1(torch.cat([d2, e1], dim=1))  # Skip connection
        
        return torch.sigmoid(d1)
\end{lstlisting}
\end{exercise}

\begin{exercise}[7.6: 3D CNN for Volumetric Data - $\bigstar\bigstar\bigstar\bigstar$]
\textbf{Goal:} Work with 3D convolutions.

\begin{enumerate}
    \item Generate synthetic 3D data: 3D Gaussian blobs as "tumors"
    \item Build a 3D CNN classifier: 3 conv layers + global pooling + FC
    \item Handle memory constraints (small batch size, fewer filters)
    \item Achieve good classification accuracy
    \item Visualize 3D volumes and predictions (use slicing)
\end{enumerate}

\textbf{Data generation:}
\begin{lstlisting}
def generate_3d_blob(size=32):
    """Generate 3D volume with central blob."""
    volume = np.zeros((size, size, size))
    center = size // 2
    
    # Create Gaussian blob
    for i in range(size):
        for j in range(size):
            for k in range(size):
                dist = np.sqrt((i-center)**2 + (j-center)**2 + 
                              (k-center)**2)
                volume[i, j, k] = np.exp(-dist**2 / 100)
    
    return torch.FloatTensor(volume).unsqueeze(0)  # Add channel dim
\end{lstlisting}

\textbf{Warning:} Use batch\_size=2 or 4 for 3D convolutions!
\end{exercise}

\clearpage
\subsection{Key Takeaways}

\textbf{Why CNNs work:}
\begin{itemize}
    \item \textbf{Parameter sharing:} Drastically fewer parameters than MLPs
    \item \textbf{Translation invariance:} Features work regardless of position
    \item \textbf{Hierarchical learning:} Low-level → high-level features
    \item \textbf{Local connectivity:} Exploits spatial structure
\end{itemize}

\textbf{Architecture design principles:}
\begin{itemize}
    \item Start with 32-64 filters, double after each pooling
    \item Use small kernels (3×3) rather than large ones
    \item Add batch normalization for deep networks
    \item Use strided convolutions or pooling for downsampling
    \item Global pooling at the end for flexibility
\end{itemize}

\textbf{Convolution variants:}
\begin{itemize}
    \item \textbf{1D:} Time series, audio, text sequences
    \item \textbf{2D:} Images (most common)
    \item \textbf{3D:} Videos, medical scans, volumetric data
    \item \textbf{Transposed:} Upsampling in decoder/generator
    \item \textbf{Dilated:} Large receptive field without losing resolution
\end{itemize}

\textbf{Common patterns:}
\begin{itemize}
    \item \textbf{VGG-style:} Stack conv blocks, double channels after pooling
    \item \textbf{Encoder-decoder:} Downsample then upsample (autoencoders, U-Net)
    \item \textbf{Skip connections:} Connect encoder to decoder (U-Net, ResNet)
\end{itemize}

\textbf{Debugging tips:}
\begin{itemize}
    \item Always check shapes at each layer
    \item Visualize feature maps to understand what network learns
    \item Start small (few layers, few filters) then scale up
    \item Use \texttt{AdaptiveAvgPool2d} for flexibility
    \item Watch out for memory with 3D convolutions
\end{itemize}

\clearpage
% =============================================
% SECTION 8: RESIDUAL NETWORKS & SKIP CONNECTIONS
% =============================================

\section{Residual Networks \& Skip Connections}

\subsection{Introduction: The Deep Network Problem}

In the early 2010s, researchers discovered something puzzling: making networks deeper didn't always make them better. Sometimes, a 56-layer network performed \textbf{worse} than a 20-layer network—even on training data!

This wasn't overfitting (test loss was also worse). It was a fundamental optimization problem.

\textbf{The Degradation Problem:}
\begin{itemize}
    \item Very deep networks are harder to optimize
    \item Gradients vanish or explode
    \item Training error increases with depth (counterintuitively)
    \item Not caused by overfitting—happens on training set too
\end{itemize}

\textbf{Residual Networks (ResNets) solved this in 2015,} enabling networks with 100+ layers that actually train better than shallow ones. The key insight: \textbf{skip connections} (also called \textbf{residual connections}).

\textbf{Impact:}
\begin{itemize}
    \item ResNet-152 won ImageNet 2015 (3.57\% error, human-level performance)
    \item Now standard in almost all vision architectures
    \item Fundamental principle used across all deep learning (Transformers, etc.)
\end{itemize}

\subsection{Theory: Why Skip Connections Work}

\subsubsection{The Problem: Vanishing Gradients}

In a deep network without skip connections:

\[
\mathbf{x}_L = f_L(f_{L-1}(...f_2(f_1(\mathbf{x}_0))))
\]

During backpropagation, gradients multiply through the chain rule:

\[
\frac{\partial \mathcal{L}}{\partial \mathbf{x}_0} = \frac{\partial \mathcal{L}}{\partial \mathbf{x}_L} \cdot \frac{\partial \mathbf{x}_L}{\partial \mathbf{x}_{L-1}} \cdot ... \cdot \frac{\partial \mathbf{x}_2}{\partial \mathbf{x}_1} \cdot \frac{\partial \mathbf{x}_1}{\partial \mathbf{x}_0}
\]

If each term is less than 1, the gradient \textbf{vanishes} exponentially as it propagates backward.

\textbf{Example:} If each layer multiplies gradient by 0.9:
\begin{itemize}
    \item After 10 layers: $0.9^{10} \approx 0.35$
    \item After 20 layers: $0.9^{20} \approx 0.12$
    \item After 50 layers: $0.9^{50} \approx 0.005$ (essentially zero!)
\end{itemize}

Early layers barely learn, making deep networks ineffective.

\subsubsection{The Solution: Residual Connections}

Instead of learning a direct mapping $\mathbf{y} = \mathcal{F}(\mathbf{x})$, learn a \textbf{residual}:

\[
\mathbf{y} = \mathcal{F}(\mathbf{x}) + \mathbf{x}
\]

The network learns the \textbf{difference} (residual) from the identity mapping.

\begin{theorybox}[Key Insight: Identity Mapping]
With a skip connection, the network can learn the identity function by simply setting $\mathcal{F}(\mathbf{x}) = 0$ (all weights to zero).

\textbf{Why this matters:}
\begin{itemize}
    \item Identity is easy to learn (do nothing)
    \item Network starts with identity, then learns refinements
    \item Even if $\mathcal{F}$ is poorly initialized, gradients still flow through skip connection
    \item Optimization becomes: "How should I modify the input?" instead of "What should the output be?"
\end{itemize}
\end{theorybox}

\subsubsection{Gradient Flow Through Skip Connections}

During backpropagation with skip connections:

\[
\frac{\partial \mathcal{L}}{\partial \mathbf{x}} = \frac{\partial \mathcal{L}}{\partial \mathbf{y}} \left( \frac{\partial \mathcal{F}}{\partial \mathbf{x}} + 1 \right)
\]

The $+1$ term creates a \textbf{gradient highway}:
\begin{itemize}
    \item Gradients always have a direct path backward (through the $+1$)
    \item Even if $\frac{\partial \mathcal{F}}{\partial \mathbf{x}} \approx 0$, gradients still flow
    \item No vanishing gradient problem!
\end{itemize}

\textbf{Analogy:} Regular network = narrow mountain trail (one path, easy to get stuck)

Skip connections = trail + highway (gradient can always get through)

\clearpage
\subsubsection{Residual Block Variants}

\textbf{Basic Residual Block (Original):}

\begin{verbatim}
Input (x)
   |
   ├─────────────┐  (skip connection)
   |             |
 Conv 3x3        |
   |             |
 BatchNorm       |
   |             |
  ReLU           |
   |             |
 Conv 3x3        |
   |             |
 BatchNorm       |
   |             |
   +─────────────┘  (addition)
   |
  ReLU
   |
Output (y)
\end{verbatim}

\textbf{Pre-Activation Residual Block (Improved):}

\begin{verbatim}
Input (x)
   |
   ├─────────────┐  (skip connection)
   |             |
 BatchNorm       |
   |             |
  ReLU           |
   |             |
 Conv 3x3        |
   |             |
 BatchNorm       |
   |             |
  ReLU           |
   |             |
 Conv 3x3        |
   |             |
   +─────────────┘  (addition)
   |
Output (y)
\end{verbatim}

\textbf{Difference:}
\begin{itemize}
    \item \textbf{Post-activation:} Conv → BN → ReLU (then add)
    \item \textbf{Pre-activation:} BN → ReLU → Conv (then add)
\end{itemize}

\textbf{Pre-activation advantages:}
\begin{itemize}
    \item Better gradient flow (identity mapping is truly unimpeded)
    \item Easier optimization
    \item Better for very deep networks (>100 layers)
\end{itemize}

\subsubsection{When Dimensions Don't Match}

If input and output have different dimensions, use a \textbf{projection shortcut}:

\[
\mathbf{y} = \mathcal{F}(\mathbf{x}) + \mathbf{W}_s\mathbf{x}
\]

where $\mathbf{W}_s$ is a linear projection (1×1 convolution).

\textbf{Two approaches:}

\begin{enumerate}
    \item \textbf{Identity shortcut + zero padding:} Pad channels with zeros (parameter-free)
    \item \textbf{Projection shortcut:} Use 1×1 conv to match dimensions (adds parameters)
\end{enumerate}

Most implementations use projection shortcuts when dimensions change.

\subsection{Implementation: Building Residual Blocks}

\subsubsection{Basic Residual Block}

\begin{lstlisting}
import torch
import torch.nn as nn

class BasicBlock(nn.Module):
    """Basic residual block with two 3x3 convolutions."""
    
    def __init__(self, in_channels, out_channels, stride=1):
        super().__init__()
        
        # Main path
        self.conv1 = nn.Conv2d(in_channels, out_channels, 
                              kernel_size=3, stride=stride, padding=1, 
                              bias=False)
        self.bn1 = nn.BatchNorm2d(out_channels)
        self.relu = nn.ReLU(inplace=True)
        
        self.conv2 = nn.Conv2d(out_channels, out_channels,
                              kernel_size=3, stride=1, padding=1,
                              bias=False)
        self.bn2 = nn.BatchNorm2d(out_channels)
        
        # Skip connection (identity or projection)
        self.skip = nn.Sequential()
        if stride != 1 or in_channels != out_channels:
            self.skip = nn.Sequential(
                nn.Conv2d(in_channels, out_channels, 
                         kernel_size=1, stride=stride, bias=False),
                nn.BatchNorm2d(out_channels)
            )
    
    def forward(self, x):
        identity = x
        
        # Main path
        out = self.conv1(x)
        out = self.bn1(out)
        out = self.relu(out)
        
        out = self.conv2(out)
        out = self.bn2(out)
        
        # Skip connection
        identity = self.skip(identity)
        
        # Add residual
        out += identity
        out = self.relu(out)
        
        return out

# Test
block = BasicBlock(64, 64)
x = torch.randn(4, 64, 32, 32)
y = block(x)
print(y.shape)  # torch.Size([4, 64, 32, 32])

# Test with dimension change
block_downsample = BasicBlock(64, 128, stride=2)
y2 = block_downsample(x)
print(y2.shape)  # torch.Size([4, 128, 16, 16])
\end{lstlisting}

\begin{pytorchtip}[Why bias=False?]
When using batch normalization immediately after convolution, the bias is redundant (BatchNorm has its own bias term). Setting \texttt{bias=False} saves parameters and computation without affecting performance.
\end{pytorchtip}

\clearpage
\subsubsection{Pre-Activation Residual Block}

\begin{lstlisting}
class PreActBlock(nn.Module):
    """Pre-activation residual block (improved version)."""
    
    def __init__(self, in_channels, out_channels, stride=1):
        super().__init__()
        
        # Pre-activation
        self.bn1 = nn.BatchNorm2d(in_channels)
        self.relu = nn.ReLU(inplace=True)
        self.conv1 = nn.Conv2d(in_channels, out_channels,
                              kernel_size=3, stride=stride, padding=1,
                              bias=False)
        
        self.bn2 = nn.BatchNorm2d(out_channels)
        self.conv2 = nn.Conv2d(out_channels, out_channels,
                              kernel_size=3, stride=1, padding=1,
                              bias=False)
        
        # Skip connection
        self.skip = nn.Sequential()
        if stride != 1 or in_channels != out_channels:
            self.skip = nn.Sequential(
                nn.Conv2d(in_channels, out_channels,
                         kernel_size=1, stride=stride, bias=False)
            )
    
    def forward(self, x):
        # Pre-activation
        out = self.bn1(x)
        out = self.relu(out)
        
        # Save for skip connection (after activation)
        identity = self.skip(out)
        
        # First conv
        out = self.conv1(out)
        
        # Second pre-activation + conv
        out = self.bn2(out)
        out = self.relu(out)
        out = self.conv2(out)
        
        # Add residual (no activation after addition!)
        out += identity
        
        return out
\end{lstlisting}

\textbf{Key difference:} In pre-activation, the final addition has no activation. The next block will apply activation first.

\subsubsection{Bottleneck Block}

For deeper networks (ResNet-50+), use bottleneck blocks to reduce parameters:

\begin{lstlisting}
class BottleneckBlock(nn.Module):
    """
    Bottleneck block: 1x1 -> 3x3 -> 1x1
    Reduces parameters by using 1x1 convolutions to reduce/expand channels.
    """
    expansion = 4  # Output channels = input channels * 4
    
    def __init__(self, in_channels, out_channels, stride=1):
        super().__init__()
        
        # Bottleneck: reduce channels
        self.conv1 = nn.Conv2d(in_channels, out_channels, 
                              kernel_size=1, bias=False)
        self.bn1 = nn.BatchNorm2d(out_channels)
        
        # Main 3x3 conv
        self.conv2 = nn.Conv2d(out_channels, out_channels,
                              kernel_size=3, stride=stride, 
                              padding=1, bias=False)
        self.bn2 = nn.BatchNorm2d(out_channels)
        
        # Expand channels
        self.conv3 = nn.Conv2d(out_channels, out_channels * self.expansion,
                              kernel_size=1, bias=False)
        self.bn3 = nn.BatchNorm2d(out_channels * self.expansion)
        
        self.relu = nn.ReLU(inplace=True)
        
        # Skip connection
        self.skip = nn.Sequential()
        if stride != 1 or in_channels != out_channels * self.expansion:
            self.skip = nn.Sequential(
                nn.Conv2d(in_channels, out_channels * self.expansion,
                         kernel_size=1, stride=stride, bias=False),
                nn.BatchNorm2d(out_channels * self.expansion)
            )
    
    def forward(self, x):
        identity = x
        
        out = self.conv1(x)
        out = self.bn1(out)
        out = self.relu(out)
        
        out = self.conv2(out)
        out = self.bn2(out)
        out = self.relu(out)
        
        out = self.conv3(out)
        out = self.bn3(out)
        
        identity = self.skip(identity)
        out += identity
        out = self.relu(out)
        
        return out

# Compare parameters
basic = BasicBlock(256, 256)
bottleneck = BottleneckBlock(256, 64)  # Output: 64*4 = 256

basic_params = sum(p.numel() for p in basic.parameters())
bottleneck_params = sum(p.numel() for p in bottleneck.parameters())

print(f"Basic block: {basic_params:,} parameters")
print(f"Bottleneck: {bottleneck_params:,} parameters")
# Basic block: ~590,000 parameters
# Bottleneck: ~70,000 parameters (much fewer!)
\end{lstlisting}

\clearpage
\subsubsection{Building a Complete ResNet}

\begin{lstlisting}
class ResNet(nn.Module):
    """ResNet architecture for CIFAR-10."""
    
    def __init__(self, block, num_blocks, num_classes=10):
        """
        Args:
            block: BasicBlock or BottleneckBlock
            num_blocks: List of number of blocks per layer
            num_classes: Number of output classes
        """
        super().__init__()
        self.in_channels = 64
        
        # Initial convolution
        self.conv1 = nn.Conv2d(3, 64, kernel_size=3, 
                              stride=1, padding=1, bias=False)
        self.bn1 = nn.BatchNorm2d(64)
        self.relu = nn.ReLU(inplace=True)
        
        # Residual layers
        self.layer1 = self._make_layer(block, 64, num_blocks[0], stride=1)
        self.layer2 = self._make_layer(block, 128, num_blocks[1], stride=2)
        self.layer3 = self._make_layer(block, 256, num_blocks[2], stride=2)
        self.layer4 = self._make_layer(block, 512, num_blocks[3], stride=2)
        
        # Global pooling and classifier
        self.avgpool = nn.AdaptiveAvgPool2d((1, 1))
        self.fc = nn.Linear(512 * block.expansion, num_classes)
    
    def _make_layer(self, block, out_channels, num_blocks, stride):
        """Create a layer with multiple residual blocks."""
        strides = [stride] + [1] * (num_blocks - 1)
        layers = []
        
        for stride in strides:
            layers.append(block(self.in_channels, out_channels, stride))
            self.in_channels = out_channels * block.expansion
        
        return nn.Sequential(*layers)
    
    def forward(self, x):
        out = self.conv1(x)
        out = self.bn1(out)
        out = self.relu(out)
        
        out = self.layer1(out)
        out = self.layer2(out)
        out = self.layer3(out)
        out = self.layer4(out)
        
        out = self.avgpool(out)
        out = out.view(out.size(0), -1)
        out = self.fc(out)
        
        return out

# ResNet-18
def ResNet18(num_classes=10):
    return ResNet(BasicBlock, [2, 2, 2, 2], num_classes)

# ResNet-34
def ResNet34(num_classes=10):
    return ResNet(BasicBlock, [3, 4, 6, 3], num_classes)

# ResNet-50 (uses bottleneck)
def ResNet50(num_classes=10):
    # Note: BottleneckBlock needs expansion=4
    BottleneckBlock.expansion = 4
    return ResNet(BottleneckBlock, [3, 4, 6, 3], num_classes)

# Test
model = ResNet18()
x = torch.randn(4, 3, 32, 32)
y = model(x)
print(y.shape)  # torch.Size([4, 10])

# Count parameters
total_params = sum(p.numel() for p in model.parameters())
print(f"ResNet-18: {total_params:,} parameters")
\end{lstlisting}

\begin{pytorchtip}[ResNet Naming]
ResNet-N refers to the total number of weight layers:

\textbf{ResNet-18:} 1 initial conv + 4 layers × 2 blocks × 2 convs + 1 FC = 18 layers

\textbf{ResNet-34:} 1 + (3+4+6+3) × 2 + 1 = 34 layers

\textbf{ResNet-50:} 1 + (3+4+6+3) × 3 + 1 = 50 layers (bottleneck has 3 convs)

Deeper = more capacity but slower training/inference.
\end{pytorchtip}

\clearpage
\subsubsection{Dense Connections (DenseNet)}

DenseNet takes skip connections further: \textbf{every layer connects to every other layer}.

\begin{lstlisting}
class DenseBlock(nn.Module):
    """Dense block where each layer receives all previous layers as input."""
    
    def __init__(self, in_channels, growth_rate, num_layers):
        """
        Args:
            in_channels: Initial number of channels
            growth_rate: Number of channels added per layer
            num_layers: Number of layers in the dense block
        """
        super().__init__()
        
        self.layers = nn.ModuleList()
        for i in range(num_layers):
            self.layers.append(
                self._make_dense_layer(in_channels + i * growth_rate, 
                                      growth_rate)
            )
    
    def _make_dense_layer(self, in_channels, growth_rate):
        return nn.Sequential(
            nn.BatchNorm2d(in_channels),
            nn.ReLU(inplace=True),
            nn.Conv2d(in_channels, growth_rate, kernel_size=3, 
                     padding=1, bias=False)
        )
    
    def forward(self, x):
        features = [x]
        for layer in self.layers:
            # Concatenate all previous feature maps
            new_features = layer(torch.cat(features, dim=1))
            features.append(new_features)
        
        # Return concatenation of all features
        return torch.cat(features, dim=1)

# Test
block = DenseBlock(in_channels=64, growth_rate=32, num_layers=4)
x = torch.randn(2, 64, 16, 16)
y = block(x)
print(y.shape)  # torch.Size([2, 192, 16, 16])
# 192 = 64 (input) + 4 * 32 (4 layers, each adds 32 channels)
\end{lstlisting}

\textbf{DenseNet characteristics:}
\begin{itemize}
    \item \textbf{Feature reuse:} All layers have access to all previous features
    \item \textbf{Fewer parameters:} Smaller growth\_rate than ResNet channel counts
    \item \textbf{Memory intensive:} Must store all intermediate features
    \item \textbf{Better gradient flow:} Even better than ResNet
\end{itemize}

\textbf{When to use DenseNet vs ResNet:}
\begin{itemize}
    \item \textbf{DenseNet:} When parameter efficiency is critical, smaller datasets
    \item \textbf{ResNet:} When memory is limited, faster training/inference, more standard
\end{itemize}

\subsubsection{Skip Connections in Practice}

\textbf{General principles for adding skip connections:}

\begin{enumerate}
    \item \textbf{Add skip every 2-3 layers:} Balance between gradient flow and complexity
    \item \textbf{Match dimensions:} Use projection when channels or spatial size change
    \item \textbf{Identity when possible:} Projection-free shortcuts are better
    \item \textbf{Add before final activation:} Preserves identity mapping
    \item \textbf{Don't overdo it:} Too many skip connections can hurt performance
\end{enumerate}

\textbf{Example: Adding skip connections to existing MLP:}

\begin{lstlisting}
class MLPWithSkipConnections(nn.Module):
    """MLP with residual connections."""
    
    def __init__(self, input_dim, hidden_dim, output_dim, num_layers=4):
        super().__init__()
        
        self.input_layer = nn.Linear(input_dim, hidden_dim)
        
        # Hidden layers with skip connections
        self.hidden_layers = nn.ModuleList()
        for _ in range(num_layers):
            self.hidden_layers.append(
                nn.Sequential(
                    nn.Linear(hidden_dim, hidden_dim),
                    nn.ReLU(),
                    nn.Linear(hidden_dim, hidden_dim)
                )
            )
        
        self.output_layer = nn.Linear(hidden_dim, output_dim)
        self.relu = nn.ReLU()
    
    def forward(self, x):
        x = self.relu(self.input_layer(x))
        
        # Apply each hidden layer with skip connection
        for layer in self.hidden_layers:
            residual = x
            x = layer(x)
            x = x + residual  # Skip connection
            x = self.relu(x)
        
        x = self.output_layer(x)
        return x
\end{lstlisting}

\clearpage
% =============================================
% SECTION 8: RESNET - PART 2 (Exercises & Debugging)
% =============================================

\subsection{Implementation: Debugging ResNets}

\subsubsection{Verifying Skip Connections}

\begin{lstlisting}
def verify_skip_connections(model, input_shape=(1, 3, 32, 32)):
    """Verify that skip connections are working."""
    x = torch.randn(*input_shape, requires_grad=True)
    
    # Forward pass
    output = model(x)
    loss = output.sum()
    
    # Backward pass
    loss.backward()
    
    # Check if gradients reach input
    if x.grad is not None:
        print(f"✓ Gradients flow to input: {x.grad.abs().mean().item():.6f}")
    else:
        print("✗ No gradients at input!")
    
    # Check gradient magnitudes throughout network
    print("\nGradient magnitudes by layer:")
    for name, param in model.named_parameters():
        if param.grad is not None:
            grad_mean = param.grad.abs().mean().item()
            grad_max = param.grad.abs().max().item()
            print(f"{name:30s}: mean={grad_mean:.6f}, max={grad_max:.6f}")

# Usage
# model = ResNet18()
# verify_skip_connections(model)
\end{lstlisting}

\subsubsection{Comparing With and Without Skip Connections}

\begin{lstlisting}
class PlainCNN(nn.Module):
    """Same architecture as ResNet but without skip connections."""
    
    def __init__(self, num_classes=10):
        super().__init__()
        
        self.conv1 = nn.Conv2d(3, 64, 3, padding=1, bias=False)
        self.bn1 = nn.BatchNorm2d(64)
        
        # Same structure as ResNet-18 but no skip connections
        self.layers = nn.Sequential(
            # Layer 1
            nn.Conv2d(64, 64, 3, padding=1, bias=False),
            nn.BatchNorm2d(64),
            nn.ReLU(),
            nn.Conv2d(64, 64, 3, padding=1, bias=False),
            nn.BatchNorm2d(64),
            nn.ReLU(),
            
            # Layer 2 (with stride=2 downsampling)
            nn.Conv2d(64, 128, 3, stride=2, padding=1, bias=False),
            nn.BatchNorm2d(128),
            nn.ReLU(),
            nn.Conv2d(128, 128, 3, padding=1, bias=False),
            nn.BatchNorm2d(128),
            nn.ReLU(),
            
            # Continue pattern...
        )
        
        self.avgpool = nn.AdaptiveAvgPool2d((1, 1))
        self.fc = nn.Linear(512, num_classes)
    
    def forward(self, x):
        x = nn.ReLU()(self.bn1(self.conv1(x)))
        x = self.layers(x)
        x = self.avgpool(x)
        x = x.view(x.size(0), -1)
        return self.fc(x)

# Compare training curves
# plain_model = PlainCNN()
# resnet_model = ResNet18()
# Train both and plot losses - ResNet should converge faster and better
\end{lstlisting}

\subsection{Exercises}

\begin{exercise}[8.1: Basic Residual Block - $\bigstar\bigstar$]
\textbf{Goal:} Implement and understand a basic residual block.

\begin{enumerate}
    \item Implement a \texttt{BasicBlock} with two 3×3 convolutions
    \item Include skip connection with projection for dimension changes
    \item Test with different input/output channels:
    \begin{itemize}
        \item Same channels (64 → 64)
        \item Different channels (64 → 128)
        \item With downsampling (stride=2)
    \end{itemize}
    \item Verify output shapes match expected dimensions
\end{enumerate}

\textbf{Starter code:}
\begin{lstlisting}
class MyBasicBlock(nn.Module):
    def __init__(self, in_channels, out_channels, stride=1):
        super().__init__()
        # Your code here
    
    def forward(self, x):
        # Your code here
        pass

# Test cases
block1 = MyBasicBlock(64, 64, stride=1)
block2 = MyBasicBlock(64, 128, stride=2)

x = torch.randn(4, 64, 32, 32)
print(block1(x).shape)  # Should be [4, 64, 32, 32]
print(block2(x).shape)  # Should be [4, 128, 16, 16]
\end{lstlisting}
\end{exercise}

\begin{exercise}[8.2: Plain vs Residual Comparison - $\bigstar\bigstar\bigstar$]
\textbf{Goal:} Empirically verify that skip connections help.

\begin{enumerate}
    \item Build two identical networks (10 layers each):
    \begin{itemize}
        \item One \textbf{without} skip connections (plain CNN)
        \item One \textbf{with} skip connections (ResNet-style)
    \end{itemize}
    \item Train both on CIFAR-10 or MNIST
    \item Track and plot:
    \begin{itemize}
        \item Training loss
        \item Validation accuracy
        \item Gradient magnitudes in early layers
    \end{itemize}
    \item Observe: ResNet should train faster and achieve better accuracy
\end{enumerate}

\textbf{Questions to answer:}
\begin{itemize}
    \item How much faster does ResNet converge?
    \item What's the final accuracy difference?
    \item How do gradient magnitudes compare?
\end{itemize}
\end{exercise}

\begin{exercise}[8.3: Pre-Activation vs Post-Activation - $\bigstar\bigstar\bigstar$]
\textbf{Goal:} Compare residual block variants.

\begin{enumerate}
    \item Implement both post-activation and pre-activation blocks
    \item Build identical ResNets with each type
    \item Train on the same dataset
    \item Compare:
    \begin{itemize}
        \item Training stability (loss curves)
        \item Final accuracy
        \item Training time
    \end{itemize}
\end{enumerate}

\textbf{Hypothesis:} Pre-activation should be slightly better, especially for deeper networks.

\textbf{Extension:} Try with different depths (18, 34, 50 layers) and see if the gap widens.
\end{exercise}

\begin{exercise}[8.4: Building ResNet-18 - $\bigstar\bigstar\bigstar$]
\textbf{Goal:} Build a complete ResNet from scratch.

\begin{enumerate}
    \item Implement ResNet-18 architecture:
    \begin{itemize}
        \item Initial 7×7 conv (or 3×3 for CIFAR)
        \item 4 residual layers with [2, 2, 2, 2] blocks
        \item Channels: 64 → 128 → 256 → 512
        \item Global average pooling + FC
    \end{itemize}
    \item Train on CIFAR-10
    \item Achieve >90\% test accuracy
    \item Save the model and load it for inference
\end{enumerate}

\textbf{Bonus:} Visualize learned features in first convolutional layer.
\end{exercise}

\begin{exercise}[8.5: Bottleneck Block - $\bigstar\bigstar\bigstar\bigstar$]
\textbf{Goal:} Implement parameter-efficient bottleneck blocks.

\begin{enumerate}
    \item Implement \texttt{BottleneckBlock}: 1×1 → 3×3 → 1×1
    \item Compare parameter count with \texttt{BasicBlock} for same input/output
    \item Build ResNet-50 using bottleneck blocks
    \item Train and compare with ResNet-18:
    \begin{itemize}
        \item Accuracy (ResNet-50 should be better)
        \item Training time (ResNet-50 slower)
        \item Memory usage
    \end{itemize}
\end{enumerate}

\textbf{Challenge:} How deep can you go before hitting memory limits? Try ResNet-101.
\end{exercise}

\begin{exercise}[8.6: Skip Connections in U-Net - $\bigstar\bigstar\bigstar\bigstar$]
\textbf{Goal:} Apply skip connections to encoder-decoder architectures.

\begin{enumerate}
    \item Take the U-Net from CNN section (Exercise 7.5)
    \item Add residual blocks in encoder and decoder paths
    \item Train on image denoising task
    \item Compare with original U-Net:
    \begin{itemize}
        \item Denoising quality (PSNR/SSIM)
        \item Training stability
        \item Convergence speed
    \end{itemize}
\end{enumerate}

\textbf{Architecture:}
\begin{itemize}
    \item Replace conv blocks with residual blocks
    \item Keep skip connections between encoder/decoder
    \item Use both residual blocks AND U-Net skip connections
\end{itemize}
\end{exercise}

\begin{exercise}[8.7: Dense Block Implementation - $\bigstar\bigstar\bigstar\bigstar$]
\textbf{Goal:} Implement and understand DenseNet.

\begin{enumerate}
    \item Implement a \texttt{DenseBlock} where each layer receives all previous layers
    \item Implement a transition layer (1×1 conv + pooling) to reduce channels
    \item Build a small DenseNet:
    \begin{itemize}
        \item 3 dense blocks with growth\_rate=12
        \item Transition layers between blocks
        \item Final global pooling + FC
    \end{itemize}
    \item Compare with ResNet of similar depth:
    \begin{itemize}
        \item Parameter count (DenseNet should be fewer)
        \item Memory usage (DenseNet higher during training)
        \item Accuracy
    \end{itemize}
\end{enumerate}

\textbf{Hint:} Use \texttt{torch.cat()} to concatenate feature maps along channel dimension.
\end{exercise}

\clearpage
\subsection{Key Takeaways}

\textbf{The fundamental problem:}
\begin{itemize}
    \item Very deep networks suffer from degradation (not just overfitting)
    \item Vanishing gradients make early layers hard to train
    \item Optimization becomes difficult, not just capacity
\end{itemize}

\textbf{How skip connections solve it:}
\begin{itemize}
    \item Create gradient highways: $\frac{\partial \mathcal{L}}{\partial \mathbf{x}}$ always has $+1$ term
    \item Learn residuals instead of direct mappings (easier optimization)
    \item Enable training of 100+ layer networks
    \item Identity mapping is easy to learn (set weights to zero)
\end{itemize}

\textbf{Residual block variants:}
\begin{itemize}
    \item \textbf{Post-activation:} Conv → BN → ReLU, then add (original)
    \item \textbf{Pre-activation:} BN → ReLU → Conv, then add (improved)
    \item \textbf{Bottleneck:} 1×1 → 3×3 → 1×1 (for deeper networks, fewer parameters)
    \item \textbf{Dense:} Every layer connects to all previous layers (extreme skip connections)
\end{itemize}

\textbf{Implementation guidelines:}
\begin{itemize}
    \item Use projection shortcuts when dimensions change
    \item Set \texttt{bias=False} when using BatchNorm
    \item Add skip connections every 2-3 layers
    \item Pre-activation is better for very deep networks (>100 layers)
    \item Use bottleneck blocks for ResNet-50+
\end{itemize}

\textbf{When to use skip connections:}
\begin{itemize}
    \item \textbf{Always use for deep networks (>20 layers)}
    \item \textbf{Vision tasks:} ResNet is standard backbone
    \item \textbf{U-Net style architectures:} Encoder-decoder with skip connections
    \item \textbf{Any architecture where gradient flow is critical}
    \item \textbf{When training is unstable or slow}
\end{itemize}

\textbf{Architecture choices:}
\begin{table}[h]
\centering
\begin{tabular}{llll}
\toprule
\textbf{Model} & \textbf{Layers} & \textbf{Use Case} & \textbf{Parameters} \\
\midrule
ResNet-18 & 18 & General purpose, fast & 11M \\
ResNet-34 & 34 & More capacity & 21M \\
ResNet-50 & 50 & High accuracy & 25M \\
ResNet-101 & 101 & Maximum accuracy & 44M \\
DenseNet-121 & 121 & Parameter efficient & 8M \\
\bottomrule
\end{tabular}
\caption{Popular ResNet/DenseNet architectures}
\end{table}

\textbf{Impact on deep learning:}
\begin{itemize}
    \item Enabled training of 100+ layer networks
    \item Now standard in almost all vision models
    \item Extended to other domains (Transformers use similar ideas)
    \item Fundamental principle: make it easy for network to learn identity
\end{itemize}

\textbf{Debugging tips:}
\begin{itemize}
    \item Verify gradients flow to early layers
    \item Compare with plain network (without skip connections)
    \item Check gradient magnitudes throughout network
    \item Visualize activations in residual blocks
    \item If training fails, check skip connection implementation
\end{itemize}

\textbf{Common mistakes:}
\begin{enumerate}
    \item Forgetting projection when dimensions change
    \item Applying activation after addition in pre-activation blocks
    \item Not using BatchNorm (skip connections alone aren't enough)
    \item Making skip connection too complex (keep it identity when possible)
    \item Adding too many skip connections (can hurt performance)
\end{enumerate}

\begin{pytorchtip}[Rule of Thumb]
\textbf{For networks >20 layers:} Always use residual connections

\textbf{For networks >50 layers:} Use bottleneck blocks

\textbf{For networks >100 layers:} Use pre-activation blocks

\textbf{When in doubt:} Use ResNet as your starting point and modify from there.
\end{pytorchtip}

\clearpage
% =============================================
% SECTION 9: BATCH NORMALIZATION & LAYER NORMALIZATION
% =============================================

\section{Batch Normalization \& Layer Normalization}

\subsection{Introduction: Why Normalization Matters}

Training deep networks is hard. As gradients flow backward, they can vanish or explode. As data flows forward, the distribution of activations can shift. These issues slow training and make networks sensitive to initialization.

\textbf{Normalization techniques} stabilize training by controlling the distribution of activations.

\textbf{Key benefits:}
\begin{itemize}
    \item \textbf{Faster training:} Can use higher learning rates (2-10× speedup)
    \item \textbf{Less sensitivity to initialization:} Network more robust
    \item \textbf{Regularization effect:} Acts like dropout (slight noise from batch statistics)
    \item \textbf{Enables deeper networks:} Makes 100+ layer networks trainable
\end{itemize}

\textbf{When you need normalization:}
\begin{itemize}
    \item Deep networks (>10 layers)
    \item Training is slow or unstable
    \item Network sensitive to learning rate or initialization
    \item Using high learning rates
\end{itemize}

\subsection{Theory: Batch Normalization}

\subsubsection{The Problem: Internal Covariate Shift}

As we train, the distribution of inputs to each layer changes. This is called \textbf{internal covariate shift}.

\textbf{Example:} Consider layer 3 of a network.
\begin{itemize}
    \item Early in training: inputs to layer 3 might have mean=0.5, std=0.3
    \item After 100 steps: mean=2.1, std=1.8 (distribution has shifted!)
\end{itemize}

Each layer must constantly adapt to the changing distribution. This slows learning.

\textbf{Batch Normalization's solution:} Normalize inputs to each layer to have consistent statistics.

\subsubsection{How Batch Normalization Works}

Given a batch of inputs $\mathbf{x} = \{x_1, x_2, \ldots, x_m\}$:

\textbf{Step 1: Compute batch statistics}
\[
\mu_B = \frac{1}{m} \sum_{i=1}^m x_i \quad \text{(batch mean)}
\]
\[
\sigma_B^2 = \frac{1}{m} \sum_{i=1}^m (x_i - \mu_B)^2 \quad \text{(batch variance)}
\]

\textbf{Step 2: Normalize}
\[
\hat{x}_i = \frac{x_i - \mu_B}{\sqrt{\sigma_B^2 + \epsilon}}
\]

where $\epsilon$ is a small constant (typically $10^{-5}$) for numerical stability.

\textbf{Step 3: Scale and shift (learnable parameters)}
\[
y_i = \gamma \hat{x}_i + \beta
\]

where $\gamma$ (scale) and $\beta$ (shift) are learned parameters.

\textbf{Why scale and shift?}
The network can learn to undo the normalization if needed! If $\gamma = \sqrt{\sigma_B^2}$ and $\beta = \mu_B$, we recover the original distribution.

\begin{theorybox}[Key Insight]
Batch norm has two modes:

\textbf{Training:} Use batch statistics ($\mu_B$, $\sigma_B^2$)
\begin{itemize}
    \item Computed from current batch
    \item Adds stochasticity (different batches → different statistics)
    \item Acts as regularization
\end{itemize}

\textbf{Inference:} Use running statistics ($\mu_{running}$, $\sigma_{running}^2$)
\begin{itemize}
    \item Exponential moving average from training
    \item Deterministic (same input → same output)
    \item No batch dependency
\end{itemize}

\textbf{During training, PyTorch maintains:}
\[
\mu_{running} \leftarrow (1 - \text{momentum}) \cdot \mu_{running} + \text{momentum} \cdot \mu_B
\]

Default momentum = 0.1.
\end{theorybox}

\clearpage
\subsubsection{Why Batch Normalization Helps}

\textbf{1. Smooths the optimization landscape}

Batch norm makes the loss surface smoother, allowing larger learning rates.

\textbf{2. Reduces internal covariate shift}

Each layer receives inputs with consistent statistics.

\textbf{3. Acts as regularization}

Noise from batch statistics acts like dropout (but usually weaker).

\textbf{4. Reduces sensitivity to initialization}

Even with poor initialization, batch norm helps normalize activations.

\textbf{5. Allows higher learning rates}

The smoothing effect permits learning rates 10× higher than without batch norm.

\subsubsection{Limitations of Batch Normalization}

\textbf{1. Requires large enough batches}

With batch\_size < 8, batch statistics are unreliable. Solution: Use LayerNorm or GroupNorm.

\textbf{2. Different behavior in train/eval}

Must remember to call \texttt{model.train()} and \texttt{model.eval()}. Forgetting this is a common bug!

\textbf{3. Doesn't work well for RNNs}

Batch statistics across sequence positions don't make sense. Solution: Use LayerNorm.

\textbf{4. Coupling between samples in batch}

Each sample's normalization depends on other samples in the batch. Can cause issues in some scenarios.

\subsection{Theory: Layer Normalization}

Layer Normalization normalizes across features instead of across the batch.

\textbf{Batch Norm:} Normalize each feature across the batch
\[
\hat{x}_{ij} = \frac{x_{ij} - \mu_j}{\sqrt{\sigma_j^2 + \epsilon}} \quad \text{where } \mu_j = \frac{1}{m}\sum_{i=1}^m x_{ij}
\]
(Averages over batch dimension $i$ for each feature $j$)

\textbf{Layer Norm:} Normalize all features for each sample
\[
\hat{x}_{ij} = \frac{x_{ij} - \mu_i}{\sqrt{\sigma_i^2 + \epsilon}} \quad \text{where } \mu_i = \frac{1}{d}\sum_{j=1}^d x_{ij}
\]
(Averages over feature dimension $j$ for each sample $i$)

\textbf{Key differences:}

\begin{table}[h]
\centering
\begin{tabular}{lll}
\toprule
\textbf{Aspect} & \textbf{Batch Norm} & \textbf{Layer Norm} \\
\midrule
Normalizes over & Batch dimension & Feature dimension \\
Batch size dependency & Yes (fails for small batches) & No (works with batch=1) \\
Train/eval modes & Different & Same \\
Learnable params & $\gamma, \beta$ per feature & $\gamma, \beta$ per feature \\
Best for & CNNs, MLPs & RNNs, Transformers \\
Running statistics & Yes & No \\
\bottomrule
\end{tabular}
\caption{Batch Norm vs Layer Norm}
\end{table}

\textbf{When to use Layer Norm:}
\begin{itemize}
    \item RNNs and sequence models
    \item Transformers (standard choice)
    \item Small batch sizes (batch\_size < 8)
    \item Online learning (single sample at a time)
    \item When you want same behavior in train/eval
\end{itemize}

\clearpage
\subsection{Implementation: Using Normalization in PyTorch}

\subsubsection{Batch Normalization Usage}

\begin{lstlisting}
import torch
import torch.nn as nn

# For 1D data (MLPs, sequences)
bn1d = nn.BatchNorm1d(num_features=128)

# For 2D data (CNNs)
bn2d = nn.BatchNorm2d(num_features=64)

# For 3D data (3D CNNs, video)
bn3d = nn.BatchNorm3d(num_features=32)

# Example: CNN with batch norm
model = nn.Sequential(
    nn.Conv2d(3, 64, 3, padding=1),
    nn.BatchNorm2d(64),  # Batch norm after conv
    nn.ReLU(),
    
    nn.Conv2d(64, 128, 3, padding=1),
    nn.BatchNorm2d(128),
    nn.ReLU(),
)

# Training mode (default)
model.train()
x = torch.randn(16, 3, 32, 32)
out = model(x)  # Uses batch statistics

# Evaluation mode
model.eval()
with torch.no_grad():
    out = model(x)  # Uses running statistics
\end{lstlisting}

\textbf{Where to place batch norm:}

\textbf{Option 1: Conv → BN → Activation (recommended)}
\begin{lstlisting}
nn.Conv2d(in_ch, out_ch, 3, padding=1),
nn.BatchNorm2d(out_ch),
nn.ReLU()
\end{lstlisting}

\textbf{Option 2: Conv → Activation → BN (less common)}
\begin{lstlisting}
nn.Conv2d(in_ch, out_ch, 3, padding=1),
nn.ReLU(),
nn.BatchNorm2d(out_ch)
\end{lstlisting}

Both work, but Option 1 is more standard.

\begin{warningbox}[Critical: Set bias=False]
When using batch norm after a linear/conv layer, set \texttt{bias=False}:

\begin{lstlisting}
# WRONG: Batch norm makes bias redundant
nn.Conv2d(3, 64, 3, padding=1, bias=True),  # Wasteful!
nn.BatchNorm2d(64),  # Has its own bias (beta)

# CORRECT: No need for conv bias
nn.Conv2d(3, 64, 3, padding=1, bias=False),  # No bias
nn.BatchNorm2d(64),  # Beta parameter serves as bias
\end{lstlisting}

Why? Batch norm subtracts the mean, so any constant bias gets canceled out. The $\beta$ parameter in batch norm serves as the bias.
\end{warningbox}

\subsubsection{Layer Normalization Usage}

\begin{lstlisting}
# Layer norm normalizes over feature dimension
ln = nn.LayerNorm(normalized_shape=128)

# For sequences: (batch, seq_len, features)
x = torch.randn(32, 100, 128)  # 32 sequences, length 100, 128 features
out = ln(x)  # Normalizes over the 128 features for each position

# Example: Transformer-style layer
class TransformerBlock(nn.Module):
    def __init__(self, d_model):
        super().__init__()
        self.attention = nn.MultiheadAttention(d_model, num_heads=8)
        self.ln1 = nn.LayerNorm(d_model)
        self.ffn = nn.Sequential(
            nn.Linear(d_model, 4*d_model),
            nn.ReLU(),
            nn.Linear(4*d_model, d_model)
        )
        self.ln2 = nn.LayerNorm(d_model)
    
    def forward(self, x):
        # Attention with layer norm
        attn_out, _ = self.attention(x, x, x)
        x = self.ln1(x + attn_out)  # Residual + norm
        
        # FFN with layer norm
        ffn_out = self.ffn(x)
        x = self.ln2(x + ffn_out)  # Residual + norm
        
        return x
\end{lstlisting}

\clearpage
\subsubsection{Other Normalization Variants}

\textbf{Instance Normalization (for style transfer):}
\begin{lstlisting}
# Normalizes each sample and channel independently
# Used in style transfer and GANs
in_norm = nn.InstanceNorm2d(num_features=64)

# Shape: (batch, channels, height, width)
# Computes mean/std for each (sample, channel) pair over (H, W)
\end{lstlisting}

\textbf{Group Normalization (hybrid approach):}
\begin{lstlisting}
# Divides channels into groups, normalizes within each group
# Works well with small batch sizes
gn = nn.GroupNorm(num_groups=8, num_channels=64)

# Example: 64 channels → 8 groups of 8 channels each
# Normalizes over spatial dimensions and within each group
\end{lstlisting}

\textbf{When to use each:}

\begin{table}[h]
\centering
\begin{tabular}{ll}
\toprule
\textbf{Normalization} & \textbf{Use Case} \\
\midrule
Batch Norm & CNNs, MLPs, large batches ($\geq 8$) \\
Layer Norm & RNNs, Transformers, any batch size \\
Instance Norm & Style transfer, GANs \\
Group Norm & Small batch sizes, object detection \\
\bottomrule
\end{tabular}
\caption{Choosing normalization type}
\end{table}

\subsubsection{Implementing Batch Norm from Scratch}

\begin{lstlisting}
class MyBatchNorm1d(nn.Module):
    """Batch normalization for understanding."""
    
    def __init__(self, num_features, eps=1e-5, momentum=0.1):
        super().__init__()
        
        # Learnable parameters
        self.gamma = nn.Parameter(torch.ones(num_features))
        self.beta = nn.Parameter(torch.zeros(num_features))
        
        # Running statistics (not trainable)
        self.register_buffer('running_mean', torch.zeros(num_features))
        self.register_buffer('running_var', torch.ones(num_features))
        
        self.eps = eps
        self.momentum = momentum
    
    def forward(self, x):
        # x shape: (batch, features)
        
        if self.training:
            # Training mode: use batch statistics
            batch_mean = x.mean(dim=0)
            batch_var = x.var(dim=0, unbiased=False)
            
            # Normalize
            x_norm = (x - batch_mean) / torch.sqrt(batch_var + self.eps)
            
            # Update running statistics (exponential moving average)
            with torch.no_grad():
                self.running_mean = (1 - self.momentum) * self.running_mean + \
                                   self.momentum * batch_mean
                self.running_var = (1 - self.momentum) * self.running_var + \
                                  self.momentum * batch_var
        else:
            # Eval mode: use running statistics
            x_norm = (x - self.running_mean) / \
                     torch.sqrt(self.running_var + self.eps)
        
        # Scale and shift
        out = self.gamma * x_norm + self.beta
        
        return out

# Test
bn = MyBatchNorm1d(10)
x = torch.randn(32, 10)

# Training mode
bn.train()
out_train = bn(x)

# Eval mode
bn.eval()
out_eval = bn(x)

print(f"Train output: {out_train.mean():.4f}, {out_train.std():.4f}")
print(f"Eval output: {out_eval.mean():.4f}, {out_eval.std():.4f}")
\end{lstlisting}

\clearpage
\subsubsection{Common Mistakes and Debugging}

\textbf{Mistake 1: Forgetting to set eval mode}

\begin{lstlisting}
# WRONG: Model still in training mode during evaluation
model.train()  # Set at beginning of training
# ... training loop ...
# Evaluation (FORGOT model.eval()!)
with torch.no_grad():
    val_loss = evaluate(model, val_loader)  # Uses batch statistics!

# CORRECT:
model.train()
# ... training ...
model.eval()  # Switch to eval mode
with torch.no_grad():
    val_loss = evaluate(model, val_loader)  # Uses running statistics
\end{lstlisting}

\textbf{Mistake 2: Small batch sizes with Batch Norm}

\begin{lstlisting}
# With batch_size=2, batch statistics are unreliable
# Symptoms: High variance in training, poor performance

# Solution 1: Increase batch size
train_loader = DataLoader(dataset, batch_size=32)  # Instead of 2

# Solution 2: Use Layer Norm or Group Norm instead
# Replace BatchNorm2d with GroupNorm
nn.GroupNorm(num_groups=8, num_channels=64)
\end{lstlisting}

\textbf{Mistake 3: Not setting bias=False}

\begin{lstlisting}
# WASTEFUL: Both conv and batch norm have biases
layer = nn.Sequential(
    nn.Conv2d(64, 128, 3, bias=True),  # Has bias
    nn.BatchNorm2d(128)  # Also has bias (beta)
)

# EFFICIENT: Only batch norm has bias
layer = nn.Sequential(
    nn.Conv2d(64, 128, 3, bias=False),  # No bias
    nn.BatchNorm2d(128)  # Beta serves as bias
)
\end{lstlisting}

\textbf{Debugging: Check if batch norm is working}

\begin{lstlisting}
def check_batch_norm_stats(model, dataloader):
    """Verify batch norm running statistics are being updated."""
    
    model.train()
    
    # Save initial running mean
    bn_layer = None
    for module in model.modules():
        if isinstance(module, nn.BatchNorm2d):
            bn_layer = module
            break
    
    if bn_layer is None:
        print("No BatchNorm found!")
        return
    
    initial_mean = bn_layer.running_mean.clone()
    
    # Run one batch
    x, _ = next(iter(dataloader))
    _ = model(x)
    
    # Check if running mean changed
    mean_changed = not torch.allclose(initial_mean, bn_layer.running_mean)
    print(f"Running mean updated: {mean_changed}")
    
    # Check train vs eval difference
    model.eval()
    with torch.no_grad():
        out_eval = model(x)
    
    model.train()
    out_train = model(x)
    
    different = not torch.allclose(out_train, out_eval)
    print(f"Train vs eval outputs differ: {different}")

# Usage
# check_batch_norm_stats(model, train_loader)
\end{lstlisting}

\clearpage
\subsection{Exercises}

\begin{exercise}[9.1: Batch Norm Impact - $\bigstar\bigstar$]
\textbf{Goal:} See batch norm's effect empirically.

Train two identical networks on CIFAR-10:
\begin{enumerate}
    \item \textbf{Without batch norm:} Plain CNN
    \item \textbf{With batch norm:} Same CNN + batch norm after each conv
\end{enumerate}

Compare:
\begin{itemize}
    \item Training speed (epochs to reach 70\% accuracy)
    \item Learning rate stability (try lr=0.01 and lr=0.1)
    \item Final test accuracy
\end{itemize}

\textbf{Expected observation:} Batch norm allows higher learning rate and converges faster.

\textbf{Starter code:}
\begin{lstlisting}
class CNNWithoutBN(nn.Module):
    def __init__(self):
        super().__init__()
        self.features = nn.Sequential(
            nn.Conv2d(3, 64, 3, padding=1),
            nn.ReLU(),
            nn.Conv2d(64, 128, 3, padding=1),
            nn.ReLU(),
            nn.MaxPool2d(2),
            # ... more layers
        )

class CNNWithBN(nn.Module):
    def __init__(self):
        super().__init__()
        self.features = nn.Sequential(
            nn.Conv2d(3, 64, 3, padding=1, bias=False),
            nn.BatchNorm2d(64),
            nn.ReLU(),
            # ... add batch norm after each conv
        )
\end{lstlisting}
\end{exercise}

\begin{exercise}[9.2: Train vs Eval Mode - $\bigstar\bigstar$]
\textbf{Goal:} Understand the importance of eval mode.

\begin{enumerate}
    \item Train a CNN with batch norm on MNIST
    \item After training, evaluate on test set \textbf{without} calling \texttt{model.eval()}
    \item Evaluate again \textbf{with} \texttt{model.eval()}
    \item Compare test accuracies
    \item Investigate: Try with different batch sizes during evaluation (1, 8, 64)
\end{enumerate}

\textbf{Questions:}
\begin{itemize}
    \item How much does forgetting eval() hurt performance?
    \item Does batch size during eval matter when in train mode?
    \item Why does this happen?
\end{itemize}
\end{exercise}

\begin{exercise}[9.3: Batch Norm from Scratch - $\bigstar\bigstar\bigstar$]
\textbf{Goal:} Implement batch norm to understand it deeply.

\begin{enumerate}
    \item Complete the \texttt{MyBatchNorm1d} implementation from above
    \item Extend it to \texttt{MyBatchNorm2d} for CNNs
    \item Test that it matches PyTorch's \texttt{nn.BatchNorm2d}:
    \begin{itemize}
        \item Initialize both with same parameters
        \item Feed same input
        \item Verify outputs are close (use \texttt{torch.allclose})
    \end{itemize}
    \item Train a small CNN using your custom batch norm
    \item Verify training works correctly
\end{enumerate}

\textbf{Hint for BatchNorm2d:}
\begin{lstlisting}
# For input shape (batch, channels, height, width)
# Compute mean and var over dimensions (0, 2, 3)
# This gives per-channel statistics
batch_mean = x.mean(dim=(0, 2, 3), keepdim=True)
batch_var = x.var(dim=(0, 2, 3), keepdim=True, unbiased=False)
\end{lstlisting}
\end{exercise}

\begin{exercise}[9.4: Layer Norm vs Batch Norm - $\bigstar\bigstar\bigstar$]
\textbf{Goal:} Compare normalization types for sequences.

\begin{enumerate}
    \item Generate a sequence classification task (e.g., classify sine vs cosine)
    \item Build an RNN with:
    \begin{itemize}
        \item Batch normalization
        \item Layer normalization
        \item No normalization
    \end{itemize}
    \item Train all three and compare:
    \begin{itemize}
        \item Training stability
        \item Final accuracy
        \item Effect of batch size (try 4, 16, 64)
    \end{itemize}
\end{enumerate}

\textbf{Expected result:} Layer norm should work better for sequences, especially with small batches.
\end{exercise}

\begin{exercise}[9.5: Placement Experiments - $\bigstar\bigstar\bigstar\bigstar$]
\textbf{Goal:} Investigate where to place batch norm.

Test different placements in a residual block:
\begin{enumerate}
    \item Conv → BN → ReLU
    \item Conv → ReLU → BN
    \item BN → ReLU → Conv (pre-activation)
\end{enumerate}

For each:
\begin{itemize}
    \item Train on CIFAR-10
    \item Measure convergence speed
    \item Check gradient flow (from Section 8)
    \item Compare final accuracy
\end{itemize}

\textbf{Questions:}
\begin{itemize}
    \item Which placement works best?
    \item Does the answer change with network depth?
    \item How does it affect gradient flow?
\end{itemize}
\end{exercise}

\begin{exercise}[9.6: Small Batch Challenge - $\bigstar\bigstar\bigstar\bigstar$]
\textbf{Goal:} Handle small batch sizes effectively.

\begin{enumerate}
    \item Train a network with batch\_size=2 (deliberately small)
    \item Try:
    \begin{itemize}
        \item Batch Norm (observe instability)
        \item Layer Norm
        \item Group Norm (8 groups)
        \item No normalization
    \end{itemize}
    \item Compare training stability and final performance
    \item Investigate: Why does batch norm fail? Look at batch statistics variance.
\end{enumerate}

\textbf{Analysis code:}
\begin{lstlisting}
# Check batch statistics variance
bn_layer = model.features[1]  # Assuming second layer is BN
means = []
vars = []

for batch in train_loader:
    x, _ = batch
    # Hook to capture batch statistics
    means.append(bn_layer.running_mean.clone())
    vars.append(bn_layer.running_var.clone())
    model(x)

# Plot variance of batch means over time
plt.plot([m.std().item() for m in means])
plt.title('Variance of Batch Means')
plt.show()
\end{lstlisting}
\end{exercise}

\clearpage
\subsection{Key Takeaways}

\textbf{Why normalization helps:}
\begin{itemize}
    \item Reduces internal covariate shift (distribution changes between layers)
    \item Smooths optimization landscape (allows higher learning rates)
    \item Provides regularization (slight noise from batch statistics)
    \item Reduces sensitivity to initialization
\end{itemize}

\textbf{Batch Normalization:}
\begin{itemize}
    \item Normalizes over batch dimension
    \item Requires large enough batches ($\geq 8$)
    \item Different behavior in train/eval modes (critical!)
    \item Best for CNNs and MLPs with large batches
    \item Can speed up training by 2-10×
\end{itemize}

\textbf{Layer Normalization:}
\begin{itemize}
    \item Normalizes over feature dimension
    \item Works with any batch size (even 1)
    \item Same behavior in train/eval modes
    \item Best for RNNs, Transformers, sequences
    \item Standard in modern NLP models
\end{itemize}

\textbf{Implementation best practices:}
\begin{itemize}
    \item Place after Conv/Linear, before activation (standard)
    \item Set \texttt{bias=False} in layer before batch norm
    \item Always call \texttt{model.eval()} during evaluation
    \item Use \texttt{torch.no\_grad()} during inference
    \item For sequences: prefer Layer Norm
    \item For small batches: use Layer Norm or Group Norm
\end{itemize}

\textbf{Common mistakes:}
\begin{itemize}
    \item Forgetting \texttt{model.eval()} (very common bug!)
    \item Using Batch Norm with tiny batches
    \item Not setting \texttt{bias=False} (wastes parameters)
    \item Wrong normalization type for task (Batch Norm for RNNs)
    \item Not understanding train vs eval mode difference
\end{itemize}

\textbf{Choosing normalization:}
\begin{itemize}
    \item \textbf{Default for CNNs:} Batch Norm
    \item \textbf{Default for Transformers/RNNs:} Layer Norm
    \item \textbf{Small batches (<8):} Layer Norm or Group Norm
    \item \textbf{Style transfer:} Instance Norm
    \item \textbf{Object detection:} Group Norm (common in modern detectors)
\end{itemize}

\textbf{When NOT to use normalization:}
\begin{itemize}
    \item Very shallow networks (<5 layers)
    \item Already training stably
    \item Certain GANs (can cause mode collapse)
    \item When exact reproducibility needed (batch stats introduce randomness)
\end{itemize}

\clearpage
% =============================================
% SECTION 10: RECURRENT NEURAL NETWORKS (RNNs)
% =============================================

\section{Recurrent Neural Networks (RNNs)}

\subsection{Introduction: Processing Sequential Data}

So far, we've seen networks that process fixed-size inputs. But many real-world problems involve \textbf{sequences}:
\begin{itemize}
    \item Time series (stock prices, sensor data, climate data)
    \item Text (words, characters, sentences)
    \item Audio (speech, music)
    \item Video (frames over time)
    \item Biological sequences (DNA, proteins)
\end{itemize}

\textbf{Challenges with sequences:}
\begin{enumerate}
    \item Variable length (sentences have different lengths)
    \item Temporal dependencies (past affects future)
    \item Need to maintain state (remember what happened before)
\end{enumerate}

\textbf{Why not use MLPs or CNNs?}

\textbf{MLPs:}
\begin{itemize}
    \item Fixed input size (can't handle variable-length sequences)
    \item No notion of order (position 1 vs position 100 treated same)
    \item Can't share parameters across time (learn same pattern at different positions)
\end{itemize}

\textbf{CNNs:}
\begin{itemize}
    \item Can handle sequences with 1D convolutions
    \item Limited temporal context (receptive field)
    \item No true memory of long-term dependencies
\end{itemize}

\textbf{RNNs solve these problems} by maintaining a \textbf{hidden state} that gets updated at each time step.

\subsection{Theory: How RNNs Work}

\subsubsection{The Basic RNN}

An RNN processes a sequence one element at a time, maintaining a hidden state:

\textbf{At each time step $t$:}
\[
h_t = \tanh(W_{hh} h_{t-1} + W_{xh} x_t + b_h)
\]
\[
y_t = W_{hy} h_t + b_y
\]

where:
\begin{itemize}
    \item $x_t$: Input at time $t$
    \item $h_t$: Hidden state at time $t$ (memory)
    \item $y_t$: Output at time $t$
    \item $W_{hh}$: Hidden-to-hidden weights (memory transformation)
    \item $W_{xh}$: Input-to-hidden weights
    \item $W_{hy}$: Hidden-to-output weights
\end{itemize}

\textbf{Key idea:} The hidden state $h_t$ summarizes information from all previous time steps.

\begin{theorybox}[The Hidden State]
The hidden state $h_t$ is the RNN's \textbf{memory}:
\begin{itemize}
    \item $h_0$: Initial memory (usually zeros)
    \item $h_1 = f(x_1, h_0)$: Combines first input with initial state
    \item $h_2 = f(x_2, h_1)$: Combines second input with previous state
    \item $h_3 = f(x_3, h_2)$: And so on...
\end{itemize}

By the end: $h_T$ contains information about the entire sequence $x_1, x_2, \ldots, x_T$.
\end{theorybox}

\textbf{Unrolled view:}

\begin{verbatim}
x1 → [RNN] → h1 → [RNN] → h2 → [RNN] → h3 → ...
      ↓            ↓            ↓
      y1           y2           y3
\end{verbatim}

All boxes share the \textbf{same weights} (parameter sharing across time).

\clearpage
\subsubsection{Parameter Sharing}

\textbf{Why parameter sharing matters:}

For a sequence of length 100:
\begin{itemize}
    \item MLP: Would need 100 separate weight matrices (millions of parameters)
    \item RNN: Uses same weights at each step (thousands of parameters)
\end{itemize}

The network learns patterns that work at \textbf{any position} in the sequence.

\subsubsection{Sequence Modeling Patterns}

\textbf{1. One-to-Many (sequence generation):}
\begin{verbatim}
x → [RNN] → [RNN] → [RNN] → ...
             ↓       ↓       ↓
             y1      y2      y3
Example: Image captioning (image → sentence)
\end{verbatim}

\textbf{2. Many-to-One (sequence classification):}
\begin{verbatim}
x1 → [RNN] → x2 → [RNN] → x3 → [RNN]
                                 ↓
                                 y
Example: Sentiment analysis (sentence → positive/negative)
\end{verbatim}

\textbf{3. Many-to-Many (same length):}
\begin{verbatim}
x1 → [RNN] → x2 → [RNN] → x3 → [RNN]
     ↓           ↓           ↓
     y1          y2          y3
Example: Part-of-speech tagging (word → tag)
\end{verbatim}

\textbf{4. Many-to-Many (different length, encoder-decoder):}
\begin{verbatim}
Encoder:  x1 → [RNN] → x2 → [RNN] → h
Decoder:  h → [RNN] → [RNN] → [RNN]
               ↓       ↓       ↓
               y1      y2      y3
Example: Machine translation (English → French)
\end{verbatim}

\subsubsection{The Vanishing Gradient Problem}

\textbf{Problem:} Vanilla RNNs struggle with long sequences.

When backpropagating through time, gradients are multiplied by $W_{hh}$ at each step:

\[
\frac{\partial h_t}{\partial h_0} = \prod_{i=1}^{t} W_{hh} \cdot \text{diag}(\tanh'(h_i))
\]

If $|W_{hh}| < 1$: Gradients vanish exponentially (network can't learn long-term dependencies)

If $|W_{hh}| > 1$: Gradients explode (training becomes unstable)

\textbf{In practice:} Vanilla RNNs can only remember ~10-20 steps back.

\textbf{Solution:} LSTM and GRU architectures (next sections).

\clearpage
\subsection{Theory: Long Short-Term Memory (LSTM)}

LSTMs solve the vanishing gradient problem through a clever gating mechanism.

\subsubsection{LSTM Architecture}

LSTM has two states:
\begin{itemize}
    \item $h_t$: Hidden state (short-term memory)
    \item $c_t$: Cell state (long-term memory)
\end{itemize}

And three gates:
\begin{itemize}
    \item $f_t$: Forget gate (what to remove from memory)
    \item $i_t$: Input gate (what new information to add)
    \item $o_t$: Output gate (what to output)
\end{itemize}

\textbf{LSTM Equations:}

\textbf{1. Forget gate (what to forget):}
\[
f_t = \sigma(W_f [h_{t-1}, x_t] + b_f)
\]

\textbf{2. Input gate (what to add):}
\[
i_t = \sigma(W_i [h_{t-1}, x_t] + b_i)
\]
\[
\tilde{c}_t = \tanh(W_c [h_{t-1}, x_t] + b_c) \quad \text{(candidate values)}
\]

\textbf{3. Update cell state:}
\[
c_t = f_t \odot c_{t-1} + i_t \odot \tilde{c}_t
\]

\textbf{4. Output gate (what to output):}
\[
o_t = \sigma(W_o [h_{t-1}, x_t] + b_o)
\]
\[
h_t = o_t \odot \tanh(c_t)
\]

where $\odot$ is element-wise multiplication, $\sigma$ is sigmoid, $[a, b]$ is concatenation.

\begin{theorybox}[LSTM Intuition]
\textbf{Cell state ($c_t$):} The "memory highway"
\begin{itemize}
    \item Runs through the entire sequence
    \item Modified only by additions and element-wise multiplications
    \item Gradients flow easily (no repeated matrix multiplications!)
\end{itemize}

\textbf{Forget gate ($f_t$):} "Should I forget the past?"
\begin{itemize}
    \item Values near 0: Forget most of $c_{t-1}$
    \item Values near 1: Keep most of $c_{t-1}$
\end{itemize}

\textbf{Input gate ($i_t$):} "Should I add this new information?"
\begin{itemize}
    \item Controls how much of $\tilde{c}_t$ to add to memory
\end{itemize}

\textbf{Output gate ($o_t$):} "How much of memory should I expose?"
\begin{itemize}
    \item Controls how much of $c_t$ becomes $h_t$
\end{itemize}
\end{theorybox}

\textbf{Why LSTM works:}

The cell state $c_t$ provides a path where gradients can flow without repeated matrix multiplications (just element-wise operations). This solves vanishing gradients!

\clearpage
\subsection{Theory: Gated Recurrent Unit (GRU)}

GRU is a simplified version of LSTM with fewer parameters.

\subsubsection{GRU Architecture}

GRU combines forget and input gates into a single \textbf{update gate} and merges cell state with hidden state.

\textbf{GRU Equations:}

\textbf{1. Update gate (how much to update):}
\[
z_t = \sigma(W_z [h_{t-1}, x_t] + b_z)
\]

\textbf{2. Reset gate (how much past to forget):}
\[
r_t = \sigma(W_r [h_{t-1}, x_t] + b_r)
\]

\textbf{3. Candidate hidden state:}
\[
\tilde{h}_t = \tanh(W_h [r_t \odot h_{t-1}, x_t] + b_h)
\]

\textbf{4. Update hidden state:}
\[
h_t = (1 - z_t) \odot h_{t-1} + z_t \odot \tilde{h}_t
\]

\textbf{Intuition:}
\begin{itemize}
    \item $z_t \approx 0$: Keep old state $h_{t-1}$ (don't update)
    \item $z_t \approx 1$: Use new candidate $\tilde{h}_t$ (full update)
    \item $z_t \approx 0.5$: Blend old and new
\end{itemize}

\textbf{LSTM vs GRU:}

\begin{table}[h]
\centering
\begin{tabular}{lll}
\toprule
\textbf{Aspect} & \textbf{LSTM} & \textbf{GRU} \\
\midrule
Gates & 3 (forget, input, output) & 2 (update, reset) \\
States & 2 ($h_t$, $c_t$) & 1 ($h_t$) \\
Parameters & More & Fewer (faster) \\
Performance & Slightly better & Nearly as good \\
Training speed & Slower & Faster \\
Memory & More & Less \\
\bottomrule
\end{tabular}
\caption{LSTM vs GRU comparison}
\end{table}

\textbf{When to use each:}
\begin{itemize}
    \item \textbf{LSTM:} Default choice, especially for complex tasks
    \item \textbf{GRU:} When speed matters or dataset is smaller
    \item \textbf{Rule of thumb:} Try both, see which works better
\end{itemize}

\subsection{Implementation: RNNs in PyTorch}

\subsubsection{Vanilla RNN}

\begin{lstlisting}
import torch
import torch.nn as nn

# Simple RNN cell
rnn_cell = nn.RNNCell(input_size=10, hidden_size=20)

# Input: (batch, input_size)
# Hidden: (batch, hidden_size)
x = torch.randn(32, 10)  # Batch of 32
h = torch.zeros(32, 20)  # Initial hidden state

# One step
h_next = rnn_cell(x, h)
print(h_next.shape)  # torch.Size([32, 20])

# Full RNN layer (processes entire sequence)
rnn = nn.RNN(input_size=10, hidden_size=20, num_layers=1, batch_first=True)

# Input: (batch, seq_len, input_size)
x = torch.randn(32, 100, 10)  # 32 sequences, length 100
h0 = torch.zeros(1, 32, 20)   # (num_layers, batch, hidden_size)

# Forward pass
output, h_n = rnn(x, h0)
# output: (32, 100, 20) - outputs at each time step
# h_n: (1, 32, 20) - final hidden state

print(output.shape, h_n.shape)
\end{lstlisting}

\begin{pytorchtip}[batch\_first Parameter]
PyTorch RNNs default to shape (seq\_len, batch, features), but this is confusing!

Always use \texttt{batch\_first=True} for shape (batch, seq\_len, features):
\begin{lstlisting}
rnn = nn.RNN(..., batch_first=True)  # Recommended
# Input: (batch, seq_len, features)
# Output: (batch, seq_len, hidden_size)
\end{lstlisting}
\end{pytorchtip}

\clearpage
\subsubsection{LSTM}

\begin{lstlisting}
# LSTM cell
lstm_cell = nn.LSTMCell(input_size=10, hidden_size=20)

x = torch.randn(32, 10)
h = torch.zeros(32, 20)
c = torch.zeros(32, 20)  # Cell state

# One step
h_next, c_next = lstm_cell(x, (h, c))  # Note: tuple of (h, c)

# Full LSTM layer
lstm = nn.LSTM(input_size=10, hidden_size=20, num_layers=2, 
               batch_first=True, dropout=0.2)

x = torch.randn(32, 100, 10)
h0 = torch.zeros(2, 32, 20)  # (num_layers, batch, hidden)
c0 = torch.zeros(2, 32, 20)  # (num_layers, batch, hidden)

# Forward pass
output, (h_n, c_n) = lstm(x, (h0, c0))
# output: (32, 100, 20) - outputs at each time step
# h_n: (2, 32, 20) - final hidden state for each layer
# c_n: (2, 32, 20) - final cell state for each layer

print(output.shape, h_n.shape, c_n.shape)
\end{lstlisting}

\begin{warningbox}[LSTM Hidden State is a Tuple!]
LSTM returns \texttt{(h, c)} as a tuple, not just \texttt{h}:

\begin{lstlisting}
# WRONG:
output, hidden = lstm(x)
h_n = hidden  # This is a tuple, not a tensor!

# CORRECT:
output, (h_n, c_n) = lstm(x, (h0, c0))
# or
output, hidden = lstm(x, (h0, c0))
h_n, c_n = hidden
\end{lstlisting}
\end{warningbox}

\subsubsection{GRU}

\begin{lstlisting}
# GRU (similar to LSTM, but only one hidden state)
gru = nn.GRU(input_size=10, hidden_size=20, num_layers=2, 
             batch_first=True, dropout=0.2)

x = torch.randn(32, 100, 10)
h0 = torch.zeros(2, 32, 20)

# Forward pass
output, h_n = gru(x, h0)  # Only h, no cell state
# output: (32, 100, 20)
# h_n: (2, 32, 20)

print(output.shape, h_n.shape)
\end{lstlisting}

\subsubsection{Building a Sequence Classifier}

\begin{lstlisting}
class SequenceClassifier(nn.Module):
    """Many-to-one: sequence → single label."""
    
    def __init__(self, input_size, hidden_size, num_classes, num_layers=2):
        super().__init__()
        
        self.lstm = nn.LSTM(
            input_size=input_size,
            hidden_size=hidden_size,
            num_layers=num_layers,
            batch_first=True,
            dropout=0.2 if num_layers > 1 else 0
        )
        
        self.fc = nn.Linear(hidden_size, num_classes)
    
    def forward(self, x):
        # x: (batch, seq_len, input_size)
        
        # LSTM forward
        output, (h_n, c_n) = self.lstm(x)
        # output: (batch, seq_len, hidden_size)
        # h_n: (num_layers, batch, hidden_size)
        
        # Use last time step's output
        last_output = output[:, -1, :]  # (batch, hidden_size)
        
        # Or use final hidden state from last layer
        # last_hidden = h_n[-1]  # (batch, hidden_size)
        
        # Classify
        logits = self.fc(last_output)
        return logits

# Test
model = SequenceClassifier(input_size=10, hidden_size=64, num_classes=3)
x = torch.randn(32, 100, 10)  # 32 sequences, length 100
output = model(x)
print(output.shape)  # torch.Size([32, 3])
\end{lstlisting}

\clearpage
\subsubsection{Sequence-to-Sequence (Many-to-Many)}

\begin{lstlisting}
class SequenceLabeler(nn.Module):
    """Many-to-many: each input → each output (same length)."""
    
    def __init__(self, input_size, hidden_size, num_classes):
        super().__init__()
        
        self.lstm = nn.LSTM(
            input_size=input_size,
            hidden_size=hidden_size,
            num_layers=2,
            batch_first=True,
            dropout=0.2
        )
        
        # Apply classifier at each time step
        self.fc = nn.Linear(hidden_size, num_classes)
    
    def forward(self, x):
        # x: (batch, seq_len, input_size)
        
        output, _ = self.lstm(x)
        # output: (batch, seq_len, hidden_size)
        
        # Apply classifier to all time steps
        logits = self.fc(output)
        # logits: (batch, seq_len, num_classes)
        
        return logits

# Test
model = SequenceLabeler(input_size=10, hidden_size=64, num_classes=5)
x = torch.randn(32, 100, 10)
output = model(x)
print(output.shape)  # torch.Size([32, 100, 5])
\end{lstlisting}

\subsubsection{Bidirectional RNNs}

Process sequence in both directions (forward and backward):

\begin{lstlisting}
class BidirectionalRNN(nn.Module):
    """Bidirectional LSTM for sequence classification."""
    
    def __init__(self, input_size, hidden_size, num_classes):
        super().__init__()
        
        self.lstm = nn.LSTM(
            input_size=input_size,
            hidden_size=hidden_size,
            num_layers=2,
            batch_first=True,
            dropout=0.2,
            bidirectional=True  # Process both directions
        )
        
        # Hidden size is doubled (forward + backward)
        self.fc = nn.Linear(hidden_size * 2, num_classes)
    
    def forward(self, x):
        # x: (batch, seq_len, input_size)
        
        output, _ = self.lstm(x)
        # output: (batch, seq_len, hidden_size * 2)
        
        # Use last time step (or all for seq-to-seq)
        last_output = output[:, -1, :]
        
        logits = self.fc(last_output)
        return logits

# Test
model = BidirectionalRNN(input_size=10, hidden_size=64, num_classes=3)
x = torch.randn(32, 100, 10)
output = model(x)
print(output.shape)  # torch.Size([32, 3])
\end{lstlisting}

\textbf{When to use bidirectional:}
\begin{itemize}
    \item Text classification (can see future context)
    \item Part-of-speech tagging
    \item Named entity recognition
    \item Any task where you have the full sequence before prediction
\end{itemize}

\textbf{Don't use bidirectional for:}
\begin{itemize}
    \item Time series prediction (can't see future!)
    \item Language generation (need causal/autoregressive)
    \item Online/streaming applications
\end{itemize}

\clearpage
\subsubsection{Handling Variable-Length Sequences}

\begin{lstlisting}
from torch.nn.utils.rnn import pack_padded_sequence, pad_packed_sequence

class VariableLengthRNN(nn.Module):
    """Efficiently handle sequences of different lengths."""
    
    def __init__(self, input_size, hidden_size, num_classes):
        super().__init__()
        self.lstm = nn.LSTM(input_size, hidden_size, batch_first=True)
        self.fc = nn.Linear(hidden_size, num_classes)
    
    def forward(self, x, lengths):
        """
        Args:
            x: Padded sequences (batch, max_len, input_size)
            lengths: Actual lengths of each sequence
        """
        # Pack sequences (skip padding in computation)
        packed = pack_padded_sequence(x, lengths, batch_first=True, 
                                     enforce_sorted=False)
        
        # LSTM forward on packed sequence
        packed_output, (h_n, c_n) = self.lstm(packed)
        
        # Unpack
        output, _ = pad_packed_sequence(packed_output, batch_first=True)
        
        # Use final hidden state
        logits = self.fc(h_n[-1])
        
        return logits

# Example usage
# Sequences of different lengths: [50, 30, 40]
lengths = torch.tensor([50, 30, 40])
max_len = lengths.max()

# Pad to max length
x = torch.randn(3, max_len, 10)  # Some values are padding

model = VariableLengthRNN(10, 64, 3)
output = model(x, lengths)
print(output.shape)  # torch.Size([3, 3])
\end{lstlisting}

\textbf{Why pack sequences?}
\begin{itemize}
    \item Skip computation on padding (faster)
    \item More accurate (padding doesn't affect hidden state)
    \item More memory efficient
\end{itemize}

\clearpage
% =============================================
% SECTION 10: RNNs - PART 2 (Advanced Topics & Exercises)
% =============================================

\subsubsection{Gradient Clipping (Essential for RNNs)}

RNNs are prone to exploding gradients. Gradient clipping is \textbf{essential}.

\begin{lstlisting}
# Method 1: Clip by norm (recommended)
torch.nn.utils.clip_grad_norm_(model.parameters(), max_norm=5.0)

# Method 2: Clip by value
torch.nn.utils.clip_grad_value_(model.parameters(), clip_value=0.5)

# In training loop
for epoch in range(num_epochs):
    for batch in dataloader:
        optimizer.zero_grad()
        
        output = model(batch)
        loss = criterion(output, target)
        loss.backward()
        
        # Clip gradients AFTER backward, BEFORE step
        torch.nn.utils.clip_grad_norm_(model.parameters(), max_norm=5.0)
        
        optimizer.step()
\end{lstlisting}

\begin{warningbox}[Always Use Gradient Clipping for RNNs]
Without gradient clipping:
\begin{itemize}
    \item Loss may suddenly spike to NaN
    \item Training becomes unstable
    \item Network may never converge
\end{itemize}

\textbf{Typical values:}
\begin{itemize}
    \item \texttt{max\_norm=1.0}: Conservative (very stable)
    \item \texttt{max\_norm=5.0}: Standard (good default)
    \item \texttt{max\_norm=10.0}: Aggressive (faster but less stable)
\end{itemize}
\end{warningbox}

\subsubsection{Training Tips for RNNs}

\textbf{1. Learning rate}
\begin{lstlisting}
# RNNs typically need lower learning rates than CNNs
optimizer = torch.optim.Adam(model.parameters(), lr=0.001)  # Good start

# Use learning rate scheduling
scheduler = torch.optim.lr_scheduler.ReduceLROnPlateau(
    optimizer, mode='min', factor=0.5, patience=5
)
\end{lstlisting}

\textbf{2. Weight initialization}
\begin{lstlisting}
def init_lstm_weights(lstm):
    """Initialize LSTM weights for better convergence."""
    for name, param in lstm.named_parameters():
        if 'weight_ih' in name:
            # Input-hidden weights: Xavier
            nn.init.xavier_uniform_(param)
        elif 'weight_hh' in name:
            # Hidden-hidden weights: Orthogonal (better for RNNs)
            nn.init.orthogonal_(param)
        elif 'bias' in name:
            # Biases: zeros, except forget gate bias = 1
            nn.init.zeros_(param)
            # Set forget gate bias to 1 (helps gradient flow)
            n = param.size(0)
            param.data[n//4:n//2].fill_(1.0)

# Apply to model
for name, module in model.named_modules():
    if isinstance(module, nn.LSTM):
        init_lstm_weights(module)
\end{lstlisting}

\textbf{3. Dropout placement}
\begin{lstlisting}
# Dropout between layers (built-in)
lstm = nn.LSTM(input_size, hidden_size, num_layers=3, dropout=0.2)
# This applies dropout between layers, not within recurrent connections

# For dropout on inputs/outputs:
class RNNWithDropout(nn.Module):
    def __init__(self, input_size, hidden_size, num_classes):
        super().__init__()
        self.input_dropout = nn.Dropout(0.2)
        self.lstm = nn.LSTM(input_size, hidden_size, num_layers=2)
        self.output_dropout = nn.Dropout(0.5)
        self.fc = nn.Linear(hidden_size, num_classes)
    
    def forward(self, x):
        x = self.input_dropout(x)
        output, _ = self.lstm(x)
        output = self.output_dropout(output[:, -1, :])
        return self.fc(output)
\end{lstlisting}

\textbf{4. Batch size considerations}

RNNs benefit from larger batch sizes:
\begin{itemize}
    \item Batch size 32-128 typically good
    \item Too small (<16): Noisy gradients, unstable training
    \item Too large (>256): May hurt generalization
\end{itemize}

\clearpage
\subsubsection{Common RNN Architectures}

\textbf{Stacked RNNs (Deep RNNs):}
\begin{lstlisting}
class DeepLSTM(nn.Module):
    """Multiple LSTM layers stacked."""
    
    def __init__(self, input_size, hidden_size, num_layers, num_classes):
        super().__init__()
        
        # num_layers > 1 creates stacked LSTM
        self.lstm = nn.LSTM(
            input_size=input_size,
            hidden_size=hidden_size,
            num_layers=num_layers,
            batch_first=True,
            dropout=0.3 if num_layers > 1 else 0  # Dropout between layers
        )
        
        self.fc = nn.Linear(hidden_size, num_classes)
    
    def forward(self, x):
        output, (h_n, c_n) = self.lstm(x)
        # h_n[-1]: hidden state from the last layer
        return self.fc(h_n[-1])

# Typical depths: 2-4 layers
# Deeper than 4 often doesn't help much
model = DeepLSTM(input_size=10, hidden_size=128, num_layers=3, 
                 num_classes=10)
\end{lstlisting}

\textbf{Encoder-Decoder (Seq2Seq):}
\begin{lstlisting}
class Seq2Seq(nn.Module):
    """Encoder-decoder for sequence-to-sequence tasks."""
    
    def __init__(self, input_size, hidden_size, output_size):
        super().__init__()
        
        # Encoder: processes input sequence
        self.encoder = nn.LSTM(input_size, hidden_size, batch_first=True)
        
        # Decoder: generates output sequence
        self.decoder = nn.LSTM(output_size, hidden_size, batch_first=True)
        
        # Output projection
        self.fc = nn.Linear(hidden_size, output_size)
    
    def forward(self, src, tgt, teacher_forcing_ratio=0.5):
        """
        Args:
            src: Source sequence (batch, src_len, input_size)
            tgt: Target sequence (batch, tgt_len, output_size)
            teacher_forcing_ratio: Probability of using true target vs prediction
        """
        batch_size = src.size(0)
        tgt_len = tgt.size(1)
        output_size = tgt.size(2)
        
        # Encode source
        _, (h_n, c_n) = self.encoder(src)
        
        # Initialize decoder hidden state with encoder final state
        decoder_hidden = (h_n, c_n)
        
        # First decoder input: start token (zeros)
        decoder_input = torch.zeros(batch_size, 1, output_size).to(src.device)
        
        outputs = []
        
        # Generate sequence step by step
        for t in range(tgt_len):
            # Decode one step
            decoder_output, decoder_hidden = self.decoder(
                decoder_input, decoder_hidden
            )
            
            # Project to output space
            prediction = self.fc(decoder_output)
            outputs.append(prediction)
            
            # Teacher forcing: use true target or prediction?
            use_teacher_forcing = torch.rand(1).item() < teacher_forcing_ratio
            
            if use_teacher_forcing:
                decoder_input = tgt[:, t:t+1, :]  # Use true target
            else:
                decoder_input = prediction  # Use prediction
        
        # Stack all outputs
        outputs = torch.cat(outputs, dim=1)
        return outputs

# Test
model = Seq2Seq(input_size=10, hidden_size=64, output_size=10)
src = torch.randn(32, 20, 10)  # 32 sequences, length 20
tgt = torch.randn(32, 15, 10)  # Target length 15
output = model(src, tgt)
print(output.shape)  # torch.Size([32, 15, 10])
\end{lstlisting}

\clearpage
\subsubsection{Time Series Forecasting Example}

\begin{lstlisting}
class TimeSeriesForecaster(nn.Module):
    """Predict future values from past observations."""
    
    def __init__(self, input_size=1, hidden_size=64, num_layers=2, 
                 forecast_horizon=10):
        super().__init__()
        
        self.hidden_size = hidden_size
        self.num_layers = num_layers
        self.forecast_horizon = forecast_horizon
        
        # Encoder LSTM
        self.lstm = nn.LSTM(
            input_size=input_size,
            hidden_size=hidden_size,
            num_layers=num_layers,
            batch_first=True,
            dropout=0.2 if num_layers > 1 else 0
        )
        
        # Decoder: predict next step
        self.fc = nn.Linear(hidden_size, input_size)
    
    def forward(self, x, future_steps=0):
        """
        Args:
            x: Past observations (batch, seq_len, input_size)
            future_steps: How many steps to predict into future
        
        Returns:
            predictions: (batch, seq_len + future_steps, input_size)
        """
        batch_size = x.size(0)
        
        # Process historical data
        output, (h, c) = self.lstm(x)
        
        # Predict on historical data
        predictions = self.fc(output)
        
        # If we need to predict future steps
        if future_steps > 0:
            future_preds = []
            
            # Use last prediction as next input
            last_pred = predictions[:, -1:, :]
            
            for _ in range(future_steps):
                # Predict next step
                output, (h, c) = self.lstm(last_pred, (h, c))
                pred = self.fc(output)
                future_preds.append(pred)
                
                # Use prediction as next input (autoregressive)
                last_pred = pred
            
            # Concatenate all predictions
            future_preds = torch.cat(future_preds, dim=1)
            predictions = torch.cat([predictions, future_preds], dim=1)
        
        return predictions

# Example usage
model = TimeSeriesForecaster(input_size=1, hidden_size=64)

# Historical data: 100 time steps
x = torch.randn(32, 100, 1)

# Predict on historical + 20 future steps
predictions = model(x, future_steps=20)
print(predictions.shape)  # torch.Size([32, 120, 1])
\end{lstlisting}

\subsection{Exercises}

\begin{exercise}[10.1: Simple Sequence Classification - $\bigstar\bigstar$]
\textbf{Goal:} Build your first RNN.

\begin{enumerate}
    \item Generate synthetic sequences: sine waves (class 0) and cosine waves (class 1)
    \item Each sequence: 50 time steps
    \item Build an LSTM classifier
    \item Train for 20 epochs
    \item Achieve >95\% accuracy
\end{enumerate}

\textbf{Starter code:}
\begin{lstlisting}
import numpy as np

def generate_sine_data(n_samples=1000, seq_len=50):
    """Generate sine wave sequences."""
    X = []
    y = []
    for _ in range(n_samples):
        # Random frequency and phase
        freq = np.random.uniform(0.5, 2.0)
        phase = np.random.uniform(0, 2*np.pi)
        t = np.linspace(0, 4*np.pi, seq_len)
        
        if np.random.rand() < 0.5:
            # Sine wave
            seq = np.sin(freq * t + phase)
            label = 0
        else:
            # Cosine wave
            seq = np.cos(freq * t + phase)
            label = 1
        
        X.append(seq)
        y.append(label)
    
    return torch.FloatTensor(X).unsqueeze(-1), torch.LongTensor(y)

# Your LSTM classifier here
\end{lstlisting}
\end{exercise}

\begin{exercise}[10.2: LSTM vs GRU Comparison - $\bigstar\bigstar\bigstar$]
\textbf{Goal:} Compare LSTM and GRU empirically.

\begin{enumerate}
    \item Generate a moderately complex sequence task
    \item Train identical architectures with:
    \begin{itemize}
        \item LSTM (2 layers, 128 hidden)
        \item GRU (2 layers, 128 hidden)
        \item Vanilla RNN (2 layers, 128 hidden)
    \end{itemize}
    \item Compare:
    \begin{itemize}
        \item Training speed (time per epoch)
        \item Convergence speed (epochs to 90\% accuracy)
        \item Final accuracy
        \item Number of parameters
    \end{itemize}
\end{enumerate}

\textbf{Expected results:}
\begin{itemize}
    \item LSTM and GRU should perform similarly
    \item GRU should be faster (fewer parameters)
    \item Vanilla RNN should struggle (vanishing gradients)
\end{itemize}
\end{exercise}

\begin{exercise}[10.3: Sequence-to-Sequence Labeling - $\bigstar\bigstar\bigstar$]
\textbf{Goal:} Many-to-many prediction.

\begin{enumerate}
    \item Task: Given a noisy sine wave, denoise it at each time step
    \item Input: sine + Gaussian noise
    \item Target: clean sine wave
    \item Build a bidirectional LSTM
    \item Output prediction at each time step
    \item Visualize: noisy input vs clean output
\end{enumerate}

\textbf{Evaluation:}
\begin{itemize}
    \item Mean squared error on test set
    \item Visual inspection of denoised signals
\end{itemize}
\end{exercise}

\begin{exercise}[10.4: Variable-Length Sequences - $\bigstar\bigstar\bigstar$]
\textbf{Goal:} Handle sequences of different lengths efficiently.

\begin{enumerate}
    \item Generate sequences with random lengths (20-100)
    \item Implement proper padding and packing:
    \begin{itemize}
        \item Pad sequences to same length
        \item Use \texttt{pack\_padded\_sequence}
        \item Use \texttt{pad\_packed\_sequence}
    \end{itemize}
    \item Train two models:
    \begin{itemize}
        \item Without packing (processes padding)
        \item With packing (skips padding)
    \end{itemize}
    \item Compare:
    \begin{itemize}
        \item Training time
        \item Accuracy
        \item Verify packing gives same results
    \end{itemize}
\end{enumerate}
\end{exercise}

\begin{exercise}[10.5: Time Series Forecasting - $\bigstar\bigstar\bigstar\bigstar$]
\textbf{Goal:} Predict future values.

\begin{enumerate}
    \item Use real time series data (or generate complex synthetic data)
    \item Sliding window approach:
    \begin{itemize}
        \item Input: past 50 steps
        \item Target: next 10 steps
    \end{itemize}
    \item Build forecasting model with LSTM
    \item Implement autoregressive prediction (use predictions as inputs)
    \item Evaluate:
    \begin{itemize}
        \item One-step-ahead prediction
        \item Multi-step-ahead prediction
        \item Compare with baseline (simple moving average)
    \end{itemize}
    \item Visualize predictions vs ground truth
\end{enumerate}

\textbf{Metrics:}
\begin{lstlisting}
def calculate_metrics(y_true, y_pred):
    """Calculate forecasting metrics."""
    mse = torch.mean((y_true - y_pred) ** 2)
    mae = torch.mean(torch.abs(y_true - y_pred))
    
    # MAPE (Mean Absolute Percentage Error)
    mape = torch.mean(torch.abs((y_true - y_pred) / (y_true + 1e-8))) * 100
    
    return {
        'MSE': mse.item(),
        'MAE': mae.item(),
        'MAPE': mape.item()
    }
\end{lstlisting}
\end{exercise}

\begin{exercise}[10.6: Gradient Clipping Investigation - $\bigstar\bigstar\bigstar\bigstar$]
\textbf{Goal:} Understand why gradient clipping is essential.

\begin{enumerate}
    \item Train an RNN on a long sequence task (100+ steps)
    \item Train multiple times with different settings:
    \begin{itemize}
        \item No gradient clipping
        \item \texttt{max\_norm=1.0}
        \item \texttt{max\_norm=5.0}
        \item \texttt{max\_norm=10.0}
    \end{itemize}
    \item Monitor and plot:
    \begin{itemize}
        \item Gradient norms over time
        \item Loss curves
        \item Percentage of batches where gradients are clipped
    \end{itemize}
    \item Observe:
    \begin{itemize}
        \item Without clipping: loss spikes, NaN values
        \item With clipping: stable training
    \end{itemize}
\end{enumerate}

\textbf{Visualization code:}
\begin{lstlisting}
def track_gradient_norms(model):
    """Track gradient norms during training."""
    total_norm = 0
    for p in model.parameters():
        if p.grad is not None:
            param_norm = p.grad.data.norm(2)
            total_norm += param_norm.item() ** 2
    total_norm = total_norm ** 0.5
    return total_norm

# In training loop
gradient_norms = []
for epoch in range(num_epochs):
    for batch in dataloader:
        optimizer.zero_grad()
        loss = ...
        loss.backward()
        
        # Track before clipping
        norm_before = track_gradient_norms(model)
        gradient_norms.append(norm_before)
        
        torch.nn.utils.clip_grad_norm_(model.parameters(), max_norm=5.0)
        optimizer.step()

# Plot
plt.plot(gradient_norms)
plt.axhline(y=5.0, color='r', linestyle='--', label='Clip threshold')
plt.xlabel('Step')
plt.ylabel('Gradient Norm')
plt.legend()
plt.show()
\end{lstlisting}
\end{exercise}

\clearpage
\subsection{Key Takeaways}

\textbf{When to use RNNs:}
\begin{itemize}
    \item Sequential data with temporal dependencies
    \item Variable-length sequences
    \item Time series forecasting
    \item Text processing (though Transformers now dominate)
    \item When order matters
\end{itemize}

\textbf{Architecture choices:}
\begin{itemize}
    \item \textbf{Vanilla RNN:} Almost never use (vanishing gradients)
    \item \textbf{LSTM:} Default choice, especially for complex tasks
    \item \textbf{GRU:} When speed matters or dataset is smaller
    \item \textbf{Bidirectional:} When you have full sequence (not streaming)
    \item \textbf{Stacked (2-4 layers):} For complex patterns
\end{itemize}

\textbf{Essential training practices:}
\begin{itemize}
    \item \textbf{Always use gradient clipping} (max\_norm=5.0 is good default)
    \item Initialize forget gate bias to 1 (helps gradient flow)
    \item Use orthogonal initialization for hidden-to-hidden weights
    \item Lower learning rates than CNNs (0.001 typical)
    \item Larger batch sizes (32-128)
    \item Pack sequences when lengths vary (efficiency)
\end{itemize}

\textbf{Common patterns:}
\begin{itemize}
    \item \textbf{Many-to-one:} Classification (use last hidden state)
    \item \textbf{Many-to-many (same length):} Sequence labeling (output at each step)
    \item \textbf{Many-to-many (different length):} Seq2seq with encoder-decoder
    \item \textbf{One-to-many:} Generation (feed output back as input)
\end{itemize}

\textbf{Why LSTMs work better than vanilla RNNs:}
\begin{itemize}
    \item Cell state provides gradient highway (no repeated matrix multiplications)
    \item Gates control information flow (selective memory)
    \item Can learn dependencies over 100+ steps
    \item Forget gate prevents exploding activations
\end{itemize}

\textbf{Limitations of RNNs:}
\begin{itemize}
    \item Sequential processing (slow, can't parallelize)
    \item Still struggle with very long sequences (>1000 steps)
    \item Transformers now preferred for many NLP tasks
    \item Require careful tuning (gradient clipping, learning rate)
\end{itemize}

\textbf{Practical tips:}
\begin{itemize}
    \item Start with 2-layer LSTM, hidden size = 128 or 256
    \item Use \texttt{batch\_first=True} always
    \item Remember LSTM returns \texttt{(h, c)} tuple
    \item Dropout between layers (0.2-0.3), higher on outputs (0.5)
    \item Monitor gradient norms (should be <10)
    \item Use early stopping (RNNs prone to overfitting)
\end{itemize}

\textbf{Debugging checklist:}
\begin{itemize}
    \item Loss explodes? → Add/increase gradient clipping
    \item Loss stuck? → Check learning rate, try lower
    \item Poor performance? → Try LSTM instead of GRU, add layers
    \item Slow training? → Use packing for variable lengths
    \item Different train/test? → Check dropout mode
\end{itemize}

\textbf{Modern alternatives:}
\begin{itemize}
    \item \textbf{Transformers:} Better for most NLP tasks (parallelizable)
    \item \textbf{Temporal CNNs:} Good for some time series (efficient)
    \item \textbf{State space models:} Emerging alternative (S4, Mamba)
\end{itemize}

However, RNNs still useful for:
\begin{itemize}
    \item Online/streaming applications (Transformers need full context)
    \item Very long sequences where Transformers too expensive
    \item Small datasets (fewer parameters than Transformers)
    \item Scientific time series (physics-informed RNNs)
\end{itemize}

\clearpage
% =============================================
% SECTION 11: ATTENTION & TRANSFORMERS
% =============================================

\section{Attention \& Transformers}

\subsection{Introduction: Beyond RNNs}

RNNs revolutionized sequence modeling, but they have fundamental limitations:

\textbf{Problems with RNNs:}
\begin{enumerate}
    \item \textbf{Sequential processing:} Can't parallelize (slow!)
    \item \textbf{Long-range dependencies:} Even LSTM struggles beyond ~100 steps
    \item \textbf{Information bottleneck:} All info must pass through hidden state
    \item \textbf{Vanishing gradients:} Still an issue for very long sequences
\end{enumerate}

\textbf{The Attention mechanism} (2014) and \textbf{Transformers} (2017) solved these problems.

\textbf{Key innovation:} Instead of compressing everything into a fixed-size hidden state, let the model \textbf{attend to} any part of the input directly.

\textbf{Impact:}
\begin{itemize}
    \item GPT, BERT, ChatGPT: All based on Transformers
    \item Transformers now dominate NLP, vision (ViT), multimodal (CLIP), and more
    \item Enabled training on massive datasets (parallelization)
    \item State-of-the-art on virtually all sequence tasks
\end{itemize}

\subsection{Theory: The Attention Mechanism}

\subsubsection{Intuition: What is Attention?}

Imagine reading a document and answering a question. You don't re-read the entire document—you \textbf{focus} on relevant parts.

\textbf{Example:}
\begin{itemize}
    \item Document: "The cat sat on the mat. The dog played in the garden."
    \item Query: "Where is the cat?"
    \item You attend to: "The cat sat on the mat" (relevant)
    \item You ignore: "The dog played in the garden" (irrelevant)
\end{itemize}

\textbf{Attention does this automatically:}
\begin{itemize}
    \item Given a query, compute relevance scores for all inputs
    \item Higher scores = more attention
    \item Output is weighted sum of values, weighted by attention scores
\end{itemize}

\subsubsection{Mathematical Formulation}

\textbf{Attention has three components:}
\begin{itemize}
    \item \textbf{Query (Q):} What we're looking for
    \item \textbf{Keys (K):} What each position represents
    \item \textbf{Values (V):} The actual content at each position
\end{itemize}

\textbf{Step 1: Compute attention scores}
\[
\text{scores} = QK^T
\]

Each element $(i, j)$ is how relevant key $j$ is to query $i$.

\textbf{Step 2: Normalize with softmax}
\[
\text{attention\_weights} = \text{softmax}(\text{scores})
\]

Converts scores to probabilities (sum to 1).

\textbf{Step 3: Weighted sum of values}
\[
\text{output} = \text{attention\_weights} \cdot V
\]

\textbf{Complete formula (scaled dot-product attention):}
\[
\text{Attention}(Q, K, V) = \text{softmax}\left(\frac{QK^T}{\sqrt{d_k}}\right)V
\]

where $d_k$ is the dimension of keys (scaling prevents large dot products).

\begin{theorybox}[Why Scaling by $\sqrt{d_k}$?]
Without scaling, dot products can become very large when $d_k$ is large:
\begin{itemize}
    \item Large dot products → softmax saturates → tiny gradients
    \item Scaling by $\sqrt{d_k}$ keeps values in reasonable range
    \item Ensures stable gradients
\end{itemize}

Example: If $d_k = 512$, dot products could be $\sim 512$ without scaling. After scaling: $\sim \sqrt{512} \approx 22.6$.
\end{theorybox}

\clearpage
\subsubsection{Self-Attention vs Cross-Attention}

\textbf{Self-Attention:} Attention within same sequence

Query, Key, Value all come from the same source.

\textbf{Use case:} Understanding relationships between words in a sentence.

\textbf{Example:}
\begin{itemize}
    \item Sentence: "The animal didn't cross the street because it was too tired"
    \item "it" attends to "animal" (not "street")
\end{itemize}

\textbf{Cross-Attention:} Attention between two sequences

Query from one sequence, Keys and Values from another.

\textbf{Use case:} Machine translation, encoder-decoder models.

\textbf{Example:}
\begin{itemize}
    \item English (Keys/Values): "The cat is black"
    \item French decoder (Query): "Le chat est..."
    \item Decoder attends to relevant English words
\end{itemize}

\subsubsection{Multi-Head Attention}

Instead of one attention mechanism, use multiple in parallel:

\[
\text{MultiHead}(Q, K, V) = \text{Concat}(\text{head}_1, \ldots, \text{head}_h)W^O
\]

where each head is:
\[
\text{head}_i = \text{Attention}(QW^Q_i, KW^K_i, VW^V_i)
\]

\textbf{Why multiple heads?}
\begin{itemize}
    \item Different heads can focus on different aspects
    \item Head 1: Syntax (subject-verb agreement)
    \item Head 2: Semantics (meaning relationships)
    \item Head 3: Position (nearby words)
    \item Richer representations than single attention
\end{itemize}

\textbf{Typical values:}
\begin{itemize}
    \item $h = 8$ or $h = 16$ heads
    \item $d_{model} = 512$ (total dimension)
    \item $d_k = d_v = d_{model} / h = 64$ per head
\end{itemize}

\subsubsection{Positional Encoding}

\textbf{Problem:} Attention has no notion of order!

The sentence "cat chased dog" and "dog chased cat" look identical to attention (same words, same attention scores).

\textbf{Solution:} Add positional information to input embeddings.

\textbf{Sinusoidal positional encoding:}
\[
PE_{(pos, 2i)} = \sin\left(\frac{pos}{10000^{2i/d_{model}}}\right)
\]
\[
PE_{(pos, 2i+1)} = \cos\left(\frac{pos}{10000^{2i/d_{model}}}\right)
\]

where $pos$ is position, $i$ is dimension index.

\textbf{Properties:}
\begin{itemize}
    \item Different frequency for each dimension
    \item Can extrapolate to longer sequences than seen during training
    \item Model can learn to attend by relative position
\end{itemize}

\textbf{Alternative:} Learned positional embeddings (fixed maximum length).

\clearpage
\subsection{Theory: The Transformer Architecture}

\subsubsection{Complete Architecture}

Transformer consists of:
\begin{itemize}
    \item \textbf{Encoder:} Processes input sequence
    \item \textbf{Decoder:} Generates output sequence
\end{itemize}

\textbf{Encoder Block (repeated $N$ times):}
\begin{enumerate}
    \item Multi-head self-attention
    \item Add \& Layer Norm (residual connection)
    \item Feed-forward network (two linear layers with ReLU)
    \item Add \& Layer Norm (residual connection)
\end{enumerate}

\textbf{Decoder Block (repeated $N$ times):}
\begin{enumerate}
    \item Masked multi-head self-attention (can't see future)
    \item Add \& Layer Norm
    \item Multi-head cross-attention (attend to encoder output)
    \item Add \& Layer Norm
    \item Feed-forward network
    \item Add \& Layer Norm
\end{enumerate}

\textbf{Diagram:}
\begin{verbatim}
Input → Embedding → + Positional Encoding
                     ↓
                [Encoder Block] x N
                     ↓
                Encoder Output
                     ↓
Target → Embedding → + Positional Encoding
                     ↓
                [Decoder Block] x N
                     ↓
                Linear → Softmax → Output
\end{verbatim}

\subsubsection{Why Transformers Work So Well}

\textbf{1. Parallelization}

RNN: Must process step-by-step (slow)

Transformer: All positions processed in parallel (fast!)

Training on long sequences: 10-100× faster than RNNs.

\textbf{2. Long-range dependencies}

RNN: Information must flow through many steps (degrades)

Transformer: Direct connection between any two positions (one attention operation)

\textbf{3. Scalability}

Transformers scale beautifully with:
\begin{itemize}
    \item More data (billions of tokens)
    \item More parameters (billions to trillions)
    \item More compute (hundreds of GPUs)
\end{itemize}

\textbf{4. Transfer learning}

Pre-train once on massive data → Fine-tune for specific tasks.

Examples: BERT, GPT, T5.

\textbf{5. Inductive biases}

Minimal assumptions about data structure:
\begin{itemize}
    \item CNNs assume local structure (good for images)
    \item RNNs assume sequential processing (good for sequences)
    \item Transformers learn structure from data (flexible!)
\end{itemize}

\clearpage
\subsection{Implementation: Building Attention from Scratch}

\subsubsection{Scaled Dot-Product Attention}

\begin{lstlisting}
import torch
import torch.nn as nn
import torch.nn.functional as F
import math

def scaled_dot_product_attention(query, key, value, mask=None):
    """
    Scaled dot-product attention.
    
    Args:
        query: (batch, seq_len_q, d_k)
        key: (batch, seq_len_k, d_k)
        value: (batch, seq_len_k, d_v)
        mask: Optional mask (batch, seq_len_q, seq_len_k)
    
    Returns:
        output: (batch, seq_len_q, d_v)
        attention_weights: (batch, seq_len_q, seq_len_k)
    """
    d_k = query.size(-1)
    
    # Compute attention scores: Q @ K^T
    scores = torch.matmul(query, key.transpose(-2, -1))  # (batch, seq_len_q, seq_len_k)
    
    # Scale by sqrt(d_k)
    scores = scores / math.sqrt(d_k)
    
    # Apply mask if provided (set masked positions to -inf)
    if mask is not None:
        scores = scores.masked_fill(mask == 0, -1e9)
    
    # Softmax to get attention weights
    attention_weights = F.softmax(scores, dim=-1)  # (batch, seq_len_q, seq_len_k)
    
    # Weighted sum of values
    output = torch.matmul(attention_weights, value)  # (batch, seq_len_q, d_v)
    
    return output, attention_weights

# Test
batch_size = 2
seq_len = 10
d_k = 64

Q = torch.randn(batch_size, seq_len, d_k)
K = torch.randn(batch_size, seq_len, d_k)
V = torch.randn(batch_size, seq_len, d_k)

output, weights = scaled_dot_product_attention(Q, K, V)
print(f"Output shape: {output.shape}")  # (2, 10, 64)
print(f"Attention weights shape: {weights.shape}")  # (2, 10, 10)
print(f"Weights sum to 1: {weights.sum(dim=-1)}")  # All ones
\end{lstlisting}

\subsubsection{Multi-Head Attention}

\begin{lstlisting}
class MultiHeadAttention(nn.Module):
    """Multi-head attention mechanism."""
    
    def __init__(self, d_model, num_heads, dropout=0.1):
        """
        Args:
            d_model: Model dimension (e.g., 512)
            num_heads: Number of attention heads (e.g., 8)
            dropout: Dropout probability
        """
        super().__init__()
        
        assert d_model % num_heads == 0, "d_model must be divisible by num_heads"
        
        self.d_model = d_model
        self.num_heads = num_heads
        self.d_k = d_model // num_heads  # Dimension per head
        
        # Linear projections for Q, K, V
        self.W_q = nn.Linear(d_model, d_model)
        self.W_k = nn.Linear(d_model, d_model)
        self.W_v = nn.Linear(d_model, d_model)
        
        # Output projection
        self.W_o = nn.Linear(d_model, d_model)
        
        self.dropout = nn.Dropout(dropout)
    
    def split_heads(self, x):
        """
        Split last dimension into (num_heads, d_k).
        
        Input: (batch, seq_len, d_model)
        Output: (batch, num_heads, seq_len, d_k)
        """
        batch_size, seq_len, d_model = x.size()
        
        # Reshape to (batch, seq_len, num_heads, d_k)
        x = x.view(batch_size, seq_len, self.num_heads, self.d_k)
        
        # Transpose to (batch, num_heads, seq_len, d_k)
        return x.transpose(1, 2)
    
    def combine_heads(self, x):
        """
        Combine heads back.
        
        Input: (batch, num_heads, seq_len, d_k)
        Output: (batch, seq_len, d_model)
        """
        batch_size, num_heads, seq_len, d_k = x.size()
        
        # Transpose to (batch, seq_len, num_heads, d_k)
        x = x.transpose(1, 2)
        
        # Reshape to (batch, seq_len, d_model)
        return x.contiguous().view(batch_size, seq_len, self.d_model)
    
    def forward(self, query, key, value, mask=None):
        """
        Args:
            query: (batch, seq_len_q, d_model)
            key: (batch, seq_len_k, d_model)
            value: (batch, seq_len_k, d_model)
            mask: Optional (batch, seq_len_q, seq_len_k)
        """
        batch_size = query.size(0)
        
        # Linear projections
        Q = self.W_q(query)  # (batch, seq_len_q, d_model)
        K = self.W_k(key)    # (batch, seq_len_k, d_model)
        V = self.W_v(value)  # (batch, seq_len_k, d_model)
        
        # Split into multiple heads
        Q = self.split_heads(Q)  # (batch, num_heads, seq_len_q, d_k)
        K = self.split_heads(K)  # (batch, num_heads, seq_len_k, d_k)
        V = self.split_heads(V)  # (batch, num_heads, seq_len_k, d_k)
        
        # Adjust mask for multiple heads
        if mask is not None:
            # Add head dimension: (batch, 1, seq_len_q, seq_len_k)
            mask = mask.unsqueeze(1)
        
        # Scaled dot-product attention
        d_k = Q.size(-1)
        scores = torch.matmul(Q, K.transpose(-2, -1)) / math.sqrt(d_k)
        
        if mask is not None:
            scores = scores.masked_fill(mask == 0, -1e9)
        
        attention_weights = F.softmax(scores, dim=-1)
        attention_weights = self.dropout(attention_weights)
        
        # Apply attention to values
        attention_output = torch.matmul(attention_weights, V)
        # (batch, num_heads, seq_len_q, d_k)
        
        # Combine heads
        attention_output = self.combine_heads(attention_output)
        # (batch, seq_len_q, d_model)
        
        # Final linear projection
        output = self.W_o(attention_output)
        
        return output, attention_weights

# Test
mha = MultiHeadAttention(d_model=512, num_heads=8)
x = torch.randn(2, 10, 512)  # (batch, seq_len, d_model)
output, weights = mha(x, x, x)  # Self-attention

print(f"Output shape: {output.shape}")  # (2, 10, 512)
print(f"Attention weights shape: {weights.shape}")  # (2, 8, 10, 10)
\end{lstlisting}

\clearpage
\subsubsection{Positional Encoding}

\begin{lstlisting}
class PositionalEncoding(nn.Module):
    """Sinusoidal positional encoding."""
    
    def __init__(self, d_model, max_len=5000, dropout=0.1):
        super().__init__()
        self.dropout = nn.Dropout(dropout)
        
        # Create positional encoding matrix
        pe = torch.zeros(max_len, d_model)
        position = torch.arange(0, max_len, dtype=torch.float).unsqueeze(1)
        
        # Compute div_term for scaling
        div_term = torch.exp(
            torch.arange(0, d_model, 2).float() * 
            (-math.log(10000.0) / d_model)
        )
        
        # Apply sine to even indices
        pe[:, 0::2] = torch.sin(position * div_term)
        
        # Apply cosine to odd indices
        pe[:, 1::2] = torch.cos(position * div_term)
        
        # Add batch dimension: (1, max_len, d_model)
        pe = pe.unsqueeze(0)
        
        # Register as buffer (not a parameter, but part of state)
        self.register_buffer('pe', pe)
    
    def forward(self, x):
        """
        Args:
            x: (batch, seq_len, d_model)
        """
        # Add positional encoding
        x = x + self.pe[:, :x.size(1), :]
        return self.dropout(x)

# Test
pe = PositionalEncoding(d_model=512, max_len=100)
x = torch.randn(2, 50, 512)  # (batch, seq_len, d_model)
x_with_pos = pe(x)
print(x_with_pos.shape)  # (2, 50, 512)

# Visualize positional encoding
import matplotlib.pyplot as plt
plt.figure(figsize=(12, 4))
plt.imshow(pe.pe[0, :100, :].detach().numpy(), aspect='auto', cmap='viridis')
plt.xlabel('Dimension')
plt.ylabel('Position')
plt.colorbar()
plt.title('Positional Encoding')
plt.show()
\end{lstlisting}

\clearpage
\subsubsection{Feed-Forward Network}

\begin{lstlisting}
class PositionWiseFeedForward(nn.Module):
    """Position-wise feed-forward network."""
    
    def __init__(self, d_model, d_ff, dropout=0.1):
        """
        Args:
            d_model: Input/output dimension (e.g., 512)
            d_ff: Hidden dimension (typically 4 * d_model = 2048)
            dropout: Dropout probability
        """
        super().__init__()
        
        self.fc1 = nn.Linear(d_model, d_ff)
        self.fc2 = nn.Linear(d_ff, d_model)
        self.dropout = nn.Dropout(dropout)
    
    def forward(self, x):
        """
        Args:
            x: (batch, seq_len, d_model)
        """
        # First linear + ReLU
        x = F.relu(self.fc1(x))
        x = self.dropout(x)
        
        # Second linear
        x = self.fc2(x)
        
        return x

# Test
ffn = PositionWiseFeedForward(d_model=512, d_ff=2048)
x = torch.randn(2, 10, 512)
output = ffn(x)
print(output.shape)  # (2, 10, 512)
\end{lstlisting}

\subsubsection{Complete Transformer Encoder Block}

\begin{lstlisting}
class TransformerEncoderBlock(nn.Module):
    """Single Transformer encoder block."""
    
    def __init__(self, d_model, num_heads, d_ff, dropout=0.1):
        super().__init__()
        
        # Multi-head attention
        self.attention = MultiHeadAttention(d_model, num_heads, dropout)
        
        # Feed-forward network
        self.ffn = PositionWiseFeedForward(d_model, d_ff, dropout)
        
        # Layer normalization
        self.norm1 = nn.LayerNorm(d_model)
        self.norm2 = nn.LayerNorm(d_model)
        
        # Dropout
        self.dropout1 = nn.Dropout(dropout)
        self.dropout2 = nn.Dropout(dropout)
    
    def forward(self, x, mask=None):
        """
        Args:
            x: (batch, seq_len, d_model)
            mask: Optional attention mask
        """
        # Multi-head self-attention with residual and layer norm
        attn_output, _ = self.attention(x, x, x, mask)
        x = x + self.dropout1(attn_output)  # Residual connection
        x = self.norm1(x)  # Layer normalization
        
        # Feed-forward with residual and layer norm
        ffn_output = self.ffn(x)
        x = x + self.dropout2(ffn_output)  # Residual connection
        x = self.norm2(x)  # Layer normalization
        
        return x

# Test
encoder_block = TransformerEncoderBlock(d_model=512, num_heads=8, d_ff=2048)
x = torch.randn(2, 10, 512)
output = encoder_block(x)
print(output.shape)  # (2, 10, 512)
\end{lstlisting}

\begin{pytorchtip}[Layer Norm Placement: Pre-Norm vs Post-Norm]
\textbf{Original Transformer (post-norm):}
\begin{lstlisting}
x = norm(x + sublayer(x))
\end{lstlisting}

\textbf{Modern practice (pre-norm):}
\begin{lstlisting}
x = x + sublayer(norm(x))
\end{lstlisting}

Pre-norm is now preferred:
\begin{itemize}
    \item More stable training
    \item Can train deeper networks
    \item Better gradient flow
\end{itemize}

The code above uses post-norm (original paper), but many modern implementations use pre-norm.
\end{pytorchtip}

\clearpage
% =============================================
% SECTION 11: TRANSFORMERS - PART 2 (PyTorch Built-ins & Exercises)
% =============================================

\subsubsection{Using PyTorch's Built-in Attention}

\begin{lstlisting}
# PyTorch provides nn.MultiheadAttention
attention = nn.MultiheadAttention(
    embed_dim=512,
    num_heads=8,
    dropout=0.1,
    batch_first=True  # Important! Use batch_first=True
)

# Self-attention
x = torch.randn(2, 10, 512)  # (batch, seq_len, embed_dim)
attn_output, attn_weights = attention(x, x, x)

print(attn_output.shape)  # (2, 10, 512)
print(attn_weights.shape)  # (2, 10, 10) if need_weights=True

# Cross-attention (e.g., encoder-decoder)
encoder_output = torch.randn(2, 20, 512)
decoder_input = torch.randn(2, 10, 512)

attn_output, _ = attention(
    query=decoder_input,
    key=encoder_output,
    value=encoder_output
)
print(attn_output.shape)  # (2, 10, 512)
\end{lstlisting}

\begin{warningbox}[batch\_first in MultiheadAttention]
PyTorch's \texttt{nn.MultiheadAttention} defaults to \texttt{batch\_first=False}, expecting shape (seq\_len, batch, embed\_dim).

\textbf{Always use batch\_first=True} for consistency:
\begin{lstlisting}
# CONFUSING (default):
attention = nn.MultiheadAttention(embed_dim=512, num_heads=8)
x = torch.randn(10, 2, 512)  # (seq_len, batch, embed_dim)

# CLEAR (recommended):
attention = nn.MultiheadAttention(embed_dim=512, num_heads=8, 
                                 batch_first=True)
x = torch.randn(2, 10, 512)  # (batch, seq_len, embed_dim)
\end{lstlisting}
\end{warningbox}

\subsubsection{Masking for Causal Attention}

For autoregressive models (e.g., language generation), prevent attending to future positions:

\begin{lstlisting}
def create_causal_mask(seq_len):
    """
    Create causal mask to prevent attending to future positions.
    
    Returns:
        mask: (seq_len, seq_len) with 1s on and below diagonal
    """
    mask = torch.tril(torch.ones(seq_len, seq_len))
    return mask

# Example
mask = create_causal_mask(5)
print(mask)
"""
tensor([[1., 0., 0., 0., 0.],
        [1., 1., 0., 0., 0.],
        [1., 1., 1., 0., 0.],
        [1., 1., 1., 1., 0.],
        [1., 1., 1., 1., 1.]])
"""

# Use with attention
seq_len = 10
causal_mask = create_causal_mask(seq_len)

# PyTorch MultiheadAttention expects mask where True = ignore
# So we invert: 0 = attend, 1 = ignore
attn_mask = (1 - causal_mask).bool()

x = torch.randn(2, seq_len, 512)
attention = nn.MultiheadAttention(embed_dim=512, num_heads=8, 
                                 batch_first=True)
output, _ = attention(x, x, x, attn_mask=attn_mask)
\end{lstlisting}

\subsubsection{Complete Transformer for Classification}

\begin{lstlisting}
class TransformerClassifier(nn.Module):
    """Transformer for sequence classification."""
    
    def __init__(self, vocab_size, d_model, num_heads, num_layers, 
                 num_classes, d_ff, max_len=512, dropout=0.1):
        super().__init__()
        
        # Token embedding
        self.embedding = nn.Embedding(vocab_size, d_model)
        
        # Positional encoding
        self.pos_encoding = PositionalEncoding(d_model, max_len, dropout)
        
        # Transformer encoder layers
        encoder_layer = nn.TransformerEncoderLayer(
            d_model=d_model,
            nhead=num_heads,
            dim_feedforward=d_ff,
            dropout=dropout,
            batch_first=True
        )
        
        self.transformer_encoder = nn.TransformerEncoder(
            encoder_layer,
            num_layers=num_layers
        )
        
        # Classification head
        self.fc = nn.Linear(d_model, num_classes)
        
        self.d_model = d_model
    
    def forward(self, x, mask=None):
        """
        Args:
            x: (batch, seq_len) - token indices
            mask: Optional padding mask (batch, seq_len)
        """
        # Embedding and scaling
        x = self.embedding(x) * math.sqrt(self.d_model)
        
        # Add positional encoding
        x = self.pos_encoding(x)
        
        # Transformer encoding
        x = self.transformer_encoder(x, src_key_padding_mask=mask)
        
        # Global average pooling
        x = x.mean(dim=1)  # (batch, d_model)
        
        # Classification
        logits = self.fc(x)
        
        return logits

# Test
model = TransformerClassifier(
    vocab_size=10000,
    d_model=512,
    num_heads=8,
    num_layers=6,
    num_classes=3,
    d_ff=2048,
    dropout=0.1
)

# Random token sequences
x = torch.randint(0, 10000, (32, 50))  # (batch, seq_len)
output = model(x)
print(output.shape)  # (32, 3)

# Count parameters
total_params = sum(p.numel() for p in model.parameters())
print(f"Total parameters: {total_params:,}")
\end{lstlisting}

\clearpage
\subsubsection{Transformer for Sequence-to-Sequence}

\begin{lstlisting}
class TransformerSeq2Seq(nn.Module):
    """Complete Transformer with encoder and decoder."""
    
    def __init__(self, src_vocab_size, tgt_vocab_size, d_model=512, 
                 num_heads=8, num_layers=6, d_ff=2048, dropout=0.1):
        super().__init__()
        
        # Source embeddings
        self.src_embedding = nn.Embedding(src_vocab_size, d_model)
        self.src_pos_encoding = PositionalEncoding(d_model, dropout=dropout)
        
        # Target embeddings
        self.tgt_embedding = nn.Embedding(tgt_vocab_size, d_model)
        self.tgt_pos_encoding = PositionalEncoding(d_model, dropout=dropout)
        
        # Transformer
        self.transformer = nn.Transformer(
            d_model=d_model,
            nhead=num_heads,
            num_encoder_layers=num_layers,
            num_decoder_layers=num_layers,
            dim_feedforward=d_ff,
            dropout=dropout,
            batch_first=True
        )
        
        # Output projection
        self.fc_out = nn.Linear(d_model, tgt_vocab_size)
        
        self.d_model = d_model
    
    def forward(self, src, tgt, src_mask=None, tgt_mask=None, 
                src_padding_mask=None, tgt_padding_mask=None):
        """
        Args:
            src: (batch, src_seq_len) - source token indices
            tgt: (batch, tgt_seq_len) - target token indices
            src_mask: Optional source attention mask
            tgt_mask: Causal mask for target (prevent seeing future)
            src_padding_mask: Padding mask for source
            tgt_padding_mask: Padding mask for target
        """
        # Embed and add positional encoding
        src = self.src_embedding(src) * math.sqrt(self.d_model)
        src = self.src_pos_encoding(src)
        
        tgt = self.tgt_embedding(tgt) * math.sqrt(self.d_model)
        tgt = self.tgt_pos_encoding(tgt)
        
        # Transformer
        output = self.transformer(
            src, tgt,
            src_mask=src_mask,
            tgt_mask=tgt_mask,
            src_key_padding_mask=src_padding_mask,
            tgt_key_padding_mask=tgt_padding_mask
        )
        
        # Project to vocabulary
        logits = self.fc_out(output)
        
        return logits
    
    def generate(self, src, max_len, start_token, end_token):
        """
        Generate sequence autoregressively.
        
        Args:
            src: (batch, src_seq_len)
            max_len: Maximum generation length
            start_token: Token to start generation
            end_token: Token to stop generation
        """
        self.eval()
        batch_size = src.size(0)
        device = src.device
        
        # Encode source
        src = self.src_embedding(src) * math.sqrt(self.d_model)
        src = self.src_pos_encoding(src)
        memory = self.transformer.encoder(src)
        
        # Start with start_token
        tgt = torch.full((batch_size, 1), start_token, 
                        dtype=torch.long, device=device)
        
        for _ in range(max_len):
            # Create causal mask
            tgt_mask = nn.Transformer.generate_square_subsequent_mask(
                tgt.size(1)
            ).to(device)
            
            # Decode
            tgt_embedded = self.tgt_embedding(tgt) * math.sqrt(self.d_model)
            tgt_embedded = self.tgt_pos_encoding(tgt_embedded)
            
            output = self.transformer.decoder(tgt_embedded, memory, 
                                             tgt_mask=tgt_mask)
            
            # Project to vocabulary
            logits = self.fc_out(output[:, -1, :])  # Last position
            
            # Greedy decoding (take argmax)
            next_token = logits.argmax(dim=-1, keepdim=True)
            
            # Append to sequence
            tgt = torch.cat([tgt, next_token], dim=1)
            
            # Stop if all sequences generated end_token
            if (next_token == end_token).all():
                break
        
        return tgt

# Test
model = TransformerSeq2Seq(
    src_vocab_size=10000,
    tgt_vocab_size=10000,
    d_model=512,
    num_heads=8,
    num_layers=6
)

src = torch.randint(0, 10000, (2, 20))  # Source sequences
tgt = torch.randint(0, 10000, (2, 15))  # Target sequences

# Training forward pass
output = model(src, tgt[:, :-1])  # Shift target by 1
print(output.shape)  # (2, 14, 10000)

# Generation
generated = model.generate(src, max_len=20, start_token=1, end_token=2)
print(generated.shape)  # (2, <=20)
\end{lstlisting}

\clearpage
\subsection{Exercises}

\begin{exercise}[11.1: Attention from Scratch - $\bigstar\bigstar$]
\textbf{Goal:} Implement and understand attention mechanism.

\begin{enumerate}
    \item Implement scaled dot-product attention from scratch (without using the provided code)
    \item Test on random Q, K, V matrices
    \item Verify:
    \begin{itemize}
        \item Attention weights sum to 1
        \item Output shape is correct
        \item Scaling by $\sqrt{d_k}$ stabilizes scores
    \end{itemize}
    \item Visualize attention weights as a heatmap
\end{enumerate}

\textbf{Visualization:}
\begin{lstlisting}
import matplotlib.pyplot as plt

# Compute attention
output, weights = scaled_dot_product_attention(Q, K, V)

# Visualize attention weights (first sample in batch)
plt.figure(figsize=(8, 6))
plt.imshow(weights[0].detach().numpy(), cmap='viridis')
plt.colorbar()
plt.xlabel('Key Position')
plt.ylabel('Query Position')
plt.title('Attention Weights')
plt.show()
\end{lstlisting}
\end{exercise}

\begin{exercise}[11.2: Multi-Head Attention - $\bigstar\bigstar\bigstar$]
\textbf{Goal:} Implement multi-head attention.

\begin{enumerate}
    \item Complete the \texttt{MultiHeadAttention} class
    \item Test with different numbers of heads (1, 2, 4, 8)
    \item Verify outputs are identical when using 1 head vs your scaled dot-product attention
    \item Visualize attention patterns from different heads
    \item Compare parameter counts: single-head vs multi-head
\end{enumerate}

\textbf{Questions:}
\begin{itemize}
    \item Do different heads learn different attention patterns?
    \item How does performance change with number of heads?
\end{itemize}
\end{exercise}

\begin{exercise}[11.3: Positional Encoding Analysis - $\bigstar\bigstar\bigstar$]
\textbf{Goal:} Understand positional encoding.

\begin{enumerate}
    \item Implement sinusoidal positional encoding
    \item Visualize the encoding for positions 0-100
    \item Experiment:
    \begin{itemize}
        \item Train a Transformer with positional encoding
        \item Train without positional encoding
        \item Use learned positional embeddings instead
    \end{itemize}
    \item Compare:
    \begin{itemize}
        \item Final accuracy
        \item Does model learn order information without PE?
        \item Which PE works best?
    \end{itemize}
\end{enumerate}

\textbf{Task:} Sequence classification (e.g., sentiment analysis)
\end{exercise}

\begin{exercise}[11.4: Complete Transformer Encoder - $\bigstar\bigstar\bigstar\bigstar$]
\textbf{Goal:} Build a full Transformer encoder for classification.

\begin{enumerate}
    \item Implement complete Transformer encoder:
    \begin{itemize}
        \item Multi-head attention
        \item Position-wise FFN
        \item Layer normalization
        \item Residual connections
        \item Positional encoding
    \end{itemize}
    \item Train on text classification task (e.g., IMDB sentiment)
    \item Use vocabulary size = 10,000 (most common words)
    \item Hyperparameters:
    \begin{itemize}
        \item $d_{model} = 256$
        \item num\_heads = 8
        \item num\_layers = 4
        \item $d_{ff} = 1024$
    \end{itemize}
    \item Achieve >85\% accuracy
    \item Visualize attention weights on example sentences
\end{enumerate}

\textbf{Dataset preparation:}
\begin{lstlisting}
from torchtext.datasets import IMDB
from torchtext.data.utils import get_tokenizer
from collections import Counter

tokenizer = get_tokenizer('basic_english')

# Build vocabulary from training data
def build_vocab(data_iter, tokenizer, max_size=10000):
    counter = Counter()
    for label, text in data_iter:
        counter.update(tokenizer(text))
    
    vocab = {word: i+2 for i, (word, _) in 
             enumerate(counter.most_common(max_size))}
    vocab['<pad>'] = 0
    vocab['<unk>'] = 1
    
    return vocab

# Encode text
def encode_text(text, vocab, tokenizer, max_len=256):
    tokens = tokenizer(text)[:max_len]
    indices = [vocab.get(token, vocab['<unk>']) for token in tokens]
    
    # Pad to max_len
    indices += [vocab['<pad>']] * (max_len - len(indices))
    
    return torch.LongTensor(indices)
\end{lstlisting}
\end{exercise}

\begin{exercise}[11.5: Causal Attention for Language Modeling - $\bigstar\bigstar\bigstar\bigstar$]
\textbf{Goal:} Implement autoregressive generation with causal masking.

\begin{enumerate}
    \item Build a decoder-only Transformer (like GPT)
    \item Use causal masking (can't attend to future tokens)
    \item Train on character-level language modeling:
    \begin{itemize}
        \item Dataset: Shakespeare text or similar
        \item Task: Predict next character
    \end{itemize}
    \item Implement generation:
    \begin{itemize}
        \item Greedy decoding (argmax)
        \item Top-k sampling
        \item Temperature sampling
    \end{itemize}
    \item Generate text samples and evaluate quality
\end{enumerate}

\textbf{Generation strategies:}
\begin{lstlisting}
def generate_text(model, start_text, max_len=100, temperature=1.0, 
                 top_k=None):
    """Generate text autoregressively."""
    model.eval()
    
    # Encode start text
    tokens = encode(start_text)
    
    with torch.no_grad():
        for _ in range(max_len):
            # Forward pass
            logits = model(tokens)
            
            # Get logits for last position
            logits = logits[-1] / temperature
            
            # Top-k filtering
            if top_k is not None:
                values, _ = torch.topk(logits, top_k)
                logits[logits < values[-1]] = -float('inf')
            
            # Sample from distribution
            probs = F.softmax(logits, dim=-1)
            next_token = torch.multinomial(probs, 1)
            
            # Append to sequence
            tokens = torch.cat([tokens, next_token])
    
    return decode(tokens)
\end{lstlisting}
\end{exercise}

\begin{exercise}[11.6: Attention Visualization - $\bigstar\bigstar\bigstar\bigstar$]
\textbf{Goal:} Understand what Transformers learn.

\begin{enumerate}
    \item Train a Transformer on sequence task
    \item Extract attention weights from different layers and heads
    \item Visualize:
    \begin{itemize}
        \item Which heads focus on local context?
        \item Which heads capture long-range dependencies?
        \item How do patterns differ across layers?
    \end{itemize}
    \item Create attention heatmaps for example sentences
    \item Analyze:
    \begin{itemize}
        \item Do lower layers learn syntax?
        \item Do higher layers learn semantics?
    \end{itemize}
\end{enumerate}

\textbf{Visualization code:}
\begin{lstlisting}
def visualize_attention_heads(model, sentence, tokenizer):
    """Visualize attention patterns from all heads."""
    
    # Encode sentence
    tokens = encode(sentence, tokenizer)
    
    # Forward pass with hooks to capture attention
    attentions = []
    
    def hook_fn(module, input, output):
        # output[1] is attention weights
        attentions.append(output[1].detach())
    
    hooks = []
    for layer in model.transformer_encoder.layers:
        hook = layer.self_attn.register_forward_hook(hook_fn)
        hooks.append(hook)
    
    # Forward
    _ = model(tokens.unsqueeze(0))
    
    # Remove hooks
    for hook in hooks:
        hook.remove()
    
    # Plot attention from each layer and head
    num_layers = len(attentions)
    num_heads = attentions[0].size(1)
    
    fig, axes = plt.subplots(num_layers, num_heads, 
                            figsize=(num_heads*3, num_layers*3))
    
    for layer in range(num_layers):
        for head in range(num_heads):
            ax = axes[layer, head]
            attn = attentions[layer][0, head].cpu().numpy()
            
            im = ax.imshow(attn, cmap='viridis')
            ax.set_title(f'L{layer+1} H{head+1}')
            ax.set_xlabel('Key')
            ax.set_ylabel('Query')
    
    plt.tight_layout()
    plt.show()

# Example
visualize_attention_heads(model, "The cat sat on the mat", tokenizer)
\end{lstlisting}
\end{exercise}

\clearpage
\subsection{Key Takeaways}

\textbf{Why Transformers dominate:}
\begin{itemize}
    \item \textbf{Parallelization:} All positions processed simultaneously (10-100× faster than RNNs)
    \item \textbf{Long-range dependencies:} Direct connections between any positions
    \item \textbf{Scalability:} Works with billions of parameters and tokens
    \item \textbf{Flexibility:} Minimal inductive biases, learns from data
\end{itemize}

\textbf{Core components:}
\begin{itemize}
    \item \textbf{Scaled dot-product attention:} $\text{Attention}(Q,K,V) = \text{softmax}(QK^T/\sqrt{d_k})V$
    \item \textbf{Multi-head attention:} Multiple attention mechanisms in parallel
    \item \textbf{Positional encoding:} Inject position information (no inherent order)
    \item \textbf{Feed-forward networks:} Position-wise transformation
    \item \textbf{Residual connections + Layer norm:} Stable training
\end{itemize}

\textbf{Architecture patterns:}
\begin{itemize}
    \item \textbf{Encoder-only:} BERT (classification, understanding)
    \item \textbf{Decoder-only:} GPT (generation, autoregressive)
    \item \textbf{Encoder-decoder:} T5, BART (translation, summarization)
\end{itemize}

\textbf{Implementation tips:}
\begin{itemize}
    \item Always use \texttt{batch\_first=True} in PyTorch
    \item Scale attention scores by $\sqrt{d_k}$ (stability)
    \item Use causal masking for autoregressive tasks
    \item Pre-norm is more stable than post-norm for deep networks
    \item Typical hyperparameters: $d_{model}=512$, $h=8$, $d_{ff}=2048$
\end{itemize}

\textbf{Training considerations:}
\begin{itemize}
    \item Transformers need lots of data (millions of examples)
    \item Use warmup learning rate schedule
    \item Gradient clipping less critical than RNNs (but still useful)
    \item Dropout on attention weights (0.1) and residual connections
    \item Label smoothing helps (0.1)
\end{itemize}

\textbf{When to use Transformers:}
\begin{itemize}
    \item NLP tasks (now standard)
    \item Long sequences (>100 tokens)
    \item When you have lots of data
    \item When parallelization matters
    \item Vision tasks (Vision Transformer - ViT)
\end{itemize}

\textbf{Limitations:}
\begin{itemize}
    \item \textbf{Quadratic complexity:} $O(n^2)$ in sequence length (memory and compute)
    \item \textbf{Data hungry:} Requires large datasets for good performance
    \item \textbf{Memory intensive:} Attention matrices can be huge
    \item \textbf{Position limit:} Most models have fixed maximum sequence length
\end{itemize}

\textbf{Solutions to limitations:}
\begin{itemize}
    \item Sparse attention (only attend to subset of positions)
    \item Linear attention approximations (e.g., Linformer)
    \item Local attention windows (e.g., Longformer)
    \item Efficient Transformers (e.g., Reformer, Performer)
\end{itemize}

\textbf{Common mistakes:}
\begin{itemize}
    \item Forgetting positional encoding (model can't distinguish positions)
    \item Wrong mask shape or type
    \item Not using causal mask for generation
    \item Forgetting to scale attention scores
    \item Using post-norm for very deep networks (use pre-norm)
\end{itemize}

\textbf{Modern developments:}
\begin{itemize}
    \item \textbf{Vision Transformers (ViT):} Transformers for images
    \item \textbf{BERT:} Bidirectional encoder (masked language modeling)
    \item \textbf{GPT:} Decoder-only (autoregressive generation)
    \item \textbf{T5:} Unified text-to-text framework
    \item \textbf{LLaMA, ChatGPT:} Large language models at scale
\end{itemize}

\clearpage
% =============================================
% PART III: PRACTICAL SKILLS
% =============================================

\part{Practical Skills}

% =============================================
% SECTION 13: DEBUGGING & BEST PRACTICES
% =============================================

\section{Debugging \& Best Practices}

\subsection{Introduction: Why Debugging Matters}

Deep learning is full of silent failures:
\begin{itemize}
    \item Model trains but doesn't learn (loss stuck at baseline)
    \item Loss is NaN after a few iterations
    \item Training loss decreases but validation doesn't
    \item Model works on toy data but fails on real data
    \item Gradients vanish or explode
\end{itemize}

Unlike traditional programming where bugs crash immediately, deep learning bugs often manifest as \textbf{poor performance}. You need systematic debugging strategies.

\subsection{Common Bugs and How to Fix Them}

\subsubsection{Bug 1: Loss is NaN}

\textbf{Symptoms:}
\begin{lstlisting}
Epoch 1: loss = 2.345
Epoch 2: loss = 1.876
Epoch 3: loss = nan
\end{lstlisting}

\textbf{Causes and fixes:}

\textbf{1. Learning rate too high}
\begin{lstlisting}
# Check: Does loss explode before NaN?
# Fix: Reduce learning rate
optimizer = torch.optim.Adam(model.parameters(), lr=1e-5)  # Try lower
\end{lstlisting}

\textbf{2. Numerical instability in loss}
\begin{lstlisting}
# BAD: Manual log can cause NaN
loss = -torch.log(predictions)  # NaN if predictions = 0

# GOOD: Use stable implementations
loss = F.cross_entropy(logits, targets)  # Numerically stable
# or
loss = F.binary_cross_entropy_with_logits(logits, targets)
\end{lstlisting}

\textbf{3. Exploding gradients (RNNs)}
\begin{lstlisting}
# Fix: Add gradient clipping
torch.nn.utils.clip_grad_norm_(model.parameters(), max_norm=1.0)
\end{lstlisting}

\textbf{4. Data contains NaN or Inf}
\begin{lstlisting}
# Check data
print(torch.isnan(data).any())
print(torch.isinf(data).any())

# Fix: Clean data or add checks
data = torch.nan_to_num(data, nan=0.0, posinf=1e6, neginf=-1e6)
\end{lstlisting}

\textbf{Detection code:}
\begin{lstlisting}
def check_for_nan(model, data, target):
    """Debug NaN issues."""
    # Check inputs
    print(f"Data has NaN: {torch.isnan(data).any()}")
    print(f"Data has Inf: {torch.isinf(data).any()}")
    
    # Forward pass
    output = model(data)
    print(f"Output has NaN: {torch.isnan(output).any()}")
    
    # Backward pass
    loss = criterion(output, target)
    print(f"Loss has NaN: {torch.isnan(loss).any()}")
    loss.backward()
    
    # Check gradients
    for name, param in model.named_parameters():
        if param.grad is not None:
            if torch.isnan(param.grad).any():
                print(f"NaN gradient in {name}")
            if torch.isinf(param.grad).any():
                print(f"Inf gradient in {name}")
\end{lstlisting}

\clearpage
\subsubsection{Bug 2: Loss Not Decreasing}

\textbf{Symptoms:}
\begin{lstlisting}
Epoch 1: loss = 2.345
Epoch 2: loss = 2.342
Epoch 3: loss = 2.340
...
Epoch 100: loss = 2.300  # Barely moving
\end{lstlisting}

\textbf{Causes and fixes:}

\textbf{1. Learning rate too low}
\begin{lstlisting}
# Fix: Increase learning rate
optimizer = torch.optim.Adam(model.parameters(), lr=1e-2)  # Try higher

# Or use learning rate finder
def find_lr(model, train_loader, init_lr=1e-8, final_lr=10):
    """Find optimal learning rate."""
    lrs = []
    losses = []
    
    optimizer = torch.optim.SGD(model.parameters(), lr=init_lr)
    lr_scheduler = torch.optim.lr_scheduler.ExponentialLR(
        optimizer, gamma=(final_lr/init_lr)**(1/100)
    )
    
    for batch in train_loader:
        optimizer.zero_grad()
        loss = compute_loss(model, batch)
        loss.backward()
        optimizer.step()
        
        lrs.append(optimizer.param_groups[0]['lr'])
        losses.append(loss.item())
        
        lr_scheduler.step()
        
        if len(lrs) >= 100:
            break
    
    # Plot
    import matplotlib.pyplot as plt
    plt.plot(lrs, losses)
    plt.xscale('log')
    plt.xlabel('Learning Rate')
    plt.ylabel('Loss')
    plt.title('Learning Rate Finder')
    plt.show()
    
    # Optimal LR is usually where loss decreases fastest
    # (steepest negative gradient)
\end{lstlisting}

\textbf{2. Wrong loss function}
\begin{lstlisting}
# BAD: Using MSE for classification
criterion = nn.MSELoss()  # Wrong for discrete labels!

# GOOD: Use appropriate loss
criterion = nn.CrossEntropyLoss()  # For classification
\end{lstlisting}

\textbf{3. Forgot to zero gradients}
\begin{lstlisting}
# BAD: Gradients accumulate
for batch in dataloader:
    loss = criterion(model(x), y)
    loss.backward()
    optimizer.step()  # FORGOT optimizer.zero_grad()!

# GOOD:
for batch in dataloader:
    optimizer.zero_grad()  # Always zero first!
    loss = criterion(model(x), y)
    loss.backward()
    optimizer.step()
\end{lstlisting}

\textbf{4. Data not normalized}
\begin{lstlisting}
# Check data range
print(f"Data min: {data.min()}, max: {data.max()}")

# Fix: Normalize
mean = data.mean()
std = data.std()
data = (data - mean) / (std + 1e-8)
\end{lstlisting}

\textbf{5. Dead ReLUs or vanishing gradients}
\begin{lstlisting}
# Check activation distributions
def check_activations(model, data):
    activations = {}
    
    def hook(name):
        def hook_fn(module, input, output):
            activations[name] = output.detach()
        return hook_fn
    
    # Register hooks
    for name, module in model.named_modules():
        if isinstance(module, nn.ReLU):
            module.register_forward_hook(hook(name))
    
    model(data)
    
    # Check what percentage is zero
    for name, act in activations.items():
        zero_pct = (act == 0).float().mean() * 100
        print(f"{name}: {zero_pct:.1f}% zeros")
        
        if zero_pct > 50:
            print(f"  WARNING: More than 50% dead neurons!")

# Fix: Use Leaky ReLU or adjust initialization
\end{lstlisting}

\clearpage
\subsubsection{Bug 3: Training Loss Decreases, Validation Doesn't}

\textbf{Symptoms:} Overfitting

\textbf{Causes and fixes:}

\textbf{1. Model too complex for data}
\begin{lstlisting}
# Fix: Reduce model size
model = SmallNet()  # Fewer layers, fewer parameters

# Or increase data
# Or add regularization (see below)
\end{lstlisting}

\textbf{2. No regularization}
\begin{lstlisting}
# Add dropout
model = nn.Sequential(
    nn.Linear(100, 128),
    nn.ReLU(),
    nn.Dropout(0.5),  # Add dropout
    nn.Linear(128, 10)
)

# Add weight decay
optimizer = torch.optim.Adam(model.parameters(), lr=1e-3, 
                            weight_decay=1e-4)  # L2 regularization

# Add data augmentation
transform = transforms.Compose([
    transforms.RandomHorizontalFlip(),
    transforms.RandomCrop(32, padding=4),
    transforms.ToTensor()
])
\end{lstlisting}

\textbf{3. Training too long}
\begin{lstlisting}
# Implement early stopping
class EarlyStopping:
    def __init__(self, patience=10, min_delta=0):
        self.patience = patience
        self.min_delta = min_delta
        self.counter = 0
        self.best_loss = None
        self.should_stop = False
    
    def __call__(self, val_loss):
        if self.best_loss is None:
            self.best_loss = val_loss
        elif val_loss > self.best_loss - self.min_delta:
            self.counter += 1
            if self.counter >= self.patience:
                self.should_stop = True
        else:
            self.best_loss = val_loss
            self.counter = 0

# Usage
early_stopping = EarlyStopping(patience=10)

for epoch in range(max_epochs):
    train_loss = train_epoch()
    val_loss = validate()
    
    early_stopping(val_loss)
    if early_stopping.should_stop:
        print(f"Early stopping at epoch {epoch}")
        break
\end{lstlisting}

\subsubsection{Bug 4: Model Works on Small Data, Fails on Full Dataset}

\textbf{Causes and fixes:}

\textbf{1. Batch size too small}
\begin{lstlisting}
# Batch norm unstable with small batches
# Fix: Increase batch size or use Layer Norm

# Or use Gradient Accumulation
accumulation_steps = 4

for i, batch in enumerate(dataloader):
    loss = criterion(model(x), y)
    loss = loss / accumulation_steps  # Scale loss
    loss.backward()
    
    if (i + 1) % accumulation_steps == 0:
        optimizer.step()
        optimizer.zero_grad()
\end{lstlisting}

\textbf{2. Data distribution shift}
\begin{lstlisting}
# Check: Are train and val distributions similar?
def check_distribution(train_data, val_data):
    print(f"Train mean: {train_data.mean()}, std: {train_data.std()}")
    print(f"Val mean: {val_data.mean()}, std: {val_data.std()}")
    
    # Visualize
    import matplotlib.pyplot as plt
    plt.hist(train_data.flatten(), bins=50, alpha=0.5, label='Train')
    plt.hist(val_data.flatten(), bins=50, alpha=0.5, label='Val')
    plt.legend()
    plt.show()

# Fix: Normalize using training statistics
mean = train_data.mean()
std = train_data.std()

train_data = (train_data - mean) / std
val_data = (val_data - mean) / std  # Use training stats!
\end{lstlisting}

\clearpage
\subsection{Debugging Strategies}

\subsubsection{Start Simple, Add Complexity}

\textbf{Step 1: Overfit on one batch}

If you can't overfit one batch, something is fundamentally wrong.

\begin{lstlisting}
# Get one batch
x, y = next(iter(train_loader))

# Try to overfit it
for epoch in range(1000):
    optimizer.zero_grad()
    output = model(x)
    loss = criterion(output, y)
    loss.backward()
    optimizer.step()
    
    if epoch % 100 == 0:
        print(f"Epoch {epoch}: Loss = {loss.item():.4f}")

# Should reach near-zero loss
# If not, check:
# - Model has enough capacity
# - Loss function is correct
# - Learning rate is reasonable
\end{lstlisting}

\textbf{Step 2: Overfit on small dataset}

Try 100 samples. Should still overfit easily.

\textbf{Step 3: Add validation}

Once overfitting works, add validation and regularization.

\textbf{Step 4: Scale to full dataset}

Now train on full data with proper regularization.

\subsubsection{Check Shapes at Every Step}

\begin{lstlisting}
def debug_shapes(model, x):
    """Print shapes through the network."""
    print(f"Input: {x.shape}")
    
    for i, (name, module) in enumerate(model.named_children()):
        x = module(x)
        print(f"{i}. {name}: {x.shape}")
    
    return x

# Usage
x = torch.randn(4, 3, 32, 32)
output = debug_shapes(model, x)
"""
Input: torch.Size([4, 3, 32, 32])
0. conv1: torch.Size([4, 64, 32, 32])
1. relu: torch.Size([4, 64, 32, 32])
2. pool: torch.Size([4, 64, 16, 16])
...
"""
\end{lstlisting}

\subsubsection{Visualize Gradients}

\begin{lstlisting}
def plot_grad_flow(named_parameters):
    """
    Plots gradient flow through the network.
    Helps identify vanishing/exploding gradients.
    """
    ave_grads = []
    max_grads = []
    layers = []
    
    for n, p in named_parameters:
        if p.requires_grad and p.grad is not None:
            layers.append(n)
            ave_grads.append(p.grad.abs().mean().item())
            max_grads.append(p.grad.abs().max().item())
    
    import matplotlib.pyplot as plt
    plt.figure(figsize=(12, 6))
    plt.bar(range(len(ave_grads)), ave_grads, alpha=0.5, label='mean')
    plt.bar(range(len(max_grads)), max_grads, alpha=0.5, label='max')
    plt.hlines(0, 0, len(ave_grads), linewidth=2, color='k')
    plt.xticks(range(len(layers)), layers, rotation='vertical')
    plt.xlabel('Layers')
    plt.ylabel('Gradient magnitude')
    plt.legend()
    plt.title('Gradient Flow')
    plt.grid(True)
    plt.tight_layout()
    plt.show()

# Usage after loss.backward()
plot_grad_flow(model.named_parameters())
\end{lstlisting}

\subsubsection{Use Hooks for Debugging}

\begin{lstlisting}
# Forward hook: inspect activations
def forward_hook(module, input, output):
    print(f"Forward: {output.shape}, mean={output.mean():.4f}")

# Backward hook: inspect gradients
def backward_hook(module, grad_input, grad_output):
    print(f"Backward: grad mean={grad_output[0].mean():.4f}")

# Register hooks
for name, module in model.named_modules():
    if isinstance(module, nn.Linear):
        module.register_forward_hook(forward_hook)
        module.register_backward_hook(backward_hook)

# Now run forward and backward
output = model(x)
loss = criterion(output, y)
loss.backward()

# Hooks will print debug info
\end{lstlisting}

\clearpage
\subsection{Best Practices for Development}

\subsubsection{Code Organization}

\textbf{Project structure:}
\begin{verbatim}
project/
├── data/
│   ├── __init__.py
│   ├── dataset.py        # Dataset classes
│   └── transforms.py     # Data augmentation
├── models/
│   ├── __init__.py
│   ├── resnet.py         # Model architectures
│   └── utils.py          # Model utilities
├── utils/
│   ├── __init__.py
│   ├── train.py          # Training utilities
│   ├── eval.py           # Evaluation utilities
│   └── metrics.py        # Custom metrics
├── configs/
│   └── config.yaml       # Hyperparameters
├── train.py              # Main training script
├── evaluate.py           # Evaluation script
└── requirements.txt      # Dependencies
\end{verbatim}

\textbf{Separate concerns:}
\begin{lstlisting}
# models/resnet.py
class ResNet(nn.Module):
    """Model definition only."""
    def __init__(self, ...):
        # Architecture
    
    def forward(self, x):
        # Forward pass
        return x

# utils/train.py
def train_epoch(model, loader, optimizer, criterion, device):
    """Training logic."""
    model.train()
    total_loss = 0
    
    for batch in loader:
        # Training step
        ...
    
    return total_loss / len(loader)

# train.py
def main():
    # Configuration
    # Data loading
    # Model creation
    # Training loop
    pass

if __name__ == "__main__":
    main()
\end{lstlisting}

\subsubsection{Configuration Management}

\begin{lstlisting}
# config.yaml
model:
  type: resnet18
  num_classes: 10

training:
  batch_size: 128
  epochs: 100
  learning_rate: 0.001
  weight_decay: 0.0001

data:
  root: ./data
  num_workers: 4
  augmentation: true

# Load config
import yaml

with open('config.yaml', 'r') as f:
    config = yaml.safe_load(f)

# Use config
batch_size = config['training']['batch_size']
lr = config['training']['learning_rate']
\end{lstlisting}

\subsubsection{Experiment Tracking}

\begin{lstlisting}
# Simple logging
import logging

logging.basicConfig(
    level=logging.INFO,
    format='%(asctime)s - %(levelname)s - %(message)s',
    handlers=[
        logging.FileHandler('training.log'),
        logging.StreamHandler()
    ]
)

logger = logging.getLogger(__name__)

# Use in training
logger.info(f"Epoch {epoch}: Train Loss = {train_loss:.4f}")

# TensorBoard (better)
from torch.utils.tensorboard import SummaryWriter

writer = SummaryWriter('runs/experiment_1')

for epoch in range(num_epochs):
    train_loss = train_epoch()
    val_loss = validate()
    
    writer.add_scalar('Loss/train', train_loss, epoch)
    writer.add_scalar('Loss/val', val_loss, epoch)
    writer.add_scalar('LR', optimizer.param_groups[0]['lr'], epoch)

writer.close()

# View: tensorboard --logdir=runs
\end{lstlisting}

\clearpage
\subsubsection{Reproducibility}

\begin{lstlisting}
import random
import numpy as np
import torch

def set_seed(seed=42):
    """Set all random seeds for reproducibility."""
    random.seed(seed)
    np.random.seed(seed)
    torch.manual_seed(seed)
    torch.cuda.manual_seed(seed)
    torch.cuda.manual_seed_all(seed)
    
    # Make cudnn deterministic
    torch.backends.cudnn.deterministic = True
    torch.backends.cudnn.benchmark = False

# Call at start of training
set_seed(42)

# Note: This makes training slower but reproducible
# For production: set benchmark=True for speed
\end{lstlisting}

\subsubsection{Checkpointing}

\begin{lstlisting}
def save_checkpoint(model, optimizer, epoch, loss, path):
    """Save model checkpoint."""
    torch.save({
        'epoch': epoch,
        'model_state_dict': model.state_dict(),
        'optimizer_state_dict': optimizer.state_dict(),
        'loss': loss,
    }, path)

def load_checkpoint(model, optimizer, path):
    """Load model checkpoint."""
    checkpoint = torch.load(path)
    model.load_state_dict(checkpoint['model_state_dict'])
    optimizer.load_state_dict(checkpoint['optimizer_state_dict'])
    epoch = checkpoint['epoch']
    loss = checkpoint['loss']
    return epoch, loss

# Save best model
best_loss = float('inf')

for epoch in range(num_epochs):
    train_loss = train_epoch()
    val_loss = validate()
    
    # Save if best
    if val_loss < best_loss:
        best_loss = val_loss
        save_checkpoint(model, optimizer, epoch, val_loss, 
                       'best_model.pth')
    
    # Save periodic checkpoints
    if epoch % 10 == 0:
        save_checkpoint(model, optimizer, epoch, val_loss,
                       f'checkpoint_epoch_{epoch}.pth')
\end{lstlisting}

\subsection{Debugging Checklist}

\textbf{When starting a new project:}
\begin{enumerate}
    \item ☐ Can model overfit one batch?
    \item ☐ Can model overfit small dataset (100 samples)?
    \item ☐ Are shapes correct at every layer?
    \item ☐ Is loss function appropriate for task?
    \item ☐ Is learning rate reasonable? (Try LR finder)
    \item ☐ Is data normalized?
    \item ☐ Are you using the right optimizer?
    \item ☐ Did you set model.train() / model.eval()?
\end{enumerate}

\textbf{When loss is NaN:}
\begin{enumerate}
    \item ☐ Check for NaN in data
    \item ☐ Reduce learning rate
    \item ☐ Add gradient clipping
    \item ☐ Use stable loss functions
    \item ☐ Check for division by zero
\end{enumerate}

\textbf{When loss doesn't decrease:}
\begin{enumerate}
    \item ☐ Try higher learning rate
    \item ☐ Check if you're zeroing gradients
    \item ☐ Verify loss function is correct
    \item ☐ Check data normalization
    \item ☐ Verify model has enough capacity
    \item ☐ Check for dead ReLUs
\end{enumerate}

\textbf{When overfitting:}
\begin{enumerate}
    \item ☐ Add dropout
    \item ☐ Add weight decay
    \item ☐ Use data augmentation
    \item ☐ Reduce model size
    \item ☐ Get more data
    \item ☐ Use early stopping
\end{enumerate}

\clearpage
\subsection{Key Takeaways}

\textbf{Systematic debugging:}
\begin{itemize}
    \item Start simple (overfit one batch)
    \item Add complexity gradually
    \item Check shapes at every step
    \item Visualize gradients and activations
    \item Use hooks for inspection
\end{itemize}

\textbf{Common bugs:}
\begin{itemize}
    \item NaN loss: LR too high, numerical instability, exploding gradients
    \item Loss not decreasing: LR too low, wrong loss, forgot zero\_grad, bad data
    \item Overfitting: Model too complex, no regularization, training too long
    \item Works on toy data, fails on real: Batch size, distribution shift
\end{itemize}

\textbf{Best practices:}
\begin{itemize}
    \item Separate code into modules (data, models, training)
    \item Use configuration files
    \item Track experiments (TensorBoard)
    \item Set random seeds for reproducibility
    \item Save checkpoints regularly
    \item Log everything
\end{itemize}

\textbf{Development workflow:}
\begin{enumerate}
    \item Start with simple baseline
    \item Verify can overfit small data
    \item Add regularization
    \item Scale to full dataset
    \item Tune hyperparameters
    \item Validate on held-out test set
\end{enumerate}

\textbf{Tools for debugging:}
\begin{itemize}
    \item Print statements (shapes, values, statistics)
    \item Hooks (forward and backward)
    \item TensorBoard (visualize training)
    \item Gradient visualization
    \item Learning rate finder
    \item Activation distribution checks
\end{itemize}

\textbf{Remember:}
\begin{itemize}
    \item Deep learning bugs are often silent (poor performance, not crashes)
    \item Always start with simplest possible setup
    \item Check your data first (garbage in = garbage out)
    \item One change at a time (isolate what works)
    \item Keep good records (what did you try, what worked)
\end{itemize}

\clearpage
% =============================================
% SECTION 14: TRAINING BEST PRACTICES
% =============================================

\section{Training Best Practices}

\subsection{Introduction: From Working to Working Well}

You've built a model that trains. Now: how do you make it train \textbf{well}?

This section covers practical strategies for:
\begin{itemize}
    \item Choosing hyperparameters
    \item Learning rate strategies
    \item Efficient training
    \item Avoiding common pitfalls
\end{itemize}

\subsection{Hyperparameter Tuning}

\subsubsection{Which Hyperparameters Matter Most?}

\textbf{Priority order:}

\textbf{1. Learning rate (MOST IMPORTANT)}

Can make 10× difference in performance. Get this right first.

\textbf{2. Batch size}

Affects both training speed and generalization.

\textbf{3. Architecture (width, depth)}

Number of layers and neurons per layer.

\textbf{4. Regularization (dropout, weight decay)}

Prevents overfitting.

\textbf{5. Learning rate schedule}

How LR changes over time.

\textbf{6. Everything else}

Optimizer choice, activation functions, initialization—usually matter less.

\subsubsection{Learning Rate Strategies}

\textbf{Strategy 1: Learning Rate Finder}

Find optimal LR before training:

\begin{lstlisting}
from torch_lr_finder import LRFinder

model = YourModel()
optimizer = torch.optim.Adam(model.parameters(), lr=1e-7)
criterion = nn.CrossEntropyLoss()

lr_finder = LRFinder(model, optimizer, criterion, device="cuda")
lr_finder.range_test(train_loader, end_lr=10, num_iter=100)
lr_finder.plot()  # Shows loss vs LR

# Pick LR where loss decreases fastest (steepest slope)
# Usually 10× smaller than where loss explodes
optimal_lr = 1e-3  # Read from plot

lr_finder.reset()  # Reset model and optimizer
\end{lstlisting}

\textbf{Strategy 2: Start Large, Decay}

\begin{lstlisting}
# Common schedules

# 1. Step decay (reduce every N epochs)
scheduler = torch.optim.lr_scheduler.StepLR(
    optimizer, step_size=30, gamma=0.1
)
# LR × 0.1 every 30 epochs

# 2. Exponential decay
scheduler = torch.optim.lr_scheduler.ExponentialLR(
    optimizer, gamma=0.95
)
# LR × 0.95 every epoch

# 3. Cosine annealing (smooth decay)
scheduler = torch.optim.lr_scheduler.CosineAnnealingLR(
    optimizer, T_max=100, eta_min=1e-6
)
# Smooth cosine decay from initial_lr to eta_min over T_max epochs

# 4. Reduce on plateau (adaptive)
scheduler = torch.optim.lr_scheduler.ReduceLROnPlateau(
    optimizer, mode='min', factor=0.5, patience=10
)
# Reduce LR by 0.5 if val loss doesn't improve for 10 epochs

# Usage
for epoch in range(num_epochs):
    train_loss = train_epoch()
    val_loss = validate()
    
    # For ReduceLROnPlateau, pass validation loss
    scheduler.step(val_loss)  # or scheduler.step() for others
\end{lstlisting}

\textbf{Strategy 3: Warmup + Cosine Decay}

State-of-the-art for Transformers:

\begin{lstlisting}
from torch.optim.lr_scheduler import LambdaLR

def get_cosine_schedule_with_warmup(optimizer, num_warmup_steps, 
                                    num_training_steps):
    """Warmup + cosine decay."""
    
    def lr_lambda(current_step):
        # Warmup
        if current_step < num_warmup_steps:
            return float(current_step) / float(max(1, num_warmup_steps))
        
        # Cosine decay
        progress = float(current_step - num_warmup_steps) / \
                  float(max(1, num_training_steps - num_warmup_steps))
        return max(0.0, 0.5 * (1.0 + math.cos(math.pi * progress)))
    
    return LambdaLR(optimizer, lr_lambda)

# Usage
num_epochs = 100
steps_per_epoch = len(train_loader)
num_training_steps = num_epochs * steps_per_epoch
num_warmup_steps = 5 * steps_per_epoch  # 5 epochs warmup

optimizer = torch.optim.AdamW(model.parameters(), lr=1e-3)
scheduler = get_cosine_schedule_with_warmup(
    optimizer, num_warmup_steps, num_training_steps
)

for epoch in range(num_epochs):
    for batch in train_loader:
        # Training step
        ...
        optimizer.step()
        scheduler.step()  # Update LR every step, not epoch!
\end{lstlisting}

\clearpage
\subsubsection{Batch Size Selection}

\textbf{Trade-offs:}

\begin{table}[h]
\centering
\begin{tabular}{lll}
\toprule
\textbf{Batch Size} & \textbf{Pros} & \textbf{Cons} \\
\midrule
Small (8-32) & Better generalization & Slower, noisy gradients \\
             & Less memory & Batch norm unstable \\
Medium (64-256) & Good balance & Standard choice \\
Large (512+) & Faster training & Worse generalization \\
             & More stable & Requires careful tuning \\
\bottomrule
\end{tabular}
\caption{Batch size trade-offs}
\end{table}

\textbf{Rules of thumb:}
\begin{itemize}
    \item \textbf{Start with 32-128:} Usually works well
    \item \textbf{Larger batches need higher LR:} Scale by $\sqrt{\text{batch size}}$
    \item \textbf{Use gradient accumulation} if memory limited
    \item \textbf{Don't go below 8} if using batch norm
\end{itemize}

\textbf{Linear scaling rule:}

If you double batch size, double learning rate (approximately).

\begin{lstlisting}
# Base setup
batch_size = 64
lr = 0.001

# Double batch size
batch_size = 128
lr = 0.002  # Double LR

# But: this is approximate! Always validate on your task
\end{lstlisting}

\subsubsection{Model Architecture Selection}

\textbf{Width vs Depth:}

\begin{lstlisting}
# Wider networks (fewer layers, more neurons)
model_wide = MLP([100, 1024, 1024, 10])  # 2 hidden layers, 1024 neurons

# Deeper networks (more layers, fewer neurons)
model_deep = MLP([100, 256, 256, 256, 256, 10])  # 4 hidden layers, 256 neurons

# Modern preference: deeper is usually better (with skip connections)
\end{lstlisting}

\textbf{Start small, scale up:}

\begin{enumerate}
    \item Start with small model that trains fast
    \item Check if it underfits (high train loss)
    \item If underfitting: increase capacity
    \item If overfitting: add regularization
\end{enumerate}

\subsection{Regularization Strategies}

\subsubsection{Dropout}

\begin{lstlisting}
# Standard dropout
model = nn.Sequential(
    nn.Linear(100, 256),
    nn.ReLU(),
    nn.Dropout(0.5),  # Drop 50% during training
    nn.Linear(256, 10)
)

# Guidelines:
# - Start with 0.5, tune if needed
# - Higher dropout = more regularization
# - Use lower dropout (0.1-0.2) for convolutional layers
# - Higher dropout (0.5-0.7) for large fully connected layers
\end{lstlisting}

\subsubsection{Weight Decay (L2 Regularization)}

\begin{lstlisting}
# Add weight decay to optimizer
optimizer = torch.optim.Adam(
    model.parameters(),
    lr=1e-3,
    weight_decay=1e-4  # L2 penalty
)

# Typical values: 1e-5 to 1e-3
# Larger weight decay = stronger regularization

# Note: Use AdamW for better weight decay
optimizer = torch.optim.AdamW(model.parameters(), lr=1e-3, 
                             weight_decay=1e-4)
\end{lstlisting}

\subsubsection{Data Augmentation}

\begin{lstlisting}
# For images
train_transform = transforms.Compose([
    transforms.RandomHorizontalFlip(),
    transforms.RandomCrop(32, padding=4),
    transforms.ColorJitter(brightness=0.2, contrast=0.2),
    transforms.RandomRotation(15),
    transforms.ToTensor(),
    transforms.Normalize(mean=[0.485, 0.456, 0.406],
                        std=[0.229, 0.224, 0.225])
])

# NO augmentation for validation/test
val_transform = transforms.Compose([
    transforms.ToTensor(),
    transforms.Normalize(mean=[0.485, 0.456, 0.406],
                        std=[0.229, 0.224, 0.225])
])
\end{lstlisting}

\clearpage
\subsection{Training Monitoring}

\subsubsection{What to Track}

\textbf{Essential metrics:}
\begin{itemize}
    \item Training loss (per epoch)
    \item Validation loss (per epoch)
    \item Validation accuracy/metric (per epoch)
    \item Learning rate (per epoch or step)
\end{itemize}

\textbf{Useful but optional:}
\begin{itemize}
    \item Gradient norms
    \item Weight norms
    \item Training time per epoch
    \item GPU memory usage
\end{itemize}

\begin{lstlisting}
# Complete training loop with monitoring
from torch.utils.tensorboard import SummaryWriter

writer = SummaryWriter('runs/experiment')

for epoch in range(num_epochs):
    # Training
    model.train()
    train_loss = 0
    train_correct = 0
    train_total = 0
    
    for batch in train_loader:
        optimizer.zero_grad()
        
        x, y = batch
        x, y = x.to(device), y.to(device)
        
        output = model(x)
        loss = criterion(output, y)
        loss.backward()
        
        # Track gradient norm
        grad_norm = torch.nn.utils.clip_grad_norm_(
            model.parameters(), max_norm=1.0
        )
        
        optimizer.step()
        
        train_loss += loss.item()
        pred = output.argmax(dim=1)
        train_correct += (pred == y).sum().item()
        train_total += y.size(0)
    
    # Validation
    model.eval()
    val_loss = 0
    val_correct = 0
    val_total = 0
    
    with torch.no_grad():
        for batch in val_loader:
            x, y = batch
            x, y = x.to(device), y.to(device)
            
            output = model(x)
            loss = criterion(output, y)
            
            val_loss += loss.item()
            pred = output.argmax(dim=1)
            val_correct += (pred == y).sum().item()
            val_total += y.size(0)
    
    # Compute metrics
    train_loss /= len(train_loader)
    train_acc = train_correct / train_total
    val_loss /= len(val_loader)
    val_acc = val_correct / val_total
    
    # Log to TensorBoard
    writer.add_scalar('Loss/train', train_loss, epoch)
    writer.add_scalar('Loss/val', val_loss, epoch)
    writer.add_scalar('Accuracy/train', train_acc, epoch)
    writer.add_scalar('Accuracy/val', val_acc, epoch)
    writer.add_scalar('LR', optimizer.param_groups[0]['lr'], epoch)
    writer.add_scalar('GradNorm', grad_norm, epoch)
    
    # Print
    print(f"Epoch {epoch}: Train Loss={train_loss:.4f}, "
          f"Val Loss={val_loss:.4f}, Val Acc={val_acc:.4f}")
    
    # Step scheduler
    scheduler.step()

writer.close()
\end{lstlisting}

\subsubsection{Recognizing Training Issues from Curves}

\textbf{Healthy training:}
\begin{verbatim}
Train loss: Decreasing smoothly
Val loss: Decreasing, tracking train loss
Gap: Small (< 10% difference)
\end{verbatim}

\textbf{Overfitting:}
\begin{verbatim}
Train loss: Very low
Val loss: High and increasing
Gap: Large and growing
→ Solution: More regularization, more data
\end{verbatim}

\textbf{Underfitting:}
\begin{verbatim}
Train loss: High
Val loss: High (similar to train)
Gap: Small
→ Solution: Bigger model, train longer
\end{verbatim}

\textbf{Unstable training:}
\begin{verbatim}
Both losses: Oscillating wildly
→ Solution: Lower learning rate, gradient clipping
\end{verbatim}

\subsection{Model Selection and Evaluation}

\subsubsection{Proper Train/Val/Test Split}

\begin{lstlisting}
# WRONG: Tune on test set
# This gives overly optimistic results!

# RIGHT: Three separate sets
# 1. Training set (70%): Train model
# 2. Validation set (15%): Tune hyperparameters
# 3. Test set (15%): Final evaluation (ONCE!)

from sklearn.model_selection import train_test_split

# Split data
X_train, X_temp, y_train, y_temp = train_test_split(
    X, y, test_size=0.3, stratify=y, random_state=42
)
X_val, X_test, y_val, y_test = train_test_split(
    X_temp, y_temp, test_size=0.5, stratify=y_temp, random_state=42
)

# Training process:
# 1. Train many models with different hyperparameters
# 2. Evaluate all on validation set
# 3. Pick best model
# 4. Train best model on train+val (optional)
# 5. Evaluate ONCE on test set
# 6. Report test performance
\end{lstlisting}

\subsubsection{Cross-Validation for Small Datasets}

\begin{lstlisting}
from sklearn.model_selection import KFold

kfold = KFold(n_splits=5, shuffle=True, random_state=42)

scores = []

for fold, (train_idx, val_idx) in enumerate(kfold.split(X)):
    print(f"Fold {fold + 1}")
    
    # Split data
    X_train, X_val = X[train_idx], X[val_idx]
    y_train, y_val = y[train_idx], y[val_idx]
    
    # Create datasets
    train_dataset = YourDataset(X_train, y_train)
    val_dataset = YourDataset(X_val, y_val)
    
    # Train model (reset for each fold!)
    model = YourModel()
    optimizer = torch.optim.Adam(model.parameters())
    
    # Train...
    val_score = train_and_evaluate(model, train_dataset, val_dataset)
    scores.append(val_score)

# Report mean and std
print(f"CV Score: {np.mean(scores):.4f} ± {np.std(scores):.4f}")
\end{lstlisting}

\clearpage
\subsection{Practical Tips}

\subsubsection{Start with Good Defaults}

\begin{lstlisting}
# Proven configuration for most tasks:

optimizer = torch.optim.AdamW(
    model.parameters(),
    lr=1e-3,          # Will likely need tuning
    weight_decay=1e-4,
    betas=(0.9, 0.999)
)

scheduler = torch.optim.lr_scheduler.CosineAnnealingLR(
    optimizer, T_max=num_epochs
)

# Initialization: PyTorch defaults are usually good
# But for RNNs, use orthogonal init for hidden-to-hidden

# Batch size: 64 or 128
# Epochs: Start with 100, use early stopping
# Gradient clipping: max_norm=1.0 (for RNNs)
\end{lstlisting}

\subsubsection{Hyperparameter Search Strategies}

\textbf{Grid search (exhaustive):}
\begin{lstlisting}
lrs = [1e-4, 1e-3, 1e-2]
weight_decays = [0, 1e-5, 1e-4, 1e-3]
dropouts = [0.2, 0.5]

best_val_loss = float('inf')
best_config = None

for lr in lrs:
    for wd in weight_decays:
        for dropout in dropouts:
            val_loss = train_and_evaluate(lr, wd, dropout)
            
            if val_loss < best_val_loss:
                best_val_loss = val_loss
                best_config = (lr, wd, dropout)

print(f"Best: LR={best_config[0]}, WD={best_config[1]}, "
      f"Dropout={best_config[2]}")
\end{lstlisting}

\textbf{Random search (more efficient):}
\begin{lstlisting}
import random

def random_config():
    lr = 10 ** random.uniform(-5, -2)  # 1e-5 to 1e-2
    wd = 10 ** random.uniform(-6, -3)  # 1e-6 to 1e-3
    dropout = random.uniform(0.1, 0.7)
    return lr, wd, dropout

num_trials = 20
results = []

for trial in range(num_trials):
    lr, wd, dropout = random_config()
    val_loss = train_and_evaluate(lr, wd, dropout)
    results.append((val_loss, lr, wd, dropout))

# Sort by validation loss
results.sort()
best = results[0]
print(f"Best: Val Loss={best[0]:.4f}, LR={best[1]:.2e}, "
      f"WD={best[2]:.2e}, Dropout={best[3]:.2f}")
\end{lstlisting}

\textbf{Bayesian optimization (most efficient):}
\begin{lstlisting}
# Use libraries like Optuna or Ray Tune
import optuna

def objective(trial):
    # Suggest hyperparameters
    lr = trial.suggest_loguniform('lr', 1e-5, 1e-2)
    wd = trial.suggest_loguniform('weight_decay', 1e-6, 1e-3)
    dropout = trial.suggest_uniform('dropout', 0.1, 0.7)
    
    # Train and return validation loss
    val_loss = train_and_evaluate(lr, wd, dropout)
    return val_loss

# Optimize
study = optuna.create_study(direction='minimize')
study.optimize(objective, n_trials=50)

print(f"Best hyperparameters: {study.best_params}")
print(f"Best validation loss: {study.best_value:.4f}")
\end{lstlisting}

\subsection{Key Takeaways}

\textbf{Hyperparameter priorities:}
\begin{enumerate}
    \item Learning rate (MOST IMPORTANT)
    \item Batch size
    \item Architecture (depth, width)
    \item Regularization (dropout, weight decay)
    \item Everything else
\end{enumerate}

\textbf{Learning rate strategies:}
\begin{itemize}
    \item Use LR finder to find good starting point
    \item Start high, decay over time
    \item Cosine annealing works well
    \item Warmup helps for Transformers
    \item ReduceLROnPlateau is safe fallback
\end{itemize}

\textbf{Batch size selection:}
\begin{itemize}
    \item Start with 32-128
    \item Larger batch = faster but may hurt generalization
    \item Scale LR with batch size (approximately)
    \item Use gradient accumulation if memory limited
\end{itemize}

\textbf{Regularization:}
\begin{itemize}
    \item Dropout: 0.5 for FC layers, 0.1-0.2 for conv layers
    \item Weight decay: 1e-5 to 1e-3
    \item Data augmentation: Always for images
    \item Early stopping: Always use
\end{itemize}

\textbf{Training monitoring:}
\begin{itemize}
    \item Track train/val loss and metrics
    \item Use TensorBoard for visualization
    \item Watch for overfitting (gap between train/val)
    \item Watch for underfitting (both losses high)
\end{itemize}

\textbf{Model selection:}
\begin{itemize}
    \item Always use separate test set
    \item Tune on validation set only
    \item Test set evaluated ONCE at the end
    \item Use cross-validation for small datasets
\end{itemize}

\textbf{Hyperparameter search:}
\begin{itemize}
    \item Random search better than grid search
    \item Bayesian optimization most efficient
    \item Start with good defaults
    \item Log everything
\end{itemize}

\textbf{Common mistakes:}
\begin{itemize}
    \item Not tuning learning rate
    \item Evaluating on test set during development
    \item Using validation statistics to normalize test data
    \item Not using early stopping
    \item Training for too many epochs
\end{itemize}

\clearpage
% =============================================
% SECTION 15: PERFORMANCE & OPTIMIZATION
% =============================================

\section{Performance \& Optimization}

\subsection{Introduction: Making Training Faster}

Training can be slow. A model that takes 1 hour vs 10 hours per epoch makes a huge difference in development speed.

This section covers:
\begin{itemize}
    \item Mixed precision training (2× speedup)
    \item GPU optimization
    \item DataLoader efficiency
    \item Memory management
    \item Profiling tools
\end{itemize}

\subsection{Mixed Precision Training}

\textbf{Idea:} Use FP16 (half precision) instead of FP32 (full precision) for most operations.

\textbf{Benefits:}
\begin{itemize}
    \item 2-3× faster training
    \item 2× less memory (can use larger batch sizes)
    \item Minimal accuracy loss
\end{itemize}

\subsubsection{Automatic Mixed Precision (AMP)}

PyTorch provides automatic mixed precision via \texttt{torch.cuda.amp}:

\begin{lstlisting}
from torch.cuda.amp import autocast, GradScaler

model = YourModel().cuda()
optimizer = torch.optim.Adam(model.parameters(), lr=1e-3)

# Create gradient scaler
scaler = GradScaler()

for epoch in range(num_epochs):
    for x, y in train_loader:
        x, y = x.cuda(), y.cuda()
        
        optimizer.zero_grad()
        
        # Forward pass in mixed precision
        with autocast():
            output = model(x)
            loss = criterion(output, y)
        
        # Backward pass with gradient scaling
        scaler.scale(loss).backward()
        
        # Unscale gradients and step
        scaler.step(optimizer)
        scaler.update()
\end{lstlisting}

\textbf{How it works:}
\begin{enumerate}
    \item \texttt{autocast()}: Automatically chooses FP16 or FP32 for each operation
    \item \texttt{GradScaler}: Scales loss to prevent underflow in FP16 gradients
\end{enumerate}

\begin{pytorchtip}[When to Use Mixed Precision]
\textbf{Use AMP when:}
\begin{itemize}
    \item Training on modern GPUs (Volta, Turing, Ampere)
    \item You want faster training
    \item You're memory-limited
\end{itemize}

\textbf{Don't use if:}
\begin{itemize}
    \item GPU doesn't support FP16 (e.g., older GPUs)
    \item Model has numerical instability issues
    \item You need exact FP32 precision (rare)
\end{itemize}

\textbf{Typical speedup:} 1.5-3× depending on model and GPU.
\end{pytorchtip}

\subsubsection{Gradient Checkpointing}

Trade computation for memory:

\begin{lstlisting}
from torch.utils.checkpoint import checkpoint

class MemoryEfficientBlock(nn.Module):
    def __init__(self, ...):
        super().__init__()
        self.layer = ExpensiveLayer()
    
    def forward(self, x):
        # Use checkpointing for this layer
        return checkpoint(self.layer, x)

# Saves activations strategically
# Recomputes forward pass during backward (uses more time, less memory)
\end{lstlisting}

\textbf{When to use:}
\begin{itemize}
    \item Very deep models (100+ layers)
    \item Running out of GPU memory
    \item Willing to trade 20-30\% slower training for 2× memory reduction
\end{itemize}

\clearpage
\subsection{GPU Optimization}

\subsubsection{Data Transfer Optimization}

\textbf{Pin memory for faster CPU→GPU transfer:}

\begin{lstlisting}
train_loader = DataLoader(
    dataset,
    batch_size=128,
    num_workers=4,
    pin_memory=True  # Faster transfers
)

# Also call .cuda(non_blocking=True)
for x, y in train_loader:
    x = x.cuda(non_blocking=True)
    y = y.cuda(non_blocking=True)
    
    # Model forward...
\end{lstlisting}

\textbf{Keep data on GPU when possible:}

\begin{lstlisting}
# BAD: Moving data back and forth
for x, y in train_loader:
    x = x.cuda()
    output = model(x)
    output = output.cpu()  # Unnecessary transfer!
    loss = criterion(output, y.cpu())  # Another transfer!

# GOOD: Keep everything on GPU
for x, y in train_loader:
    x, y = x.cuda(), y.cuda()
    output = model(x)  # Stays on GPU
    loss = criterion(output, y)  # All on GPU
\end{lstlisting}

\subsubsection{Batch Size and GPU Utilization}

\begin{lstlisting}
import torch

# Check GPU utilization
# nvidia-smi in terminal

# If GPU utilization < 80%, try:
# 1. Increase batch size
train_loader = DataLoader(dataset, batch_size=256)  # Larger

# 2. Increase num_workers
train_loader = DataLoader(dataset, batch_size=128, num_workers=8)

# 3. Use larger model (if not memory-limited)
\end{lstlisting}

\subsubsection{Multi-GPU Training}

\textbf{DataParallel (simple but not recommended):}

\begin{lstlisting}
if torch.cuda.device_count() > 1:
    print(f"Using {torch.cuda.device_count()} GPUs")
    model = nn.DataParallel(model)

model = model.cuda()

# Training as usual
# Batch automatically split across GPUs
\end{lstlisting}

\textbf{DistributedDataParallel (better, recommended):}

\begin{lstlisting}
import torch.distributed as dist
from torch.nn.parallel import DistributedDataParallel as DDP

def setup(rank, world_size):
    dist.init_process_group("nccl", rank=rank, world_size=world_size)

def cleanup():
    dist.destroy_process_group()

def train(rank, world_size):
    setup(rank, world_size)
    
    # Create model and move to GPU
    model = YourModel().to(rank)
    model = DDP(model, device_ids=[rank])
    
    # Create distributed sampler
    sampler = torch.utils.data.distributed.DistributedSampler(
        dataset, num_replicas=world_size, rank=rank
    )
    
    train_loader = DataLoader(dataset, batch_size=128, sampler=sampler)
    
    # Training loop as usual
    for epoch in range(num_epochs):
        sampler.set_epoch(epoch)  # Important for shuffling!
        for x, y in train_loader:
            # Training step...
            pass
    
    cleanup()

# Launch with torch.multiprocessing
import torch.multiprocessing as mp
mp.spawn(train, args=(world_size,), nprocs=world_size, join=True)

# Or use torchrun:
# torchrun --nproc_per_node=4 train.py
\end{lstlisting}

\textbf{DDP vs DataParallel:}
\begin{itemize}
    \item DDP: Faster, more efficient, recommended
    \item DP: Simpler but slower, single-process
\end{itemize}

\clearpage
\subsection{DataLoader Optimization}

\subsubsection{Optimal Number of Workers}

\begin{lstlisting}
import time

def benchmark_dataloader(num_workers):
    """Find optimal num_workers."""
    loader = DataLoader(dataset, batch_size=128, 
                       num_workers=num_workers, pin_memory=True)
    
    start = time.time()
    for _ in loader:
        pass  # Just iterate
    elapsed = time.time() - start
    
    return elapsed

# Test different values
for num_workers in [0, 2, 4, 8, 16]:
    elapsed = benchmark_dataloader(num_workers)
    print(f"num_workers={num_workers}: {elapsed:.2f}s")

# Typical optimal: 4-8 workers
# More workers = more CPU usage but faster
# Too many workers = overhead dominates
\end{lstlisting}

\textbf{Guidelines:}
\begin{itemize}
    \item Start with \texttt{num\_workers=4}
    \item If CPU usage low: increase
    \item If CPU usage 100\%: decrease
    \item Typical range: 2-8 workers
\end{itemize}

\subsubsection{Prefetching}

PyTorch DataLoader automatically prefetches, but you can optimize:

\begin{lstlisting}
# Use persistent_workers to avoid recreating workers
train_loader = DataLoader(
    dataset,
    batch_size=128,
    num_workers=4,
    pin_memory=True,
    persistent_workers=True,  # Keep workers alive between epochs
    prefetch_factor=2  # Number of batches to prefetch per worker
)
\end{lstlisting}

\subsection{Memory Management}

\subsubsection{Monitoring Memory Usage}

\begin{lstlisting}
import torch

def print_memory_usage():
    """Print current GPU memory usage."""
    if torch.cuda.is_available():
        allocated = torch.cuda.memory_allocated() / 1e9
        reserved = torch.cuda.memory_reserved() / 1e9
        print(f"Allocated: {allocated:.2f} GB")
        print(f"Reserved: {reserved:.2f} GB")

# Check memory before and after operations
print("Before model creation:")
print_memory_usage()

model = LargeModel().cuda()

print("After model creation:")
print_memory_usage()
\end{lstlisting}

\subsubsection{Reducing Memory Usage}

\textbf{1. Use gradient accumulation instead of large batches:}

\begin{lstlisting}
# Instead of batch_size=512 (OOM)
# Use batch_size=128 with accumulation

accumulation_steps = 4
effective_batch_size = 128 * 4  # = 512

optimizer.zero_grad()

for i, (x, y) in enumerate(train_loader):
    output = model(x)
    loss = criterion(output, y)
    loss = loss / accumulation_steps  # Scale loss
    loss.backward()  # Accumulate gradients
    
    if (i + 1) % accumulation_steps == 0:
        optimizer.step()
        optimizer.zero_grad()
\end{lstlisting}

\textbf{2. Clear cache periodically:}

\begin{lstlisting}
# If encountering OOM during long training
if epoch % 10 == 0:
    torch.cuda.empty_cache()
\end{lstlisting}

\textbf{3. Delete unnecessary variables:}

\begin{lstlisting}
# In evaluation, don't keep gradients
with torch.no_grad():
    for x, y in val_loader:
        output = model(x)
        # ...

# If you need to keep large tensors temporarily
large_tensor = compute_something()
result = process(large_tensor)
del large_tensor  # Free memory immediately
torch.cuda.empty_cache()
\end{lstlisting}

\textbf{4. Use smaller data types:}

\begin{lstlisting}
# FP32 (default)
x = torch.randn(1000, 1000)  # 4 MB

# FP16 (half)
x = torch.randn(1000, 1000, dtype=torch.float16)  # 2 MB

# For labels, use appropriate type
labels = torch.randint(0, 10, (1000,), dtype=torch.long)  # Not float!
\end{lstlisting}

\clearpage
\subsection{Profiling and Bottleneck Analysis}

\subsubsection{PyTorch Profiler}

\begin{lstlisting}
from torch.profiler import profile, ProfilerActivity

model = YourModel().cuda()

# Profile training
with profile(
    activities=[ProfilerActivity.CPU, ProfilerActivity.CUDA],
    record_shapes=True,
    profile_memory=True,
    with_stack=True
) as prof:
    for i, (x, y) in enumerate(train_loader):
        if i >= 10:  # Profile first 10 batches
            break
        
        x, y = x.cuda(), y.cuda()
        
        optimizer.zero_grad()
        output = model(x)
        loss = criterion(output, y)
        loss.backward()
        optimizer.step()

# Print profiling results
print(prof.key_averages().table(sort_by="cuda_time_total", row_limit=10))

# Export for Chrome tracing
prof.export_chrome_trace("trace.json")
# View at chrome://tracing
\end{lstlisting}

\subsubsection{Simple Timing}

\begin{lstlisting}
import time

# Time entire epoch
start = time.time()
for x, y in train_loader:
    # Training step...
    pass
elapsed = time.time() - start
print(f"Epoch time: {elapsed:.2f}s")

# Time individual operations
torch.cuda.synchronize()  # Wait for GPU
start = time.time()

output = model(x)

torch.cuda.synchronize()
elapsed = time.time() - start
print(f"Forward pass: {elapsed*1000:.2f}ms")
\end{lstlisting}

\subsubsection{Identify Bottlenecks}

\begin{lstlisting}
def profile_training_loop(model, train_loader, num_batches=10):
    """Profile different parts of training."""
    
    times = {
        'data_loading': 0,
        'to_gpu': 0,
        'forward': 0,
        'backward': 0,
        'optimizer': 0,
        'total': 0
    }
    
    total_start = time.time()
    data_start = time.time()
    
    for i, (x, y) in enumerate(train_loader):
        if i >= num_batches:
            break
        
        times['data_loading'] += time.time() - data_start
        
        # Transfer to GPU
        gpu_start = time.time()
        x, y = x.cuda(), y.cuda()
        torch.cuda.synchronize()
        times['to_gpu'] += time.time() - gpu_start
        
        # Forward
        forward_start = time.time()
        output = model(x)
        torch.cuda.synchronize()
        times['forward'] += time.time() - forward_start
        
        # Backward
        backward_start = time.time()
        loss = criterion(output, y)
        loss.backward()
        torch.cuda.synchronize()
        times['backward'] += time.time() - backward_start
        
        # Optimizer
        opt_start = time.time()
        optimizer.step()
        optimizer.zero_grad()
        torch.cuda.synchronize()
        times['optimizer'] += time.time() - opt_start
        
        data_start = time.time()
    
    times['total'] = time.time() - total_start
    
    # Print breakdown
    print("Time breakdown:")
    for key, value in times.items():
        if key != 'total':
            pct = (value / times['total']) * 100
            print(f"  {key}: {value:.3f}s ({pct:.1f}%)")
        else:
            print(f"  {key}: {value:.3f}s")

# Usage
profile_training_loop(model, train_loader)
"""
Time breakdown:
  data_loading: 0.421s (25.3%)
  to_gpu: 0.089s (5.3%)
  forward: 0.567s (34.1%)
  backward: 0.423s (25.4%)
  optimizer: 0.165s (9.9%)
  total: 1.665s
"""

# If data_loading is high: increase num_workers
# If forward/backward is high: model is slow (expected)
# If to_gpu is high: use pin_memory, non_blocking transfers
\end{lstlisting}

\clearpage
\subsection{Production Deployment}

\subsubsection{Model Export}

\textbf{TorchScript (JIT compilation):}

\begin{lstlisting}
# Option 1: Tracing
model = YourModel()
model.eval()

example_input = torch.randn(1, 3, 224, 224)
traced_model = torch.jit.trace(model, example_input)

# Save
traced_model.save("model_traced.pt")

# Load and use
loaded_model = torch.jit.load("model_traced.pt")
output = loaded_model(example_input)
\end{lstlisting}

\begin{lstlisting}
# Option 2: Scripting (handles control flow)
scripted_model = torch.jit.script(model)
scripted_model.save("model_scripted.pt")
\end{lstlisting}

\textbf{ONNX (cross-framework):}

\begin{lstlisting}
import torch.onnx

model = YourModel()
model.eval()

dummy_input = torch.randn(1, 3, 224, 224)

torch.onnx.export(
    model,
    dummy_input,
    "model.onnx",
    export_params=True,
    opset_version=11,
    input_names=['input'],
    output_names=['output'],
    dynamic_axes={'input': {0: 'batch_size'},
                  'output': {0: 'batch_size'}}
)

# Can now use in other frameworks (TensorFlow, etc.)
\end{lstlisting}

\subsubsection{Inference Optimization}

\begin{lstlisting}
# 1. Set to eval mode
model.eval()

# 2. Disable gradient computation
with torch.no_grad():
    output = model(input)

# 3. Use FP16 if possible
with torch.cuda.amp.autocast():
    output = model(input)

# 4. Batch inference
# Process multiple inputs together
batch = torch.stack([input1, input2, input3])
outputs = model(batch)

# 5. Use torch.jit for production
model = torch.jit.load("model.pt")
\end{lstlisting}

\subsection{Key Takeaways}

\textbf{Mixed precision training:}
\begin{itemize}
    \item Use \texttt{torch.cuda.amp} for 2× speedup
    \item Minimal code changes required
    \item Works on modern GPUs (Volta+)
    \item Typical speedup: 1.5-3×
\end{itemize}

\textbf{GPU optimization:}
\begin{itemize}
    \item Keep data on GPU (avoid CPU↔GPU transfers)
    \item Use \texttt{pin\_memory=True} and \texttt{non\_blocking=True}
    \item Maximize batch size (within memory limits)
    \item Use DistributedDataParallel for multi-GPU
\end{itemize}

\textbf{DataLoader optimization:}
\begin{itemize}
    \item Start with \texttt{num\_workers=4}
    \item Use \texttt{persistent\_workers=True}
    \item Benchmark different worker counts
    \item Typical optimal: 4-8 workers
\end{itemize}

\textbf{Memory management:}
\begin{itemize}
    \item Use gradient accumulation for large effective batch sizes
    \item Clear cache periodically if needed
    \item Delete unnecessary variables
    \item Use gradient checkpointing for very deep models
\end{itemize}

\textbf{Profiling:}
\begin{itemize}
    \item Use PyTorch Profiler for detailed analysis
    \item Identify bottlenecks (data loading, forward, backward)
    \item Focus optimization efforts on slowest parts
    \item Profile regularly during development
\end{itemize}

\textbf{Production deployment:}
\begin{itemize}
    \item Use TorchScript for inference
    \item Export to ONNX for cross-framework compatibility
    \item Always use \texttt{model.eval()} and \texttt{torch.no\_grad()}
    \item Batch inference when possible
    \item Consider FP16 inference
\end{itemize}

\textbf{Common performance issues:}
\begin{itemize}
    \item Data loading too slow → increase \texttt{num\_workers}
    \item GPU underutilized → increase batch size or model size
    \item Out of memory → reduce batch size, use gradient accumulation, or AMP
    \item Training too slow → use mixed precision, profile bottlenecks
\end{itemize}

\clearpage
% =============================================
% PART IV: INTEGRATIVE CHALLENGES
% =============================================

\part{Integrative Challenges}

% =============================================
% SECTION 16: GRAND CHALLENGES
% =============================================

\section{Grand Challenges}

\subsection{Introduction: Putting It All Together}

You've learned the fundamentals. Now it's time to combine everything into complete, real-world projects.

These challenges integrate multiple concepts:
\begin{itemize}
    \item Multiple architecture types (CNNs, RNNs, Transformers)
    \item Proper training pipelines (data loading, optimization, monitoring)
    \item Best practices (regularization, debugging, performance)
    \item Scientific computing applications
\end{itemize}

\textbf{How to approach these challenges:}
\begin{enumerate}
    \item Start simple (baseline model)
    \item Get it working (overfit small data)
    \item Make it good (full dataset + regularization)
    \item Make it great (hyperparameter tuning + tricks)
    \item Document everything (what worked, what didn't)
\end{enumerate}

\subsection{Challenge 1: Time Series Forecasting System}

\textbf{Difficulty:} $\bigstar\bigstar\bigstar$

\textbf{Goal:} Build a complete forecasting system for multivariate time series.

\textbf{Dataset:} Use real data (weather, stock prices, sensor readings) or generate complex synthetic data.

\textbf{Requirements:}

\textbf{1. Data Pipeline}
\begin{itemize}
    \item Load and preprocess time series data
    \item Create sliding windows (past 100 steps → predict next 10)
    \item Handle missing values
    \item Normalize using training statistics
    \item Split into train/val/test (temporal split, not random!)
\end{itemize}

\textbf{2. Model Architecture}

Implement and compare three approaches:
\begin{itemize}
    \item \textbf{LSTM-based:} Stacked LSTM with attention
    \item \textbf{CNN-based:} 1D CNN with residual connections
    \item \textbf{Transformer-based:} Encoder-only Transformer
\end{itemize}

\textbf{3. Training}
\begin{itemize}
    \item Proper learning rate schedule (warmup + decay)
    \item Early stopping on validation set
    \item Gradient clipping (essential for RNNs)
    \item Mixed precision training
    \item TensorBoard logging
\end{itemize}

\textbf{4. Evaluation}
\begin{itemize}
    \item Multiple metrics: MSE, MAE, MAPE
    \item One-step ahead vs multi-step ahead prediction
    \item Visualization: predicted vs actual
    \item Confidence intervals (optional)
\end{itemize}

\textbf{5. Analysis}
\begin{itemize}
    \item Which architecture works best?
    \item How far ahead can you predict accurately?
    \item What patterns does the model learn?
    \item Where does it fail?
\end{itemize}

\textbf{Starter template:}
\begin{lstlisting}
class TimeSeriesDataset(Dataset):
    def __init__(self, data, window_size, forecast_horizon):
        # Your implementation
        pass
    
    def __len__(self):
        return len(self.data) - self.window_size - self.forecast_horizon + 1
    
    def __getitem__(self, idx):
        # Return (input_window, target_window)
        pass

class LSTMForecaster(nn.Module):
    def __init__(self, input_size, hidden_size, num_layers, forecast_horizon):
        super().__init__()
        self.lstm = nn.LSTM(input_size, hidden_size, num_layers, 
                           batch_first=True)
        self.fc = nn.Linear(hidden_size, forecast_horizon)
    
    def forward(self, x):
        # LSTM encoding
        output, (h_n, c_n) = self.lstm(x)
        # Predict from final hidden state
        prediction = self.fc(h_n[-1])
        return prediction

# Your training loop with all best practices
\end{lstlisting}

\textbf{Bonus challenges:}
\begin{itemize}
    \item Add attention mechanism to LSTM
    \item Implement probabilistic forecasting (predict distribution, not point estimate)
    \item Handle multiple related time series (multivariate)
    \item Deploy as REST API
\end{itemize}

\clearpage
\subsection{Challenge 2: Hybrid Vision-Language Model}

\textbf{Difficulty:} $\bigstar\bigstar\bigstar\bigstar$

\textbf{Goal:} Build a model that combines visual and textual information.

\textbf{Task:} Image captioning or visual question answering.

\textbf{Requirements:}

\textbf{1. Architecture}

Implement encoder-decoder with:
\begin{itemize}
    \item \textbf{Image encoder:} ResNet or Vision Transformer
    \item \textbf{Text decoder:} LSTM or Transformer decoder
    \item \textbf{Cross-modal attention:} Decoder attends to image features
\end{itemize}

\begin{lstlisting}
class ImageCaptioningModel(nn.Module):
    def __init__(self, vocab_size, embed_dim=512, hidden_dim=512):
        super().__init__()
        
        # Image encoder (pretrained ResNet)
        resnet = models.resnet50(pretrained=True)
        # Remove final classification layer
        self.encoder = nn.Sequential(*list(resnet.children())[:-2])
        
        # Freeze encoder (transfer learning)
        for param in self.encoder.parameters():
            param.requires_grad = False
        
        # Project image features
        self.image_proj = nn.Linear(2048, embed_dim)
        
        # Text decoder
        self.embedding = nn.Embedding(vocab_size, embed_dim)
        self.lstm = nn.LSTM(embed_dim, hidden_dim, num_layers=2, 
                           batch_first=True)
        
        # Attention mechanism
        self.attention = nn.MultiheadAttention(embed_dim, num_heads=8)
        
        # Output projection
        self.fc_out = nn.Linear(hidden_dim, vocab_size)
    
    def forward(self, images, captions):
        # Encode images: (batch, 2048, 7, 7)
        img_features = self.encoder(images)
        
        # Reshape: (batch, 49, 2048)
        batch_size = img_features.size(0)
        img_features = img_features.view(batch_size, 2048, -1)
        img_features = img_features.transpose(1, 2)
        
        # Project: (batch, 49, embed_dim)
        img_features = self.image_proj(img_features)
        
        # Embed captions
        caption_embeds = self.embedding(captions)
        
        # LSTM decoding with attention
        lstm_out, _ = self.lstm(caption_embeds)
        
        # Attend to image features
        attn_out, _ = self.attention(lstm_out, img_features, img_features)
        
        # Combine and project
        combined = lstm_out + attn_out
        output = self.fc_out(combined)
        
        return output
\end{lstlisting}

\textbf{2. Data Preparation}
\begin{itemize}
    \item Use COCO dataset or similar
    \item Build vocabulary from captions
    \item Tokenize captions
    \item Implement proper data augmentation for images
    \item Handle variable-length captions (padding + masking)
\end{itemize}

\textbf{3. Training}
\begin{itemize}
    \item Teacher forcing during training
    \item Proper cross-entropy loss (ignore padding tokens)
    \item Learning rate warmup + decay
    \item Mixed precision training
    \item Gradient accumulation (if memory limited)
\end{itemize}

\textbf{4. Generation}

Implement multiple decoding strategies:
\begin{lstlisting}
def generate_caption(model, image, vocab, max_len=20, 
                    strategy='greedy', temperature=1.0, beam_width=5):
    """
    Generate caption from image.
    
    Args:
        strategy: 'greedy', 'sampling', or 'beam_search'
    """
    model.eval()
    
    # Encode image
    with torch.no_grad():
        img_features = model.encode_image(image)
    
    # Start token
    caption = [vocab['<start>']]
    
    if strategy == 'greedy':
        # Greedy decoding (take argmax at each step)
        for _ in range(max_len):
            logits = model.decode_step(img_features, caption)
            next_word = logits.argmax()
            caption.append(next_word.item())
            if next_word == vocab['<end>']:
                break
    
    elif strategy == 'sampling':
        # Sample from distribution
        for _ in range(max_len):
            logits = model.decode_step(img_features, caption)
            logits = logits / temperature
            probs = F.softmax(logits, dim=-1)
            next_word = torch.multinomial(probs, 1)
            caption.append(next_word.item())
            if next_word == vocab['<end>']:
                break
    
    elif strategy == 'beam_search':
        # Beam search (keep top-k candidates)
        # Your implementation
        pass
    
    return caption
\end{lstlisting}

\textbf{5. Evaluation}
\begin{itemize}
    \item BLEU score (text similarity metric)
    \item Qualitative: Visual inspection of generated captions
    \item Error analysis: Where does model fail?
    \item Attention visualization: What does model look at?
\end{itemize}

\textbf{Bonus challenges:}
\begin{itemize}
    \item Implement beam search decoding
    \item Add image-text contrastive learning (CLIP-style)
    \item Fine-tune encoder (not just freeze)
    \item Add copy mechanism for rare words
\end{itemize}

\clearpage
\subsection{Challenge 3: Denoising Scientific Data}

\textbf{Difficulty:} $\bigstar\bigstar\bigstar$

\textbf{Goal:} Build a U-Net style architecture to denoise 2D scientific data.

\textbf{Application:} Medical images, microscopy, sensor data, astronomical images.

\textbf{Requirements:}

\textbf{1. Data Generation}

Create noisy data:
\begin{lstlisting}
def generate_noisy_data(clean_images, noise_level=0.1):
    """Add Gaussian noise to clean images."""
    noise = torch.randn_like(clean_images) * noise_level
    noisy = clean_images + noise
    return noisy, clean_images

# Use real dataset (medical images, etc.) or synthetic
\end{lstlisting}

\textbf{2. U-Net Architecture}

Implement complete U-Net with:
\begin{itemize}
    \item Encoder: Downsampling path with residual blocks
    \item Decoder: Upsampling path with skip connections
    \item Attention gates (optional but recommended)
    \item Batch normalization throughout
\end{itemize}

\begin{lstlisting}
class UNet(nn.Module):
    """U-Net for image denoising/reconstruction."""
    
    def __init__(self, in_channels=1, out_channels=1):
        super().__init__()
        
        # Encoder
        self.enc1 = self.conv_block(in_channels, 64)
        self.enc2 = self.conv_block(64, 128)
        self.enc3 = self.conv_block(128, 256)
        self.enc4 = self.conv_block(256, 512)
        
        # Bottleneck
        self.bottleneck = self.conv_block(512, 1024)
        
        # Decoder with skip connections
        self.upconv4 = nn.ConvTranspose2d(1024, 512, 2, stride=2)
        self.dec4 = self.conv_block(1024, 512)  # 1024 = 512 + 512 from skip
        
        self.upconv3 = nn.ConvTranspose2d(512, 256, 2, stride=2)
        self.dec3 = self.conv_block(512, 256)
        
        self.upconv2 = nn.ConvTranspose2d(256, 128, 2, stride=2)
        self.dec2 = self.conv_block(256, 128)
        
        self.upconv1 = nn.ConvTranspose2d(128, 64, 2, stride=2)
        self.dec1 = self.conv_block(128, 64)
        
        # Output
        self.out = nn.Conv2d(64, out_channels, 1)
        
        self.pool = nn.MaxPool2d(2, 2)
    
    def conv_block(self, in_ch, out_ch):
        """Double convolution block."""
        return nn.Sequential(
            nn.Conv2d(in_ch, out_ch, 3, padding=1),
            nn.BatchNorm2d(out_ch),
            nn.ReLU(inplace=True),
            nn.Conv2d(out_ch, out_ch, 3, padding=1),
            nn.BatchNorm2d(out_ch),
            nn.ReLU(inplace=True)
        )
    
    def forward(self, x):
        # Encoder
        enc1 = self.enc1(x)
        enc2 = self.enc2(self.pool(enc1))
        enc3 = self.enc3(self.pool(enc2))
        enc4 = self.enc4(self.pool(enc3))
        
        # Bottleneck
        bottleneck = self.bottleneck(self.pool(enc4))
        
        # Decoder with skip connections
        dec4 = self.upconv4(bottleneck)
        dec4 = torch.cat([dec4, enc4], dim=1)
        dec4 = self.dec4(dec4)
        
        dec3 = self.upconv3(dec4)
        dec3 = torch.cat([dec3, enc3], dim=1)
        dec3 = self.dec3(dec3)
        
        dec2 = self.upconv2(dec3)
        dec2 = torch.cat([dec2, enc2], dim=1)
        dec2 = self.dec2(dec2)
        
        dec1 = self.upconv1(dec2)
        dec1 = torch.cat([dec1, enc1], dim=1)
        dec1 = self.dec1(dec1)
        
        return self.out(dec1)
\end{lstlisting}

\textbf{3. Training}
\begin{itemize}
    \item Loss function: MSE + Perceptual loss (optional)
    \item Data augmentation: Random flips, rotations
    \item Learning rate schedule
    \item Monitor both MSE and PSNR metrics
\end{itemize}

\textbf{4. Evaluation}
\begin{lstlisting}
def calculate_psnr(img1, img2):
    """Calculate Peak Signal-to-Noise Ratio."""
    mse = torch.mean((img1 - img2) ** 2)
    if mse == 0:
        return float('inf')
    max_pixel = 1.0
    psnr = 20 * torch.log10(max_pixel / torch.sqrt(mse))
    return psnr.item()

def calculate_ssim(img1, img2):
    """Structural Similarity Index (use skimage or pytorch)."""
    from skimage.metrics import structural_similarity
    return structural_similarity(
        img1.cpu().numpy(), img2.cpu().numpy(),
        data_range=1.0
    )

# Evaluate
model.eval()
psnrs = []
ssims = []

with torch.no_grad():
    for noisy, clean in test_loader:
        denoised = model(noisy)
        
        for i in range(len(noisy)):
            psnr = calculate_psnr(denoised[i], clean[i])
            ssim = calculate_ssim(denoised[i], clean[i])
            psnrs.append(psnr)
            ssims.append(ssim)

print(f"Average PSNR: {np.mean(psnrs):.2f} dB")
print(f"Average SSIM: {np.mean(ssims):.4f}")
\end{lstlisting}

\textbf{5. Visualization}
\begin{lstlisting}
def visualize_denoising(model, noisy_images, clean_images):
    """Visualize denoising results."""
    model.eval()
    
    with torch.no_grad():
        denoised = model(noisy_images)
    
    fig, axes = plt.subplots(len(noisy_images), 3, figsize=(12, 4*len(noisy_images)))
    
    for i in range(len(noisy_images)):
        # Noisy
        axes[i, 0].imshow(noisy_images[i].squeeze().cpu(), cmap='gray')
        axes[i, 0].set_title('Noisy')
        axes[i, 0].axis('off')
        
        # Denoised
        axes[i, 1].imshow(denoised[i].squeeze().cpu(), cmap='gray')
        axes[i, 1].set_title('Denoised')
        axes[i, 1].axis('off')
        
        # Clean
        axes[i, 2].imshow(clean_images[i].squeeze().cpu(), cmap='gray')
        axes[i, 2].set_title('Clean (Target)')
        axes[i, 2].axis('off')
    
    plt.tight_layout()
    plt.show()
\end{lstlisting}

\textbf{Bonus challenges:}
\begin{itemize}
    \item Add attention gates between encoder and decoder
    \item Implement progressive training (start with low resolution)
    \item Handle different noise types (Gaussian, salt-and-pepper, Poisson)
    \item Add perceptual loss using pretrained VGG
\end{itemize}

\clearpage
\subsection{Challenge 4: Custom Transformer for Sequence Tasks}

\textbf{Difficulty:} $\bigstar\bigstar\bigstar\bigstar$

\textbf{Goal:} Build a complete Transformer from scratch for sequence classification or generation.

\textbf{Task:} Text classification, sequence generation, or time series classification.

\textbf{Requirements:}

\textbf{1. Complete Transformer Implementation}

Build all components from scratch (no \texttt{nn.Transformer}):
\begin{itemize}
    \item Multi-head attention
    \item Position-wise feed-forward
    \item Positional encoding
    \item Layer normalization
    \item Encoder and/or decoder blocks
\end{itemize}

\textbf{2. Training Infrastructure}
\begin{itemize}
    \item Warmup learning rate schedule
    \item Gradient clipping
    \item Label smoothing
    \item Mixed precision training
    \item Checkpointing and recovery
\end{itemize}

\textbf{3. Advanced Features}

Implement at least two of:
\begin{itemize}
    \item Learning rate warmup + cosine annealing
    \item Custom learning rate schedule
    \item Gradient accumulation
    \item Model ensemble
    \item Knowledge distillation
\end{itemize}

\textbf{4. Comprehensive Evaluation}
\begin{itemize}
    \item Validation metrics
    \item Attention visualization
    \item Error analysis
    \item Compare with baseline (LSTM, simple MLP)
    \item Ablation studies (remove components, measure impact)
\end{itemize}

\textbf{5. Analysis}

Answer these questions:
\begin{itemize}
    \item How many heads are optimal?
    \item What does each attention head learn?
    \item How deep should the model be?
    \item What's the effect of positional encoding?
    \item Where does the model fail?
\end{itemize}

\clearpage
\subsection{Challenge 5: End-to-End ML Pipeline}

\textbf{Difficulty:} $\bigstar\bigstar\bigstar\bigstar$

\textbf{Goal:} Build a production-ready machine learning system.

\textbf{Task:} Choose your own (classification, regression, generation).

\textbf{Requirements:}

\textbf{1. Complete Pipeline}

\begin{verbatim}
project/
├── data/
│   ├── raw/              # Raw data
│   ├── processed/        # Preprocessed data
│   └── preprocessing.py  # Data processing scripts
├── models/
│   ├── architectures.py  # Model definitions
│   └── pretrained/       # Saved models
├── training/
│   ├── train.py          # Training script
│   ├── evaluate.py       # Evaluation script
│   └── utils.py          # Training utilities
├── configs/
│   ├── base_config.yaml  # Base configuration
│   └── experiment_*.yaml # Experiment configs
├── notebooks/
│   ├── exploration.ipynb # Data exploration
│   └── analysis.ipynb    # Results analysis
├── tests/
│   └── test_*.py         # Unit tests
├── requirements.txt
├── README.md
└── main.py               # Entry point
\end{verbatim}

\textbf{2. Code Quality}
\begin{itemize}
    \item Type hints
    \item Docstrings for all functions
    \item Unit tests for critical functions
    \item Configuration via YAML
    \item Logging (not just print statements)
    \item Error handling
\end{itemize}

\textbf{3. Experiment Tracking}
\begin{itemize}
    \item TensorBoard or Weights \& Biases
    \item Log hyperparameters
    \item Log metrics over time
    \item Save model checkpoints
    \item Version control (Git)
\end{itemize}

\textbf{4. Reproducibility}
\begin{itemize}
    \item Set random seeds
    \item Save exact hyperparameters
    \item Save training logs
    \item Document environment (requirements.txt)
    \item Save data preprocessing steps
\end{itemize}

\textbf{5. Deployment}
\begin{itemize}
    \item Export model (TorchScript or ONNX)
    \item Create inference script
    \item Docker container (optional)
    \item REST API using FastAPI (optional)
    \item Documentation for usage
\end{itemize}

\textbf{Evaluation criteria:}
\begin{itemize}
    \item Model performance
    \item Code quality and organization
    \item Reproducibility
    \item Documentation
    \item Completeness of pipeline
\end{itemize}

\subsection{Evaluation and Next Steps}

\textbf{How to know you've succeeded:}

\begin{enumerate}
    \item \textbf{It works:} Model trains without errors
    \item \textbf{It learns:} Loss decreases, metrics improve
    \item \textbf{It generalizes:} Validation performance is good
    \item \textbf{You understand it:} Can explain why it works
    \item \textbf{It's reproducible:} Others can replicate your results
\end{enumerate}

\textbf{Beyond this guide:}

\textbf{Keep learning:}
\begin{itemize}
    \item Read recent papers (arXiv, conferences)
    \item Implement papers from scratch
    \item Contribute to open source projects
    \item Join competitions (Kaggle, etc.)
    \item Build projects in your domain
\end{itemize}

\textbf{Advanced topics to explore:}
\begin{itemize}
    \item Self-supervised learning
    \item Meta-learning
    \item Neural architecture search
    \item Efficient networks (quantization, pruning)
    \item Diffusion models
    \item Large language models
    \item Multimodal learning
    \item Reinforcement learning
\end{itemize}

\textbf{Stay updated:}
\begin{itemize}
    \item Follow PyTorch blog and releases
    \item Join ML communities (Reddit, Discord, forums)
    \item Attend conferences (NeurIPS, ICML, ICLR)
    \item Read technical blogs
    \item Experiment with new techniques
\end{itemize}

\subsection{Final Thoughts}

You now have a comprehensive toolkit for deep learning with PyTorch. You've learned:

\begin{itemize}
    \item \textbf{Foundations:} Tensors, autograd, modules, training loops, data loading
    \item \textbf{Architectures:} MLPs, CNNs, ResNets, RNNs, Transformers
    \item \textbf{Techniques:} Normalization, regularization, optimization
    \item \textbf{Practical skills:} Debugging, training best practices, performance optimization
\end{itemize}

\textbf{Remember:}
\begin{itemize}
    \item Start simple, add complexity gradually
    \item Always check if your model can overfit small data
    \item Visualization and monitoring are essential
    \item Good engineering matters as much as good models
    \item Deep learning is iterative—experiment and learn
\end{itemize}

The best way to master these concepts is to \textbf{build things}. Pick a challenge that excites you and start coding!

Good luck, and happy building!

\clearpage
\end{lstlisting}
\end{document}

% % =============================================
% SECTION 5: DATA LOADING & PREPROCESSING (COMPACT)
% =============================================

\section{Data Loading \& Preprocessing}

\subsection{Dataset and DataLoader}

PyTorch uses \texttt{Dataset} (defines how to access samples) and \texttt{DataLoader} (handles batching/shuffling).

\textbf{Custom Dataset:}
\begin{lstlisting}
from torch.utils.data import Dataset, DataLoader

class SimpleDataset(Dataset):
    def __init__(self, X, y):
        self.X = torch.FloatTensor(X)
        self.y = torch.FloatTensor(y)
    
    def __len__(self):
        return len(self.X)
    
    def __getitem__(self, idx):
        return self.X[idx], self.y[idx]

# Usage
dataset = SimpleDataset(X, y)
loader = DataLoader(dataset, batch_size=32, shuffle=True, num_workers=2)

for batch_X, batch_y in loader:
    # Training code
    pass
\end{lstlisting}

\textbf{Key DataLoader parameters:}
\begin{itemize}
    \item \texttt{batch\_size}: 32-256 typically
    \item \texttt{shuffle}: True for training, False for eval
    \item \texttt{num\_workers}: 2-4 for parallel loading
    \item \texttt{pin\_memory}: True if using GPU
\end{itemize}

\subsection{Common Dataset Patterns}

\textbf{Time Series Dataset:}
\begin{lstlisting}
class TimeSeriesDataset(Dataset):
    def __init__(self, data, window_size, forecast_horizon):
        self.data = torch.FloatTensor(data)
        self.window = window_size
        self.horizon = forecast_horizon
    
    def __len__(self):
        return len(self.data) - self.window - self.horizon + 1
    
    def __getitem__(self, idx):
        x = self.data[idx:idx+self.window]
        y = self.data[idx+self.window:idx+self.window+self.horizon]
        return x, y
\end{lstlisting}

\textbf{Loading from files:}
\begin{lstlisting}
class FileDataset(Dataset):
    def __init__(self, file_list, transform=None):
        self.files = file_list
        self.transform = transform
    
    def __len__(self):
        return len(self.files)
    
    def __getitem__(self, idx):
        data = np.load(self.files[idx])  # Load on-demand
        if self.transform:
            data = self.transform(data)
        return torch.FloatTensor(data)
\end{lstlisting}

\subsection{Normalization}

\begin{warningbox}[Always Use Training Statistics]
Compute mean/std from training data only, then apply to val/test.
\end{warningbox}

\begin{lstlisting}
class StandardScaler:
    def __init__(self, data):
        self.mean = data.mean(dim=0)
        self.std = data.std(dim=0)
    
    def transform(self, data):
        return (data - self.mean) / (self.std + 1e-8)

# Usage
scaler = StandardScaler(X_train)
X_train_norm = scaler.transform(X_train)
X_test_norm = scaler.transform(X_test)  # Use training stats!
\end{lstlisting}

\subsection{Train/Val/Test Split}

\begin{lstlisting}
from torch.utils.data import random_split

train_size = int(0.7 * len(dataset))
val_size = int(0.15 * len(dataset))
test_size = len(dataset) - train_size - val_size

train_ds, val_ds, test_ds = random_split(dataset, 
                                          [train_size, val_size, test_size])
\end{lstlisting}

\subsection{Exercises}

\begin{exercise}[5.1: Custom Dataset - $\bigstar\bigstar$]
Create a dataset for synthetic regression data, implement \texttt{\_\_len\_\_} and \texttt{\_\_getitem\_\_}, create a DataLoader, and iterate through batches.
\end{exercise}

\begin{exercise}[5.2: Time Series Dataset - $\bigstar\bigstar\bigstar$]
Implement a windowed time series dataset for predicting future values from past observations.
\end{exercise}

\begin{exercise}[5.3: Normalization Impact - $\bigstar\bigstar\bigstar$]
Train a model with and without normalization. Compare convergence speed and final performance.
\end{exercise}

\clearpage

% =============================================
% SECTION 6: FULLY CONNECTED NETWORKS (COMPACT)
% =============================================

\section{Fully Connected Networks (MLPs)}

\subsection{Core Concepts}

MLPs are universal function approximators. Each layer: $\mathbf{h} = \sigma(\mathbf{W}\mathbf{x} + \mathbf{b})$

\textbf{When to use:} Tabular data, function approximation, small input dimensions

\textbf{Architecture choices:}
\begin{itemize}
    \item \textbf{Depth vs Width:} Start with 2-3 layers; deeper = more parameter-efficient but harder to train
    \item \textbf{Layer sizes:} First hidden layer 1-2× input size, then decrease
\end{itemize}

\subsection{Activation Functions}

\begin{lstlisting}
# ReLU (default choice)
nn.ReLU()

# Leaky ReLU (prevents dead neurons)
nn.LeakyReLU(0.01)

# GELU (for Transformers)
nn.GELU()

# Tanh (for RNNs, zero-centered)
nn.Tanh()
\end{lstlisting}

\textbf{Rule:} Use ReLU unless you have a specific reason not to. Never use Sigmoid in hidden layers (vanishing gradients).

\subsection{Implementation}

\begin{lstlisting}
class MLP(nn.Module):
    def __init__(self, input_dim, hidden_dims, output_dim):
        super().__init__()
        layers = []
        prev_dim = input_dim
        
        for hidden_dim in hidden_dims:
            layers.append(nn.Linear(prev_dim, hidden_dim))
            layers.append(nn.ReLU())
            prev_dim = hidden_dim
        
        layers.append(nn.Linear(prev_dim, output_dim))
        self.network = nn.Sequential(*layers)
    
    def forward(self, x):
        return self.network(x)

# Usage
model = MLP(10, [64, 32], 1)
\end{lstlisting}

\subsection{Initialization}

\begin{lstlisting}
import torch.nn.init as init

def init_weights(m):
    if isinstance(m, nn.Linear):
        init.kaiming_normal_(m.weight, nonlinearity='relu')  # He init for ReLU
        if m.bias is not None:
            init.constant_(m.bias, 0)

model.apply(init_weights)
\end{lstlisting}

\textbf{Rules:}
\begin{itemize}
    \item ReLU networks: He/Kaiming initialization
    \item Tanh/Sigmoid: Xavier/Glorot initialization
\end{itemize}

\subsection{Batch Normalization}

\begin{lstlisting}
class MLPWithBatchNorm(nn.Module):
    def __init__(self, input_dim, hidden_dims, output_dim):
        super().__init__()
        layers = []
        prev_dim = input_dim
        
        for hidden_dim in hidden_dims:
            layers.append(nn.Linear(prev_dim, hidden_dim))
            layers.append(nn.BatchNorm1d(hidden_dim))  # After Linear
            layers.append(nn.ReLU())
            prev_dim = hidden_dim
        
        layers.append(nn.Linear(prev_dim, output_dim))
        self.network = nn.Sequential(*layers)
    
    def forward(self, x):
        return self.network(x)
\end{lstlisting}

\textbf{BatchNorm benefits:} Stabilizes training, allows higher learning rates, acts as regularization

\textbf{Important:} Use \texttt{model.train()} and \texttt{model.eval()} properly!

\subsection{Dropout}

\begin{lstlisting}
layers.append(nn.Dropout(0.5))  # After activation
\end{lstlisting}

Typical values: 0.2-0.5. Higher = more regularization.

\subsection{Common Issues}

\begin{itemize}
    \item \textbf{Dead ReLUs:} Use Leaky ReLU or proper initialization
    \item \textbf{Vanishing gradients:} Use ReLU, batch norm, or skip connections
    \item \textbf{Overfitting:} Add dropout, weight decay, or more data
\end{itemize}

\subsection{Exercises}

\begin{exercise}[6.1: Build and Train MLP - $\bigstar\bigstar$]
Create a 3-layer MLP, train on synthetic data ($y = x^2$), plot loss curve.
\end{exercise}

\begin{exercise}[6.2: Activation Comparison - $\bigstar\bigstar\bigstar$]
Train identical MLPs with ReLU, Leaky ReLU, Tanh, and Sigmoid. Compare convergence.
\end{exercise}

\begin{exercise}[6.3: Regularization - $\bigstar\bigstar\bigstar\bigstar$]
Small dataset (200 samples), large model. Add dropout, batch norm, weight decay. Compare overfitting.
\end{exercise}

\clearpage

% =============================================
% SECTION 7: CONVOLUTIONAL NEURAL NETWORKS
% =============================================

\section{Convolutional Neural Networks}

\subsection{Why Convolutions?}

\textbf{Three key properties:}
\begin{enumerate}
    \item \textbf{Local connectivity:} Each neuron connects to small region
    \item \textbf{Parameter sharing:} Same filter across entire input
    \item \textbf{Translation invariance:} Detect features anywhere
\end{enumerate}

\textbf{Use CNNs for:} Images, spatial data, time series (1D conv)

\subsection{Convolution Operation}

\textbf{2D Convolution:}
\[
y[i,j] = \sum_{m}\sum_{n} x[i+m, j+n] \cdot w[m,n] + b
\]

\textbf{Key parameters:}
\begin{itemize}
    \item \textbf{kernel\_size}: Size of filter (3×3, 5×5, etc.)
    \item \textbf{stride}: Step size (1 = slide by 1 pixel)
    \item \textbf{padding}: Add zeros around border
    \item \textbf{dilation}: Spacing between kernel elements
\end{itemize}

\subsection{Output Size Calculation}

\[
\text{output\_size} = \left\lfloor \frac{\text{input\_size} + 2 \times \text{padding} - \text{kernel\_size}}{\text{stride}} \right\rfloor + 1
\]

\textbf{Example:} Input 28×28, kernel 3×3, stride 1, padding 1 → Output 28×28

\begin{pytorchtip}[Same Padding]
To keep spatial size: \texttt{padding = (kernel\_size - 1) // 2}
\end{pytorchtip}

\subsection{Implementation}

\textbf{Basic CNN:}
\begin{lstlisting}
class SimpleCNN(nn.Module):
    def __init__(self):
        super().__init__()
        self.conv1 = nn.Conv2d(1, 32, kernel_size=3, padding=1)  # 28x28 -> 28x28
        self.conv2 = nn.Conv2d(32, 64, kernel_size=3, padding=1) # 28x28 -> 28x28
        self.pool = nn.MaxPool2d(2, 2)  # 28x28 -> 14x14
        self.fc1 = nn.Linear(64 * 7 * 7, 128)
        self.fc2 = nn.Linear(128, 10)
    
    def forward(self, x):
        x = self.pool(torch.relu(self.conv1(x)))  # 28x28 -> 14x14
        x = self.pool(torch.relu(self.conv2(x)))  # 14x14 -> 7x7
        x = x.view(x.size(0), -1)  # Flatten
        x = torch.relu(self.fc1(x))
        x = self.fc2(x)
        return x
\end{lstlisting}

\textbf{Conv2d parameters:}
\begin{lstlisting}
nn.Conv2d(in_channels, out_channels, kernel_size, stride=1, padding=0)
\end{lstlisting}

\subsection{Pooling}

\textbf{Max Pooling} (most common):
\begin{lstlisting}
nn.MaxPool2d(kernel_size=2, stride=2)  # Downsample by 2
\end{lstlisting}

\textbf{Average Pooling:}
\begin{lstlisting}
nn.AvgPool2d(kernel_size=2, stride=2)
\end{lstlisting}

\textbf{Why pool?}
\begin{itemize}
    \item Reduce spatial dimensions (fewer parameters)
    \item Increase receptive field
    \item Translation invariance
\end{itemize}

\subsection{Receptive Field}

The receptive field is the region of input that affects a neuron.

\textbf{Growth:} Each layer increases receptive field
\begin{itemize}
    \item Layer 1 (3×3 kernel): 3×3 receptive field
    \item Layer 2 (3×3 kernel): 5×5 receptive field
    \item With pooling: grows even faster
\end{itemize}

\subsection{1D and 3D Convolutions}

\textbf{1D Conv for sequences:}
\begin{lstlisting}
nn.Conv1d(in_channels, out_channels, kernel_size)

# Example: time series with 10 features, window size 5
conv1d = nn.Conv1d(10, 32, kernel_size=5)
x = torch.randn(batch, 10, 100)  # (batch, channels, length)
out = conv1d(x)  # (batch, 32, 96)
\end{lstlisting}

\textbf{3D Conv for volumes:}
\begin{lstlisting}
nn.Conv3d(in_channels, out_channels, kernel_size)

# Example: 3D medical image
conv3d = nn.Conv3d(1, 32, kernel_size=3)
x = torch.randn(batch, 1, 64, 64, 64)  # (batch, channels, D, H, W)
out = conv3d(x)
\end{lstlisting}

\subsection{Transposed Convolutions}

For upsampling (e.g., in autoencoders):

\begin{lstlisting}
nn.ConvTranspose2d(in_channels, out_channels, kernel_size, stride=2)

# Example: 7x7 -> 14x14
upconv = nn.ConvTranspose2d(64, 32, kernel_size=3, stride=2, padding=1)
\end{lstlisting}

\subsection{Complete CNN Example}

\begin{lstlisting}
class VGGStyleCNN(nn.Module):
    def __init__(self, num_classes=10):
        super().__init__()
        
        self.features = nn.Sequential(
            # Block 1
            nn.Conv2d(3, 64, 3, padding=1),
            nn.ReLU(),
            nn.Conv2d(64, 64, 3, padding=1),
            nn.ReLU(),
            nn.MaxPool2d(2, 2),
            
            # Block 2
            nn.Conv2d(64, 128, 3, padding=1),
            nn.ReLU(),
            nn.Conv2d(128, 128, 3, padding=1),
            nn.ReLU(),
            nn.MaxPool2d(2, 2),
        )
        
        self.classifier = nn.Sequential(
            nn.Linear(128 * 8 * 8, 256),
            nn.ReLU(),
            nn.Dropout(0.5),
            nn.Linear(256, num_classes)
        )
    
    def forward(self, x):
        x = self.features(x)
        x = x.view(x.size(0), -1)
        x = self.classifier(x)
        return x
\end{lstlisting}

\subsection{Exercises}

\begin{exercise}[7.1: Simple CNN - $\bigstar\bigstar$]
Build and train a CNN on MNIST: 2 conv layers + 2 FC layers. Achieve >95\% accuracy.
\end{exercise}

\begin{exercise}[7.2: Output Shape Calculation - $\bigstar\bigstar$]
Given input 32×32, calculate output shapes through: Conv(3×3, stride=1, padding=1) → MaxPool(2×2) → Conv(5×5, stride=2, padding=2).
\end{exercise}

\begin{exercise}[7.3: 1D CNN for Time Series - $\bigstar\bigstar\bigstar$]
Use 1D convolutions to smooth/denoise a noisy sine wave. Compare with moving average.
\end{exercise}

\begin{exercise}[7.4: Build U-Net - $\bigstar\bigstar\bigstar\bigstar$]
Implement a simple U-Net with encoder (downsampling) and decoder (upsampling) using ConvTranspose2d.
\end{exercise}

\clearpage

% =============================================
% SECTION 8: RESIDUAL NETWORKS & SKIP CONNECTIONS
% =============================================

\section{Residual Networks \& Skip Connections}

\subsection{The Degradation Problem}

Deeper networks should perform at least as well as shallow ones (can learn identity mapping). In practice, they perform \textbf{worse} due to optimization difficulties.

\textbf{ResNets solve this} with skip connections.

\subsection{Residual Blocks}

\textbf{Key idea:} Learn residual $F(x) = H(x) - x$ instead of $H(x)$ directly.

\[
y = F(x) + x
\]

If optimal mapping is identity, it's easier to learn $F(x) = 0$ than $H(x) = x$.

\begin{lstlisting}
class ResidualBlock(nn.Module):
    def __init__(self, channels):
        super().__init__()
        self.conv1 = nn.Conv2d(channels, channels, 3, padding=1)
        self.bn1 = nn.BatchNorm2d(channels)
        self.conv2 = nn.Conv2d(channels, channels, 3, padding=1)
        self.bn2 = nn.BatchNorm2d(channels)
    
    def forward(self, x):
        residual = x  # Save input
        
        out = torch.relu(self.bn1(self.conv1(x)))
        out = self.bn2(self.conv2(out))
        
        out += residual  # Add skip connection
        out = torch.relu(out)
        return out
\end{lstlisting}

\subsection{Projection Shortcuts}

When dimensions change:

\begin{lstlisting}
class ResidualBlockProjection(nn.Module):
    def __init__(self, in_channels, out_channels, stride=1):
        super().__init__()
        self.conv1 = nn.Conv2d(in_channels, out_channels, 3, 
                              stride=stride, padding=1)
        self.bn1 = nn.BatchNorm2d(out_channels)
        self.conv2 = nn.Conv2d(out_channels, out_channels, 3, padding=1)
        self.bn2 = nn.BatchNorm2d(out_channels)
        
        # Projection shortcut if dimensions change
        self.shortcut = nn.Sequential()
        if stride != 1 or in_channels != out_channels:
            self.shortcut = nn.Sequential(
                nn.Conv2d(in_channels, out_channels, 1, stride=stride),
                nn.BatchNorm2d(out_channels)
            )
    
    def forward(self, x):
        residual = self.shortcut(x)
        
        out = torch.relu(self.bn1(self.conv1(x)))
        out = self.bn2(self.conv2(out))
        
        out += residual
        out = torch.relu(out)
        return out
\end{lstlisting}

\subsection{Building a ResNet}

\begin{lstlisting}
class SimpleResNet(nn.Module):
    def __init__(self, num_classes=10):
        super().__init__()
        
        self.conv1 = nn.Conv2d(3, 64, 7, stride=2, padding=3)
        self.bn1 = nn.BatchNorm2d(64)
        self.pool = nn.MaxPool2d(3, stride=2, padding=1)
        
        # Residual blocks
        self.layer1 = self._make_layer(64, 64, 2)
        self.layer2 = self._make_layer(64, 128, 2, stride=2)
        self.layer3 = self._make_layer(128, 256, 2, stride=2)
        
        self.avgpool = nn.AdaptiveAvgPool2d((1, 1))
        self.fc = nn.Linear(256, num_classes)
    
    def _make_layer(self, in_channels, out_channels, num_blocks, stride=1):
        layers = []
        layers.append(ResidualBlockProjection(in_channels, out_channels, stride))
        for _ in range(1, num_blocks):
            layers.append(ResidualBlock(out_channels))
        return nn.Sequential(*layers)
    
    def forward(self, x):
        x = self.pool(torch.relu(self.bn1(self.conv1(x))))
        x = self.layer1(x)
        x = self.layer2(x)
        x = self.layer3(x)
        x = self.avgpool(x)
        x = x.view(x.size(0), -1)
        x = self.fc(x)
        return x
\end{lstlisting}

\subsection{Why Skip Connections Work}

\begin{itemize}
    \item \textbf{Gradient flow:} Gradients flow directly through shortcuts
    \item \textbf{Easier optimization:} Easier to learn small adjustments than full transformation
    \item \textbf{Ensemble effect:} Network becomes ensemble of shallower networks
\end{itemize}

\subsection{Exercises}

\begin{exercise}[8.1: Residual Block - $\bigstar\bigstar$]
Implement a basic residual block and verify the skip connection works.
\end{exercise}

\begin{exercise}[8.2: Deep Network Comparison - $\bigstar\bigstar\bigstar$]
Train a 20-layer network with and without skip connections. Compare convergence and final accuracy.
\end{exercise}

\begin{exercise}[8.3: Build Mini-ResNet - $\bigstar\bigstar\bigstar\bigstar$]
Implement a small ResNet with 3 stages. Train on CIFAR-10.
\end{exercise}

\clearpage

% =============================================
% SECTION 9: BATCH & LAYER NORMALIZATION
% =============================================

\section{Normalization Techniques}

\subsection{Batch Normalization}

\textbf{Problem:} Internal covariate shift (distribution of activations changes during training)

\textbf{Solution:} Normalize within each batch

\[
\hat{x} = \frac{x - \mu_B}{\sqrt{\sigma_B^2 + \epsilon}}, \quad y = \gamma \hat{x} + \beta
\]

\begin{lstlisting}
nn.BatchNorm1d(num_features)  # For fully connected
nn.BatchNorm2d(num_features)  # For convolutions
\end{lstlisting}

\textbf{Training vs Eval:}
\begin{lstlisting}
model.train()  # Uses batch statistics
model.eval()   # Uses running mean/std
\end{lstlisting}

\subsection{Layer Normalization}

Normalize across features instead of batch:

\begin{lstlisting}
nn.LayerNorm(normalized_shape)
\end{lstlisting}

\textbf{Use LayerNorm for:}
\begin{itemize}
    \item RNNs and Transformers
    \item Small batch sizes (< 8)
    \item Batch size varies
\end{itemize}

\subsection{Comparison}

\begin{table}[h]
\centering
\begin{tabular}{lll}
\toprule
\textbf{Property} & \textbf{BatchNorm} & \textbf{LayerNorm} \\
\midrule
Normalizes over & Batch & Features \\
Batch size sensitive & Yes & No \\
Train/eval difference & Yes & No \\
Common in & CNNs, MLPs & Transformers, RNNs \\
\bottomrule
\end{tabular}
\end{table}

\subsection{Instance Normalization}

For style transfer and GANs:

\begin{lstlisting}
nn.InstanceNorm2d(num_features)  # Normalize each sample independently
\end{lstlisting}

\subsection{Group Normalization}

Middle ground between Layer and Instance:

\begin{lstlisting}
nn.GroupNorm(num_groups, num_channels)
\end{lstlisting}

\subsection{Exercises}

\begin{exercise}[9.1: BatchNorm Impact - $\bigstar\bigstar$]
Train a deep MLP with and without BatchNorm. Compare training stability and speed.
\end{exercise}

\begin{exercise}[9.2: Small Batch Issue - $\bigstar\bigstar\bigstar$]
Train with batch\_size=2. Compare BatchNorm vs LayerNorm performance.
\end{exercise}

\clearpage

% =============================================
% SECTION 10: ATTENTION & TRANSFORMERS
% =============================================

\section{Attention Mechanisms \& Transformers}

\subsection{The Attention Mechanism}

\textbf{Core idea:} Focus on relevant parts of input when producing output.

\textbf{Scaled Dot-Product Attention:}
\[
\text{Attention}(Q, K, V) = \text{softmax}\left(\frac{QK^T}{\sqrt{d_k}}\right)V
\]

\begin{itemize}
    \item $Q$ (Query): What we're looking for
    \item $K$ (Key): What each position contains
    \item $V$ (Value): Actual information
    \item $d_k$: Key dimension (scaling factor)
\end{itemize}

\subsection{Implementation}

\begin{lstlisting}
def scaled_dot_product_attention(Q, K, V, mask=None):
    """
    Q, K, V: (batch, seq_len, d_k)
    Returns: (batch, seq_len, d_k)
    """
    d_k = Q.size(-1)
    scores = torch.matmul(Q, K.transpose(-2, -1)) / torch.sqrt(torch.tensor(d_k, dtype=torch.float32))
    
    if mask is not None:
        scores = scores.masked_fill(mask == 0, -1e9)
    
    attention_weights = torch.softmax(scores, dim=-1)
    output = torch.matmul(attention_weights, V)
    return output, attention_weights
\end{lstlisting}

\subsection{Multi-Head Attention}

Run attention in parallel with different learned projections:

\begin{lstlisting}
class MultiHeadAttention(nn.Module):
    def __init__(self, d_model, num_heads):
        super().__init__()
        assert d_model % num_heads == 0
        
        self.d_model = d_model
        self.num_heads = num_heads
        self.d_k = d_model // num_heads
        
        self.W_q = nn.Linear(d_model, d_model)
        self.W_k = nn.Linear(d_model, d_model)
        self.W_v = nn.Linear(d_model, d_model)
        self.W_o = nn.Linear(d_model, d_model)
    
    def forward(self, Q, K, V, mask=None):
        batch_size = Q.size(0)
        
        # Linear projections and split into heads
        Q = self.W_q(Q).view(batch_size, -1, self.num_heads, self.d_k).transpose(1, 2)
        K = self.W_k(K).view(batch_size, -1, self.num_heads, self.d_k).transpose(1, 2)
        V = self.W_v(V).view(batch_size, -1, self.num_heads, self.d_k).transpose(1, 2)
        
        # Scaled dot-product attention
        d_k = Q.size(-1)
        scores = torch.matmul(Q, K.transpose(-2, -1)) / torch.sqrt(torch.tensor(d_k, dtype=torch.float32))
        
        if mask is not None:
            scores = scores.masked_fill(mask == 0, -1e9)
        
        attention = torch.softmax(scores, dim=-1)
        output = torch.matmul(attention, V)
        
        # Concatenate heads and project
        output = output.transpose(1, 2).contiguous().view(batch_size, -1, self.d_model)
        output = self.W_o(output)
        
        return output
\end{lstlisting}

\subsection{Positional Encoding}

Transformers have no notion of position. Add positional information:

\begin{lstlisting}
class PositionalEncoding(nn.Module):
    def __init__(self, d_model, max_len=5000):
        super().__init__()
        
        pe = torch.zeros(max_len, d_model)
        position = torch.arange(0, max_len).unsqueeze(1)
        div_term = torch.exp(torch.arange(0, d_model, 2) * 
                            -(torch.log(torch.tensor(10000.0)) / d_model))
        
        pe[:, 0::2] = torch.sin(position * div_term)
        pe[:, 1::2] = torch.cos(position * div_term)
        
        self.register_buffer('pe', pe.unsqueeze(0))
    
    def forward(self, x):
        return x + self.pe[:, :x.size(1)]
\end{lstlisting}

\subsection{Transformer Block}

\begin{lstlisting}
class TransformerBlock(nn.Module):
    def __init__(self, d_model, num_heads, d_ff, dropout=0.1):
        super().__init__()
        
        self.attention = MultiHeadAttention(d_model, num_heads)
        self.norm1 = nn.LayerNorm(d_model)
        self.norm2 = nn.LayerNorm(d_model)
        
        self.ffn = nn.Sequential(
            nn.Linear(d_model, d_ff),
            nn.ReLU(),
            nn.Dropout(dropout),
            nn.Linear(d_ff, d_model)
        )
        
        self.dropout = nn.Dropout(dropout)
    
    def forward(self, x, mask=None):
        # Multi-head attention with residual
        attn_output = self.attention(x, x, x, mask)
        x = self.norm1(x + self.dropout(attn_output))
        
        # Feed-forward with residual
        ffn_output = self.ffn(x)
        x = self.norm2(x + self.dropout(ffn_output))
        
        return x
\end{lstlisting}

\subsection{Complete Transformer Encoder}

\begin{lstlisting}
class TransformerEncoder(nn.Module):
    def __init__(self, vocab_size, d_model, num_heads, num_layers, d_ff, max_len=5000):
        super().__init__()
        
        self.embedding = nn.Embedding(vocab_size, d_model)
        self.pos_encoding = PositionalEncoding(d_model, max_len)
        
        self.layers = nn.ModuleList([
            TransformerBlock(d_model, num_heads, d_ff)
            for _ in range(num_layers)
        ])
        
        self.dropout = nn.Dropout(0.1)
    
    def forward(self, x, mask=None):
        x = self.embedding(x)
        x = self.pos_encoding(x)
        x = self.dropout(x)
        
        for layer in self.layers:
            x = layer(x, mask)
        
        return x
\end{lstlisting}

\subsection{Causal Masking}

For autoregressive models (GPT-style):

\begin{lstlisting}
def create_causal_mask(seq_len):
    """Prevent attention to future positions."""
    mask = torch.triu(torch.ones(seq_len, seq_len), diagonal=1)
    return mask == 0  # True where attention is allowed
\end{lstlisting}

\subsection{Using PyTorch's Built-in Transformer}

\begin{lstlisting}
# PyTorch provides nn.Transformer
model = nn.Transformer(
    d_model=512,
    nhead=8,
    num_encoder_layers=6,
    num_decoder_layers=6,
    dim_feedforward=2048,
    dropout=0.1
)

# For encoder-only (BERT-style)
encoder_layer = nn.TransformerEncoderLayer(d_model=512, nhead=8)
encoder = nn.TransformerEncoder(encoder_layer, num_layers=6)
\end{lstlisting}

\subsection{Exercises}

\begin{exercise}[10.1: Scaled Dot-Product Attention - $\bigstar\bigstar$]
Implement attention from scratch. Test on random Q, K, V matrices.
\end{exercise}

\begin{exercise}[10.2: Positional Encoding - $\bigstar\bigstar\bigstar$]
Implement and visualize positional encodings. Plot the encoding for different positions and dimensions.
\end{exercise}

\begin{exercise}[10.3: Transformer for Sequence Classification - $\bigstar\bigstar\bigstar\bigstar$]
Build a transformer encoder for sequence classification. Use on synthetic sequential data.
\end{exercise}

\begin{exercise}[10.4: Attention Visualization - $\bigstar\bigstar\bigstar\bigstar$]
Extract and visualize attention weights. Show which parts of input the model focuses on.
\end{exercise}

\clearpage

% =============================================
% SECTION 11: DEBUGGING & BEST PRACTICES
% =============================================

\section{Debugging \& Best Practices}

\subsection{Gradient Checking}

\begin{lstlisting}
from torch.autograd import gradcheck

# Check if gradients are correct
x = torch.randn(3, 4, requires_grad=True, dtype=torch.float64)
test = gradcheck(lambda t: (t**2).sum(), x, eps=1e-6)
print(f"Gradient check: {test}")
\end{lstlisting}

\subsection{Finding NaN/Inf}

\begin{lstlisting}
# Check for NaN in loss
if torch.isnan(loss):
    print("NaN loss detected!")
    # Check inputs, outputs, gradients
    print(f"Input NaN: {torch.isnan(inputs).any()}")
    print(f"Output NaN: {torch.isnan(outputs).any()}")

# Register hook to catch NaN in backward
def check_nan_hook(module, grad_input, grad_output):
    if any(torch.isnan(g).any() for g in grad_output if g is not None):
        print(f"NaN gradient in {module.__class__.__name__}")

for module in model.modules():
    module.register_backward_hook(check_nan_hook)
\end{lstlisting}

\subsection{Gradient Flow Visualization}

\begin{lstlisting}
def check_gradient_flow(model):
    """Check which layers have vanishing/exploding gradients."""
    for name, param in model.named_parameters():
        if param.grad is not None:
            grad_norm = param.grad.norm().item()
            print(f"{name}: {grad_norm:.6f}")
            if grad_norm < 1e-6:
                print(f"  WARNING: Vanishing gradient!")
            if grad_norm > 1e3:
                print(f"  WARNING: Exploding gradient!")
\end{lstlisting}

\subsection{Model Checkpointing}

\begin{lstlisting}
# Save checkpoint
checkpoint = {
    'epoch': epoch,
    'model_state_dict': model.state_dict(),
    'optimizer_state_dict': optimizer.state_dict(),
    'loss': loss,
    'best_val_loss': best_val_loss
}
torch.save(checkpoint, 'checkpoint.pth')

# Load checkpoint
checkpoint = torch.load('checkpoint.pth')
model.load_state_dict(checkpoint['model_state_dict'])
optimizer.load_state_dict(checkpoint['optimizer_state_dict'])
epoch = checkpoint['epoch']
loss = checkpoint['loss']
\end{lstlisting}

\subsection{Random Seed for Reproducibility}

\begin{lstlisting}
import random
import numpy as np

def set_seed(seed=42):
    random.seed(seed)
    np.random.seed(seed)
    torch.manual_seed(seed)
    torch.cuda.manual_seed_all(seed)
    torch.backends.cudnn.deterministic = True
    torch.backends.cudnn.benchmark = False

set_seed(42)
\end{lstlisting}

\subsection{Performance Optimization}

\begin{lstlisting}
# Use mixed precision training
from torch.cuda.amp import autocast, GradScaler

scaler = GradScaler()

for data, target in dataloader:
    optimizer.zero_grad()
    
    with autocast():  # Automatic mixed precision
        output = model(data)
        loss = criterion(output, target)
    
    scaler.scale(loss).backward()
    scaler.step(optimizer)
    scaler.update()

# Pin memory for faster GPU transfer
loader = DataLoader(dataset, batch_size=32, pin_memory=True)

# Use multiple workers
loader = DataLoader(dataset, batch_size=32, num_workers=4)
\end{lstlisting}

\subsection{Common Debugging Checklist}

\begin{itemize}
    \item ✓ Check data shapes at each layer
    \item ✓ Verify data normalization
    \item ✓ Start with small model and small dataset
    \item ✓ Overfit single batch first
    \item ✓ Check learning rate (too high/low?)
    \item ✓ Monitor train and val loss
    \item ✓ Visualize predictions periodically
    \item ✓ Check for NaN/Inf values
    \item ✓ Verify gradient flow
    \item ✓ Use tensorboard for monitoring
\end{itemize}

\clearpage

% =============================================
% SECTION 12: GRAND CHALLENGES
% =============================================

\section{Grand Challenges: Putting It All Together}

\begin{profnote}
These challenges combine multiple concepts. They're designed to be challenging—expect to debug and iterate!
\end{profnote}

\begin{exercise}[Challenge 1: Logistic Regression from Scratch - $\bigstar\bigstar\bigstar$]
\textbf{Goal:} Understand optimization without high-level APIs.

Build binary classifier without \texttt{nn.Module}:
\begin{enumerate}
    \item Create 2-class dataset (200 points)
    \item Manually initialize $W$ and $b$ with \texttt{requires\_grad=True}
    \item Implement forward pass: $z = Wx + b$, $\hat{y} = \sigma(z)$
    \item Compute BCE loss manually
    \item Update parameters with gradient descent (no optimizer!)
    \item Achieve >90\% accuracy
\end{enumerate}

\textbf{Key learning:} Understanding what PyTorch does under the hood.
\end{exercise}

\begin{exercise}[Challenge 2: Custom Autograd Function - $\bigstar\bigstar\bigstar$]
\textbf{Goal:} Extend PyTorch with custom operations.

Implement custom activation: $f(x) = x \cdot \sin(x)$

\begin{lstlisting}
class SinusoidalActivation(torch.autograd.Function):
    @staticmethod
    def forward(ctx, input):
        ctx.save_for_backward(input)
        return input * torch.sin(input)
    
    @staticmethod
    def backward(ctx, grad_output):
        input, = ctx.saved_tensors
        grad_input = grad_output * (torch.sin(input) + input * torch.cos(input))
        return grad_input

# Wrap in nn.Module
class MySinActivation(nn.Module):
    def forward(self, x):
        return SinusoidalActivation.apply(x)

# Test with gradcheck
\end{lstlisting}

Use it in a network and verify gradients are correct.
\end{exercise}

\begin{exercise}[Challenge 3: CNN for CIFAR-10 - $\bigstar\bigstar\bigstar\bigstar$]
\textbf{Goal:} Build a production-quality image classifier.

Requirements:
\begin{enumerate}
    \item Build CNN with residual connections
    \item Use data augmentation (random crop, flip, etc.)
    \item Implement learning rate scheduling
    \item Add early stopping
    \item Achieve >80\% test accuracy
    \item Visualize some misclassified examples
\end{enumerate}

\textbf{Bonus:} Use techniques like mixup or cutout for better accuracy.
\end{exercise}

\begin{exercise}[Challenge 4: Transformer for Sequence-to-Sequence - $\bigstar\bigstar\bigstar\bigstar$]
\textbf{Goal:} Build a complete transformer.

Task: Learn to reverse sequences

\begin{enumerate}
    \item Input: sequence of integers [1, 5, 3, 8, 2]
    \item Target: reversed [2, 8, 3, 5, 1]
    \item Build transformer encoder-decoder
    \item Use teacher forcing during training
    \item Implement beam search for inference (optional)
    \item Achieve 100\% accuracy on sequences up to length 20
\end{enumerate}
\end{exercise}

\begin{exercise}[Challenge 5: Function Approximation for PDEs - $\bigstar\bigstar\bigstar\bigstar\bigstar$]
\textbf{Goal:} Apply deep learning to scientific computing.

Approximate solution to heat equation:
\[
\frac{\partial u}{\partial t} = \alpha \frac{\partial^2 u}{\partial x^2}
\]

\begin{enumerate}
    \item Generate training data by solving PDE numerically
    \item Build MLP or CNN to learn $(x, t) \to u(x,t)$
    \item Train on multiple initial conditions
    \item Test on unseen initial condition
    \item Visualize solution as heatmap over space and time
    \item Compare with numerical solution
\end{enumerate}

\textbf{Success criterion:} Relative error < 5\% on test set
\end{exercise}

\begin{exercise}[Challenge 6: Complete ML Pipeline - $\bigstar\bigstar\bigstar\bigstar\bigstar$]
\textbf{Goal:} Build production-ready system.

Choose any dataset (e.g., from Kaggle) and build complete pipeline:

\begin{enumerate}
    \item Data loading with custom Dataset
    \item Train/val/test split with stratification
    \item Data normalization and augmentation
    \item Model with appropriate architecture
    \item Training with:
    \begin{itemize}
        \item Learning rate scheduling
        \item Early stopping
        \item Model checkpointing
        \item Gradient clipping if needed
        \item Mixed precision training
    \end{itemize}
    \item Comprehensive evaluation:
    \begin{itemize}
        \item Confusion matrix
        \item ROC curve (if classification)
        \item Error analysis
    \end{itemize}
    \item Save final model and document hyperparameters
\end{enumerate}

\textbf{Bonus:} Deploy with ONNX or TorchScript for inference.
\end{exercise}

\subsection{Final Certification}

If you can complete Challenges 3, 4, and 6, you have strong practical PyTorch skills for scientific computing and research.

You're ready to:
\begin{itemize}
    \item Implement papers from scratch
    \item Build custom architectures for your domain
    \item Debug complex training issues
    \item Apply deep learning to scientific problems
\end{itemize}

Keep learning, keep experimenting, and most importantly—keep building!

\clearpage

\end{document}


\appendix
% \input{appendices/appendixA_math_prerequisites}
% \input{appendices/appendixB_pytorch_setup}
% \input{appendices/appendixC_solutions}
% \input{appendices/appendixD_quick_reference}
% \input{appendices/appendixE_resources}

\end{document}